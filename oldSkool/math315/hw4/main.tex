\documentclass{article}
\usepackage{amsmath}
\usepackage{amssymb}
\usepackage{amsthm}
\usepackage[utf8]{inputenc}
\usepackage{amsmath}
\usepackage{amsfonts}
\usepackage[]{graphicx}
\usepackage[a4paper, portrait, margin = 1in]{geometry}
\usepackage{enumitem}
\usepackage{xcolor}

%darkmode
%\pagecolor[rgb]{0.2,0.19,0.18} 
%\color[rgb]{0.92,0.86,0.7}

\newenvironment*{alphenum}{\begin{enumerate}[label= (\alph*)]}{\end{enumerate}}


\usepackage{euler}

\begin{document}
    \huge Problem Set 4 - Thomas Boyko - 30191728
    \normalsize
\begin{enumerate} 

\item %Let $(R, +, \cdot)$ be a ring. Recall, an element $x \in R$ is called idempotent if $x^2 = x$. Suppose every
element in R is an idempotent. Such a ring is called a Boolean ring.
\begin{enumerate}[label= (\alph*)] 
    \item Show that $char(R) = 2$.

        Recalling that $char(R)$ is the order of the multiplicative identity with respect to addition in $R$, we can 
        immediately rule out $1$ from being the characteristic of $R$; $1^1=1\neq 0$.

        And from distributive laws we can see that:
        \[
        1+1=(1+1)^2=(1+1)(1+1)=1(1+1)+1(1+1)=1+1+1+1
        .\] 
        And using additive inverses,
        $1+1=0$ and $char R=2$.

    \item Show that R must be commutative.

        Take $a,b\in R$. (Since $char R=2$, a prime, $R$ is an integral domain.)
        \begin{align*}
            (a+b)&= (a+b)^2 \\
            &= a^2+ab+ba+b^2 \\
            &= a+ab+ba +b\\
            0&= ab+ba \\
            ab&= -ba 
        .\end{align*}

        Well it sure would be convenient to show that each element is its own additive inverse. If this is true
        for $1$ why wouldn't it be true for any element? 

        Take $x\in R$. $x+x=1x+1x=x(1+1)=x 0 =0$. Wow! That was easy. So any element in $R$ is its own inverse, and
        since $ab=-ba$, $ab=ba$ and $R$ is commutative.

    \item For any non-empty set $X$, let $P(X)$ denote its power
    set. Consider the ring $(P(X), \Delta, \cap )$. Show that it is Boolean ring.

    We already know from Problem Set 1 that $(P(X),\Delta)$ is a group. So we must show that $\cap $ is associative
    , maintains closure, has identity and that it distributes over $\Delta$.

    Our identity for $\cap $ is $X$. Since any element $A\in P(X)$ must be a subset of $X$, every element of 
    $A$ is also in $X$. From this we can see that $X\cap A\subseteq A\subseteq X\cap A$ So, $X\cap A=A=A\cap X$ and $X$ is identity under $\cap $.

    Now to show associativity, take $A,B,C\in P(X)$. Let $x\in A\cap (B\cap C)$. Then $x$ must be in $A,B$, and $C$.
    From this we can say $x\in (A\cap B)\cap C$, and the same logic works the other way. So $(A\cap B)\cap C=A\cap (B\cap C)$.

    Now we show the distributive property. Start by showing $(A\cap B) \Delta (A\cap C)\subseteq A\cap (B\Delta C)$. 
    Let $x\in A\cap (B\Delta C)$. Then $x$ is in $A$ and $x$ is
    in $B$ or $C$, but not both. Suppose without loss of generality that $x\in B\setminus C$. then $x\in A\cap B$ but $x\not\in A\cap C$.
    Since $x$ is in one of these sets but not both, it is in their symmetric
    difference, and $x\in (A\cap B)\Delta(A\cap C)$. The other case is identical. So $A\cap (B\Delta C)\subseteq (A\cap B)\Delta(A\cap C)$.

    Now to show the other way. Suppose $x\in (A\cap B)\Delta(A\cap C)$. Then $x$ must be in 
    $A\cap B$ or $A\cap C$ but not both. Since both sets require $x\in A$, we know
    $x\in A$ either way. From this we infer that $x$ must be in $B$ or $C$ but not both. So $x\in B\Delta C$.
    Combining these, $x\in A\cap (B\Delta C)$. So our sets are equal and $\cap $ distributes over $\Delta$. 

    Therefore $(P(X), \Delta, \cap )$ is a ring.

    To show that $P(X)$ is a boolean ring simply requires showing that $A\cap A=A$.
    If $a\in A$, then $a$ is in $A$ and $A$, so $a\in A\cap A$, $A\subseteq A\cap A$. And if $a\in A\cap A$, then
    $a$ is in $A$, $A\cap A\subseteq A$.
    
    So $P(X)$ is a boolean ring.

\end{enumerate}

\item Let R be a commutative ring and I be an ideal of R.
    \begin{enumerate}[label= (\alph*)] 
        \item Define the radical of I as $\sqrt{I} = \{a \in  R : a^n \in  I \text{ for some integer }n > 1\}$. Show that $\sqrt{I} $
        is an ideal of $R$, containing $I$.

        Subgroup: Let $x,y\in \sqrt{I} $. Then there exist $m,n\in Z_{>1}$ so that $x^{m}=0$ and $y^{n}=0$.
        Consider the following binomial expansion, since $R$ is commutative.
        \begin{align*}
            %(x-y)^{mn}&=\sum_{k=1}^{mn} {mn\binom k}x^{mn-k}(-y)^{k}\\
                      %&=x^{mn}+ {mn\binom 1}x^{mn-1}(-y)+\ldots+  {mn\binom mn-1}x^{}(-y)^{mn-1}+(-y)^{mn}\\
                      %&=(x^{m})^{n}+ {mn\binom 1}x^{mn-1}(-y)+\ldots+  {mn\binom mn-1}x^{}(-y)^{mn-1}+(-1)^{mn}(y^{n})^{m}\\
                      %&={mn\binom 1}x^{mn-1}(-y)+\ldots+  {mn\binom mn-1}x^{}(-y)^{mn-1}
        .\end{align*}
        And either $mn-k>m$ or $n$, otherwise $k>m$ or $k>n$, and so one of our two coefficients will become zero in each term of the
        expansion. So $(\sqrt{I},+)$ is a subgroup of $(R,+)$.

        Now we show that $I\subseteq \sqrt{I} $. Let $i\in I$. Then $i^1=i$ must be in $\sqrt{I} $.

        Let $a\in \sqrt{I} $ and $r\in R$. Then by definition of $\sqrt{I} $, we know there exists some $n\in \mathbb{Z}_{>1}$
        so that $a^n\in I$. Then consider $(ar)^{n}=a^{n}r^{n}$ since $R$ is commutative. Since $a^{n}$ is in $I$, 
        an ideal, $(ar)^{n}\in I$, and by definition of $\sqrt{I} $, $ar\in \sqrt{I} $. So 
        $\sqrt{I} $ is an ideal of $R$ containing $I$.

        \item Show that if $I$ is a maximal ideal, then $\sqrt{ I} = I$.

        Let $I$ be maximal. Then since $I\subseteq \sqrt{I} \subseteq R$, either $\sqrt{I}=R$ or
        $\sqrt{I}=I$. If $\sqrt{I} =R$, then $1\in \sqrt{1} $ which would mean for some
        $n\in \mathbb{Z}_{>1},$ $1^{n}\in I$, which would have $I=R$, a contradiction by the definition of ideal.

        \item The set of all prime ideals of $R$ is denoted by Spec(R). Show that
        \[ \sqrt{\{0\} } \subseteq \bigcap_{P\in Spec(R)} P .\] 

        Let $a\in \sqrt{\{0\} } $. Then $a^{n}=0$ for some $n\in \mathbb{Z}_{>1}$. To show the above,
        we must show that $a$ is any prime ideal of $R$. Let $P$ be a prime ideal in $R$.
        Then $0\in P$ since $P$ is a subgroup of $(R,+)$ and must contain additive identity.
        And $a^{n-1}a=0$, so since $P$ is a prime ideal, $a^{n-1}$ or $a$ must be in $P$.

        If $a^{n-1}\in P$, then we split off another $a$, writing $aa^{n-2}\in P$. Again, one of these 
        must be in $P$, and we can continue until this happens, or untill we obtain $n-k=1$, 
        since $n> 1$.

    \end{enumerate}
\item Let R be a commutative ring and R[x] denote the ring of polynomials with coefficients in R.
    \begin{enumerate}[label= (\alph*)]
        \item For $\alpha \in  R$, define the evaluation map, $ev_\alpha : R[x] \to  R$ by $ev_\alpha(f(x)) = f(\alpha)$. Show that it
        is a ring homomorphism.

        Let $f(x),g(x)\in R[x]$, so that $f(x)=a_0+a_1x+\ldots$, $g(x)=b_0+b_1x+\ldots$. Then:
        \begin{align*}
            ev_0(f(x)+g(x))&=ev_0(a_0+b_0+(a_1+b_1)x+\ldots)\\
            &= a_0+b_0+(a_1+b_1)\alpha+\ldots \\
            &= a_0+a_1\alpha+\ldots+b_0+b_1\alpha+\ldots \\
            &= ev_\alpha(f(x))+ev_\alpha(g(x))
        .\end{align*}
        So $ev_\alpha$ preserves addition.
        \begin{align*}
            ev_\alpha(f(x)g(x))&= ev_0(\sum_{k=1}^{\max \{m,n\} }\sum_{i+j=k}^{} x^{k}a_ib_j \\
            &= ev_\alpha(\sum_{k=0}^{\max \{m,n\} }\sum_{i+j=k}^{} x^{k}a_ib_j) \\
            &= ev_\alpha(\sum_{k=0}^{\max \{m,n\} }x^{k}\sum_{i+j=k}^{} a_ib_j) \\
            &= \sum_{k=0}^{\max \{m,n\} }\alpha^{k}\sum_{i+j=k}^{} a_ib_j 
            &= \sum_{k=0}^{\max \{m,n\} }\sum_{i+j=k}^{}\alpha^{k} a_ib_j 
            &= f(\alpha)g(\alpha) \\
            &= ev_\alpha(f)ev_\alpha(g) \\
        .\end{align*}
        So $ev_\alpha$ is a ring homomorphism.

        \item For $\alpha = 0$, what is the $\ker(ev_0)$ ?

            Claim: $\ker(ev_0)=\{a_1x+a_2x^2+\ldots\in R[x]\}$, or the set of all polynomaials with a zero constant
            coefficient.

            Let $f(x)\in \{a_1x+a_2x^2+\ldots\in R[x]\}$. Then $f(x)=a_1x+a_2x^2+\ldots$ where $a_i\in R$.
            Then $f(0)=a_10+a_2 0^2+\ldots=0$ and $f\in \ker(ev_0)$.

        \item Is $\ker(ev_0)$ a prime ideal? Is it maximal? What extra condition do you need to impose
        on R, for this ideal to be prime or, maximal?

        $\ker(ev_0)$ is a prime ideal when $R$ is a domain. To show this, let $f(x)g(x)\in \ker(ev_0)$.
        Then we know that the constant term of $fg$ must be zero. We know the constant term of
        $fg$ to be $\sum_{i+j=0}^{} a_ib_j$, assuming that coefficients of $f$ are given by $a_i$ and
        $g$ given by $b_j$. Then $a_0b_0$ must be zero, which is true for all $f$ and $g$ only 
        in a domain. 

        $\ker(ev_0)$ is maximal when $\mathrm{Im}\,(ev_0)$ is a field. From the first isomorphism theorem, and since
        $ev_0$ is a homomorphism, we know that $R[x] / \ker(ev_0)\cong \mathrm{Im}\,(ev_0)$. And when $\mathrm{Im}\,(ev_0)$ is
        a field, we know that $\ker(ev_0)$ must be maximal.

    \end{enumerate}
\item 
    \begin{enumerate}[label= (\alph*)] 
        \item   Let $\varphi:R\to  S$ be a ring homomorphism. Show that for any ideal $J \subseteq S$, the preimage
            $\varphi^{-1}(J) = {r \in  R : \varphi(r) \in  J}$ is an ideal of $R$. (That is, the preimage of 
            an ideal under a ring homomorphism is an ideal.)

            First we show that $\varphi^{-1}(J)$ is a subgroup of $R$. Clearly $0\in \varphi^{-1}(J)$
            since $\phi(0)=0$.

            Let $a,b\in \varphi^{-1}(J)$.
            Then $\varphi(a-b)=\varphi(a)-\varphi(b)$ since $\varphi$ is a homomorphism.
            And since $\varphi(a),\varphi(b)$ are in $J$, $\varphi(a-b)\in J$.
            So $\varphi^{-1}(J)$ is a group w.r.t $+$.

            Let $\varphi:R\to  S$ be a ring homomorphism and $J\subseteq S$.
            Then let $i\in \varphi^{-1}(J)$. Then for some $j\in J$, 
            $\varphi(i)=j$. Let $r\in R$, and suppose $\varphi(r)=s$. Since $J$ is an ideal of $S$, 
            $\varphi(ri)=\varphi(r)\varphi(i)=js\in J$. So $ri \in \varphi^{-1}(J)$, and 
            $\varphi^{-1}$ is an ideal of $J$.

        \item Show that the image of an ideal under an onto ring homomorphism is an ideal. (That
        is, if $\varphi : R \to  S$ is an onto ring homomorphism, then for any ideal I of R the image
        $\varphi(I) = {\varphi(r) : r \in I}$ is an ideal of $A$.)

        Begin by showing that $\varphi(I)$ is a subgroup of $(S,+)$. Clearly since $0\in I$ (since
        $I$ is a subgroup of $(R,+)$. So $\varphi(0)=0\in \varphi(I)$.

        Now let $a,b\in \varphi(I)$. Then there exists $c,d\in I$ so that $\varphi(c)=a$ and
        $\varphi(d)=b$. And since $\varphi$ is a homomorphism 
        $a-b=\varphi(c)-\varphi(d)=\varphi(c-d)\in \varphi(I)$, and $\varphi(I)$ is a subgroup of 
        $(S,+)$.

        Let $i\in I$ so that $\phi(i)=j\in S$. Then let $s\in S$. We know since $\varphi$ is onto
        that there exists $r\in R$ so that $\varphi(r)=s$. Then $js=\varphi(i)\varphi(r)=\varphi(ir)$
        since $\varphi$ is a homomorphism. And $ir\in I$ since $i$ is in the ideal $I$.
        And since $js$ is the image of $ir$ under $\varphi$, $js\in \varphi(I)$ and $\varphi(I)$ is
        an ideal of $S$.

        \item Give an example which shows that the image of an ideal under a ring homomorphism
        need not be an ideal if the map is not onto. 

        Consider the given mapping $f:\mathbb{Z}\to \mathbb{Q}, f(x)=x$, where $\mathrm{Im}\,f=\mathbb{Z}$,
        which is not an ideal for $\mathbb{Q}$ since given $\frac{1}{2}\in \mathbb{Q}$, and $3\in \mathbb{Z}$,
        $\frac{3}{2}\not\in \mathbb{Q}$. So the image of $\mathbb{Z}$, which is an ideal
        for $\mathbb{Z}$, is not an ideal of $\mathbb{Q}$.

        \item Prove that if $I$ is an ideal of a ring $R$, there is an inclusion preserving bijection between
        the ideals of $R/I$ and the ideals of $R$ which contain $I$.
\iffalse
Translation : Show that there is a bijection Γ : { ideals J of R such that I ⊆ J} → { ideals of R/I}
such that Γ is inclusion-preserving i.e., J ⊆ J′ if and only if Γ (J) ⊆ Γ (J′).
Hint : Consider the map π : R → R/I, given by π(r) = r + I. Apply (a) and (b).
\fi

\begin{proof}
    Let $I$ be an ideal of $R$. Consider $\pi:R\to R /I\, \pi(r)=r+I$, and 
    $\Gamma: \{ideals J of R such that I \subseteq J \} \to 
    \{ideals of R/I\}$ 

\end{proof}
    my statistics group will be more mad at me if i dont finish that assignment than i will be at myself not finishing this one so i think im done here :p
    \end{enumerate}
\end{enumerate}
\end{document}
