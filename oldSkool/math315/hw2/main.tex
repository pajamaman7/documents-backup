\documentclass{article}
\usepackage[most,many,breakable]{tcolorbox}
\usepackage{amsmath}
\usepackage{amssymb}
\usepackage{amsthm}
\usepackage[]{thmbox}
\usepackage{blindtext}
\usepackage[utf8]{inputenc}
\usepackage{amsmath}
\usepackage{amsfonts}
\usepackage[]{graphicx}
\usepackage[legalpaper, portrait, margin = 1in]{geometry}
\usepackage{enumitem}


\usepackage{xcolor}

%\pagecolor[rgb]{0.2,0.19,0.18} 
%\color[rgb]{0.92,0.86,0.7}

\newtheorem[L]{le}{Lemma}[subsection]
\newtheorem[L]{th}[le]{Theorem}
\newtheorem[L]{df}[le]{Definition}
\newtheorem[L]{ex}[le]{Example}
\newtheorem[L]{pf}[le]{Proof}


\newcommand{\nl}{\newline}

\newcommand{\real}{\mathbb{R}}
\newcommand{\complex}{\mathbb{C}}
\newcommand{\integer}{\mathbb{Z}}
\newcommand{\rational}{\mathbb{Q}}
\newcommand{\lxor}{\oplus}
\newcommand{\then}{\Rightarrow}
\usepackage{scsnowman}

\begin{document}
    \huge Problem Set 2 - Thomas Boyko - 30191728
    \normalsize
\begin{enumerate} 

    \item A group $G$ and a subset $H \subseteq  G$ is given below. Show that $H$ is a subgroup of $G$.
    \begin{enumerate}[label= (\alph*)] 
        \item $G = GL_n(\mathbb{R})$ and $H$ is the subset of all diagonal matrices
            in $G$.

            Clearly the identity $I$ is in $H$ since the identity matrix is
            a diagonal matrix.

            Take $A,B\in H$. Then $A,B$ are diagonal. Consider $B^{-1}$.
            This must be diagonal since the inverse of a diagonal matrix is 
            also diagonal.

            And since $A,B^{-1}$ are diagonal their product must be diagonal since 
            the product of diagonal matrices is also diagonal.
            So $AB^{-1}$ is a diagonal matrix, and $AB^{-1}\in H$. Therefore, $H
            $ is a subgroup of $GL_N(\mathbb{R}).$

        \item Fix a positive integer $n$. Let $G$ be any Abelian group, and 
            $H = \{ g \in G : g^n = e\}$.

            The identity of $G$ must be in $H$ because $e^{n}=e$ for any integer $n$.

            Let $a,b\in H$. Then $a^{n}=e=b^{n}$. This also gives us:
            \begin{align*}
                e&= a^{n}b^{-n}\\
                &= a^{n}(b^{-1})^{n} \\
                &= (ab^{-1})^{n} &\text{Since G is abelian}
            .\end{align*}
            And since $ab^{-1}\in H$, $H$ is a subgroup of $G$.

        \item $G$ is any Abelian group, $H$ is the subset of all elements of
            finite order.

            Trivially, the identity of $G$ is in $H$ since $o(e)=1$ is finite.

            Then let $a,b\in H$. Both $a,b$ have finite order. So write 
            $o(a)=m, \,o(b)=n$. Let $l=lcm(m,n)$, so $m|l$, and $n|l$.

            Write $mc=l$ and $nd=l$, for some $c,d\in \mathbb{Z}$.

            \begin{align*}
                (ab^{-1})^{l}&= a^{l}b^{-l}&\text{Since $G$ is abelian} \\
                &= a^{mc}b^{-nd} \\
                &= (a^{m})^{c}(b^{n})^{-d} \\
                &= e^{c}e^{-d} \\
                &= e 
            .\end{align*}

            From this we can say the order of $ab^{-1}$ must divide $l$, $o(ab^{-1})$ is less than or equal to $l$, meaning the order is finite and $H$ is a subgroup of $G$.
    \end{enumerate}

\item For a positive integer $n$, recall the group $n$-th roots of unity,
    denoted by $\mu_n$.
    \begin{enumerate}[label= (\alph*)] 
        \item Show that if d|n, then $\mu_d$ is a subgroup of $\mu_n$.
            
            Begin with showing $\mu_d\subseteq \mu_n$. Suppose $d|n$, so $dk=n$ for some 
            $k\in \mathbb{Z}$. 

            Suppose $z\in \mu_d$. Then $z^{d}=1$. Then:
            \begin{align*}
                1=1^{k}&=(z^{d})^{k} \\
                &= z^{dk} \\
                &= z^{n}
            .\end{align*}
            So $z^{n}=1$, $z\in \mu_n$, and $\mu_d\subseteq \mu_n$.

            We can show easily that $1\in \mu_d$, since $1^{d}=d$ for any $d\in \mathbb{Z}$.

            Now take some $a,b\in  \mu_d$. We say that $a^{d}=b^{d}=1$. This gives $a^{d}b^{-d}=1$, and since multiplication in $\mathbb{C}$ is commutative, $(ab^{-1})^{d}=1$, and $ab^{-1}\in \mu_d$. So $\mu_d$ is a subgroup of $\mu_n$.

        \item Describe all subgroups of $\mu_n$.

            By the fundamental theorem of finite groups, we know that since $\mu_n$ is generated
            by $e^{\frac{2\pi i}{n}}$, any subgroup $H=\left<e^{\frac{2\pi id}{n}} \right>$, with
            $d|n$. So $n=dk$ and $H=\left<\frac{2\pi i}{k} \right>$. So any subgroup of 
            $\mu_n$ is $\mu_k$ for some $k|n$.

        \item If $H$ is any finite subgroup of $(C \setminus \{0\} , \cdot )$,
            then show that there exists a positive
            integer k such that H = $\mu_k$.

            Let $H$ be a finite subgroup of $(C \setminus \{0\})$, and write $|H|=n$.

            Since all elements of $H$ have finite order, let $l$ be the $lcm$ of all elements of $H$.
            Then for any $h\in H, h^{l}=1$, since the order of $h$ must divide $l$.

            So any $h\in H$ must be an $l$th root of unity. Since $H$ is a subgroup of 
            $(\mathbb{C}\setminus \{0\} ,\cdot )$, it is closed under $\cdot $,
            and it must be a subset of $\mu_l$, we can say from part b that $H$ must be $\mu_k$ for 
            some $k|l$.
    \end{enumerate}
\item Let $f : G \to H$ be a group homomorphism.
    \begin{enumerate}[label= (\alph*)] 
        \item Show that if $G = \langle g\rangle$, then $\mathrm{Im} (f) = \langle f (g)\rangle$.

            We begin by showing $\mathrm{Im}(f)\subseteq \left<f(g) \right>$

            Suppose $y\in \mathrm{Im}(f)$. Then there exists some $x\in G$
            so that $f(x)=y$. Since $G$ is cyclic, we can write $g^{k}=x$ 
            for some integer $k$ and $g\in G$. Now we can apply $f$
            on both sides, giving $f(g^{k})=f(x)$. Since $f$ is a homomorphism, 
            $f(g^{k})=f(g)^{k}=f(x)=y$, and $y\in \left<f(g) \right>$.

            Now we show $\left<f(g) \right>\subseteq \mathrm{Im}(f)$.

            Suppose $y\in \left<f(g) \right>$. Then there exists some $k\in \mathbb{Z}$ so that 
            $f(g)^{k}=y$. Since $f$ is a homomorphism, $f(g)^{k}=f(g^{k})=y$ and $y$ is in 
            $\mathrm{Im}(f)$.

        \item Let $G \cong H$, show that $G$ is cyclic, if and only if $H$ is cyclic.
            
            We show $G$ is cyclic $\implies$ $H$ is cyclic. The converse is identical.

            Let $G$ be cyclic, $\left<g \right>= G$.
            And since $G\cong H$, there exists a bijection $f:G\to H$. 

            In order to show that $H$ is cyclic, we will show that any $h\in H$ can be generated
            as some power of $f(g)$, or that $H=\left<f(g) \right>$.

            Let $h\in H$. Since $f$ is bijective, there exists a uniques $x\in G$ so that 
            $f(x)=h$. And since $g$ generates $G$, for some $k\in \mathbb{Z}$, $g^{k}=x$.
            If we apply $f$ to both sides of this equality, we get $f(g^{k})=f(x)$.

            Since $f$ is a homomorphism, $f(g)^{k}=h$. So any $h\in  H$ is some power of
            $f(g)$, and $H$ is cyclic.

        \item Recall, $N$ is a normal subgroup if $gNg^{-1} = N$ for all $g\in G$.
            Show that $\ker( f )$ is a normal subgroup of $G$.

            $\ker(f)$ is trivially a subset of $G$, since by definition each element of $\ker(f)$ 
            must be in $G$. 

            First we show $\ker(f)$ is a subgroup, then we will show that it is a normal subgroup. 

            Let $a,b\in \ker(f)$. Then $f(a)=e=f(b)$. Next note that $f(b^{-1})=e=f(b)^{-1}$ since $f$
            is a homomorphism. Now we note that $f(a)f(b^{-1})=e$, and again by homomorphism, 
            $f(ab^{-1})=e$, and $\ker(f)$ is a subgroup of $G$.

            Let $g\in G$ and $h\in \ker(f)$. We want to show that $ghg^{-1}\in \ker(f)$.

            Now we have
            \begin{align*}
                f(ghg^{-1})&=f(g)f(h)f(g^{-1}) &\text{by homomorphism} \\
                           &= f(g)ef(g)^{-1}&\text{since }h\in \ker(f)\text{ and by homom.} \\
                           &= f(g)f(g)^{-1} \\
                           &= e
            .\end{align*}
            Therefore $ghg^{-1}\in \ker(f)$ and $\ker(f)$ is a normal subgroup of $G$.

    \end{enumerate}
\item For any two groups $G_1$ and $G_2$, consider the set of all homomorphisms:
    $Hom(G_1, G_2) := \{ f : G_1 \to G_2 : f\text{ is a group homomorphism}\}.$
    \begin{enumerate}[label= (\alph*)] 
        \item  Show that for any group G, there is a bijective map $\alpha : Hom(\mathbb{Z}, G) \to G$

            Choose the map $\alpha:Hom(\mathbb{Z},G)\to G$ given by 
            $\alpha (f)=f(1)$ for any $f \in  Hom(\mathbb{Z},G)$.

            First we show that $\alpha$ is injective. Suppose $\alpha(f)=\alpha(g)$ for some
            homomorphisms $f,g$ from $\mathbb{Z}\to G$. Then:
            \begin{align*}
                \alpha(f)&=\alpha(g)\\
                f(1)&= g(1) \\
                f(1)^{k}&= g(1)^{k} &\text{For some }k\in \mathbb{Z}\\
                f(k)&=g(k) &\text{By Homomorphism}
            .\end{align*}

            And since $f, g$ agree on all inputs from $\mathbb{Z}$, they are equal and
            $\alpha$ is injective.

            Now we show surjectivity. Let $y\in G$. We want to show that there exists some 
            $f:\mathbb{Z}\to G$ so that $f$ is a homomorphism, and $\alpha(f)=y$.

            So we choose $f(1)=y,$ and for any $k\in \mathbb{Z}$, $f(k)=y^{k}$ This satisifies our
            second condition, and we only need to show that 
            $f$ is a homomorphism.

            For $a,b\in \mathbb{Z}$, we get $f(a+b)=y^{a+b}=y^{a}y^{b}=f(a)f(b)$. So $f$ is a 
            homomorphism and $\alpha$ is surjective

            So $\alpha$ is bijective and we are done.
            
        \item List all elements of $Hom(\mathbb{Z}, S_3)$. How many elements does it have?

            As we saw above, a homomorphism from a cyclic group is dependent on where we send
            the generator, and the image of the homomorphism is generated by the element we map
            the generator to. So we have our six homomorphisms (all for $x\in \mathbb{Z}$):
            \begin{align*}
                \phi_1: x&\mapsto e\\
                \phi_2: x&\mapsto (1\,2)^{x}\\
                \phi_3: x&\mapsto (1\,3)^{x}\\
                \phi_4: x&\mapsto (2\,3)^{x}\\
                \phi_5: x&\mapsto (1\,2\,3)^{x}\\
                \phi_6: x&\mapsto (1\,3\,2)^{x}
            .\end{align*}
    \end{enumerate}

\item Let $\varphi(n)$ denote the Euler’s totient function.
Give a group theoretic argument to show the famous
identity from elementary number theory:
\[
\sum_{d|n}^{} \varphi(d)=n
.\] 
Hint : Let $G$ be a cyclic group of order $n$. Partition $G$ into subsets $S_d$ of elements of order 
$d$.
Then count the number of elements in each $S_d$.

Suppose $G$ is a cyclic group with $|G|=n$, and that $G$ is generated by $g$. Then for each divisor $d|n$, we have a unique cyclic
subgroup $S_d$ of order $d$. And each subgroup $S_d$ is generated by $g^{d}$. 

Stuck here :p
\end{enumerate}
\end{document}
