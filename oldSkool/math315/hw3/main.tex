\documentclass{article}
\usepackage{amsmath}
\usepackage{amssymb}
\usepackage{amsthm}
\usepackage[utf8]{inputenc}
\usepackage{amsmath}
\usepackage{amsfonts}
\usepackage[]{graphicx}
\usepackage[a4paper, portrait, margin = 1in]{geometry}
\usepackage{enumitem}
\usepackage{xcolor}

%darkmode
%\pagecolor[rgb]{0.2,0.19,0.18} 
%\color[rgb]{0.92,0.86,0.7}

\newenvironment*{alphenum}{\begin{enumerate}[label= (\alph*)]}{\end{enumerate}}


\begin{document}
    \huge Problem Set 3 - Thomas Boyko - 30191728
    \normalsize
\begin{enumerate} 

    \item Consider the following subset of $S_4$:
$$K = \{e, (1 2)(3 4), (1 3)(2 4), (1 4)(2 3)\}.$$
\begin{enumerate}[label= (\alph*)] 
    \item  Show that $K \trianglelefteq  S_4$.

        We write the Cayley table for $K$, and we observe closure which is sufficient to show that
        $K$ is a subgroup thanks to the fact that $S_4$ and $K$ are finite groups.
        \[
\begin{array}{c|cccc}
\cdot & e & (1\,2)(3\,4) & (1\,3)(2\,4) & (1\,4)(2\,3) \\
\hline
e & e & (1\,2)(3\,4) & (1\,3)(2\,4) & (1\,4)(2\,3) \\
(1\,2)(3\,4) & (1\,2)(3\,4) & e & (1\,4)(2\,3) & (1\,3)(2\,4) \\
(1\,3)(2\,4) & (1\,3)(2\,4) & (1\,4)(2\,3) & e & (1\,2)(3\,4) \\
(1\,4)(2\,3) & (1\,4)(2\,3) & (1\,3)(2\,4) & (1\,2)(3\,4) & e
\end{array}
\]

        Take some $\tau\in K$, and any $\sigma\in S_4$. Since conjugation in $S_n$ preserves cycle structure, and $K$ contains all the products of disjoint $2-$cycles in $S_4$, $\sigma\tau\sigma^{-1}$ is a product of disjoint $2-$cycles and $\sigma\tau\sigma^{-1}\in K$.

    \item  Write down the set of distinct left cosets of $K$, i.e. list the elements of $\frac{S_4}{K}$.

        \begin{align*}
            eK&=K\\
            (1\,2)K&= \{(1\,2),(3\,4),(1\,3\,2\,4),(1\,4\,2\,3)\}  \\
            (2\,3)K&= \{(2\,3),(1\,3\,4\,2),(1\,2\,4\,3),(1\,4)\}  \\
            (1\,3)K&= \{(1\,3),(1\,2\,3\,4),(1\,4),(4\,2)\}  \\
            (1\,2\,3)K&= \{(1\,2\,3),(1\,3\,4),(2\,4\,3),(1\,4\,2)\}  \\
            (1\,3\,2)K&= \{(1\,3\,2),(1\,4\,3),(3\,4\,2),(2\,4\,1)\}  \\
        .\end{align*}

    \item  Show that $S_4/K \cong S_3$.

        Above we have written each coset of $K$ using a cycle free of $4$. Knowing 
        $S_3=\{e,(1\,2),(1\,3),(1,2),(1\,2\,3),(1\,3\,2)\} $ we can see each element of $S_3$
        has one coset representation in $K$,
        and since cosets partition $S_4$ we know that each element of $S_3$ will appear in exactly 
        one coset.

        So take the mapping $f:S_4/K\to S_3,\, f(\sigma K)=\sigma$. Thanks to how we wrote our
        above cosets we can see that this is both onto and one-to-one. As well since each coset has
        only one representation in terms of a cycle in $S_3$, our mapping is well-defined.
        All that remains is to check homomorphism.

        Take $\sigma K,\tau K\in S_4 /K$. Then:
        \[
        f(\sigma K\tau K)=f(\sigma\tau K)=\sigma\tau=f(\sigma K \tau K)
        .\] 
        So $f$ is a homomorphism, and since $f$ is a bijection, it is an isomorphism, and therefore
        $S_4/K\cong S_3$.

\end{enumerate}
\item Let $G$ be a group and H be a subgroup of G.
    \begin{enumerate}[label= (\alph*)] 
        \item Show that $H \trianglelefteq  G$ if and only if $H$ is a union of conjugacy classes.

            \paragraph{$\implies$:} Suppose $H\trianglelefteq G$ and let $C=\bigcup_{h\in H} Cl(h)$.

            We will show that $H=C$. Clearly every $h\in H$ is in $C$, since the conjugacy class for
            $h$ will contain $h=ehe^{-1}$. So $H\subseteq C$

            Now let some $c\in C$, where $c\in Cl(h_i)$ for some $h_i\in H$. Then for some $g\in G$,
            $c=gh_ig^{-1}$ which means that $c\in H$ since $H\trianglelefteq G$. Therefore 
            $C\subseteq H$, $C=H$ and $H$ is a union of conjugacy classes.

            \paragraph{$\impliedby$:} Suppose $H$ is a union of $n$ conjugacy classes.
            So for any $h\in H$, $h\in Cl(h_i)$, where $i\in \{1,2,\ldots,n\} $.
            This means that for any $g\in G$, we can write $h=gh_ig^{-1}$, and $g^{-1}hg=h_i$. 
            Since $h_i\in H$, for any $h\in H$ and any $g\in G$, $g^{-1}hg\in H$,
            and $g^{-1}Hg\subseteq H$ and $H\trianglelefteq G$.

        \item Suppose $|G| = 20$, and the class equation of $G$ is given by $20 = 1 + 4 + 5 + 5 + 5$
            Does $G$ have a subgroup of order 4 ? what about order 5 ? Can $G$
            have a normal subgroup of order 4? what about order 5 ? Justify.

            By Cauchy's Theorem, $G$ must have an element of order 5, and a subgroup generated by
            that element of order 5. $G$ may not necessarily have a subgroup of order $4$ since $4$
            is not prime.

            By what we proved above, we can make a subgroup of order 5 from the singleton $\{e\} $ and
            the class of order $4$. So $G$ has a normal subgroup of order 5. However there is no way
            to make a subgroup of order $4$, since the only way to make $4$ is with the class
            containing $4$ elements, which would not contain identity. So $G$ does not have a normal 
            subgroup of order $4$.

    \end{enumerate}
\item \begin{enumerate}[label= (\alph*)] 
        \item  Deduce with proper justification, the class equation of the dihedral group $D_4$.
    
            Write out the center and conjugacy classes for $D_4$.
            \begin{align*}
                Z(D_4)&= \{e,r^2\}  \\
                Cl(r)&= \{r,r^3\}  \\
                Cl(s)&=\{s,s r^2\} \\
                Cl(sr)&= \{s r,s r^3\}  
            .\end{align*}

            And so our class equation is $|D_4|=1+1+2+2+2$.

        \item Deduce with proper justification, the class equation of $S_4$.

            Write out the center and conjugacy classes for $S_4$, this is made easier by 
            the fact that cycle structure is maintained by conjugation.

            \begin{align*}
                Z(S_4)&= \{e\}  \\
                Cl((1\,2))&=\{(1\,2),(1\,3),(1\,4),(2\,3),(2\,4),(3\,4)\} \\
                Cl((1\,2\,3))&=\{(1\,2\,3),(1\,3\,2),(1\,2\,4),(1\,4\,2)
                            ,(1\,3\,4),(1\,4\,3),(2\,3\,4),(2\,4\,3)\} \\
                Cl((1\,2\,3\,4))&=\{(1\,2\,3\,4),(1\,2\,4\,3),(1\,3\,2\,4),
                            (1\,3\,4\,2),(1\,4\,2\,3),(1\,4\,3\,2)\} \\
                Cl((1\,2)(3\,4))&=\{(1\,2)(3\,4),(1\,3)(2\,4),(1\,4)(2\,3)\} 
            .\end{align*}
            So our class equation is given by:
            \[
            |S_4|=1+6+8+6+3
            .\] 

    \end{enumerate}
\item This question is all about finding an appropriate homomorphism and directly applying the
first isomorphism theorem. Show that
\begin{enumerate}[label= (\alph*)] 
\item $S_n/A_n \cong \mathbb{Z}_2$.

    Consider the function:
    \[
    f:S_n\to\mathbb{Z}_2,\quad f(\tau)=\begin{cases}
        [0],\quad &\text{If $\sigma$ is an even permutation}\\
        [1],\quad &\text{If $\sigma$ is an odd permutation}
    \end{cases}
    .\] 

    We begin by checking well-definedness. Suppose $\sigma_1=\sigma_2$, and $\sigma_1,\sigma_2\in S_n$.
    Since the two are equal, they will be both even or both odd. So they will both be mapped to 
    $[1]$ or both mapped to $[0]$.

    Now for homomorphism: Suppose $\sigma_1,\sigma_2\in S_n$. Take $f(\sigma_1,\sigma_2)$.
    If both $\sigma_1, \sigma_2$ are even or odd cycles, their product will be even. So in this case
    $f(\sigma_1\sigma_2)=[0]=f(\sigma_1)f(\sigma_2)$. And if exactly one of $\sigma_1,\sigma_2$ is
    odd, then their product will be odd. So $f(\sigma_1\sigma_2)=[0]=f(\sigma_1)f(\sigma_2)$, and since
    this covers every case, $f$ is a homomorphism.

    What is the kernel of this transformation? All elements mapped to $[0]$ will be even permutations, 
    so $\ker f=A_n$ since $A_n$ is the group of only even permutations.

    And the image of this homomorphism is $\mathbb{Z}_2$, since we will be able to find both an even and
    an odd cycle in any $S_n, $ when $n\ge 2$.

    So by the first isomorphism theorem, $S_n/A_n \cong \mathbb{Z}_2$.
    
\item $GL_n(\mathbb{Q})/SL_n(\mathbb{Q}) \cong (\mathbb{Q} ,  \cdot)$.

    Consider:
    \[
    f:GL_n(\mathbb{Q})\to\mathbb{Q},\quad f(A)=\det A
    .\] 

    First we check well-definedness of $f$. Let $A=B$ where $A,B\in GL_n(\mathbb{Q})$. Then
    $f(A)=\det A=\det B=f(B)$ so $f$ is well-defined.

    Now we check homomorphism. Let $A,B\in GL_n(\mathbb{Q}).$ Then 
    $f(AB)=\det AB=\det A \det B=f(A)f(B)$

    The kernel of this homomorphism is given by all $A$ so that $\det A=1$, which is the definition
    of $SL_n(\mathbb{Q})$, so $\ker f = SL_n(\mathbb{Q})$.

    And the image of this homomorphism is $\mathbb{Q}$, since all the entries of a matrix in 
    $GL_n(\mathbb{Q})$ are rational, so their products and sums will be in $\mathbb{Q}$ since
    $\mathbb{Q}$ is closed under multipliation and addition.

    So by the first isomorphism theorem, 
    $GL_n(\mathbb{Q})/SL_n(\mathbb{Q}) \cong (\mathbb{Q} ,  \cdot)$.

\item $\mathbb{R}/\mathbb{Z} \cong C^0$.

    Consider the mapping:
    \[
    f: \mathbb{R}\to\mathbb{C}^{0},\quad f(\theta))=e^{2\pi i\theta}, \quad k\in \mathbb{Z}
    .\] 
    We check well-definedness, let $\theta=\varphi\in \mathbb{R}$. Then
    \[
    f(\theta)=e^{2\pi i\theta}=e^{2\pi i\varphi}=f(\varphi)
    .\] 
    So $f$ is well-defined.

    Next we check homomorphism. Let $\theta,\varphi\in \mathbb{R}$.
    $$f(\theta+\varphi)=e^{2\pi i(\theta+\phi)}=e^{2\pi i\theta+2\pi i\varphi}=
    e^{2\pi i\theta}e^{2\pi i\varphi }=f(\theta)f(\varphi)$$

    So $f$ is a homomorphism.

    Next observe the image of $f$. Let $z=f(\theta)$ for some $\theta\in \mathbb{R}$.
    Then $z=e^{i 2\pi \theta}$, and by taking the modulus of both sides we see that 
    $|z|=1$. So the image of $f$ is all elements in $C$ with modulus 1, or 
    $\mathrm{Im}\,f=\mathbb{C}^{0}$.

    Now we check the kernel of $f$. Suppose $f(\theta)=1$ for some $\theta\in \mathbb{R}$.
    Then $e^{2\pi i\theta}=1$, and $\cos 2\pi\theta+i\sin 2\pi \theta=1$, which is true only
    for $\theta\in \mathbb{Z}$, meaning $\ker f=\mathbb{Z}$.

    So by the first isomorphism theorem, $\mathbb{R}/ \mathbb{Z}\cong \mathbb{C}^{0}$.
\end{enumerate}
\end{enumerate}
\end{document}
