\documentclass{article}
\usepackage[most,many,breakable]{tcolorbox}
\usepackage{amsmath}
\usepackage{amssymb}
\usepackage{amsthm}
\usepackage[]{thmbox}
\usepackage{blindtext}
\usepackage[utf8]{inputenc}
\usepackage{amsmath}
\usepackage{amsfonts}
\usepackage[]{graphicx}
\usepackage[legalpaper, portrait, margin = 1in]{geometry}
\usepackage{enumitem}


\usepackage{xcolor}

%\pagecolor[rgb]{0.2,0.19,0.18} 
%\color[rgb]{0.92,0.86,0.7}

\newtheorem[L]{le}{Lemma}[subsection]
\newtheorem[L]{th}[le]{Theorem}
\newtheorem[L]{df}[le]{Definition}
\newtheorem[L]{ex}[le]{Example}
\newtheorem[L]{pf}[le]{Proof}


\newcommand{\nl}{\newline}

\newcommand{\real}{\mathbb{R}}
\newcommand{\complex}{\mathbb{C}}
\newcommand{\integer}{\mathbb{Z}}
\newcommand{\rational}{\mathbb{Q}}
\newcommand{\lxor}{\oplus}
\newcommand{\then}{\Rightarrow}

\begin{document}
    \huge Title - Thomas Boyko - 30191728
    \normalsize
\begin{enumerate} 

    \item Let X be any set and let P (X ) denotes its power set, i.e.
$P(X) = \{A : A \subset X \}$, all subsets of $X$ .
Define the operation on $P (X ) : A\Delta B = (A \cup B) \setminus (A \cap B)$.
\begin{enumerate}[label= (\alph*)] 
    \item Show that $(P (X ), \Delta)$ forms a group. Is it Abelian ?
        
        \begin{enumerate}[label= (\roman*)]
        \item   We begin with closure. Take two sets $A,B$ in $P(X)$. Suppose we have an 
            element $x \in A\Delta B$. 
            We must show that $x$ is also in $X$, hence that $A\Delta B\in P(X)$.

            Since $x\in A\Delta B,$ $x$ must be in $A\cup B$ but not $A\cap B$. In other words,
            $x$ is in $A$ or $B$ but not both. Since $A$ and $B$ are both subsets of $X$,
            $x\in X$. So our group is closed.

        \item Next is identity. We must show that there exists some $E\subseteq X$ so that for 
            any $A\in P(X)$, $E\Delta A=A=A\Delta E$.

            We can choose the null set $\varnothing$. Of course $\varnothing\subseteq X$ since 
            the null set is a subset of any set. Now we examine 
            $\varnothing\Delta A=(A\cup \varnothing)\setminus(A\cap \varnothing)$ for any
            $A\in P(X)$. Notice that $\varnothing\cup A=A$ and $\varnothing\cap A=\varnothing$, as 
            well as $\varnothing\cup A=A\cup \varnothing$ and $\varnothing\cap A=A\cap \varnothing$.
            So we now have $\varnothing\Delta A=A\setminus\varnothing$, in both the cases 
            ($A\Delta \varnothing$ and $\varnothing\Delta A$). 

            So for an element to be in $A\Delta \varnothing$ or $\varnothing\Delta A$, it must 
            be in $A$ but not in $\varnothing$. This is true for every element of $A$ so
            $\varnothing$ is our identity
            
        \item Next we find inverses for any $A\subseteq X$. The inverse
            in this context $A$. So we must show that $A\Delta A=\varnothing$.

            $A\Delta A=(A\cap A)\setminus(A\cup A)$. Notice that $A\cap A=A$ and $A\cup A=A$.
            So this becomes $A\Delta A= A\setminus A$. And since $A\setminus A$ represents all the
            elements of $A$ that are not in $A$, this is simply the empty set.

        \item Now we check associativity. We must show that for all $A,B\subseteq X$, 
            $A\Delta (B\Delta C)=(A\Delta B)\Delta C$.

            For the following, let $A,B,C\in  P(X)$.

            Consider $(A\cup B)\cup C$. This is the set of all elements that are in $A$ or $B$,
            or those that are in $C$. Clearly this is the same as the set $A\cup (B\cup C)$.

            Now consider $(A\cap B)\cap C$. This is the set of all elements in $A$ and $B$,
            as well as those in $C$. Again this is the same as $A\cap (B\cap C)$.

            This part is incomplete, I was not able to keep track of the expressions this created
            for more than a couple of lines.

        \item Finally we check if the group is abelian.

            $$A\Delta B=(A\cup B)\setminus(A\cap B)=(B\cup A)\setminus(B\cap A)=B\Delta A$$

            Which comes from the fact that set union and intersection are both commutative.
        \end{enumerate}

    \item Take $X = \{1, 2\}$ and write the Cayley table for this group. Compare it to the tables
    we did in lectures, try to determine “upto isomorphism” which group it is.

    \begin{tabular}{c|c c c c}
        $\Delta$ & $\varnothing$ & $\{1\}$ & $\{2\}$ & $\{1,2\}$ \\
        \hline
        $\varnothing$ & $\varnothing$ & $\{1\}$ & $\{2\}$ & $\{1,2\}$ \\
        $\{1\} $&$\{1\} $&$\varnothing$&$\{1,2\} $&$\{2\} $\\
        $\{2\} $ &$\{2\} $&$\{1,2\}$&$\varnothing$&$\{1\} $\\
        $\{1,2\}$&$\{1,2\}$&$\{2\} $&$\{1\}$&$\varnothing$ 
    \end{tabular}

This set maintains the same structure as the group we saw in class with $|G|=4$ and with every element as its own inverse. We called it the Kliens four group $K_4$.
\end{enumerate}

\item Consider the set
    $$GL_2(\mathbb{Z})=\left\{ \begin{pmatrix} a&b\\c&d \end{pmatrix} 
    a,b,c,d\in \mathbb{Z}_2, ad-bc\neq 0 \right\} $$
    \begin{enumerate}[label= (\alph*)] 
        \item Show that $GL_2(\mathbb{Z}_2)$ forms a group under matrix multiplication.
            \begin{enumerate}[label=(\roman*)]
                \item First we show closure. Let $A,B\in GL_2(\mathbb{Z}_2)$. Then by the laws of 
                matrix multiplication, $AB$ is a $2\times 2$ matrix, by properties of the
                determinant, $\det(AB)=\det(A)\det(B)$. Since $\det(A),\det(B)$ are both
                nonzero, $\det(AB)\neq 0$, and since $\mathbb{Z}_2$ is closed under multiplication and addition. So $AB$ is a $2\times 2$ matrix with entries in 
                $\mathbb{Z}_2$ and nonzero determinant, $AB\in GL_{2}(\mathbb{Z}_2)$.
                So $GL_2(\mathbb{Z}_2)$ is closed.

                \item Next we show identity. We choose $I=\begin{bmatrix} 1&0\\0&1 \end{bmatrix} $. 
                For any $A\in GL_2(\mathbb{Z}_2)$, $AI=IA=A$.

                \item Now we show inverses. For some $A=\begin{bmatrix} a&b\\c&d \end{bmatrix}$, let 
                    $A^{-1}=\begin{bmatrix} d&-b\\-c&a \end{bmatrix}$. Then multiplying:
                    \[
                        AA^{-1}=\begin{bmatrix} ad-bc&-ba+ba\\cd-cd&ad-bc \end{bmatrix} 
                        =\begin{bmatrix} 1&0\\0&1 \end{bmatrix} =I
                    .\] 
                \item Finally we show associativity.

                        Let $A=\begin{bmatrix} a&b\\c&d \end{bmatrix} ,\,
                        B=\begin{bmatrix} e&f\\g&h \end{bmatrix},\,
                        C=\begin{bmatrix} i&j\\k&l \end{bmatrix} $.

                        Then:
                        \begin{align*}
                            (AB)C&= \left(\begin{bmatrix} a&b\\c&d \end{bmatrix}\begin{bmatrix} e&f\\g&h \end{bmatrix} \right) C\\
                                 &= \begin{bmatrix} ae+bg&af+bh\\ce+dg&cf+dh\end{bmatrix} C\\
                                 &= \begin{bmatrix} i(ae+bg)+k(af+hb)&j(ae+bg)+l(af+hb)\\
                                 i(ce+dg)+l(fe+dh)&j(ce+dg)+l(fe+dh)\end{bmatrix}\\
                                 &= \begin{bmatrix} aei+bgi+afk+bhk&aej+bgj+afl+hbl\\cei+dgi+efl+dhl&cdj+dgj+fel+dhl\end{bmatrix}  \\
                            A(BC)&=A\begin{bmatrix} e&f\\g&h \end{bmatrix}\begin{bmatrix} i&j\\k&l \end{bmatrix} \\
                                &=A\begin{bmatrix}ie+kf&je+lf\\ig=kh&jg+lh\end{bmatrix}\\
                                &= \begin{bmatrix} a(ie+kf)+b(ig+kh)&a(je+lf)+b(jg+lh)\\
                                c(ie+kf)+d(ig+kh)&c(je+lf)+d(jg+lh)\end{bmatrix}\\
                                &= \begin{bmatrix} aei+afk+bgi+bhk&aej+afl+bgj+bhl\\
                                cei+cfk+dgi+dhk& cej+cfl+dgj+dhl\end{bmatrix} 
                        \end{align*}

                        And from this disgusting expansion of matricies we see $A(BC)=(AB)C$ and 
                        $GL_2(\mathbb{Z}_2)$ is associative.
            \end{enumerate}
        \item Compute the order of this group with justification.

            Looking for matricies where $ad\neq c b$, and since each entry must be $0 $ or $1$,
            we are interested in all matricies where one diagonal has a product of $1$ and the other
            has a product of $0$.

            So we create some arbitrary $A\in GL_2(\mathbb{Z}_2)$. First we must choose which diagonal
            will have a product of  $0$ and which will be $1$. We have $2$ ways to do this.
            Both of the entries on this diagonal must be $1$. Then we choose the elements on the other
            diagonal. These can both be $0$ or $1$, so long as they are not both $1$. So there are $3$
            ways to choose this diagonal.

            Therefore, there are $6$ ways to make a matrix in $GL_2(\mathbb{Z}_2)$, and the order of
            $GL_2(\mathbb{Z}_2)$ is $6$.
        \item Show that the group is not Abelian.

            Choose $A=\begin{bmatrix} 1&1\\0&1 \end{bmatrix}, B=\begin{bmatrix}0&1\\1&1\end{bmatrix}$.
            Notice that $AB=\begin{bmatrix} 1&0\\1&1 \end{bmatrix}\neq
            \begin{bmatrix} 0&1\\1&0 \end{bmatrix} =BA$ so $B$ is not abelian.
    \end{enumerate}
\item Let $G$ be a group with identity $e$.
    \begin{enumerate}[label= (\alph*)] 
        \item Show that if $(ab)^2 = a^2 b^2$ for all $a, b \in  G$, then $G$ must be Abelian.

            Suppse $(ab)^2=a^2b^2$ for any $a,b\in G$. This means:
            \begin{align*}
                (ab)^2&=a^2b^2\\
                (ab)(ab)&=a^2b^2\\
                a(ba)b&= a(ab)b & \text{By Associativity}\\
                (a^{-1}a)(ba)(bb^{-1})&= (a^{-1}a)(ab)(bb^{-1})&\text{By inverses and Asoociativity}\\ 
                e(ba)e&= e(ab)e \\
                be&= ab&\text{By Identity} 
            \end{align*}

            And since $ab=ba$ for any $a,b$ in $G$, $G$ is abelian.

        \item  Show that if $g^2 = e$ for all $g \in  G$, then $G$ must be Abelian.

            Suppose $a,b\in G$ so that $a^2=e=b^2$.

            \begin{align*}
                ab&=ab  \\
                (ab)(ab)&=(ab)^2\\
                a(ba)b&= e &\text{By Associativity}\\ 
                (aa)(ba)(bb)&= ab&\text{Multiplying left by $a$, right by $b$.} \\
                e(ba)e&= ab \\
                ba&=ab
            \end{align*}

            And since $ab=ba$ for any $a,b$ in $G$, $G$ is abelian.

        \item  Show that if $|G|$ is even, then there exists an element $h \in G$ such that $h^2=e.$

            Suppose that $|G|=n$ is even.
            Argue by pairing. Trivially, $e$ satisfies this property. But we have $n-1$ other elements 
            in $G$, and each element has an inverse in $G$.

            So for every element in $G$, we can pair it with its inverse. However there are two elements we cannot pair. One is trivial, the identity, which is its own inverse. Then we have another element in $G$ that has an inverse in $G$, which cannot be paired with any other element than itself (since inverses are unique). So there exists some $h\in  G$ so that $h^2=e$.
    \end{enumerate}
\item Let $S_n$ be the symmetric group of degree $n$.
\begin{enumerate}[label= (\alph*)] 
    \item Take $n = 4$. In $S_4$, list the elements as cycles and determine the order of each element.

        The list of elements in $S_4$ with order $1$ is simply the element $e$, or the map $\sigma$
        given by $\sigma(x)=x$ for any $x \in X_4$

        The list of elements in $S_4$ with order $2$: $(1\,2),(1\,3),(1\,4),(2\,3),(2\,4),(3\,4),(1\,2)\\(3\,4),(1\,3)(2\,4),(1\,4)(2\,3)$

        The list of elements in $S_4$ with order $3$: $(1\,2\,3),(3\,2\,1),(2\,3\,4),(4\,3\,2),(1\,3\,4),\\(4\,3\,1),(1\,2\,4),(4\,2\,1)$

        The list of elements in $S_4$ with order $4$: $(1\,2\,3\,4),(1\,2\,4\,3),(1\,3\,2\,4),(1\,3\,4\,2),(1\,4\,2\,3),(1\,4\,3\,2)$

    \item What is the highest possible order of an element in $S_6$? Give an example. 
        What about $S_7$? Give an example.

        The highest possible order of an element in $S_6$ is $6$. An example of this is the cycle $(1\,2\,3\,4\,5\,6)$. We can make a cycle of order $6$ with any cycle of length $6$ or two disjoint cycles of length $3$ and $2$.

        The highest possible order of an element in $S_7$ is $12$. An example of this would be the cycle $(1\,2\,3\,4)(5\,6\,7)$. Any permutation created by two disjoint cycles of length $3$ and $4$ will satisfy this.
\end{enumerate}
\end{enumerate}
\end{document}
