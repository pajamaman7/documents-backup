\documentclass{article}
\usepackage[most,many,breakable]{tcolorbox}
\usepackage{amsmath}
\usepackage{amssymb}
\usepackage{amsthm}
\usepackage[]{thmbox}
\usepackage{blindtext}
\usepackage[utf8]{inputenc}
\usepackage{amsmath}
\usepackage{amsfonts}
\usepackage[]{graphicx}
\usepackage[legalpaper, portrait, margin = 1in]{geometry}
\usepackage{enumitem}


\usepackage{xcolor}

%\pagecolor[rgb]{0.2,0.19,0.18} 
%\color[rgb]{0.92,0.86,0.7}

\newtheorem[L]{le}{Lemma}[subsection]
\newtheorem[L]{th}[le]{Theorem}
\newtheorem[L]{df}[le]{Definition}
\newtheorem[L]{ex}[le]{Example}
\newtheorem[L]{pf}[le]{Proof}


\newcommand{\nl}{\newline}

\newcommand{\real}{\mathbb{R}}
\newcommand{\complex}{\mathbb{C}}
\newcommand{\integer}{\mathbb{Z}}
\newcommand{\rational}{\mathbb{Q}}
\newcommand{\lxor}{\oplus}
\newcommand{\then}{\Rightarrow}

\begin{document}
\huge Assignment 2 - Thomas Boyko - 30191728

\normalsize 
\begin{enumerate}
\item Consider the following probability distribution table for a discrete random variable $X$.
\begin{center}
\begin{tabular}{|c|c|c|c|}
\hline
x & -10& 0 &5\\
\hline
P(X=x)& 0.3 &0.5& 0.2\\
\hline
\end{tabular}
\end{center}

\begin{enumerate}[label= (\alph*)] 
 
\item Find $E[X]$.

\begin{align*} 
    E[X]&=\sum\limits xp(x)\\
    &=-10(0.3)+0(0.5)+0.2(5)\\
    &=-3+1\\
    &=-2
\end{align*}
\item Find $VAR[X]$.
    \begin{align*} 
    E[X^2]&=\sum\limits x^2p(x)\\
    &=100(0.3)+0(0.5)+25(0.2)\\
    &=30+0+5\\
    &=35\\
    VAR[X]&=E[X^2]-E[X]^2\\
    &=35-(-2)^2\\
    &=31
    \end{align*}
\item Find $E[3X]$.
    \begin{align*} 
    E[3X]&=3E[X]\\
    &=3(-2)\\
    &=-6
    \end{align*}
\end{enumerate}

\item In an experiment two weighted (unfair) coins are being flipped simultaneously. The first
coin will land “heads” with a probability of $0.7$ and the second coin will land “heads” with
a probability of $0.8$. Assume that the results of the flips are independent. Let $X$ represent
the total number of “heads” that result from this experiment.

\begin{enumerate}[label= (\alph*)] 
\item Find $E[X]$.

We can find the expected value from the formula $\sum xp(x)$.

\[E[X]=0(0.3\cdot0.2)+1(0.3\cdot0.8+0.7\cdot0.2)+2(0.7\cdot0.8)=1.5\]

So the expected number of heads is $1.5$.

\item Find $SD[X]$.

\begin{align*}
    SD[X]&=\sqrt{E[X^2]-E[X]^2}\\
    &=\sqrt{0(0.3\cdot0.2)+1(0.3\cdot0.8+0.7\cdot0.2)+4(0.7\cdot0.8)-1.5^2}\\
    &=\sqrt{2.62-2.25}\\
    &=0.6082762
\end{align*}

So $SD[X]=0.60828$.

\end{enumerate}
\newpage
\item Four fair coins are flipped. Let $X$ represent the number of “heads” observed.
\begin{enumerate}[label= (\alph*)] 
 
\item Construct a probability distribution table for $X$.

\begin{verbatim}
    x=0:4
    px=dbinom(x,4,.5)
\end{verbatim}

\begin{center}
\begin{tabular}{|c|c|c|c|c|c|}
\hline
x & 0& 1 &2&3&4\\
\hline
P(X=x)&  0.0625 & 0.2500 & 0.3750 & 0.2500 & 0.0625 \\
\hline
\end{tabular}
\end{center}
\item Create a probability distribution graph for $X$ (a simple sketch is fine!). Include the
graph/sketch in your assignment submission. How would you describe the shape of the graph?

\begin{verbatim}
    ##Continuing from a)
    plot(x,px)
\end{verbatim}

\includegraphics*{Rplot.png}

The shape of the graph is symmetric around the mean of $2$. The graph is roughly bell-shaped

\item What is the probability that at least two “heads” are observed when these four fair
coins are flipped?

\begin{verbatim}
    ##Continuing from a),b)
    sum(dbinom(2:4,4,0.5))
    #0.6875
\end{verbatim}

So the probability of at least two heads is $0.6875$.

\end{enumerate}

\item A factory produces a type of screw that is sold in small boxes of 20 screws each.
Suppose the probability of any given screw being defective is 0.01.
\begin{enumerate}[label= (\alph*)] 
 
\item What is the probability that a box contains no defective screws?
\begin{verbatim}
    dbinom(0,20,0.01)
    #0.8179069
\end{verbatim}
So the probability that the box contains no defective screws is $0.81791$.
\item What is the probability that a box contains exactly one defective screw?
\begin{verbatim}
    dbinom(1,20,0.01)
    #0.1652337
\end{verbatim}
So the probability of there being exactly one screw that is defective is $0.0.16523$.
\newpage
\item The company that sells the boxes offers a full refund on the purchase of a box if the
box contains more than one defective screw. What is the probability that the
company would have to offer a refund for a randomly selected box?
\begin{verbatim}
    sum(dbinom(2:20,20,0.01))
    #0.01685934
\end{verbatim}
So the probability that the company will have to refund a randomly selected box is $0.01686$.
\end{enumerate}

\item Suppose $Y\sim binomial(120,0.3)$.
\begin{enumerate}[label= (\alph*)] 
\item Find $E[4Y-2]$.

First we will find $E[Y]$.

\[E[Y]=np=120(0.3)=36\]

From this,

\[E[4Y-2]=4E[Y]-2=4(36)-2=142\]

So $E[4Y-2]=142$.

\item Find $VAR[4Y-2]$.

First we find $VAR[Y]$.

\[VAR[Y]=npq=120(0.7)(0.3)=25.2\]

And we know that:

\[VAR[4Y-2]=4^2 VAR[Y]=16(25.2)=403.2
\]
So $VAR[4Y-2]=403.2$.

\end{enumerate}

\item An urn contains two red balls, four blue balls, and three green balls. You randomly
select a ball from the urn, record its color, and return it to the urn. You do this process
twice (for a total of two recorded colors). Let $X$ represent the number of red balls that
you observed.
\begin{enumerate}[label= (\alph*)] 
 
\item Create a probability distribution table for $X$.
\begin{verbatim}
    dbinom(0:2,2,2/9)
\end{verbatim}
\begin{center}
    \begin{tabular}{|c|c|c|c|}
        \hline
        x&0&1&2\\
        \hline
        P(X=x)&0.60493827& 0.34567901& 0.04938272\\
        \hline
    \end{tabular}
\end{center}
\item How many red balls should you expect to select in this process?
\[E[X]=np=2\frac{2}{9}=\frac{4}{9}\]
\end{enumerate}
\item Bob is playing a game in which a fair die is rolled. If the die lands on a 6, Bob wins the
round. If it lands on any other number, he loses the round.
\begin{enumerate}[label= (\alph*)] 
 
\item What is the probability that Bob's first win will occur on the fourth round?

\begin{verbatim}
    dbinom(1-1,4-1,1/6)/6
    #0.09645062
\end{verbatim}

So the probability that his first win occurs on the fourth round is $0.09645$.

\item What is the probability that Bob's second win will occur on the third round?

\begin{verbatim}
    dbinom(2-1,3-1,1/6)/6
    #0.0462963
\end{verbatim}

So the probability that Bob's second win occurs on the third round is $0.04630$.

\end{enumerate}
\item According to the M\&M's website, $24\%$ of peanut M\&M's are yellow. Consider a package
containing 23 peanut M\&M's.
\begin{enumerate}[label= (\alph*)] 
 
\item What is the probability that more than 10 of the M\&M's in the pack are yellow?

\begin{verbatim}
    sum(dbinom(11:23,23,0.24))
    #0.01087513
\end{verbatim}

The probability of more than 10 yellow M\&M's in the pack is $0.01088$.

\item What is the probability that between five and 10 (inclusive) of the M\&M's in the pack
are yellow?

\begin{verbatim}
    sum(dbinom(5:10,23,0.24))
    #0.6674023
\end{verbatim}
The probability of between five and 10 (inculsive) M\&M's are in the pack is $0.66740$.
\item If the number of yellow M\&M's in the pack is between five and 10 (inclusive), what is
the probability that there are exactly seven yellow M\&M's in the pack?
\[P(x=7|5\leq x\leq10)=\frac{P(x=7\cap 5\leq x\leq 10)}{P(5\leq x\leq10)}=\frac{P(x=7)}{P(5\leq x\leq10)}\]
\begin{verbatim}
    dbinom(7,23,0.24)/sum(dbinom(5:10,23,0.24))
    #0.20871
\end{verbatim}
So, given that there are between 5 and 10 inclusive yellow M\&M's in the pack, there is a $0.20871$ chance that there are 7 yellow M\&M's.
\end{enumerate}
\end{enumerate}
\end{document}