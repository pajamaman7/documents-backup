\documentclass[]{article}
\usepackage{amsmath}
\usepackage{amssymb}
\usepackage{amsthm}
\usepackage[utf8]{inputenc}
\usepackage{amsmath}
\usepackage{amsfonts}
\usepackage[]{graphicx}
\usepackage[a4paper, portrait, margin = 1in]{geometry}
\usepackage{enumitem}
\usepackage{xcolor}

%darkmode
%\pagecolor[rgb]{0.2,0.19,0.18} 
%\color[rgb]{0.92,0.86,0.7}

\newenvironment*{alphenum}{\begin{enumerate}[label= (\alph*)]}{\end{enumerate}}



\begin{document}
\huge Written Assignment 2 - Thomas Boyko - 30191728 \nl \nl
\normalsize 
Consider the power series: 
\[\sum_{n=0}^{\infty}\frac{3^n}{n^2+1}(2x-1)^n\]

\begin{alphenum}
    \item Explain how you know that the centre of the series is $\frac{1}{2}$.
            \par We can set $2x-1$ equal to zero,
            yielding $x=\frac{1}{2}$, and at at that value,
            each term of the series will equal zero.
    \item Show that the radius of convergence is $\frac{1}{6}$.
            \par To show this, we can use the ratio test and see where $L<1$.
            \begin{align*}
            \lim_{n\to\infty}
            \left|\frac{3^{n+1} (2x-1)^{n+1}}{(n+1)^2+1}\cdot
            \frac{n^2+1}{3^n(2x-1)^n}
            \right| &= L\\
            \lim_{n\to\infty}\left|\frac{3(2x-1)(n^2+1)}{n^2+2n+2}\right| &= L \\
            |6x| &= L\\
            |6x| &<1 \\
            |x| &<\frac{1}{6}
            \end{align*}
            So the radius of convergence is $\frac{1}{6}$.
    \item Show that $x = \frac{1}{3}$ is included in the interval of covergence of the power series.
    \par Our series will now become: 
    \begin{align*}
    &\sum_{n=0}^{\infty}\frac{3^n}{n^2+1}(\frac{2}{3}-1)^n\\
    &\sum_{n=0}^{\infty}\frac{(-1)^n }{n^2+1}\\
    \end{align*}
    \par Now we can use the alternating series test.
        \[\lim_{n\to\infty}\frac{(-1)^n}{n^2+1}=0\]
    \par(since the denominator tends to infinity.)
    \par Now we must show $a_n\leq a_{n+1}$.
    \begin{align*}
        a_n &= \frac{1}{n^2+1}\\
        a_{n+1} &= \frac{1}{n^2+2n+2}
    \end{align*}
    \par Since $n^2+2n+2>n^2+1$ for all positive $n$, $a_{n+1}\geq a_n$, we can 
    use the Alternating Series test to say that this series converges.
    \newpage 
    \item Show that $x = \frac{2}{3}$ is included in the interval of convergence of the power series.
    \par Our series becomes: 
    \begin{align*}
      &\sum_{n=0}^{\infty}\frac{3^n}{n^2+1}(\frac{4}{3}-1)^n\\  
      &\sum_{n=0}^{\infty}\frac{1}{n^2+1}\\  
    \end{align*}
    Now we can use the Comparison Test with $\frac{1}{n^2}$.
    \[\lim_{n\to\infty}\frac{n^2}{n^2+1}=1\]
    Since the limit of the ratio of the two test equals a positive, finite number,
    both series must converge (since $\frac{1}{n^2}$ is a p-series with $p=2$).
\end{alphenum}

\end{document}
