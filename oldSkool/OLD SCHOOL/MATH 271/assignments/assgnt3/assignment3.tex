\documentclass{article}
\usepackage{amsmath}
\usepackage{amssymb}
\usepackage{amsthm}
\usepackage[utf8]{inputenc}
\usepackage{amsmath}
\usepackage{amsfonts}
\usepackage[]{graphicx}
\usepackage[a4paper, portrait, margin = 1in]{geometry}
\usepackage{enumitem}
\usepackage{xcolor}

%darkmode
%\pagecolor[rgb]{0.2,0.19,0.18} 
%\color[rgb]{0.92,0.86,0.7}

\newenvironment*{alphenum}{\begin{enumerate}[label= (\alph*)]}{\end{enumerate}}


\begin{document}
\huge Assignment 3 - Thomas Boyko - 30191728
\normalsize 
\begin{enumerate}
\item For each of the following statements: If the statement is true, then give a proof;
if the statement is false, then write out the negation and prove that.
\nl \textbf{Notation: } Let $F$ be the set of all functions from $\integer$ to $\integer$.
\begin{alphenum}
   \item For all $f,g,h \in F$, if $ f\circ g = f\circ h$, then $g=h$.
        \nl The statement is false. The negation is: 
        "There exist functions $f,g,h\in F$ so that $f\circ g = f\circ h$ but $g\neq h$."
        \begin{proof}
            Choose the functions $f,g,h \in F$ where $f(x)= x^2$, $g(x) = x$ and $h(x)= -x$ for all $x\in\integer$.
            \nl Note that 
            \[f\circ g = f(g(x)) = f(x)=x^2=(-x)^2=f(-x)=f(h(x))=f\circ h\]
            So for all $x\in\integer$, $f(g(x))=f(h(x))$
            \nl However, 
            \[g(1)=1\neq-1=h(1)\]
            So $g\neq h$.
            \nl Therefore, 
            there exist functions $f,g, h\in F$ so that $f\circ g = f\circ h$ but $g\neq h$.
        \end{proof}

   \item For all $f,g,h \in F$, if $ g\circ f = h\circ f$, then $g=h$.
   The statement is false. The negation is: "There exist $f,g,h \in F$ so 
   that $g\circ f=h\circ f$ but $g\neq h$."

    \begin{proof}
        Choose the following functions $f,g,h \in F$.
        \begin{align*}
            f(x) &= x^2\\
            g(x) &= \begin{cases}
                y & \exists y\in\integer \text{ so that } y^2=x\\
                0 & \forall y\in\integer, y^2\neq x
            \end{cases}\\
            h(x) &= \begin{cases}
                y & \exists y\in\integer \text{ so that } y^2=x\\
                1 & \forall y\in\integer, y^2\neq x
            \end{cases}
        \end{align*}

        Note that the entire codomain of $f$ will take the first case for $g,h$
        since every element in the codomain is a perfect square.

        So:
        \[h(f(x))=x=g(f(x))\]

        But, $h(2)=1\neq0=g(2)$, so $h\neq g$.

        Therefore, there exist $f,g,h \in F$ so that $g\circ f=h\circ f$ but $g\neq h$.

    \end{proof}

   \item For all $f,g,h \in F$, if $ f\circ g = f\circ h$ and $f$ is one-to-one,
   then $g=h$.

        \begin{proof}
            Suppose $f,g,h\in F$. Further suppose $f\circ g(x)= f\circ h(x)$ for all $x\in\integer$.
            Finally, suppose $f$ is one-to-one.
            \nl Let $a=h(x)$ and $b=g(x)$ where $a,b\in\integer$.
            \nl Then f(a)=f(b).
            \nl Then by the definition of one-to-one, since $f(a)=f(b)$, $a=b$.
            \nl Therefore, for all $f,g,h \in F$, if $ f\circ g = f\circ h$ and $f$ is one-to-one,
            then $g=h$.
        \end{proof}
   
   \item For all $f,g,h \in F$, if $ g\circ f = h\circ f$ and $f$ is onto, then $g=h$.

   \begin{proof}
    Suppose $f,g,h\in F$, and that $f\circ g = f\circ h$.

    Since $f$ is onto, for all $b\in\integer$, there exists an $a\in\integer$ so that
    $f(a)=b$. So, the codomain of $f$ is $\integer$.

    This means that for all $c\in\integer$, $g(c)=h(c)$.

    Therefore, $g=f$.

   \end{proof}
\end{alphenum}
\newpage

\item Let $f:\integer\to\integer$ be the function defined by $f(x) = 3x^2+x$ for every $x\in\integer$.
\begin{alphenum}
    

    \item Is $f$ one-to-one? Prove your answer.
    \nl $f$ is one-to-one.

    \begin{proof}
        Let $f:\integer\to\integer$, $f(x)=3x^2+x$, where $x\in\integer$.
        Suppose $a,b\in \integer$ and that $f(a)=f(b)$. From this we will attempt to prove that $a=b$.
        \nl So we have:
        \begin{align*}
            3a^2+a&=3b^2+b\\
            3a^2+a-3b^2-b&=0\\
            3(a^2-b^2)+(a-b)&=\\
            3(a-b)(a+b)+(a-b)&=\\
            (a-b)(3(a+b)+1)&=\\
        \end{align*}
        So, either $(a-b)=0$ or $3(a+b)+1=0$. However, if $3(a+b)+1=0$, then $a+b=-\frac{1}{3}$ which means $a,b$ cannot
        both be integers. 
        \nl Since $3(a+b)+1\neq 0$, $a-b=0$ and $a=b$.\nl
        Therefore, for all $a,b\in\integer$, if $f(a)=f(b)$, then $a=b$.
        So $f$ is one-to-one.
        \nl 
    \end{proof}


    \item Is $f$ onto? Prove your answer.
    \nl $f$ is not onto. So, we must prove: "There exists some $b\in\integer$ so that for all $a\in\integer$, $f(a)\neq b$."

    \begin{proof}
        Choose $b=-1$. We must prove that for all $a\in\integer$, $f(a)> -1$.

        Note that since $a\in\integer$, we know that $(a+\frac{1}{6})^2\geq0$.

        We can follow this to show that $f$ is greater than a certain value.
        
        \begin{align*}
            (a+\frac{1}{6})^2&\geq0\\
            3(a+\frac{1}{6})^2&\geq0\\
            3(a^2+\frac{a}{3}+\frac{1}{36})&\geq0\\
            (3a^2+a+\frac{1}{12})&\geq0\\
            3a^2+a&\geq-\frac{1}{12}>-1
        \end{align*}

        So the function $f(a)\neq-1$ for all $a\in\integer$.

        Since there exists some $b\in\integer$ so that for all $a\in\integer$, $f(a)\neq b$, 
        $f$ is not onto.

        
    \end{proof}

    \item Is there a function $g:\integer\to\integer$ so that $g\circ f =I_{\integer}$
    where $I_\integer$ is the identity function from $\integer$ to $\integer$? Prove your answer.

    \begin{proof}
        Choose the following function $g:\integer\to\integer$. 
        \[
        g(x)=\begin{cases}
            y& \text{If }3y^2+y=x \text{ for some } y\in\integer\\
            0 & \text{If }\forall y\in\integer, \; 3y^2+y\neq x
        \end{cases}\]
        Note that $g\circ f = g(f(x))=g(3x^2+x)$.
        \nl Looking back at the definition for $g(x)$ we can see that every input
        for $g\circ f$ will take the form $3y^2+y$ for some integer $y$, where $y$ will always equal $x$.
        \nl So, for all integers $x$, $g(f(x))=x$ which is the identity function $I_{\integer}$.
        \nl Therefore, there exists a function $g:\integer\to\integer$ so that $g\circ f = I_{\integer}$ 

        
    \end{proof}

\end{alphenum}

\newpage 

\item Let $S=\{1,2,3,4,5\}$.\nl 
Let $f: S\to S$ be the function defined by $f=\{(1,1),(2,1),(3,3),(4,3),(5,5)\}$.
\begin{alphenum}
    \item How many functions $g:S\to S$ are there so that $g\circ f(2) = 1$? Explain.

    Note that $f(2)=1$ so $g(f(2))=g(1)=1$. So $(1,1)\in g$. 
    
    Now we must map the remaining 4 elements of $S$ to something in $S$. Since there
    are $5$ elements in $S$, there are $5^4$ ways to do this.

    So the total number of functions $g:S\to S$ so that $g\circ f(2)=1$ is $5^4=625$

    \item How many functions $g:S\to S$ are there so that $f \circ g(2) = 1$? Explain.
    
    Note that for $f(i)$ to equal 1, $i$ must equal 1 or 2. So, either $g(2)=1$ or 
    $g(2)=2$.

    So, to count the number of functions, we will first choose whether $g(2)$ equals
    1 or 2. (there are two ways to do this). Then, for the remaining elements of $S$,
    we can choose it to map to any other element of $S$ under $g$. There are $5^4$ ways
    to do this.

    So, the total number of functions $g:S\to S$ so that $f\circ g(2)=1$ is $2\times 5^4=1250$

    \item How many functions $g:S\to S$ are there so that $g\circ f(i) = 1$ for some $i\in S$? Explain.

    Note that the codomain of $f$ is $\{1,3,5\}$. So the number of functions $g$ so that $g(f(i))=1$ equals
    the total number of functions $g:S\to S$ minus those where $1, 3, 5$ are all not mapped
    to 1.

    We can build the former section with $5^5$, since we are choosing an element from $S$
    for each element of $S$.

    The latter can be built by choosing for 1,3 and 5, to map it to any element that is not
    1. There are $4^3$ ways to do this. Then we can map 2 and 4 to any element,
    and there are $5^2$ ways to do this. So there are $5^2\times4^3$ ways to build a function
    so that 1, 3, and 5 are all not mapped to 1.

    Subtracting these, we find there are $5^5-5^2\times4^3=3125-1600=1525$ ways to build a function $g:S\to S$
    so that $g\circ f (i)=1$

    \item How many functions $g:S\to S$ are there so that $f \circ g(i) = 1$ for some $i\in S$? Explain.

    Note that $f(n)$ equals 1 if $n=1$ or $n=2$. Therefore, $g(i)$ must equal 1 or 2.

    We can count these functions by taking the total number of functions $g:S\to S$ and subtracting the number
    of functions where no element is mapped to 1 or 2. 

    The total number of functions is $5^5$.

    We can count the functions where no element is mapped to 1 or 2 by mapping each element in $S$ to
    $3,4$ or $5$. There are $3^5$ ways to choose this.

    So the total number of ways to choose a function $g:S\to S$ so that $f\circ g(i)=1 $ for some 
    $i\in S$ is $5^5-3^5=3125-243=2882$.

\end{alphenum}

\end{enumerate}
\end{document}