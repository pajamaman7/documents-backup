\documentclass[]{article}
\usepackage[most,many,breakable]{tcolorbox}
\usepackage{amsmath}
\usepackage{amssymb}
\usepackage{amsthm}
\usepackage[]{thmbox}
\usepackage{blindtext}
\usepackage[utf8]{inputenc}
\usepackage{amsmath}
\usepackage{amsfonts}
\usepackage[]{graphicx}
\usepackage[legalpaper, portrait, margin = 1in]{geometry}
\usepackage{enumitem}


\usepackage{xcolor}

%\pagecolor[rgb]{0.2,0.19,0.18} 
%\color[rgb]{0.92,0.86,0.7}

\newtheorem[L]{le}{Lemma}[subsection]
\newtheorem[L]{th}[le]{Theorem}
\newtheorem[L]{df}[le]{Definition}
\newtheorem[L]{ex}[le]{Example}
\newtheorem[L]{pf}[le]{Proof}


\newcommand{\nl}{\newline}

\newcommand{\real}{\mathbb{R}}
\newcommand{\complex}{\mathbb{C}}
\newcommand{\integer}{\mathbb{Z}}
\newcommand{\rational}{\mathbb{Q}}
\newcommand{\lxor}{\oplus}
\newcommand{\then}{\Rightarrow}
\begin{document}
    \huge Thomas Boyko - Math 271 Assignment 4 - 30191728
    \normalsize 
    \begin{enumerate} 
        \item Let $A=\{1,2,3,4,5,6,7,8,9\}$. Let $\mathcal{P}=\mathcal{P}(A)$ be the power set of $A$

        Let $\mathrel{R}$ be the relation on $A$ defined by:
        \[\text{For all } X,Y\in\mathcal{P}, X\mathrel{R}Y\iff|X\cup Y|\leq|X\cup\{1,2\}|.\]
        \begin{enumerate}[label= (\alph*)]
        \item Is $R$:
        \begin{enumerate}[label= (\roman*)] 
            \item Reflexive?
                $\mathrel{R}$ is reflexive. We must prove that for all $X\in \mathcal{P}, X\mathrel{R}X$.
                \begin{proof}
                    Suppose $X\in \mathcal{P}$ and that $X$ has $x$ elements.

                    So $|X\cup X|=|X|=x$.

                    Note that $|X\cup \{1,2\}| = x+2-|X\cap \{1,2\}|$. We can see this through summing
                    the size of both sets, and then subtracting their intersection since it will be 
                    counted twice.

                    For an element to be in $|X\cap \{1,2\}|$, it must be $1$ or $2$. So the set $X\cap \{1,2\}$
                    has between $0$ and $2$ elements.

                    Since $0\leq|X\cap \{1,2\}|\leq2$, we can say that 
                    $|X\cup \{1,2\}| = x+n$ where $n\in\mathbb{Z}, 0\leq n\leq2$.

                    $|X\cup X|=x\leq x+n=|X\cup \{1,2\}|$

                    Therefore, for all $X\in\mathcal{P},X\mathrel{R}X$
                \end{proof}
            \item Symmetric?
            
            $R$ is not symmetric. Or, there exist $X,Y\in \mathcal{P}$ so that $X\mathrel{R}Y$ but $Y\not\mathrel{R} X$.
            \begin{proof}

                Choose the following sets $X,Y\in\mathcal{P}$.

                \begin{align*} 
                    X&=\{1,2,3\}\\
                    Y&=\varnothing
                \end{align*}

                Note the following:

                \begin{align*} 
                    X\cup Y &= \{1,2,3\}\\
                    X\cup \{1,2\}&=\{1,2,3\}\\
                    Y\cup \{1,2\}&=\{1,2\}\\
                    |X\cup Y| =3&\leq3=|X\cup \{1,2\}|
                \end{align*}
                So, $X\mathrel{R}Y$.
                However, 
                
                \[|X\cup Y|=3>2=|Y\cup\{1,2\}|\] 
                
                So, $Y\not\mathrel{R}X$.
                Therefore, there exist $X,Y\in \mathcal{P}$ so that 
                $X\mathrel{R}Y$ but $Y\not\mathrel{R} X$, and $\mathrel{R}$ is not symmetric.
            \end{proof}
            \item Antisymmetric?

                $\mathrel{R}$ is not antisymmetric. We must prove that there exist 
                $X,Y\in\mathcal{P}$ so that $X\mathrel{R}Y$ and $Y\mathrel{R}X$ but
                $Y\neq X$.

                \begin{proof}
                    Choose the following sets $X,Y\in\mathcal{P}$:

                    \begin{align*} 
                        X&=\{1\}\\
                        Y&=\varnothing
                    \end{align*}

                    So, 
                    \begin{align*} 
                        X\cup Y &= \{1\}\\
                        X \cup \{1,2\}&=\{1,2\}\\
                        Y \cup \{1,2\}&=\{1,2\}\\
                        |X\cup Y| = 1&\leq2=X \cup \{1,2\}\\
                        |X\cup Y| = 1&\leq2=Y \cup \{1,2\}
                    \end{align*}
                    So $Y\mathrel{R}X$ and $X\mathrel{R}Y$.

                    But since $1\in X$ and $1\not\in Y$, $Y\neq X$.

                Therefore, there exist $X,Y\in\mathcal{P}$ so that $X\mathrel{R}Y$ and $Y\mathrel{R}X$ but
                $Y\neq X$, and $\mathrel{R}$ is not antisymmetric.

                \end{proof}
            \item Transitive?

            $\mathrel{R}$ is not transitive. So we must find some $X,Y,Z\in\mathcal{P}$ so that 
            $X\mathrel{R}Y$ and $ Y\mathrel{R}Z, $ but $X\not\mathrel{R}Z$.

            \begin{proof}
                Choose the following sets $X,Y,Z\in\mathcal{P}$.

                \begin{align*} 
                    X&=\varnothing\\
                    Y&=\{3,4\}\\
                    Z&=\{3,4,5\}
                \end{align*}

                Therefore we can see that:
                \begin{align*} 
                    X\cup Y &= \{3,4\}\\
                    Y\cup Z &= \{3,4,5\}\\
                    X\cup Z &= \{3,4,5\}\\
                    X\cup\{1,2\}&=\{1,2\}\\
                    Y\cup\{1,2\}&=\{1,2,3,4\}\\
                    |X\cup Y|=2&\leq 2 =|X\cup \{1,2\}|\\
                    |Y\cup Z|=3&\leq 4 =|Y\cup \{1,2\}|
                \end{align*}
                So $X\mathrel{R}Y$ and $Y\mathrel{R} Z$.

                    \[|X\cup Z| = 3>2=|X\cup\{1,2\}|\]
                So we know that $X\not\mathrel{R}Z$.
                
                Therefore there exist sets $X,Y,Z\in\mathcal{P}$ so that 
                $X\mathrel{R}Y$ and $ Y\mathrel{R}Z, $ but $X\not\mathrel{R}Z$.

            \end{proof}


        \end{enumerate}
        \item How many $S\in \mathcal{P}$ are there so $\{3\}RS$?

            Note that $|\{3\}\cup\{1,2\}|=3$, so:
            \[\{3\}\mathrel{R}S\iff|\{3\}\cup S|\leq3\]
            To create a set $S$ so that $\{3\}\mathrel{R}S$, we first can select whether 
            or not $3\in S$. Whether this is the case or not will not change if $\{3\}\mathrel{R}S$
            There are $2$ ways to do this.

            Then, we choose $2$ elements from $A$ excluding $3$, without replacement. There are $9\choose2$
            ways to do this. (We cannot choose $3$, but we can choose to not add anything to $S$).

            So there are $2 {9\choose2 }=72$ ways to choose a set $S\in\mathcal{P}$ so that $\{3\}\mathrel{R}S$.
        \item How many $S\in \mathcal{P}$ are there so $S\mathrel{R}\{3\}$?

        To count the number of sets $S$ so that $S\mathrel{R}\{3\}$, first we must
        separate into two cases.
        
        Case 1: $3\in S$.

        If $3$ is in $S$, then we can choose any elements from $A$ to put in $S$ 
        and it will still be related to $\{3\}$.

        So, if $3\in S$, there are $2^8$ ways to choose the rest of $S$ so that
        $S\mathrel{R}\{3\}$.

        Case 2: $3\not\in S$. 

        If $3$ is not in $S$, then at most one of $\{1,2\}$ can be in $S$.

        So first we choose whether $1\in S$, $2\in S$, or neither. There are 3 ways to do this.

        Then for each remaining element of $A$, we can choose if it is in or out.
        There are $2^6$ ways to do this.

        Therefore, there are $3(2^6)$ ways to choose a set $S$ so that $S\mathrel{R}\{3\}$ when $3\not\in S$.

        So the total number of ways to choose a set $S\in\mathcal{P}$ so that $S\mathrel{R}\{3\}$
        is $2^8+3(2^6)=448$.

        \end{enumerate}

        \newpage 
    
        \item Let $A=\{1,2,3,4,5\}$. Let $\mathcal{F}$ be the set of all functions
        from $A$ to $A$. Define a relation $R$ on $\mathcal{F}$ as follows:
        \[ \forall f,g\in\mathcal{F},fRg\iff  \forall i\in A, f(i)\leq g(i)\]
        \begin{enumerate}[label= (\alph*)] 
        \item Is R:
            \begin{enumerate}[label=(\roman*)] 
                \item Reflexive?

                $\mathrel{R}$ is reflexive.
                \begin{proof}
                    Suppose $f\in\mathcal{F}$.

                    Notice that $\forall i\in A, \: f(i)=f(i)$ since $f$ is a function.

                    So $f(i)\leq f(i)$.

                    So $f\mathrel{R}f$ and $\mathrel{R}$ is reflexive.
                \end{proof}
                \item Symmetric?
                
                $\mathrel{R}$ is not symmetric. So, there exist $f,g\in\mathcal{F}$ so that
                $f\mathrel{R}g$ but $g\not\mathrel{R}f$.

                \begin{proof}
                    Choose $f=\{(x,1):x\in A\}$ and $g=\{(x,2):x\in A\}$

                    So, $\forall i\in A, f(i)=1$ and $g(i)=2$.

                    \[f(i)=1\leq2=g(i)\]

                    So $f\mathrel{R}g$

                    But, $\forall i\in A, $
                
                    \[g(i)=2>1=f(i)\]

                    So $f\not\mathrel{R}g$.

                    Therefore, there exist $f,g\in\mathcal{F}$ so that
                $f\mathrel{R}g$ but $g\not\mathrel{R}f$, and $\mathrel{R}$ is not symmetric.

                \end{proof}

                \item Antisymmetric?

                $\mathrel{R}$ is antisymmetric. So, $\forall f,g\in\mathcal{F}$ and $i\in A$,
                if $f\mathrel{R}g$ and $g\mathrel{R}f$, then $g=f$.

                \begin{proof}
                    Suppose that $g,f\in\mathcal{F}$ and that $f\mathrel{R}g$ and $g\mathrel{R}f$.

                    So, for all $i\in A,\: f(i)\leq g(i), \: g(i)\leq f(i)$

                    So we have: 
                    \[f(i)\leq g(i)\leq f(i)\]

                    So $f(i)=g(i)$ and $f=g$.

                    Therefore, $\mathrel{R}$ is antisymmetric.
                \end{proof}

                \item Transitive?

                $\mathrel{R}$ is transitive. So, for all $f,g,h\in\mathcal{F}$, 
                if $f\mathrel{R}g$ and $g\mathrel{R}h$, then $f\mathrel{R}h$.

                \begin{proof}
                    Suppose that $f,g,h\in\mathcal{F}$.

                    Further suppose that $f\mathrel{R}g$ and $g\mathrel{R}h$.

                    So for all $i\in A$, $f(i)\leq g(i)$ and $g(i)\leq h(i)$

                    \[f(i)\leq g(i)\leq h(i)\]
                    \[f(i)\leq h(i)\]

                    So $f\mathrel{R}h$.

                    Therefore, for all $f,g,h\in\mathcal{F}$, if $f\mathrel{R}g$ and $g\mathrel{R}h$, 
                    $f\mathrel{R}h$ and $\mathrel{R}$ is transitive.
                \end{proof}


            \end{enumerate}


        \item Prove or disprove: For all $f\in\mathcal{F}$, there exists $g\in \mathcal{F}$ so that $f\mathrel{R}g$
        \begin{proof}
            Suppose $f\in\mathcal{F}$. 

            Choose $g=f$.

            So, for all $i\in A$, $f(i)=g(i)$.

            This satisfies the condition for the relation, so $f\mathrel{R}g$.
        \end{proof}
        \newpage
        \item Prove or disprove: There exists $g\in\mathcal{F}$ so that for all $f\in\mathcal{F}$, $f\mathrel{R}g$.

        \begin{proof}
            Choose $g=\{(a,5):a\in A\}$
            
            Suppose $f\in\mathcal{F}$.

            So, for all $i\in A$, $g(i)=5$.

            Note that since all values in $A$ are between $0$ and $5$, 
            $0\leq f(i)\leq 5$.

            Since $g(i)=5$, $f(i)\leq g(i)$.

            So, $f\mathrel{R}g$.

            Therefore, there exists $g\in\mathcal{F}$ so that for all $f\in\mathcal{F}$, $f\mathrel{R}g$.

        \end{proof}
        
        \end{enumerate}
        Let $F\in\mathcal{F}$ be the function $f=\{(1,3),(2,3),(3,3),(4,1),(5,5)\}$.
        \begin{enumerate}[resume*]
        \item How many functions $g\in\mathcal{F}$ are there so that $fRg$? Explain.
        
        We can count the number of functions $g$ so that $f\mathrel{R}g$ by counting
        the number of functions where $g(i)$ is greater than or equal $f(i)$ for each $i\in A$.

        First, $f(1)=3\leq g(1)$. So $g$ must equal 3, 4, or 5. (3 ways to choose
        g(1)).

        The same applies for $2,3$ since $3=f(3)=f(2)$ so there are 3 ways 
        each to choose $g(2), \: g(3)$.

        There are $5$ ways to choose $g(4)$ since $f(4)=1\leq g(4)$ so $g(4)$ can be 
        any element of $A$.

        Finally, there is only one way to choose $g(5)$ since it must equal 5
        in order for it to be greater than or equal to $f(5)=5$.

        So there are $3^3\times4=108$ ways to choose a function $g$ so that $f\mathrel{R}g$.



        \item How many functions $g\in\mathcal{F}$ are there so that $gRf$? Explain.

        We can count the number of functions $g$ so that $g\mathrel{R}f$ by counting
        the number of functions where $g(i)\leq f(i)$ for each $i\in A$.
        
        First we choose values for $g(1),g(2)$, and $g(3)$. (Like in the last
        case, these all have the same number of options since $f(1)=f(2)=f(3)=3$).
        There are 3 ways to choose all of these since each can equal $1,2$, or $3$.

        There is only one way to choose $g(4)$ since it must equal 1 in order to 
        be less than or equal to $f(4)=1$.

        Finally, there are $5$ ways to choose $g(5)$ since all $5$ elements of 
        $A$ are less than or equal to $5=f(5)$.

        So there are $3^3\times4=108$ ways to choose a function $g$ so that $f\mathrel{R}g$.

        \end{enumerate}
        
        \newpage
        \item Prove or disprove the following statements by using the definitions of “congruence modulo
        n” and “divides.”

        \begin{enumerate}[label= (\alph*)]
        \item For all positive integers $a,x$ and $y$, if $(a+x)\equiv (a+y)\pmod{12}$,
        then $x\equiv y\pmod {12}$

        \begin{proof}
            Suppose $x,y,a\in\mathbb{Z}$, and that $a+x\equiv a+y\pmod{12}$.

            Then $12k=(a+x)-(a+y)$ for some $k\in\mathbb{Z}$.

            So $12k=a-y$

            Therefore, $x\equiv y\pmod{12}$.
        \end{proof}

        \item For all positive integers $a,x$ and $y$, if $ax\equiv ay\pmod{12}$,
        then $x\equiv y\pmod{12}$

        The statement is false. The negation is: "There exist positive integers
        $a,x$ and $y$ so that $ax\equiv ay\pmod{12}$,
        but $x \not\equiv y\pmod{12}$"

        \begin{proof}
            Choose $a=12$, $x=2$, $y=1$.

            Note that $12(2)\equiv12(1)\pmod{12}$ since 
            $12|24-12$ because $12=(1)(12)$.

            However, $1\not\equiv2\pmod{12}$ since $12\nmid2-1$, because $12>1$.

            Therefore, there exist positive integers
            $a,x$ and $y$ so that $ax\equiv ay\pmod{12}$,
            but $x \not\equiv y\pmod{12}$
        \end{proof}

        \item There exists a positive integer $a>1$ so that for all $x,y\in\mathbb{Z}$, 
        if $ax\equiv ay\pmod{12}$, then \\$x\equiv y\pmod{12}$

        \begin{proof}
            Choose $a=7$, and suppose that $7x\equiv 7y\pmod{12}$.

            So $12|7x-7y$.

            It will prove helpful to check that $\gcd(12,7)=1$.
            \begin{align*} 
                \gcd(12,7)&=\gcd(7,5)\\
                &=\gcd(5,2)\\
                &=\gcd(2,1)\\
                &=\gcd(1,0)\\
                &=1
            \end{align*}

            By Euclid's Lemma, since $\gcd(12,7)=1$ and $12|7(x-y)$, we know that
            $12|x-y$ and $x\equiv y\pmod{12}$.

            Therefore, there exists a positive integer $a>1$ so that for all $x,y\in\mathbb{Z}$, 
            if $ax\equiv ay\pmod{12}$, then \\$x\equiv y\pmod{12}$.
        \end{proof}
        
        \item For all positive integers $a$ and $b$, if $a^2\equiv b^2\pmod{12}$,
        then $a\equiv b\pmod{12}$.

        The statement is false. The negation is: 
        "There exist positive integers $a$ and $b$ so that \\ $a^2\equiv b^2\pmod{12}$
        but $a\not\equiv b\pmod{12}$."

        \begin{proof}
            Choose $a=12$, $b=6$.

            Then $144\equiv36\pmod{12}$ since $12|144-36$, because
            $12(9)=108$.

            However, $12$ is not congruent to $6$ modulo $12$ because
            $12\nmid12-6$ since $12>6$.
            
            Therefore, there exist positive integers $a$ and $b$ so that \\ $a^2\equiv b^2\pmod{12}$
            but $a\not\equiv b\pmod{12}$.
        \end{proof}

        \end{enumerate}


    \end{enumerate}

\end{document}
