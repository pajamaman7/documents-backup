\documentclass{article}
\usepackage[most,many,breakable]{tcolorbox}
\usepackage{amsmath}
\usepackage{amssymb}
\usepackage{amsthm}
\usepackage[]{thmbox}
\usepackage{blindtext}
\usepackage[utf8]{inputenc}
\usepackage{amsmath}
\usepackage{amsfonts}
\usepackage[]{graphicx}
\usepackage[legalpaper, portrait, margin = 1in]{geometry}
\usepackage{enumitem}


\usepackage{xcolor}

%\pagecolor[rgb]{0.2,0.19,0.18} 
%\color[rgb]{0.92,0.86,0.7}

\newtheorem[L]{le}{Lemma}[subsection]
\newtheorem[L]{th}[le]{Theorem}
\newtheorem[L]{df}[le]{Definition}
\newtheorem[L]{ex}[le]{Example}
\newtheorem[L]{pf}[le]{Proof}


\newcommand{\nl}{\newline}

\newcommand{\real}{\mathbb{R}}
\newcommand{\complex}{\mathbb{C}}
\newcommand{\integer}{\mathbb{Z}}
\newcommand{\rational}{\mathbb{Q}}
\newcommand{\lxor}{\oplus}
\newcommand{\then}{\Rightarrow}

\begin{document}
    \huge Problem Set 2 - Thomas Boyko - 30191728
    \normalsize
\begin{enumerate} 

    \item \begin{enumerate}[label= (\alph*)] 
        \item  Solve $5x \equiv 11 \pmod{37}$ and $11 y \equiv 5 \pmod{37}$.

            First we will solve for $x$. Note that this equation has a solution since $GCD(5,37)=1$.
            Transform the equivalence into a Diophantine equation. We have $5x+37a=11$. We now can use 
            the Euclidean algorithm to find $x,a$. 

            This gives the equation $1=-2(37)+15(5)$, which we can multiply by $11$ to obtain 
            $11=-22(37)+165(5)$.

            We can take this $\pmod{37}$ to give us $11\equiv 165(5)\equiv 17(5) \pmod{37}$.

            So $x\equiv 17\pmod{37}$.

            Now we can find $y$. Note that this equation is solvable since $GCD(11,37)=1$
            and it provides the Diophantine equation $11y+37b=5$. 

            Solving this equation gives the solution $1=3(37)-10(11)$, which can be multiplied by $5$ to
            give us $5=15(37)-50(11)$. Reducing this $\pmod 37$ again gives 
            $5\equiv -50(11)\equiv 24(11)\pmod{37}$, which shows that $x\equiv 24\pmod{37}$.

        \item  Suppose your solutions are $x_0$ and $y_0$. What is the relationship between $[x_0]$ and
            $[y_0]$ in $\mathbb{Z}_{37}$?

            We can see computationally that $[x_0][y_0] =[x_0y_0]=[408]=[1]$. So $x_0$ and $y_0$ are 
        multiplicative inverses for each other in $\mathbb{Z}_{37}$.
\end{enumerate}

\item Use repeated squaring method to simplify $12^{149}\pmod{15}$.

First we will calcluate repeated squares, until we obtain $12^{128}\pmod{15}$.

\begin{align*}
    12&\equiv 12\pmod{15}\\
    12^2\equiv 144 &\equiv 9\pmod{15}\\
    12^{4}\equiv 81&\equiv 6\pmod{15}\\
    12^{8}\equiv 36&\equiv 6\pmod{15}\\
    12^{16}\equiv 36&\equiv 6\pmod{15}\\
    12^{32}\equiv 36&\equiv 6\pmod{15}\\
    12^{64}\equiv 36&\equiv 6\pmod{15}\\
    12^{128}\equiv 36&\equiv 6\pmod{15}\\
.\end{align*}

We can see from above that $6^n$ for any positive $n$ is equivalent to $6\pmod{15}$.

Now we must find the binary representation for $149$. We find that $149=128+16+4+1$.

So we may write $12^{149}\equiv12^{128}12^{16}12^{4}12^{1}\equiv (6)(6)(6)(12)\equiv (6)(12)\equiv 72\equiv 12\pmod{15}$.

So $12^{149}\equiv 12\pmod{15}$.

\item Let $p$ be a prime number.
\begin{enumerate}[label= (\alph*)] 
    \item Show that $\binom{p}{k}\equiv 0\pmod{p}$ for all $k\in\mathbb{Z}$ with $1\leq k<p$.
    \begin{proof} 
        Consider:
        \begin{align*}
            \binom{p}{k}  &= \frac{p!}{ k! (p-k)!}=\frac{p(p-1)!}{k!(p-k)!} \\
            \binom{p}{k}k!(p-k)!&= p(p-1)! \\
        .\end{align*}
        
        Since $\binom{p}{k}$ is an integer, $k!(p-k)!|p(p-1)!$. So $k!(p-k)!$ must divide $(p-1)!$ since 
        $p$ is prime. This means $\frac{(p-1)!}{k!(p-k)!}\in\mathbb{Z}$, and $p|\binom{p}{k}$ which means 
        $\binom{p}{k}\equiv 0\pmod{p}$.
    \end{proof}
 
\item Show that for all integers $x,y$, $(x + y)^{p} \equiv  x^p + y^p \pmod{p}$.
    \begin{proof} 
        We begin by writing $(x+y)^{p}$ according to the binomial expansion. 
        
        \begin{align*}
            (x+y)^{p} &=\sum_{k=0}^{p} \binom{p}{k} x^{p-k}y^{k}\\
                     &= x^p+y^p+\sum_{k=1}^{p-1} \binom{p}{k}x^{p-k}y^{k} \\
        \end{align*}

        $\pmod{p}$, this gives us:
        \[
            (x+y)^{p} \equiv  x^p+y^p+\sum_{k=1}^{p-1} \binom{p}{k}x^{p-k}y^{k}\pmod{p}
        .\] 
        And since we know that $\binom{p}{k}\equiv 0$ for $1\leq k\leq p-1$, we can say that
        \[
            (x+y)^{p}\equiv x^{p}+y^{p}\pmod{p}
        .\] 
    \end{proof}
\end{enumerate}
\item Deduce from the previous problem : for all integers $a, a^p \equiv  a \pmod{p}$.

    \begin{proof} 
        Argue by induction. 
        Let $p$ be prime and suppose $(a+b)^{p}\equiv a^{p}+b^{p}\pmod{p}$.

        Base case: $a\equiv 0\pmod{p}$: $a^{p}w\equiv 0^{p}\equiv 0\equiv a\pmod{p}$.

        So the base case holds.

        Inductive Hypothesis: 
        Let $k\in\mathbb{Z}_{p}$ and suppose $k^{p}\equiv k\pmod{p}$. We must show that 
        $(k+1)^{p}\equiv k+1\pmod{p}$.

        $(k+1)^{p}\equiv k^{p}+1^{p}\equiv k+1\pmod{p}$.

        So for a prime $p$ and any integer $a$, $a^{p}\equiv a\pmod{p}$.
    \end{proof}
\item Show that the polynomial $x^{6}+45x^{4}-10x^2+5x-2$ has no integer solution.

    \begin{proof} 
        Consider $f(x)\pmod{5}$ 
        \[
        x^{6}+45x^{4}-10x^2+5x-2\equiv x^{6}-2\pmod{5}
        .\] 
        So we have
        \[
        x^{6}\equiv 2\pmod{5}
        .\] 
        We can check by cases:
        \begin{align*}
            0^{6}&\equiv 0\pmod{5}\\
            1^{6}&\equiv 1\pmod{5}\\
            2^{6}&\equiv 4\pmod{5}\\
            3^{6}&\equiv 4\pmod{5}\\
            4^{6}&\equiv 1\pmod{5}
        .\end{align*}
        And since $\exists m\in \mathbb{Z}$ so that $f(x)\pmod{m}$ has no integer solutions, 
        $f$ has no integer solutions.
    \end{proof}

\end{enumerate}
\end{document}
