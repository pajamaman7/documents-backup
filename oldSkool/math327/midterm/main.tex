\documentclass{article}
\usepackage{amsmath}
\usepackage{amssymb}
\usepackage{amsthm}
\usepackage[utf8]{inputenc}
\usepackage{amsmath}
\usepackage{amsfonts}
\usepackage[]{graphicx}
\usepackage[a4paper, portrait, margin = 1in]{geometry}
\usepackage{enumitem}
\usepackage{xcolor}

%darkmode
%\pagecolor[rgb]{0.2,0.19,0.18} 
%\color[rgb]{0.92,0.86,0.7}

\newenvironment*{alphenum}{\begin{enumerate}[label= (\alph*)]}{\end{enumerate}}


\begin{document}
    \huge Take Home Midterm - Thomas Boyko - 30191728
    \normalsize
\begin{enumerate} 
\item Let $m, n$ be positive, coprime integers i.e. $GCD(m, n) = 1$. Show that if $k|mn$ then there
exists integers $a, b$ such that $k = ab$ with $a|m$ and $b|n$.
\begin{proof} 
Let $m,n\in\mathbb{Z}$ and $GCD(m,n)=1$. Let $k\in\mathbb{Z}$ so that $k|mn$. 
Write $m$ and $n$ according to the Fundamental Theorem of Arithmetic.

\begin{align*}
    m&=p_1^{\alpha_1}p_2^{\alpha_2}\ldots p_i^{\alpha_i}\\
    n&=q_1^{\beta_1}q_2^{\beta_2}\ldots q_j^{\beta_j}  
.\end{align*}

With $\alpha,\beta\in 0,1,2\ldots$ and all $p\neq q$ (since $GCD(m,n)=1$. Now we can write:

\[
mn=p_1^{\alpha_1}p_2^{\alpha_2}\ldots p_i^{\alpha_i}q_1^{\beta_1}q_2^{\beta_2}\ldots q_j^{\beta_j}  
   .\] 
   And since $k|mn$:
   \[
   k=p_1^{\gamma_1}p_2^{\gamma_2}\ldots p_i^{\gamma_i}q_1^{\delta_1}q_2^{\delta_2}\ldots q_j^{\delta_j} 
   .\] 
   With $0\le \gamma\le \alpha$ and $0\le \delta\le\beta$.

   If we choose $a=p_1^{\gamma_1}p_2^{\gamma_2}\ldots p_i^{\gamma_i}$ and 
   $b=q_1^{\delta_1}q_2^{\delta_2}\ldots q_j^{\delta_j}$, then 
   $ab=p_1^{\gamma_1}p_2^{\gamma_2}\ldots p_i^{\gamma_i}q_1^{\delta_1}q_2^{\delta_2}\ldots q_j^{\delta_j}
   = k$, with both $a|m$ and $b|n$.

   So there exist $a,b\in \mathbb{Z}$ so that $ab=k$ and $a|m$, $b|n$.
\end{proof}
\item  Sam-I-am wants exactly 600 of his daily calories to come from green eggs and ham. Each slice
of ham has 102 calories, and each green egg has 18 calories. Find all combinations of green
eggs and ham, through which this can be done?

To begin we find $GCD(102,18)$ and checking that it divides $600$.
\begin{align*}
    102&=5\times 18+12\\
    18&= 12\times 1+6 \\
    12&=6\times 2+0  
.\end{align*}

So $GCD(102,18)=6$, and $6|600$. Now we can setup the Diophantine Equation: 
\[
18e+102h=600
.\] 
We obtain $(e_0,h_0)=(600,-100)$ by reversing the steps we used to find $GCD$, 
and from this we can find general solutions:
\[
h=-100+3n,\quad e=600-17n
.\] 

Now we should check which values of $n$ give us positive numbers of both eggs and ham. 
If $n\le 33$, we have negative $h$. And if $n\ge 36$ we have 
negative $e$. So the only valid solutions are when $n=35$ and $n=34$. Substituting this back into our formula for $h$ and
$e$, we find that either $e=22$ and $h=2$ or $e=5$ and $h=5$.

So Sam-I-am can have either 22 eggs and two slices of ham, or he can have 5 eggs and 5 slices of ham.

\item Find all integers $n$ such that $106n$ and $n$ has the same last two digits.
\begin{proof} 
    We can create an equivalence which considers only the last two digits of $106n$ and $n$:
    $106n\equiv n\pmod{100}$. Note that $106\equiv 6\pmod{100}$. So we are given the new equivalence:
    $6n\equiv n\pmod{100}\implies5n\equiv 0\pmod{100}$. Converting this to a divisibility statement, we get:
    $100|5n$. So $100k=5n$ for some $k\in\mathbb{Z}$.

    Finally, we obtain $n=20k$. So for any $n$ of the form $n=20k$, $n\equiv 106n\pmod{100}$. 
\end{proof}
\newpage
\item Let $a, b$ and $n$ be positive integers. Prove that if $a^n|b^n$ then $a|b$.
\begin{proof} 
    Let $a,b,n\in \mathbb{Z}^{+}$, and $a^n|b^n$.

    Write the prime decomposition of $a,b$:
    \begin{align*}
        a&=p_1^{\alpha_1}p_2^{\alpha_2}\ldots p_i^{\alpha_i}\\
        b&=p_1^{\beta_1}p_2^{\beta_2}\ldots p_i^{\beta_i} \\ 
        a^{n}&=p_1^{n\alpha_1}p_2^{n\alpha_2}\ldots p_i^{n\alpha_i}\\
        b^{n}&=p_1^{n\beta_1}p_2^{n\beta_2}\ldots p_i^{n\beta_i}  
    .\end{align*}
    Where all $p$ are primes, and all $\alpha,\beta\in \{0,1,2,\ldots\} $. Since $a^{n}|b^{n}$,
    $n\alpha_j\le n\beta_j$ for all $1\le j\le i$. Dividing by $n$ we can see that $\alpha_j\le \beta_j$,
    which means $a|b$.
    
    So for positive integers $a,b,n$, if $a^{n}|b^{n}$ then $a|n$.

\end{proof}

\item Show that $x^{10} + y^{10} - 11z^{10} = 5$ has no integer solution.
\begin{proof} 
    Let the equation above be $E$.
    To show that our equation has no integer solutions, we must find a modulus $m$ so that $E\pmod{m}$ has 
    no solutions.

    Choose $m=11$. Then our equation becomes $x^{10}+y^{10}\equiv 5\pmod{11}$. Using Fermat's little
    Theorem, we can say that $x^{10}\equiv 1\pmod{11}$ and we obtain $1+1\equiv 2\equiv 5\pmod{11}$. So no
    matter our choices for $(x,y,z)$, our left hand side of our equation $E$ will be equivalent to $2$
    while the right is $5$. So the equation has no solution $\pmod{11}$ and therefore no solutions in 
    $\mathbb{Z}$.
\end{proof}
\end{enumerate}
\end{document}
