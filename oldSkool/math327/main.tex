\documentclass{article}
\usepackage{amsmath}
\usepackage{amssymb}
\usepackage{amsthm}
\usepackage[utf8]{inputenc}
\usepackage{amsmath}
\usepackage{amsfonts}
\usepackage[]{graphicx}
\usepackage[a4paper, portrait, margin = 1in]{geometry}
\usepackage{enumitem}
\usepackage{xcolor}

%darkmode
%\pagecolor[rgb]{0.2,0.19,0.18} 
%\color[rgb]{0.92,0.86,0.7}

\newenvironment*{alphenum}{\begin{enumerate}[label= (\alph*)]}{\end{enumerate}}


\begin{document}
    \huge Title - Thomas Boyko - 30191728
    \normalsize
\begin{enumerate} 
    \item Prove that if $p$ and $q$ are distinct prime numbers, then 
        $\sqrt{p} +\sqrt{q}$ is irrational.
        \begin{proof} 
            Let $p,q$ be distinct primes and $\alpha=\sqrt{p} +\sqrt{q} $. Suppose for the sake
            of contradiction that $\alpha\in \mathbb{Q}$. Squaring both sides,
            $\alpha^2=p+q+2\sqrt{pq} $, and $\frac{\alpha^2-p-q}{2}=\sqrt{pq}$.
            The left side of this equation is rational, so if we can show that $\sqrt{pq} \not\in 
            \mathbb{Q}$, we have a contradiction.

            Suppose for the sake of contradiction that $\sqrt{pq} =\frac{a}{b}$ for some coprime $a,b$ with $b>0$.
            Then $pq=\frac{a^2}{b^2}$ and $b^2pq=a^2$. So $b^2|a^2$, which means that $b|a$, and $GCD(a,b)=b$. But 
            $GCD(a,b)=1$, so we have our contradiction.
        \end{proof}
    \item The number e is defined as the sum of the reciprocals of the factorials, If $e$ were rational, let $n$ be its denominator when represented as a fraction, let $x$ be the sum of the terms up to $\frac{1}{n!}$,
and let $y$ be the sum of the rest of the terms. Demonstrate in
this case that $n! x$ is an integer and $n! e$ is an integer, and that
$0 < n!  y < 1$. Use this to achieve a contradiction, and fill in the
steps to prove that e is irrational. 
\begin{proof} 
    Let $e\in \mathbb{Q}$ for the sake of contradiction. That is, 
    $e=\frac{m}{n}$ for some $m,n\in \mathbb{Z}$, $n\neq 0$.
    
    Write $e=1+\frac{1}{1!}+\frac{1}{2!}+\frac{1}{3!}\ldots=x+y$, where 
    $x=1+\frac{1}{1!}+\frac{1}{2!}+\frac{1}{3!}+\ldots +\frac{1}{n!}$ and 
    $y$ is the remaining terms of the sum.

    We see that: 
    \begin{align*}
        n!x&= n!\left(1+\frac{1}{1!}+\frac{1}{2!}+\ldots+ \frac{1}{n!}\right) \\
        &= n!+ \frac{n!}{1!} +\frac{n!}{2!}+\ldots +1 
    .\end{align*}
    And since each term in this sum is an integer, we can say that 
    $n!x\in \mathbb{Z}$.

    Since $n!e=n!x+n!y$, if we can show that $0\le n!y\le 1$, the left side of 
    the equation is an integer and the right is between two integers, which is  
    a contradiction.

    Consider: 
    \begin{align*}
    n!y=n!\left( \frac{1}{(n+1)!} +\frac{1}{(n+2)!}+\ldots\right) \\
    &=\frac{1}{(n+1)} +\frac{1}{(n+2)(n+1)}+\ldots\\
    &\le \frac{1}{n+1}+\frac{1}{(n+1)^2}\ldots \\
    .\end{align*}
     
    Comparing this to the geometric series $a_n=\left(  \frac{1}{3}^{n} \right)$, 
    since $n\ge2$ (if $n=1$, $e$ is an integer),
    we get
    $n!y\le \frac{1}{3}+\frac{1}{3^2}+\ldots$.

    Using the formula for the sum of an infinite geometric series,
    $S=\frac{1}{1-r}$, we know that this sum must be less than $\frac{1}{2}$. 
    So we have reached our contradiction.

\end{proof}
\item
\begin{enumerate}[label= (\alph*)] 
    \item Let $z$ be a complex number. Prove that if $z$ is a Gaussian integer
    and an Eisenstein integer, then $z$ is an ordinary integer. 
    \begin{proof} 
        Let $z\in \mathbb{Z}[i]$ and $z\in \mathbb{Z}[\omega]$. Then for some 
        $a,b,c,d\in \mathbb{Z}$:
        \begin{align*}
            z=a+bi&=c+d\omega\\
            &= c+d(1-\sqrt{3} i)\\
            &= (c+d) -d\sqrt{3} i
        .\end{align*}

        We equate the imaginary parts and the real parts. So 
        $bi=-d\sqrt{3} i$, and $b=-d\sqrt{3} $. Since $\sqrt{3}\not \in\mathbb{Q}$,
        the only way for this equality to hold is if $b=d=0$. Therefore
        $z=a+0i$ and so $a\in \mathbb{Z}$.
    \end{proof}
    \item  Let $n$ be a positive integer. Let($\zeta= e^{2i\pi/n}$. Prove that:
     \[
        1+\zeta+\zeta^2+\ldots+\zeta^{n-1}
        .\] 
        Note: If $n=1$ then the sum is equal to $1$, i will take the case 
        $n>1$.
        \begin{proof} 
            Let $n\in \mathbb{Z}_{>1}$ and $\zeta=e^{2i\pi /2}$, and let: 
            \[
            \alpha=1+\zeta+\zeta^2+\ldots+\zeta^{n-1}
            .\] 
            So $\zeta\alpha=\zeta+\zeta^2+\zeta^3+\ldots+\zeta^{n}$.

            Consider $\zeta^{n}=e^{n2i\pi /n}=e^{2i\pi}=1$.

            Therefore $\zeta\alpha=1+\zeta+\zeta^2+\ldots+\zeta^{n-1}=\alpha$.
            
            And so $\alpha(\zeta-1)=0$. So either $\alpha$ or $\zeta-1$ is zero. $\zeta-1$ can only be $0$ if 
            $\zeta=1$, but since $n\ge 2$, this cannot be the case. Therefore $\alpha=0$.


        \end{proof}
\end{enumerate}

\item Factorize 85 into Gaussian primes.

    First let's take the ordinary prime factorization, $85=5\times 17$. 
    Conveniently both these primes are one more than a perfect square, so we can
    write $17=(4+i)(4-i)$ and $5 = (2+i)(2-i)$. 

    We can check using the results in the bonus that these are Gaussian Primes.

    So the prime factorization of $85\in \mathbb{Z}[i]$ is $85=(2+i)(2-i)
    (4-i)(4+i)$.
\item Let $x$ be an integer. Prove that $x + i$ is a Gaussian prime number
if and only if $x^{2} + 1$ is an ordinary prime number 

\begin{proof} 
    $\implies$: Let $x+i$ be a prime Gaussian integer. Then if $x+i=uv$ for some 
    Gaussian integers $u,v$, then $u$ or $v$ is a unit.
    Take the square modulus of each side.

    $x^2+1=|u|^2|v|^2$. Since one of these must be a unit, we can say that $x^2+1$
    must be one times the square modulus of the other. Therefore $x^2+1$ is prime.

    $\impliedby$: Let $x^2+1$ be a prime integer. $x$ must be even since $x^2+1$
    is odd. Since even squares are $\equiv 0\pmod{4}$, $x^2+1\equiv 1\pmod{4}$.
    So we can use Fermat's Christmas Theorem, to say that 
    $x^2+1=a^2+b^2=(a+bi)(a-bi)$. (could not finish this part).

\end{proof}
\end{enumerate}
\end{document}
