\documentclass{article}
\usepackage{amsmath}
\usepackage{amssymb}
\usepackage{amsthm}
\usepackage[utf8]{inputenc}
\usepackage{amsmath}
\usepackage{amsfonts}
\usepackage[]{graphicx}
\usepackage[a4paper, portrait, margin = 1in]{geometry}
\usepackage{enumitem}
\usepackage{xcolor}

%darkmode
%\pagecolor[rgb]{0.2,0.19,0.18} 
%\color[rgb]{0.92,0.86,0.7}

\newenvironment*{alphenum}{\begin{enumerate}[label= (\alph*)]}{\end{enumerate}}


\begin{document}
    \huge Problem Set 3 - Thomas Boyko - 30191728
    \normalsize
\begin{enumerate} 
    \item \begin{enumerate}[label= (\alph*)] 
        \item Use the Chinese Remainder Theorem to determine all the solutions to $x^2+1\equiv 0\pmod{1313}$.

            Rework the equation to obatin $x^2\equiv -1\pmod{1313}$. So $1313|x^2+1$. Then $101|x^2+1$
            and $13|x^2+1$.

            Now we are given the modular equations $x^2\equiv -1\pmod{101}$ and $x^2\equiv -1\pmod{13}$
            and if we can solve for both of these squares we can use Chinese Remainder theorem to 
            find the general solution $\pmod{1313}$.

            By trial and error we can find: $5^2\equiv (-5)^2\equiv-1\pmod{13}$; as well 
            $10^2\equiv (-10)^2\equiv 100\equiv -1\pmod{101}$. Convenient!

            To use Chinese Remainder Theorem, we can do most of the setup once and extend it to the 
            rest of our cases.

            \begin{align*}
                x&\equiv a\pmod{101}\\
                x&\equiv b\pmod{13}\\
                m&= 101 \\
                n&= 13 \\
                t&= 70 \\
                s&= 4 
            \end{align*}

            And using $x_0=sna+tmb$, we can get our four solutions:
            \begin{align*}
                x_0&=(4)(101)(5)+(70)(13)(10)=11120&\equiv 616\pmod{1313}\\
                x_1&=(4)(101)(-5)+(70)(13)(10)=7080&\equiv 515\pmod{1313}\\
                x_2&=(4)(101)(5)+(70)(13)(-10)=-7080&\equiv 798\pmod{1313}\\
                x_3&=(4)(101)(-5)+(70)(13)(-10)=-11120&\equiv 697\pmod{1313}
            \end{align*}


        \item Is 17 a square modulo 104?

        Let $x^2\equiv 17\pmod{104}$. Then $104|x^2-17$, so $13|x^2-17$ and $8|x^2-17$. We can 
        transform both of these divisiblilty statements into congruences:
    
        First, $x^2\equiv 17\equiv 4\pmod{13}$. Clearly a solution is given by 
        $x\equiv \pm 2\pmod{13}$. And since $13$ is an odd prime, these are our only solutions.

        Next, $x^2\equiv 17\equiv 1\pmod{8}$ Write $8=2^3$ and we can see our solutions will be given
        by $\pm 1$, and $p^{3-1}\pm1=4\pm 1$. So our solutions to $x^2\equiv 17\pmod{8}$ are $-1,1,3,$
        and $5$.

        Since we have a set of congruences, and $8,13$ are coprime, we know from Chinese Remainder 
        Theorem that each of these pairs of congruences has a unique solution, and therefore 
        $x^2\equiv 17\pmod{104}$ has a solution and $17$ is a square mod $104$.
    \end{enumerate}

\item Let $p=47$, $q=59$, $N=pq=2273$, and $e=157$.
    \begin{enumerate}[label= (\alph*)] 
        \item Compute a multiplicative inverse $d$, modulo $\phi(N)$.

            We must first find $\phi(2273)$ with its prime factorization, $N=47\times 59$.
            \begin{align*}
            \phi(N)&=\left( 2773\left( 1-\frac{1}{47} \right) \left( 1-\frac{1}{59} \right)  \right) \\
            &= 2668 
            .\end{align*}
            $d,e$ will be inverses $\pmod{2668}$.
            So we want to find $d$ so that $157d\equiv 1\pmod{2668}$. We obtain the Diophantine 
            Equation $157d+2668y=1$.

            Using the inverse Euclidean Algorithm we obtain $d=17$. So the inverse of $157\pmod{2668}$
            is $17$.
            
        \item Every two-letter string (including A-Z and spaces)
            can be converted to a number-message between 0 and 2626, by replacing
            a space by 00, A: by 01, B by 02, etc ... For example, 
            ME becomes 1305. Encrypt the two-letter string HI by computing its
            number-message $m$, and the ciphertext $m^{e}$ mod $N$.

            HI becomes $1309$, and $1309^{157}\equiv 840\pmod{2773}$.

        \item Decrypt the sequence of ciphertexts $0802,2179,2657,1024$ to find a message.

            We use our decryption key $d=17$, and $M^{d}\pmod{N}$ to find:
            \begin{align*}
                802^{17}\equiv 2305\pmod{2773}\\
                2179^{17}\equiv 1212\pmod{2773}\\
                2657^{17}\equiv 269\pmod{2773}\\
                1024^{17}\equiv 1405\pmod{2773}
            .\end{align*}

            Some of these numbers do not correspond to letters but the message seems to be 
            WELLB?NE.
            

    \end{enumerate}

\item Prove that if $p$ is a prime and $a\equiv b\pmod{p^2-p}$, then $a^{a}\equiv b^{b}\pmod{p}$.
    \begin{proof} 
        Since $a\equiv b\pmod{p^2-p}$, $p^2-p|a-b$. From this we can say that $p|a-b$ and $p-1|a-b$.
        So $a\equiv b\pmod{p}$ and $a\equiv b\pmod{p-1}$.

        From $a\equiv b\pmod{p-1}$ we can say that $k\phi(p)=a-b$. This allows us to derive:
        \[
            a^{a-b}\equiv a^{\phi(p)k}\equiv a^{\phi(p)^{k}}\equiv 1^{k}\equiv 1\pmod{p}
        .\]  
        And since $a^{a-b}\equiv 1\pmod{p}$, $a^{a}\equiv a^{b}\pmod{p}$.

        Finally, because $a\equiv b\pmod{p}$, $a^{b}\equiv b^{b}\pmod{p}$.
        And therefore, $a^{a}\equiv b^{b}\pmod{p}$.
    \end{proof}

\item Let p be a prime. Define the map $v_p : \mathbb{Z}^{*} \to \mathbb{Z}$ by $v_p(n) = e$
    , if $e$ is the highest exponent
    with which $p$ occurs in the factorization of $n$. For example $v_2(20) = 2, v_5(-20) = 1$
    etc. We can extend this map to $v_p: \mathbb{Q}^* \to \mathbb{Z}$ by 
    $ v_p(\frac{m}{n})=v_p(m)-v_p(n),$
    where $m$ and $n$ are at lowest terms, i.e. $GCD(m, n) = 1$. For example, for $p = 3$, 
    $v_3(\frac{18}{17})=2$
Show that:
\begin{enumerate}[label= (\alph*)] 
    \item $v_p(r \cdot s) = v_p(r) + v_p(s)$ for any $r, s \in \mathbb{Q}^*$.

        Let $r,s\in \mathbb{Q}^{*}$. Write the prime factorizations of $r,s$:
        \begin{align*}
            r&= p_1^{\alpha_1}p_2^{\alpha_2}\ldots p_r^{\alpha_j}\\
            s&= p_1^{\beta_1}p_2^{\beta_2}\ldots p_r^{\beta_j} 
        .\end{align*}
        With all $\alpha,\beta\in \mathbb{Z}$ since $r,s\in \mathbb{Q}$. If a prime is not in the 
        prime factorization, then it will simply be raised to the power of $0$.
        Suppose we are taking $v_{p_i}(r)$ for some $1\le i\le j$. 
        Since a prime can only appear in either the numerator or denominator, we write 
        $v_{p_i}(r)=\alpha_i$, and for $s$, $v_{p_i}(s)=\beta_i $.

        Now we write
        $rs= p_1^{\alpha_1+\beta_1}p_2^{\alpha_2+\beta_2}\ldots p_r^{\alpha_r +\beta_r} $.
        Therefore, $v_{p_i}(rs)=\alpha_i+\beta_i=v_{p_i}(r)+v_{p_i}(s)$.

    \item $v_p(r + s) \ge \min\{v_p(r), v_p(s)\}$ for any $r, s \in \mathbb{Q}^{*}.$

        Take the prime factorizations of $r,s$ as in part (a). Again, as above, $v_{p_i}(r)=\alpha_i$
        and $v_{p_i}(s)=\alpha_i$, for some $1\le i\le j$.

        Then write:
            $$r+s= p_1^{\alpha_1}p_2^{\alpha_2}\ldots p_r^{\alpha_j}
            + p_1^{\beta_1}p_2^{\beta_2}\ldots p_r^{\beta_j} $$.
        If $\alpha_i<\beta_i$, we can factor out $p_i^{\alpha_i}$, and the opposite holds as well.
        In other words, we can factor out whichever is lower, $p_i^{\alpha_i}$ or $p_i^{\beta_i}$.

        So $r+s$ becomes
            $$r+s= p_i^{\min \{\alpha_i, \beta_i\} }(p_1^{\alpha_1}p_2^{\alpha_2}\ldots 
            p_i^{\alpha_i-\min \{\alpha_i, \beta_i\} }\ldots p_r^{\alpha_j}
            + p_1^{\beta_1}p_2^{\beta_2}\ldots p_i^{\beta_i-\min \{\alpha_i, \beta_i\} }
            \ldots p_r^{\beta_j}) .$$
        And therefore $v_p(r+s)$ is greater than or equal to $\min \{\alpha_i,\beta_i\}$;
        $p_i^{\min \{\alpha_i, \beta_i\} }$ is guaranteed to divide $r+s$, but 
        $p_i$ might also divide the other part of our product.
        Therefore, $v_p(r + s) \ge \min\{v_p(r), v_p(s)\}$
    \item The map $v_p$ is onto

        Suppose $y\in \mathbb{Z}$. In order to show that our map is onto, we must show that for any 
        choice of $y$, we have a choice of $r$ so that $y=v_p(r)$. Simply, we can choose
        $r=p^{y}$. Then $p^y$ is the highest exponent of $p$ that appears in the prime factorization
        of $r$, and $y=v_p(r)$. 

        So the map is onto.
\end{enumerate}
\item Define the p-adic absolute value $|.|_p$ as follows : $|.|_p : \mathbb{Q}^{*} \to \mathbb{R}$
    , given by, $|q|_p = p^{-v_p(q)}$.
In the example above, $v_3(\frac{18}{17})=2$ thus, $\left| \frac{18}{17} \right| =3^{-2}=\frac{1}{9}$.
Show that,
\begin{enumerate}[label= (\alph*)] 
\item $|r + s|_p \le  max\{|r|_p, |s|_p\}.$
    \begin{proof} 
        Let $p\in \mathbb{Z}$ and $r,s\in \mathbb{Q}^{*}$.

        Consider $\left| r+s \right| _p =p^{-v_p(r+s)}$. We know from question 4 b) that 
        $v_p(r + s) \ge \min\{v_p(r), v_p(s)\}$, so $-v_p(r + s) \le -\min\{v_p(r), v_p(s)\}$.

        From this we can write $p^{-v_p(r+s)}\leq p^{-\min \{v_p(r),v_p(s)\} }$.

        There are two cases. First the case that $v_p(s)\leq v_p(r)$. In this case, 
        $\min \{v_p(r),v_p(s)\}=v_p(s)$. Therefore,
        \[
            |r+s|_p=p^{-\min \{v_p(r),v_p(s)\}}=p^{-v_p(s)}=|s|_p\leq \max \{|r|_p,|s|_p\} 
        .\] 
        The other case is where $v_p(r)<v_p(s)$.
        \[
            |r+s|_p=p^{-\min \{v_p(r),v_p(s)\}}=p^{-v_p(r)}=|r|_p\leq \max \{|r|_p,|s|_p\} 
        .\] 
        So in either case, $|r+s|_p\leq \max \{v_p(r),v_p(s)\}$.
    \end{proof}
\item Show that if $r \in  \mathbb{Z}$, then $|r|_p \le 1$. In fact, describe the set 
    $\{r\in \mathbb{Q}^{*}: \, |r|_p \le 1\} $.

    \begin{proof} 
        Let $r\in \mathbb{Z}^{*}$ in lowest terms, and suppose that $p$ is some chosen prime.

        If $p$ is not in the prime factorization of $r$, then $v_p(r)=0$, and 
        $\left| r \right|_p=p^{-0}=1$.

        If $p$ is in the prime factorization, it is raised to some $\alpha\in \{0,1,\ldots\} $, and
        $v_p(r)=\alpha$.

        So $\left| r \right| _p=p^{-\alpha}=\frac{1}{p^\alpha}$. Since $\alpha\ge 0$, this must be 
        less than or equal to $1$.
    \end{proof}

    Considering rationals, we see that if $p$ divides the denominator of some $r=\frac{a}{b}$, then
$v_p(r)$ will be negative, and so $|r|_p$ will be a prime $p$ to some positive power, which must be
$\geq 1$.
\end{enumerate}
\end{enumerate}
\end{document}
