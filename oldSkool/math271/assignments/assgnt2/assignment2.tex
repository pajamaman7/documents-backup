\documentclass{article}

\usepackage{amsmath}
\usepackage{amssymb}
\usepackage{amsthm}
\usepackage[utf8]{inputenc}
\usepackage{amsmath}
\usepackage{amsfonts}
\usepackage[]{graphicx}
\usepackage[a4paper, portrait, margin = 1in]{geometry}
\usepackage{enumitem}
\usepackage{xcolor}

%darkmode
%\pagecolor[rgb]{0.2,0.19,0.18} 
%\color[rgb]{0.92,0.86,0.7}

\newenvironment*{alphenum}{\begin{enumerate}[label= (\alph*)]}{\end{enumerate}}

\begin{document}
\huge Assignment 2 - Thomas Boyko - 30191728
\normalsize
\begin{enumerate}
\item Let $a_n = \sum\limits_{i=1}^{n}\frac{2i-1}{2^i}$ where n is a positive integer.
\begin{alphenum}
\item Find $a_1, a_2, a_3,$ and $ a_4$.
\begin{align*}
    a_1 &= \frac{1}{2}\\
    a_2 &= \frac{5}{4}\\
    a_3 &= \frac{15}{8}\\
    a_4 &= \frac{37}{16}
\end{align*}
\item Guess a simple formula for $a_n$ for all integers $n\geq 1$.
\nl Guess: \[a_n = 3- \frac{2n+3}{2^n}\]

\item Prove by induction on $n$ that your guess in part (b) is correct.

\begin{proof}
    We will prove by induction.
    Suppose $n\in\integer$ and $n\geq1$.
    \nl Base case: $n=1$.
    \[\sum_{i=1}^{1}\frac{2i-1}{2^i} = \frac{2(1)-1}{2^1}=\frac{1}{2}
    =3-\frac{5}{2}=3-\frac{2(1)+3}{2^1}\]
    Induction Hypothesis: \nl 
    Suppose $k\in\integer$ and $k \geq 1$.
    Further suppose: 
    \[\sum_{i=1}^{k}\frac{2i-1}{2^i}=3-\frac{2k+3}{2^k}\] 
    We want to prove: 
    \[\sum_{i=1}^{k+1}\frac{2i-1}{2^i}=3-\frac{2k+5}{2^{k+1}}\] 
    Let's begin. 
    \[\begin{split}
        \sum_{i=1}^{k+1}\frac{2i-1}{2^i} 
        & = \frac{2(k+1)-1}{2^{k+1}}+\sum_{i=1}^{k}\frac{2i-1}{2^i} \\
        & = \frac{2(k+1)-1}{2^{k+1}}+3-\frac{2k+3}{2^k} \text{  (By the Induction Hypothesis)}\\
        & = 3 +\frac{2k+1}{2^{k+1}} - \frac{2(2k+3)}{2^{k+1}} \\
        & = 3 +\frac{2k+1-4k-6}{2^{k+1}} \\
        & = 3-\frac{2k+5}{2^{k+1}}
    \end{split}\]
        Therefore, by induction on $n$, $\sum\limits_{i=1}^{n}\frac{2i-1}{2^i}= 3- \frac{2n+3}{2^n}$
        for all positive integers $n$.
\end{proof}

\end{alphenum}

\newpage

\item The Fibonacci numbers $f_1, f_2, f_3, \ldots$ are defined by 
setting $f_1 = f_2 = 1$ and $f_n = f_{n-1} +f_{n-2}$
for all integers $n\geq 1$.
\par A fact that will become helpful in these proofs is that $\gcd(a+b,b) = \gcd(a,b)$.
\nl Here is a proof, so that we can use it in the following questions.
\begin{proof}
    Suppose $\gcd(a,b)=d$ where $d\in\integer$. $d$ is the greatest 
    integer that divides both $a$ and $b$.
    \nl So $d|b$ and $d|a$ which means $a=dk$ and $b=dl$ for some $k,l\in\integer$. \nl
    $a+b =kd+ld = d(l+k)$ where $l+k\in\integer$, which means $d |a+b$. \nl 
    \nl Now we will let $e=\gcd(a+b,b)$. $e$ is the greatest integer that divides
    both $a+b$ and $b$.
    \nl$em=a+b$ and $en=b$ for some $m,n\in\integer$.
    \nl$em=a+en$
    \nl$e(m-n)=a$ where $m-n\in\integer$ so $e$ divides $a$.
    \nl Since d divides both $a+b$ and $b$, $d\leq e$.
    \nl And since $e$ divides both $a$ and $b$, $e\leq d$.
    \nl So $e\leq d\leq e$ which means $d=\gcd(a,b)=\gcd(a+b,b)=e$.

\end{proof}
\begin{alphenum}
\item Prove that for all integers $n\geq3$, $\gcd(f_n,f_{n-1}) = \gcd(f_{n-1},f_{n-2})$
\begin{proof}
    Suppose $n\in\integer$, and $n\geq3$.\nl 
    Because $n\geq3$, $\gcd(f_{n},f_{n-1})=\gcd(f_{n-1}+f_{n-2},f_{n-2})$.\nl 
    As proved above, this is equal to $\gcd(f_{n-1},f_{n-2}).$\nl 
    Therefore, for all integers $n\geq3$, $\gcd(f_n,f_{n-1}) = \gcd(f_{n-1},f_{n-2})$
\end{proof}
\item Prove by induction on $n$ that $\gcd(f_n,f_{n-1}) = 1$ for all integers $n\geq2$.
\begin{proof}
    Suppose $n\in\integer,n\geq2$.\nl 
    Base cases: (For this we need to define $f_3=f_2+f_1=1+1=2$)\nl 
    $n=2$: $\gcd(f_2,f_1)=\gcd(1,1)=1$.\nl 
    $n=3$: $\gcd(f_3,f_2)=\gcd(2,1)=1$.\nl 
    Induction Hypothesis:
    \nl Suppose $k\in\integer, k>3$ \nl 
    Further suppose that for all $m\in\integer$ where $2\leq m < k$: 
    \[\gcd(f_m,f_{m-1})=1\]
    We want to prove: $\gcd(f_k,f_{k-1})=1$.
    \nl $\gcd(f_k,f_{k-1}) = \gcd(f_{k-1},f_{k-2})$ as proved in part (a). 
    \nl By the Induction Hypothesis, since $2\leq k-1<k\leq3$, $1=\gcd(f_k,f_{k-1})$
    \nl So, by induction on n, $\gcd(f_n,f_{n-1})=1$ for all $n\geq2$.
    
\end{proof}
\item Prove by induction on $n$ that $\sum\limits^{n}_{i=1}f_i^2=f_{n+1}f_n$ for all integers $n\geq 1$.
\begin{proof}
    Suppose $n\in\integer$, $n\geq1$.
    \nl Base case: $n=1$.
    \[\sum\limits_{i=1}^{1}f_i^2=1^2=1=1\times1=f_2\times f_1\]
    \nl Induction Hypothesis:
    \nl Suppose $k\in\integer$, $k\geq1$.
    \nl Further suppose that $\sum\limits_{i=0}^{k}f_i^2=f_{k+1}f_k$.
    \nl We want to prove that: $\sum\limits_{i=0}^{k+1}f_i^2=f_{k+2}f_{k+1}$.
    \nl Let's begin.
    \[\begin{split}
        \sum\limits_{i=0}^{k+1}f_i^2 &= {f_{k+1}}^2\sum\limits_{i=0}^{k}f_i^2\\
        &= {f_{k+1}}^2+f_{k+1}f_k \text{ By the Induction Hypothesis}\\
        &= f_{k+1}(f_k + f_{k+1})\\
        &= f_{k+1}(f_{k+2})
    \end{split}
    \]
    Therefore, by induction on $n$, $\sum\limits^{n}_{i=1}f_i^2=f_{n+1}f_n$ for all integers $n\geq 1$.
\end{proof}
\end{alphenum}
\newpage 
\item For each of the following statements, if the statement is true
, then give a proof, and if the statement is false, 
then write out its negation and prove that.
\begin{alphenum}
\item For all sets $A,B$, and $C$, $A \cup (B\cap C)$ is a subset of 
$(A\cup B)\cap (A\cup C)$.

\begin{proof}
Suppose $A,B$ and $C$ are sets. Let $x\in A\cup(B\cap C)$. \nl
So, $x$ is in $A$, or $x$ is in both $B$ and $C$.\nl
\nl Case 1: $x\in A$
\nl If $x\in A$, x is in both $A\cup B $ and $A \cup C$.
\nl So, $x\in(A\cup B)\cap (A\cup C)$ if $x \in A$.\nl 
\nl Case 2: $x\not\in A$
\nl If $x\not\in A$, $x\in B\cap C$. 
\nl This means that $x\in C$ and $x\in B$.
\nl So x must be in both $A\cup B$ and $A\cup C$.
\nl Therefore, if $x\not\in A$, $x\in (A\cup B)\cap (A\cup C)$
\nl\nl Since we know that all $x$ must be in $A$ or ($B$ and $C$),
\nl And we know that all $x$ in $A \cup (B\cap C)$ must be in $(A\cup B)\cap (A\cup C)$, 
\nl $A \cup (B\cap C)$ is a subset of 
$(A\cup B)\cap (A\cup C)$.

\end{proof}

\item For all sets $A,B$, and $C$, if $A\subset B$, then $(C-A)\subset(C-B)$.
\nl The statement is false, so we will negate it and prove that.
\nl The negation is: "There exist sets $A,B$, and $C$ so that $A\subset B$ but $(C-A)\not\subset(C-B)$."

\begin{proof}
Choose the following sets.\nl 
$A=\varnothing$\nl$B=\{1\}$\nl $C=\{1\}$ \nl 
Note that $A\subset B$, because there are no elements in $A$.
\nl $C-A = \{1\}$
\nl $C-B = \varnothing $\nl 
So, $A\subset B$ but $(C-A)\not\subset(C-B)$ because the element 1 is in 
$C-A$ but is not in $C-B$.
Therefore, there exist sets $A,B$, and $C$ so that $A\subset B$ but 
$(C-A)\not\subset(C-B)$.
\end{proof}

\item For all sets $A,B$, and $C$, $(A \cup B)-C=(A-C)\cup(B-C)$

\begin{proof}
    Suppose $A,B$, and $C$ are sets.
    \nl Suppose $x\in(A \cup B)-Ccases$.
    \nl So $x$ is in $A$ or $B$, but $x$ is not in $C$.
    \nl We will consider two cases.
    \nl Case: $x \in A$.
    \nl Since $x\in A$, and $x\not\in C$, $x\in (A-C)$
    \nl So, when $x\in A$, $x\in(A-C)\cup(B-C)$.
    \nl Case: $x\not\in A$.
    \nl Since $x\not\in A$, $x$ must be in $B$.
    \nl Therefore $x\in(B-C)$ because $x$ is in $B$ but not $C$.
    \nl So $x \in (A-C)\cup(B-C)$.
    \nl All elements of the left side of the equation must be in the right 
    side, so $(A \cup B)-C\subset(A-C)\cup(B-C)$

    Let $y\in(A-C)\cup(B-C)$.
    \nl So $y$ must be in either $A-C$ or $B-C$. This means $y$ must be in $A$ or $B$.
    \nl Either way, $y\not\in C$
    \nl Since $y$ must be in $A$ or $B$, it will always be in $A\cup B$.
    \nl And since $y\not\in C$, $y$ must always be in $(A \cup B)-C$.
    \nl Therefore, $(A-C)\cup(B-C)\subset(A \cup B)-C$
    \nl Since the sets $(A-C)\cup(B-C)$ and $(A \cup B)-C$ are subsets of each other, they are equal.


\end{proof}
\newpage
\item For all sets $A$ and $B$, $A\subset B \iff A-B=\varnothing$.

\begin{proof} 
    Suppose $A$, $B$ are sets.\nl
    To prove that $A\subset B \iff A-B=\varnothing$, we will first prove $A\subset B \implies A-B=\varnothing$
    \nl We will prove this by contradition. Suppose $A\subset B$ and $\exists x\in(A-B)$.
    \nl So, $x\in A$ but $x\not\in B$.
    \nl But all elements of $A$ must be in $B$ since $A\subset B$. A contradiction!
    \nl So, $A\subset B\implies A-B=\varnothing$.
    \nl Next we will prove that $A-B=\varnothing \implies A\subset B$.
    \nl Since there are no elements in $A-B$, $\forall y\in A, y\in B$
    \nl This is the same as saying "A is a subset of B"
    \nl So, $A-B=\varnothing \implies A\subset B$
    \nl $A\subset B\implies A-B=\varnothing$ and $A-B=\varnothing \implies A\subset B$
    \nl Which means that $A\subset B \iff A-B=\varnothing$.

\end{proof}

\end{alphenum}
\end{enumerate}
\end{document}