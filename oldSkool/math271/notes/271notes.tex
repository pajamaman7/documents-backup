\documentclass{article}


\usepackage[most,many,breakable]{tcolorbox}
\usepackage{amsmath}
\usepackage{amssymb}
\usepackage{amsthm}
\usepackage[]{thmbox}
\usepackage{blindtext}
\usepackage[utf8]{inputenc}
\usepackage{amsmath}
\usepackage{amsfonts}
\usepackage[]{graphicx}
\usepackage[legalpaper, portrait, margin = 1in]{geometry}
\usepackage{enumitem}


\usepackage{xcolor}

%\pagecolor[rgb]{0.2,0.19,0.18} 
%\color[rgb]{0.92,0.86,0.7}

\newtheorem[L]{le}{Lemma}[subsection]
\newtheorem[L]{th}[le]{Theorem}
\newtheorem[L]{df}[le]{Definition}
\newtheorem[L]{ex}[le]{Example}
\newtheorem[L]{pf}[le]{Proof}


\newcommand{\nl}{\newline}

\newcommand{\real}{\mathbb{R}}
\newcommand{\complex}{\mathbb{C}}
\newcommand{\integer}{\mathbb{Z}}
\newcommand{\rational}{\mathbb{Q}}
\newcommand{\lxor}{\oplus}
\newcommand{\then}{\Rightarrow}

\begin{document}
\section{Statements}
A \textbf{Statement or Proposition} is a sentence that is true or false but not both.
Here's some examples:
\begin{ex}
\begin{itemize}
    \item The sky is blue.
    \item 5=2
    \item All prime numbers are divisible by two.
\end{itemize}
\end{ex}
\par 

A \textbf{Conditional Statement} is one of the form ”If p, then q.” We denote this correlation with a
$\then$ symbol. If we have a statement $ p \then q \equiv r$, then $r$ is always true unless $p$ is false and $q$ is true.
In a conditional statement, we consider $p$ to be the hypothesis, and $q$ to be the conclusion.
\par 

Note that the hypothesis of a conditional statement can be false, as long as the conclusion is as well.
This means that $0=1 \to 1=2$ is true.

\par 
Statements can take other forms, such as using and ($\land$), or ($\lor$), not ($\lnot$), as well as some other forms 
seen especially when writing proofs.
\par

Two mathematical forms often seen are: "There exists / For some" ($\exists$), and "For all
 / For every" ($\forall$).
\par 
The \textbf{negation} of a Statement is the opposite of the statement. For all true values of $P$, $\lnot P $ will be false.
and vice versa.


\begin{df}
    Negations of some basic statements:
    \[\lnot (P \land Q ) \equiv \lnot P \lor \lnot Q\]
    \[\lnot(P \lor Q) \equiv \lnot P \land \lnot Q\]
    \[\lnot(\exists x, P(x)) \equiv \forall x, \lnot P(x)\]
    \[\lnot(\forall x , Q(x))\equiv \exists x, \lnot Q(x)\]
    \[\lnot(P \to Q) \equiv P \land \lnot Q\]

\end{df}

\begin{df}
    \begin{itemize}
        \item The converse of $P \to Q$ is $Q \to P$.
        \item The contrapositive of $P \to Q$ is $\lnot P \to \lnot Q$.
    \end{itemize}
    Note that the contrapositive of a statement is logically equivalent, but the converse may not be.
\end{df}

\section{Integers}

Integers posess the following properties:
\par 
Remember that theorems can be used in assignment proofs, but lemmas
must be used more carefully with explanation.
\begin{th}
    $\forall n \in \integer $, n is either even or odd but not both.
\end{th}
\begin{th}
    $\forall a,b \in \integer$, $ a|b\implies b\geq a$
\end{th}
\begin{th}
    Products and sums of integers are integers.
\end{th}

\begin{le}
    Suppose m, n are integers. If m and n are both odd, mn is odd.
\end{le}
\begin{pf}
    Assume $m, n \in \integer$.
    Suppose m and n are odd. 
    \newline
    Then, $m = 2k+1, n = 2l+1$ for some
     $l, k \in \integer$.
    \newline
    $mn = (2k+1)(2l+1)$\newline
    $mn = 2(4kl + k + l) + 1$\newline
    So, $mn$ can be expressed in the form $2q+1$, where $q = 4kl + k +l$, and q is an integer.
\end{pf}
\begin{df}
    Let $n \in \integer$.\newline
    $n$ is prime $\iff \exists r,s $ so that $rs = n$, and $r = 1, s=n$ or $r=n, s =1$. \newline
    n is composite $\iff \exists r,s $ so that $rs =n$ and nither $r$ or $s$ is 
    equal to 1.
\end{df}
\begin{df}
    Let $a,b \in \integer$. $a$ divides $b$ ($a|b$) if $a\neq 0$ and $ak = b$
    for some $k \in \integer$.
\end{df}

\section{Rationals}
\begin{df}
    Let $x \in \real$. $x$ is rational $(x \in \rational) \iff x = \frac{a}{b}$ for some
    $a,b \in \integer, b\neq 0$
\end{df}
\begin{le}
    If $x \in \integer$, $x \in \rational$.
    \begin{proof}
        Any $x \in \integer$ can be written as $x=\frac{x}{1}$. Since both 1 and $x$
        are integers, x is rational.
    \end{proof}
\end{le}
\begin{le}
    Like with integers, the sum and product of any rationals is also rational.
\end{le}

\end{document}