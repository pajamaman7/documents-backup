\documentclass{article}
\usepackage[most,many,breakable]{tcolorbox}
\usepackage{amsmath}
\usepackage{amssymb}
\usepackage{amsthm}
\usepackage[]{thmbox}
\usepackage{blindtext}
\usepackage[utf8]{inputenc}
\usepackage{amsmath}
\usepackage{amsfonts}
\usepackage[]{graphicx}
\usepackage[legalpaper, portrait, margin = 1in]{geometry}
\usepackage{enumitem}


\usepackage{xcolor}

%\pagecolor[rgb]{0.2,0.19,0.18} 
%\color[rgb]{0.92,0.86,0.7}

\newtheorem[L]{le}{Lemma}[subsection]
\newtheorem[L]{th}[le]{Theorem}
\newtheorem[L]{df}[le]{Definition}
\newtheorem[L]{ex}[le]{Example}
\newtheorem[L]{pf}[le]{Proof}


\newcommand{\nl}{\newline}

\newcommand{\real}{\mathbb{R}}
\newcommand{\complex}{\mathbb{C}}
\newcommand{\integer}{\mathbb{Z}}
\newcommand{\rational}{\mathbb{Q}}
\newcommand{\lxor}{\oplus}
\newcommand{\then}{\Rightarrow}
\pagestyle{fancy}
\lhead{Assignment \# $6$}
\rhead{Name: Thomas Boyko; UCID: 30191728}
\chead{}

\begin{document}
\begin{enumerate} 

\item Exercise 10.1.20. Let $\mathbb{F}$ be a field.
    \begin{enumerate}
        \item Show that the following are equivalent for a polynomial $f (x) \in  F[x].$
            \begin{enumerate}
                \item $f (x)$ has no repeated root in any extension field of $\mathbb{F}$.
                \item $f (x)$ has no repeated root in some splitting field over $\mathbb{F}$.
                \item $f (x)$ and $f '(x)$ are relatively prime in $\mathbb{F}[x]$
                    \paragraph{i.$\implies$ ii.} Suppose $f$ has no repeated root in any extension 
                    of $\mathbb{F}$. $f$ has a splitting field, and by assumption it must have no repeated
                    roots in this field. 
                    \paragraph{ii.$\implies$ iii.} Suppose $f$ has no repeated roots in an extension
                    $\mathbb{E}$ in which it splits. Then suppose for the sake of 
                    contradiction that there exists some $g\in \mathbb{F}[x]$ so that $g|f$ and $g|f'$,
                    and that $g$ is nonconstant. By Kronecker's Theorem, take a root $\alpha\in \mathbb{E}$
                    of $g$. Then $x-a|g$ and so $x-\alpha|f$, $x-\alpha|f'$. But if $x-\alpha|f'$, then
                    $(x-\alpha)^2|f$, a contradiction since we assumed that $f$ had no repeated root.

                    \paragraph{iii. $\implies$ i. } Suppose that $f,f'$ are relatively prime in
                    $\mathbb{F}[x]$. Then suppose for the sake of contradiction that there is some
                    extension $\mathbb{E}$ of $\mathbb{F}$ so that $f$ has a repeated root in $\mathbb{E}$.
                    Then $(x-\alpha)^2|f$. But then $(x-\alpha)$ would divide $f'$, contradicting
                    $f,f'$ being coprime. 
                    %TODO was this an iff??
                    
            \end{enumerate}   
            \item If $f (x)$ is as in (a), show that $f (x)$ is separable, but not conversely.

              \paragraph{Solution:} Let $f$ have no repeated roots in its splitting field. Then clearly
              none of its factors can have repeated roots, so it must be separable.

              \paragraph{Counterexample:}  Take $f(x)=x^2+2x+1$ which has a repeated root in the trivial extension
              $\mathbb{F}$ of $\mathbb{F}$, but its irreducible factor $(x+1)$ has no repeated root in any
              extension $\mathbb{E}$ of $\mathbb{F}$. 
    \end{enumerate}
\item Exercise 10.1.26 (a) (b)
    \begin{enumerate}
        \item Show that the following conditions are equivalent for a field $\mathbb{F}$ 
            (then called a perfect field):
        \begin{enumerate}
            \item Every algebraic extension of $\mathbb{F}$ is separable.
            \item Every finite extension of $\mathbb{F}$ is separable.
            \item Every irreducible polynomial in $\mathbb{F}[x]$ is separable.
        \end{enumerate}

          \paragraph{i. $\implies$ ii. } Suppose that every algebraic extension of $\mathbb{F}$ 
          is separable. Then, if $\mathbb{E}$ were a finite extension, it would have to be
          algebraic and as such it would be separable.

          \paragraph{ii. $\implies$ iii. } Suppose that every finite extension of $\mathbb{F}$ is separable, and
          that $f$ is irreducible in $\mathbb{F}[x]$. Let $\mathbb{E}$ be the splitting field of $f$ over
          $\mathbb{E}$. Then $\mathbb{E}$ is a finite extension, and by hypothesis 
          $\mathbb{E}$ must be separable. Since the minimal monic polynomial for  any root  $u_i$ of $f$ 
          is $x-u_i\in E[x]$, and these are all seperable since $\mathbb{E}$ is seperable. we can say
          $f$ is seperable since it is the product of all these and a constant in $\mathbb{F}$. 

          \paragraph{iii. $\implies$ i. } Suppose all irreducible polynomials in $\mathbb{F}[x]$ are 
          seperable, and $\mathbb{E}$ be an algebraic extension of $\mathbb{F}$.
          Let $u\in \mathbb{E}$, and $m(x)$ be the minimal monic polynomial
          for $u$. Then since $m$ is irreducible, it is separable by hypothesis. Therefore the algebraic
          extension $\mathbb{E}$ is seperable. 


        \item Show that every field of characteristic 0 is perfect.
          \paragraph{Solution:} Let $f$ be of characteristic $0$. Then an irreducible $p$ is separable 
          (Nicholson Chapter 10, Theorem 4), satisfying iii. Therefore $\mathbb{F}$ is perfect.
      \end{enumerate}
      \newpage
\item Exercise 10.2.12 If $\mathbb{E}$ is a finite extension of $\mathbb{F}$ and 
    $G = gal( \mathbb{E} : \mathbb{F})$, show that the extension $E$ of $F$ is Galois if and only if 
    $|G| = [\mathbb{E} : \mathbb{F}]$.

    \paragraph{$\implies$:} Since the Galois group is a group of automorphisms fixing $\mathbb{F}$, this 
    follows from Dedekind-Artin since $\mathbb{F}=\mathbb{E}_G$.

    \paragraph{$\impliedby:$} Let $\mathbb{E}$ be a finite extension of $\mathbb{F}$ and 
    $|G| = [\mathbb{E} : \mathbb{F}]$. We want to show that the fixed set of $\mathbb{E}$ under $G$ is
    precisely $\mathbb{F}$ in order to show that the extension is Galois. If the extension is degree $1$,
    the result is clear, since the fixed set of $G=\{e\}$ is $\mathbb{F}$. So suppose the degree of the
    extension is $\ge 2$.

    Then let $u\in \mathbb{E}\setminus\mathbb{F}$, and suppose for the sake of contradiction that 
    $u\in \mathbb{E}_{G}$. Then for any $\sigma\in G,$ $\sigma(u)=u$. But since $\sigma$ is uniquely 
    defined by where it sends $u$, this must mean that $\sigma=e$ and therefore $G=\{e\} $. But then
    $|G|=[\mathbb{E}:\mathbb{F}]=1<2$, a contradiction. Therefore $\mathbb{E}_G=\mathbb{F}$, and the 
    extension is Galois.

\end{enumerate}
\end{document}
