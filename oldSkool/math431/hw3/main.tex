\documentclass{article}
\usepackage{amsmath}
\usepackage{amssymb}
\usepackage{amsthm}
\usepackage[utf8]{inputenc}
\usepackage{amsmath}
\usepackage{amsfonts}
\usepackage[]{graphicx}
\usepackage[a4paper, portrait, margin = 1in]{geometry}
\usepackage{enumitem}
\usepackage{xcolor}

%darkmode
%\pagecolor[rgb]{0.2,0.19,0.18} 
%\color[rgb]{0.92,0.86,0.7}

\newenvironment*{alphenum}{\begin{enumerate}[label= (\alph*)]}{\end{enumerate}}

\pagestyle{fancy}
\lhead{Assignment \# $3$}
\rhead{Name: Thomas Boyko; UCID: 30191728}
\chead{}

\begin{document}
\begin{enumerate} 

    \item Excercises 6. 1$\#$4 ( a)  and ( b) .
        \begin{enumerate}
            \item  Show that $\{u,v,w\}=$ span$\{u+v,u+w,v+w\}$ in any $\mathbb{F}$-vector space $V$ where 2$\neq0$ in $\mathbb{F}.$

                \paragraph{$\supseteq $: }Let $x\in $span$ \{u+v,u+w,v+w\} $. Then for some $a,b,c\in \mathbb{F},$
                $x=a(u+v)+b(u+w)+c(v+w)=(a+b)u+(a+c)v+(b+c)w\in $span$\{u,v,w\} $, and 
                $\{u,v,w\}\supseteq $ span$\{u+v,u+w,v+w\}$

                \paragraph{$\subseteq $:} Let $x\in$ span$\{u,v,w\} $. Then $x=au+bv+cw$ for some $a,b,c\in \mathbb{F}$. Rewrite $u,v,w:$
                 \begin{align*}
                    u&= \frac{(u+v)+(u+w)-(v+w)}{2} \\
                    v&= \frac{(u+v)+(v+w)-(u+w)}{2} \\
                    w&= \frac{-(u+v)+(v+w)+(u+w)}{2}\\
                    x&=a\frac{(u+v)+(u+w)-(v+w)}{2} +b\frac{(u+v)+(v+w)-(u+w)}{2}+c\frac{-(u+v)+(v+w)+(u+w)}{2}\\
                    &= \frac{a+b-c}{2}(u+v)+\frac{a+c-b}{2}(u+w)+\frac{b+c-a}{2}(v+w) 
                .\end{align*}
                So we can write $x$ as a linear combination of $\{u+v,u+w,v+w\}$, and $x\in $span$ \{u+v,u+w,v+w\}$. Therefore
                $\{u,v,w\}\subseteq $ span$\{u+v,u+w,v+w\}$ and $\{u,v,w\}=$ span$\{u+v,u+w,v+w\}$

            \item  Is (a) true if $\mathbb{F}=\mathbb{Z}/2\mathbb{Z}$? Support your answer.

                In this case it is not true. Take $\mathbb{Z}_2$ as its own vector space, and let $u=v=w=1$. Then the sum of any two of these vectors
                is  $0$ and span $\{u,v,w\} =\{0,1\} \neq \{0\} =$span$ \{u+v,u+w,v+w\} $.
        \end{enumerate}
    \item Show that $u=\sqrt{2}+\sqrt{3}\in\mathbb{C}$ is algebraic over $\mathbb{Q}$ and find its
minimal polynomial.

\paragraph{Solution: }Use algebraic trickery on $u.$  
\begin{align*}
    u&=  \sqrt{3} +\sqrt{2} \\
    u^2&= 2\sqrt{6} +5 \\
    u^2-5&= 2\sqrt{6}  \\
    (u^2-5)^2&= 24 \\
    u^4-10u^2+25&= 24 \\
    u^{4}-10u^2+1&= 0 
.\end{align*}
And we have found a polynomial in $\mathbb{Q}$ so that $u$ is a root, and $u $ is algebraic over $\mathbb{Q}$.

Though this polynomial is monic and has $u$ as a root, we have yet to show that it is irreducible. Eisenstein's criterion fails us since
the constant term is  $1$, no prime can divide it.

Recall the Modular Irreducibility Theorem (Theorem 4.2.7). It states that if $p$ is a prime not dividing the leading coefficient of a polynomial
$f$ in $\mathbb{Z}$, and the reduction of $f$ in $\mathbb{Z}_p[x]$, $\bar{f}$ has no root in $\mathbb{Z}_p$, then $f$ is irreducible in $\mathbb{Q}[x]$.

Take $p=3$. Then the reduction of  $m$ is $\bar{m}=u^{4}-u+1= 0$. Check each element, 
\begin{align*}
    \bar{m}(0)\equiv 0^4-0^2+1&\equiv 1\pmod{3} \\
    \bar{m}(1)\equiv 1^{4}-1^2+1&\equiv 1\pmod{3} \\
    \bar{m}(2)\equiv 2^{4}-2^2+1&\equiv 1\pmod{3} 
.\end{align*}
And so $\bar{m}$ has no roots in $\mathbb{Z}_3$, and by the modular irreducibility theorem $m$ is irreducible in $\mathbb{Q}$.

\item Show that if $u\in\mathbb{C}$ and $u\notin\mathbb{R}$ then $\mathbb{C}=\mathbb{R}(u).$
    \paragraph{Solution: } Let $u\in \mathbb{C}\setminus\mathbb{R}$. Then $u=a+bi$ with $b\neq 0$.

    \paragraph{$\subseteq $: }Let $x\in \mathbb{C}$. Then $x=c+di$, and rewriting, we can also see that  $i=\frac{u-a}{b}$, so 
    $x=c+d\left( \frac{u-a}{b} \right) =(c-\frac{a}{b})+u\left( \frac{d}{b} \right) \in $ span$ \{1,u\} =\mathbb{R}(u)$. so $\mathbb{C}\subseteq \mathbb{R}(u)$

    \paragraph{$\supset :$ } $\mathbb{C}$ is a field containing both $\mathbb{R}$ and $u$. Then since $\mathbb{R}(u)$ must be contained in any such field,
    $\mathbb{R}(u)\subseteq \mathbb{C}$.

    Therefore, $\mathbb{C}=\mathbb{R}(u).$

\item Show that if $u\in\mathbb{E}$ is transcendental in $\mathbb{F}$, then
\[\mathbb{F}(u)=\{f(u)g(u)^{-1}\mid f(x),g(x)\in\mathbb{F}[x],g(x)\neq0\}.\]

\paragraph{Solution: } Let $u$ be transcendental in $\mathbb{F}$.
For simplicity, write \[ \mathcal{F}=\{f(u)g(u)^{-1}\mid f(x),g(x)\in\mathbb{F}[x],g(x)\neq0\} .\] 

\paragraph{$\subseteq$: } If we can show that $\mathcal{F}$ is a field containing $\mathbb{F},u$, then we can say $\mathbb{F}(u)\subseteq \mathcal{F}$ since
$\mathbb{F}(u)$ is contained in every field which contains $\mathbb{F}$ and $u$.

Then let $a\in \mathbb{F}$. The constant polynomial $a=\frac{a}{1}$ is in $\mathcal{F}$ since, when evaluated at $u$ it is $u$. So $\mathbb{F}(u)\subseteq \mathcal{F}$
Similarly, the polynomial $\frac{x}{1}$ is simply $u$ when evaluated at $u$, and $u\in \mathcal{F}$.
Finally $\mathcal{F}$ is a field, since it is a field of fractions (Field of Quotients in Nicholson 3.2.5) for the integral domain $\{f(u):f(x)\in \mathcal{F}[x]\} $.

\paragraph{$\supseteq $: }Let $a\in \mathcal{F}$. 
Write
\begin{align*}
    f(x)&=f_0+f_1 x+\ldots+f_nx^{n}\\
    g(x)&= g_0+g_1x+\ldots+g_mx^{m}
.\end{align*}
Evaluating at $u$,
\begin{align*}
    f(u)&=f_0+f_1 u+\ldots+f_nu^{n}\\
    g(u)&= g_0+g_1u+\ldots+g_mu^{m}
.\end{align*}

We know that each term in each polynomial is in $\mathbb{F}(u) $, and so the sum of the terms must be, since $\mathbb{F}(u)$ is closed under $\cdot,+$.
And since $g(x)\neq 0$, we know the quotient must also be.

So $a=\frac{f(u)}{g(u)}\in \mathbb{F}(u)$, and $\mathbb{F}(u)\supseteq \mathcal{F}.$ Therefore $\mathbb{F}(u)=\mathcal{F}$.
\end{enumerate}
\end{document}
