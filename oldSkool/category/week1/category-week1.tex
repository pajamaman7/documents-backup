\documentclass{article}
\usepackage{amsmath}
\usepackage{amssymb}
\usepackage{amsthm}
\usepackage[utf8]{inputenc}
\usepackage{amsmath}
\usepackage{amsfonts}
\usepackage[]{graphicx}
\usepackage[a4paper, portrait, margin = 1in]{geometry}
\usepackage{enumitem}
\usepackage{xcolor}

%darkmode
%\pagecolor[rgb]{0.2,0.19,0.18} 
%\color[rgb]{0.92,0.86,0.7}

\newenvironment*{alphenum}{\begin{enumerate}[label= (\alph*)]}{\end{enumerate}}

\usepackage{commutative-diagrams}
\usepackage{tikz-cd}
\pgfqkeys{/codi}{
every arrow/.append style={
/ektropi/add=/tikz/commutative diagrams
}
}
\pagestyle{fancy}
\lhead{Week 1}
\rhead{Name: Thomas Boyko; UCID: 30191728}
\chead{}

\begin{document}
\begin{enumerate} 
    \item Let $S$ be a set. The indiscrete topological space $I(S )$ is the space whose set of points is $S$ and whose only open subsets are $\varnothing$ and $S$ itself. Imitating Example 0.5, find a universal property satisfied by the space $I(S )$.
    \item Find three examples of categories not mentioned above.

        \begin{enumerate}
            \item  $\mathrm{Mat}_R$ is the category whose objects are positive integers, and where the set of morphisms from $n$ to $m$ is the set of $m\times n $ matrices with values in $\mathbb{R}$. Composition is by matrix multiplication, and identity for $n\in \mathbb{Z}_{> 0}$ is the $n\times n$ identity matrix.
            \item We can form a category out of regular languages, since strings form a monoid under concatenation.
            \item $\mathrm{Meas} $ has measurable spaces as objects and measurable functions as morphisms.
        \end{enumerate}

    \item Show that a map in a category can have at most one inverse. That is, given a map $f:A\to B$, show that there is at most one map $g:B\to A$ such that $gf=1_A$ and $fg=1_B$.
         \paragraph{Solution: }Suppose there are two such mappings, $g,h:B\to A$ so that $fg=fh=1_B$ and $gf=hf=1_A$. Then left-compose with $g$:
         \begin{align*}
             fg&=fh\\
             gfg&= gfh \\
             1_Ag&= 1_Ah \\
             g&=h
         .\end{align*}
         So an inverse for $f$ must be unique.

    \item Let $\mathscr{A},\mathscr{B}$ be categories. The construction of the product category:
         \[
         \mathrm{ob}(\mathscr{A}\times \mathscr{B})=\mathrm{ob}(\mathscr{A})\times \mathrm{ob}(\mathscr{B})
         \] \[ \mathrm{Hom}(\mathscr{A}\times \mathscr{B})= \mathrm{Hom}(\mathscr{A})\times\mathrm{Hom}( \mathscr{B})
         \] 
         has only one choice for compositions and identities. Give both.
         \paragraph{Solution: }Let $f,g,h$ be morphisms in $\mathscr{A}\times \mathscr{B}$.
         Write $f=(f_1,f_2)$, $g=(g_1,g_2)$, $h=(h_1,h_2)$. Then the sensible composition is $gf=(g_1f_1,g_2f_2)$. And associativity follows; 
         \[
         h(gf)=h(g_1f_1,g_2f_2)=(h_1(g_1f_1),h_2(g_2f_2))=((h_1g_1)f_1,(h_2g_2)f_2)=(hg)(f)
         .\] 
         Then for an object $(a,b)\in \mathscr{A}\times \mathscr{B}$, the sensible identity is $1_{(a,b)}=(1_a,1_b)$. Then for a morphism $f=(f_1,f_2)$ with domain $(a,b)$, we have 
         \[
         f1_{(a,b)}=(f_1 1_a, f_2 1_b)=(f_1,f_2)=f
         .\] 
         And likewise for some $g=(g_1,g_2)$ with codomain $(a,b)$, we have:
          \[
         1_{(a,b)}g=(1_ag_1,1_bg_2)=(g_1,g_2)=g
         .\] 
    \item Find three examples of functors not mentioned above.

    \item Show that functors preserve isomorphism. That is, prove that if $F:\mathscr{A}\to \mathscr{B}$ is a functor and $A,A'\in \mathscr{A}$ with $A\cong A'$, then $F(A)\cong F(A')$.
        \begin{proof} 
            Suppose $F:\mathscr{A}\to \mathscr{B} $ is a functor, and $A\cong A'$ in $\mathscr{A}.$ Then there exists a pair of morphisms $f:A\to A'$ and $g:A'\to A$ with $fg=1_{A'}$ and $gf=1_A$. And, the functor $F$ gives another pair of morphisms $Ff,Fg$. Verify:
            \[
            ( Ff)( Fg)= F(fg)=F1_{A'}=1_{FA'}
            \] 
            and likewise:
            \[
                (Fg)(Ff)=F(gf)=F 1_A=1_{FA}
            .\] 
            And so we have $FA\cong FA'$. 
        \end{proof}

    \item Two categories $\mathscr{A}$ and $\mathscr{B}$ are isomorphic, written as $\mathscr{A}\cong \mathscr{B}$, if they are isomorphic as objects of $\mathrm{Cat}$.
        \begin{enumerate}
            \item Let $G$ be a group, regarded as a one-object category all of whose maps are isomorphisms. Then its opposite $G^{op}$ is also a one-object category all of whose maps are isomorphisms, and can therefore be regarded as a group too. What is $G^{op},$ in purely group-theoretic terms? Prove that $G$ is isomorphic to $G^{op}$.
                \begin{proof} 
                Take the functors $F:G\to G^{op}$, and $F':G^{op}\to G$. Define, for $g\in G$ and $h^{op}\in G^{op}$:
                \[
                F(g)=(g^{-1})^{op},\quad \quad F'(h^{op})=h^{-1}
                .\] 
                We first check that these functors compose to identity:
                \begin{align*}
                    FF'(g^{op})&= F(g^{-1}) \\
                    &= ((g^{-1})^{-1})^{op} \\
                    &= g^{op} \\
                    F F'&= 1_{G^{op}} \\
                    F'F(g)&= F'((g^{-1})^{op} )\\
                    &= (g^{-1})^{-1} \\
                    &= g \\
                    F'F&=1_{G}
                .\end{align*}
                And then we check that these mappings are indeed functors. Clearly $F,F'$ map the single object in $G$ to $G^{op}$, and vice versa. Then we check the morphism identities for $F$ and $F'$. Let $g,h \in G$;
                \begin{align*}
                    F(gh)&= \left( \left( gh \right) ^{-1} \right) ^{op} \\
                    &= \left( h^{-1} g ^{-1} \right) ^{op} \\
                    &= \left( g^{-1} \right)^{op}\left( h^{-1} \right) ^{op} \\
                    &= F(g)F(h)
                .\end{align*}
                Then, if $g^{op},h^{op}\in G^{op}$;
                \begin{align*}
                    F'(g^{op}h^{op})&= F'( (hg)^{op}) \\
                    &= (hg)^{-1} \\
                    &= g^{-1}h^{-1} \\
                    &= F(g^{op})F(h^{op})
                .\end{align*}
                And all that is left to verify is that $F,F'$ send identities to identities. Let $g\in G$, and $g^{op}\in G^{op}$. We wish to show that $F(1_G)= (1_G)^{op} =1_{G^{op}}$, and that $F'(1_{G^{op}}) =1_G$. Take $g^{op}\in G^{op}$, which we know to have a preimage $g^{-1}$ under $F$.
                \begin{align*}
                    (1_G)^{op}g^{op}&= F(1_G) g^{op}\\
                    &= F(1_G)F(g^{-1}) \\
                    &= F(1_Gg^{-1}) \\
                    &= F(g^{-1}) \\
                    &= g^{op} 
                .\end{align*}
                And so $1_{G^{op}}=(1_G)^{op}=F(1_G)$ (Since identity of right composition follows from the same argument). Now for $g\in G$,
                \begin{align*}
                    F'(1_{G^{op}})&= F'((1_G)^{op}) \\
                    &= 1_G^{-1} \\
                    &=1_G
                .\end{align*}
                So $F$ and $F'$ are functors which serve as inverses for one another, and $G\cong G'$.
                \end{proof}
            \item Find a monoid which is not isomorphic to its opposite.
                \paragraph{Solution: }Take $\mathbb{N}$, %TODO
        \end{enumerate}
        
    \item Find a universal property of the indiscrete topology, defined on any set by letting the open sets be exactly $\varnothing,S$.
        \paragraph{Solution: }We observe that if $f:X\to S$ is a function on a topological space $X$, that the preimage of $\varnothing$ is $\varnothing$ and the preimage of $S$ is $X$. So the preimage of our two open sets are both open, and $f$ is continuous. So given any function from $f:X\to S$ we define a function $\bar{f}:X\to I(S)$ which is continuous:
        \[ \begin{codi}
            \obj{ S & |(I)| I(S) \\ & X \\ };
            \mor I \iota:hookrightarrow S;
            \mor X f:-> S; 
            \mor X \bar{f}:dashrightarrow I;
        \end{codi} \] 
\end{enumerate}
\end{document}
