\section{The Extended Real Number System}\label{sec:extended-real-number-system}

In this section we are going to talk about a fascinating extension of the real
number system called the extended real number system. To get this new object, we
take the existing real numbers $\R$ and add two more objects to it which we will
denote using $-\infty$ and $+\infty$. You should recognize these symbols and
immediately realize that we are going to do what a bunch of math teachers have
told you can \emph{can't} do\footnote{After this, you'll be able to decide if
    that was for better or worse.}. This new object we create is often denoted
$\overline{\R}$ (the notation we will use), but you may also see $[-\infty, +\infty]$,
and $\R \cup \{-\infty, +\infty\}$.

To start off we still want this new object to retain the ordering from $\R$.
Therefore we define
\begin{equation*}
    -\infty < x < +\infty
\end{equation*}
for all $x\in\R$ which essentially sticks these two objects on opposite sides of
the number line.

There are a few things that adding these two additional object gets us, but the
one we really care about is the fact that all sets now have upper bounds (and
lower bounds). If you take $A$ to be a set of real number which is not bounded
above in $\R$, then $\sup{A} = +\infty$ in our new number system(This
is also true with lower bounds, but with $-\infty$).

Now unfortunately $\overline{\R}$ does not form a field\footnote{Can you figure out
    what properties are lacking, or possibly undefined?}, but it is still helpful to have the following
``conventions'':
\begin{itemize}
    \item $x\pm\infty = \pm\infty$ and $\frac{x}{\pm\infty} = 0$
    \item If $x > 0$ then $x\cdot(\pm\infty) = \pm\infty$
    \item If $x < 0$ then $x\cdot(\pm\infty) = \mp\infty$ (note the switched sign)
\end{itemize}

This is an incredibly cool object, and shows one of the great things about
mathematics which is that you can pretty much do whatever you want. Have the
reals, but you need an object that's bigger than everything? Just add it and see
what kind of properties that new thing has. Maybe it gets really boring, but
maybe not. While there is some lack of rigour in this section, it's still
insightful and can be formalized to a higher level if needed.

