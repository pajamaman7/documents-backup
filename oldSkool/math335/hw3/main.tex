\documentclass{article}
\usepackage{amsmath}
\usepackage{amssymb}
\usepackage{amsthm}
\usepackage[utf8]{inputenc}
\usepackage{amsmath}
\usepackage{amsfonts}
\usepackage[]{graphicx}
\usepackage[a4paper, portrait, margin = 1in]{geometry}
\usepackage{enumitem}
\usepackage{xcolor}

%darkmode
%\pagecolor[rgb]{0.2,0.19,0.18} 
%\color[rgb]{0.92,0.86,0.7}

\newenvironment*{alphenum}{\begin{enumerate}[label= (\alph*)]}{\end{enumerate}}

\pagestyle{fancy}
\lhead{Assignment \# $3$}
\rhead{Name: Thomas Boyko; UCID: 30191728}
\chead{}
\scsnowmandefault{nose=orange,arms,broom=brown,hat=red,buttons,muffler,scale=2}
\renewcommand\qedsymbol{\scsnowman}

\begin{document}
\begin{enumerate} 
\item \begin{enumerate}
        \item Prove that the series $$\sum_{n=2}^\infty\frac1{(\log_2n)^{p(\log_2n)}}$$
        is convergent for all $p>1.$ Here $\log_2x$ denotes the logarithm base 2 of $x.$ You may assume
        that $\log_2n$ is increasing in $n.$

        \begin{proof} 
            We attempt to satisfy the criterion in Rudin Theorem 3.27. Rewrite the series; 
            and let $c_n:\mathbb{N}\to \mathbb{R}$:
            %TODO 3.27 requires decreasing so defining c1 this wy is not right, need another way to get
            %there
            \[
            c_n=\begin{cases}
                1&n=1\\
                \frac1{(\log_2n)^{p(\log_2n)}}&n\ge 2
            \end{cases}
            .\] 
            Then our summation becomes
            \[
                1+\sum_{n=2}^\infty\frac1{(\log_2n)^{p(\log_2n)}}=
                 \sum_{n=1}^{\infty} c_n
            .\] 

            Next we want to show that the general term is decreasing so we can use Theorem 3.27. Let $1<x<y$
            be natural numbers, and thanks to $\log_2$ being increasing in $\mathbb{N}$,
            \begin{align*}
                x&\leq y\\
                \log_2 x&\leq\log_2 y\\
                (\log_2 x)^{p}&\leq(\log_2 y)^{p}\\
                (\log_2 x)^{p\log_2 x}&\leq(\log_2 y)^{p\log_2 y}\\
                \frac{1}{(\log_2 x)^{p\log_2 x}}&\geq\frac{1}{(\log_2 y)^{p\log_2 y}}
            .\end{align*}
            Note above we can only take the inverse when $\log_2$ is positive, so the series is decreasing 
            for $x\geq 2$, but we have defined $c_1=c_2$ so that our series decreases regardless.
            Now that our sum is indexed from $1$ and we have shown that the general term is decreasing,
            we can apply Rudin Theorem 3.27. Our series of $c_n$ converges if and only if the following 
            series converges.
            \begin{align*}
                \sum_{k=0}^{\infty} 2^kc_{2^k}
                &= 1\cdot 2^0+  \sum_{k=1}^{\infty} \frac{2^k}{\left(\log_2 2^k\right)^{p\log_2 2^k}} 
            .\end{align*}   
            By Rudin Theorem 3.3 (b), this is convergent $\iff \sum\limits_{k=1}^{\infty} \frac{2^k}{\left(\log_2 2^k\right)^{p\log_2 2^k}} $ is convergent.

            Now consider:
            \begin{align*}
            \limsup\limits_{k \to \infty}\sqrt[k]{\left| \frac{2}{k^p} \right| ^k}  
            &=\limsup\limits_{k \to \infty}\left| \frac{2}{k^{p}} \right| \\
            &=\limsup\limits_{k \to \infty}\frac{2}{k^{p}}\\
            &=\lim\limits_{k \to \infty}\frac{2}{k^{p}}&\text{By Rudin Theorem 3.18}\\
            &=2\lim\limits_{k \to \infty}\frac{1}{k^{p}}\\
            &=2(0)&\text{By Rudin Theorem 3.20}\\
            &= 0 \\
            &<1
            .\end{align*}
            So by Rudin Theorem 3.33 this series is convergent, and so our original series must be.

        \end{proof}

        \newpage
        \item For $a>0$ find the sum of the series
        $$\sum_{k=2}^\infty\left(\frac a{a+1}\right)^k$$ (show your work)

        \paragraph{Solution: }Since $a>0$,we can say $0<a<a+1$ and $0<\frac{a}{a+1}<1$, satisfying one 
        condition of Rudin Theorem 3.26. 
        Then we must reindex the summation in order to use the theorem:
        \begin{align*}
            \sum_{k=2}^\infty\left(\frac a{a+1}\right)^k&= 
            \sum_{k=1}^\infty\left(\frac a{a+1}\right)^k-\left( \frac{a}{a+1} \right) -1\\
            &= \frac{1}{1-\frac{a}{a+1}}-\frac{a}{a+1} -1\\
            &= \frac{1}{\frac{a+1}{a+1}-\frac{a}{a+1}}-\frac{a}{a+1} -1\\
            &= \frac{1}{\frac{1}{a+1}}-\frac{a}{a+1} -1\\
            &= a+1-\frac{a}{a+1} -1\\
            &= \frac{(a+1)^2}{a+1}-\frac{a}{a+1} -1\\
            &= \frac{a^2+2a+1}{a+1}-\frac{a}{a+1} -\frac{a+1}{a+1}\\
            &= \frac{a^2}{a+1}
        .\end{align*}

    \end{enumerate}
        \newpage
\item \begin{enumerate}
        \item Prove that $f\left(x\right)=\sin\left(x^{2}\right)$ is not uniformly continuous in $[0,\infty).$
            \begin{proof} 

                Choose $\varepsilon=1$, and let $\delta>0$. Then let $k\in \mathbb{Z}, $ and $k>\frac{1}{\delta^2}$ 
                which we can do by the Archimedian Property.

                We attempt to choose $x,y$ so that the function's value on one is $0$, and on the other
                is $\pm 1$. Then let $x^2=k\pi$ for some $k\in \mathbb{N}$, and $y^2= k\pi+\frac{\pi}{2}$, 
                and our final choice is 
                \[
                y=\sqrt{k\pi+\frac{\pi}{2}},\quad x=\sqrt{k\pi}
                .\] 
                Then regardless of our choice of $k$, 
                \[
                |f(x)-f(y)|=
                \left|\sin\left(\left(\sqrt{ k\pi}\right)^2\right)
                -\sin\left(\left(\sqrt{ k\pi+\frac{\pi}{2}}\right)^2\right)\right|
                =\left|\sin(k\pi)-\sin\left(k\pi+\frac{\pi}{2}\right)\right|.
                \] 
                If $n$ is odd, then 
                $|\sin(k\pi)-\sin\left(k\pi+\frac{\pi}{2}\right)|=|\pm 1 -0|=1$, and if $k$ is even,

                $|\sin(k\pi)-\sin\left(k\pi+\frac{\pi}{2}\right)|=|0 -\pm 1|=1$.

                We now have guaranteed that $|f(x)-f(y)|=1$ for any $k$. 
                It aids us to note that thanks to our choice of $k$, we can say that 
                $\frac{1}{k}<\delta^2$ and $\frac{1}{\sqrt{k} }<\delta$. So then we proceed
                on $|y-x|$.

                \begin{align*}
                    |y-x|&=y-x&\text{Since }y>x\\
                    &= \sqrt{\pi k+\frac{\pi}{2}} -\sqrt{\pi k}  \\
                    &= \frac{\left( \sqrt{\pi k+\frac{\pi}{2}} -\sqrt{\pi k} \right) 
                    \left( \sqrt{\pi k+\frac{\pi}{2}} +\sqrt{\pi k} \right)}
                    {\left( \sqrt{\pi k+\frac{\pi}{2}} +\sqrt{\pi k} \right)} \\
                    &=\frac{\pi k +\frac{\pi}{2}-\pi k}
                    {\left( \sqrt{\pi k+\frac{\pi}{2}} +\sqrt{\pi k} \right)} \\
                    &=\frac{\frac{\pi}{2}} { \sqrt{\pi k+\frac{\pi}{2}} +\sqrt{\pi k} } \\
                    &=\frac{\sqrt{\pi} } {2\left( \sqrt{ k+\frac{1}{2}} +\sqrt{ k}\right) } \\
                    &< \frac{\sqrt{ \pi}} {2\left(\sqrt{k} +\sqrt{  k}\right) }&
                    \sqrt{x} \text{ is monotinically increasing}  \\
                    &<\frac{\sqrt{\pi} }{4\sqrt{k} } \\
                    &< \frac{1}{\sqrt{k}}&\text{Since }\frac{\sqrt{\pi} }{4}<1\\
                    &<\delta
                .\end{align*}
            \end{proof}
            Therefore $\sin x^2$ is not uniformly continuous on $[0,\infty)$

        \item Show an example of a continuous function in $(0,1)$ which is not uniformly
            continuous (no proof necessary).

            \paragraph{Solution:} $f(x)=\sin\left( \frac{1}{x^2} \right) $ is continuous in $(0,1)$
            since it is the composition of continuous functions. However
            it is not uniformly continuous (as shown in class).
    \end{enumerate}

\end{enumerate}
\end{document}
