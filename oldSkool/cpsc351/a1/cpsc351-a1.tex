\documentclass{article}
\usepackage{textcomp}
\newcommand{\cent}{\ensuremath{\,\not\!\!c\,\,}}
\usepackage[most,many,breakable]{tcolorbox}
\usepackage{amsmath}
\usepackage{amssymb}
\usepackage{amsthm}
\usepackage[]{thmbox}
\usepackage{blindtext}
\usepackage[utf8]{inputenc}
\usepackage{amsmath}
\usepackage{amsfonts}
\usepackage[]{graphicx}
\usepackage[legalpaper, portrait, margin = 1in]{geometry}
\usepackage{enumitem}


\usepackage{xcolor}

%\pagecolor[rgb]{0.2,0.19,0.18} 
%\color[rgb]{0.92,0.86,0.7}

\newtheorem[L]{le}{Lemma}[subsection]
\newtheorem[L]{th}[le]{Theorem}
\newtheorem[L]{df}[le]{Definition}
\newtheorem[L]{ex}[le]{Example}
\newtheorem[L]{pf}[le]{Proof}


\newcommand{\nl}{\newline}

\newcommand{\real}{\mathbb{R}}
\newcommand{\complex}{\mathbb{C}}
\newcommand{\integer}{\mathbb{Z}}
\newcommand{\rational}{\mathbb{Q}}
\newcommand{\lxor}{\oplus}
\newcommand{\then}{\Rightarrow}
\pagestyle{fancy}
\lhead{Assignment \# $1$}
\rhead{Name: Thomas Boyko; UCID: 30191728}
\chead{}

\begin{document}
\begin{enumerate} 
    \item  An office building has 6 floors, numbered 0 through 5. 3 people enter the elevator on the ground floor. Each person independently selects a destination floor, choosing uniformly at random from floors 1 through 5.
        \begin{enumerate}
            \item Describe the sample space $\Omega$ for this situation. What is the total number of possible outcomes?
                \paragraph{Solution: }The sample space is $\Omega=\{1,2,3,4,5\}\times \{1,2,3,4,5\}\times \{1,2,3,4,5\}$, since three people choose a floor from 1-5 separately. We have $|\Omega|=5^3=125$ as the total number of outcomes.
            \item Define the following events and compute their probabilities: 
                \begin{enumerate}
                    \item All three people get off at different floors ($A$).
                        \paragraph{Solution: }Describe the set $A=\{(a,b,c)\in \Omega:a\neq b,a\neq c,b\neq c\} $, then count the number of outcomes that satisfy $A$ by considering an arbitrary $(a,b,c)\in a$. $a$ can be chosen arbitrarily, giving us 5 choices. Then $b$ must be chosen differently from $a$, giving us 4 choices. Finally $c$ is chosen differently from $a,b$, giving us $3$ choices. Then combining all possibilities, $|a|=3\cdot 4\cdot 5=60$, and our probability becomes:
                        \[
                        P(A)=\frac{|A|}{|\Omega|}=\frac{60}{125}=\frac{12}{25}
                        .\] 
                    \item At least two people get off at the same floor ($B$).
                        \paragraph{Solution: }Describe the set $B=\{(a,b,c)\in \Omega:a=b \text{ or } b=c \text{ or }a=c\} $

                        Count the number of outcomes satisfying $B$. The first person can get off at any floor, so there are 5 possibilities for them. Then we split into two cases, $C_1$ and $C_2$. If the second person gets off at the same floor as the first $(C_1)$, they have only one choice, and the third person can choose any floor since there are already two people on the same floor. So in this choice we have $1\cdot 5\cdot 5=25$ choices.

                        Then in the other case ($C_2$), we suppose the second person gets off at a different floor from the first. So they have four choices. Then the third person must be on the same floor as one of the first two, and they have two choices, whether to join the first or the second person. So in this case we have $5\cdot 4\cdot 2=40$ choices.

                        Thanks to our disjoint events (in the first case, every event has person 1 and 2 on the same floor, and in case two they must always be on different floors) we can have:
                         \[
                        P(B)=P(C_1\cup C_2)=P(C_1)+P(C_2)=\frac{25}{125}+\frac{40}{125}=\frac{65}{125}=\frac{13}{60}
                        .\] 
                    \item They get off at 3 consecutive floors ($C$).
                        \paragraph{Solution: }First we note that the lowest of the three floors must be at most 3. For if the lowest floor were greater than 3, we would have the highest floor greater than 5, more than the number of floors in the building. 

                        So denote the lowest floor as $l$. There are three choices for $l$, and three choices for which person will go to floor $l$. Then for floor $l+1$, there are $2$ people who could go to that floor, and one person is left to go to floor $l+2$.

                        So the set $C$ has $|C|=3\cdot 3\cdot 2=18$, and we get a probability $P(C)=\frac{18}{125}$
                \end{enumerate}
            \item Find the probability of event B in terms of a complement.
                \paragraph{Solution: }Begin by finding the set: 

                $B^{c}=\{\text{No two people get off at different floors}\} $. 
                Observe that this is simply the probability we found in part (a), so $P(B)=1-P(A)=1-\frac{12}{25}=\frac{13}{25}$
            \item Let E be the event that at least one person gets off at floor 2, and F the event that at least one person gets off at floor 3. Use the inequality of union to find an upper bound for the probability that at least one person gets off at floor 2 or floor 3.
                \paragraph{Solution: }First describe $E$. Rather than describing the events where somebody is on floor three, it may be simpler to describe those where nobody is on floor three. So, for an arbitrary event in $E^{c}$, each person has four choices of which floor to choose, and $|E^{c}|=4^3=64$, and $|E|=61$. The same reasoning clearly works for $F$, all that changes is the floor which our people cannot land on. Then we use the union inequality:
                    \[
                        P(E\cup F)\leq P(E)+P(F)=\frac{61}{125}+\frac{61}{125}=\frac{122}{125}
                    .\] 
                    Therefore the probability of at least one person landing on either floor 2 or 3 is bounded above by $\frac{122}{125}$.
        \end{enumerate}
        \newpage
\item Alice and Bob are taking turns doing somersaults. Every time they try, they can either succeed (complete the move correctly) or fail. They take two turns each - Alice tries, then Bob tries, then Alice tries again, then Bob tries again.
\begin{enumerate}
    \item Describe how to model this situation using a simple sample space. Suppose that Alice and Bob are equally skilled, and they don't get very tired or hurt during this exercise. They also don't get very encouraged or discouraged about past attempts - so that, every time each one of them attempts a somersault, they are just as likely to succeed as they are to fail.
        \paragraph{Solution: }Model a single trial as a bit string, with 0 representing a fail and 1 representing a success. Then order the trials: Alice, Bob, Alice, Bob. Then we have $\Omega=\{\{ 0,1\}^{4}\} $.
    \item Give a probability distribution for your sample space, assuming each somersault succeeds with probability 0.5. When they start, they each have two pieces of candy. Every time that Alice succeeds, Bob gives Alice a piece of candy. Every time that Bob succeeds, Alice gives Bob a piece of candy. No candy is exchanged when someone fails. There are no ways to get more candy, and no one eats any candy during this process.
        \paragraph{Solution: }

    Write out our list of bit strings, and the number of candies we will expect Bob to have at the end of the trials. 
        \begin{align*}
            &0000,  &b=2\\
            &0001,  &b=3\\
            &0010,  &b=1\\
            &0011,  &b=2\\
            &0100,  &b=3\\
            &0101,  &b=4\\
            &0110,  &b=2\\
            &0111,  &b=3\\
            &1000,  &b=1\\
            &1001,  &b=2\\
            &1010,  &b=0\\
            &1011,  &b=1\\
            &1100,  &b=2\\
            &1101,  &b=3\\
            &1110,  &b=1\\
            &1111,  &b=2
        \end{align*}
    From this we need simply count the number of times each outcome appears, giving us the probability distribution:
    \begin{align*}
        P(b=0)&=\frac{1}{16}\\
        P(b=1)&=\frac{4}{16}\\
        P(b=2)&=\frac{6}{16}\\
        P(b=3)&=\frac{4}{16}\\
        P(b=4)&=\frac{1}{16}
    .\end{align*}
    \item What is the probability of Alice ending up with more candies than Bob?
        \paragraph{Solution: }We computed all our probabilities above, and the two cases where Alice has more candies are the cases where $b=0$ or $b=1$. Since these are disjoint (Bob cannot have 0 and 1 candies), we can say:
          \[
            P(b=0 \text{ or }b=1)=P(b=1)+P(b=0)=\frac{1}{16}+\frac{3}{16}=\frac{1}{4}
          .\] 
    \item If Alice completes her first round successfully, what is the probability of her ending up with more candies?

        \paragraph{Solution: } For this example, we refer back to the table created in (a), and use our formula for conditional probability. Recall that any bit string representing Alice winning first will begin with 1 (Denote this event $A$), and that for Alice to have more candies, Bob must be finishing a trial with 0 or 1 candies.
        \[
        P(b\in \{0,1\} |A)=\frac{P(A\cap b\in \{0,1\} )}{P(A)}=\frac{5/16}{1 /2}=\frac{5}{8}
        .\] 
        So the probability that Alice ends with more candies, given she wins her first round, is $\frac{5}{8}$.
    \item A fortune teller says that Alice will end up with more candies, what is the probability that Alice finishes her first round successfully?
        \paragraph{Solution: }Suppose the fortune teller can actually predict the future. We use the same notation as $(d)$.
         \[
        P(A|b\in \{0,1\} )=\frac{P(A\cap b\in \{0,1\} )}{P(b\in \{0,1\} )}=\frac{3/16}{5 /16}=\frac{3}{5}
        .\] 
        If the fortune teller cannot predict the future, Alice's chances are 1 in 2 as always.

\end{enumerate}
\item  A man plays roulette and believes that the numbers 7, 13, and 21 are his ”lucky numbers.” He always bets on all three of these numbers in every round. The roulette wheel has 38 equally likely outcomes, numbered 1 through 36, plus 0 and 00.
\begin{enumerate}
    \item What is the probability that he wins on any given spin (i.e., that the ball lands on one of his lucky numbers)?
        \paragraph{Solution: } Take the probability:
        \begin{align*}
            P(\{7,13,21\})&= P(\{7\} \cup \{13\} \cup \{21\} ) \\
            &= P(7)+P(13)+P(21)&\text{Since all the events are disjoint} \\
            &= \frac{1}{38}+ \frac{1}{38}+\frac{1}{38}\\
            &=\frac{3}{38}
        .\end{align*}

    \item Suppose he can afford to play the game 10 times. What is the probability that he never wins?
    \paragraph{Solution: }We seek the probability of a $\frac{3}{38}$ probability event fails 10 times consecutively. Since the spins are independent, we simply multiply the probability of a fail $\frac{35}{38}$ by itself 10 times, giving a probability of 10 losses in a row; $\left( \frac{35}{38} \right) ^{10}\approx 0.43938$.


        \iffalse
        Model with a Binomial distribution, with $p=\frac{3}{38}$ and $n=10$. Then our probability is given by: 
        \begin{align*}
            P(0)&=\binom{10}{0}\left( \frac{3}{38} \right)^0\left( 1-\frac{3}{38} \right)^{10-0}\\
            &= 1\cdot 1\cdot \left( \frac{35}{38} \right) ^{10} \\
            &\approx 0.43938 
        .\end{align*}
        So there is approximately a $43.9\%$ chance the man will lose ten times consecutively.
    \fi

    \item Let the random variable $X$ represent the number of spins he plays until he wins for the first time. What is the expected value $E[X]$?
        \paragraph{Solution: } 
        We use our geometric distribution, with a probability of success $q=\frac{3}{38}$, and take $E[X]$, which we know for a geometric distribution to be $\frac{1}{q}$, in this case $\frac{38}{3},$ meaning on average the man will have to play $12$ and $\frac{2}{3}$ spins to win.
    \item Let $X_1,\dots X_{10}$ be independent random variables, each representing whether he wins (1) or loses (0) on spin $i$. Let $Y=\sum_{n=1}^{10} X_i$ be the total number of wins in 10 spins.
        \paragraph{Solution: }Let $p=\frac{3}{38}$ 
        be the probability the man wins a given trial. Take:
        \begin{align*}
            E[Y]&=E\left[ \sum_{i=1}^{10} X_i \right] \\
                &= \sum_{i=1}^{10} E[X_i]&\text{By linearity of expectation} \\
                &= \sum_{i=1}^{10} p \\
                &= 10p \\
                &= \frac{30}{38}\\
                &= \frac{15}{19} 
        .\end{align*}
\end{enumerate}
\item A sack contains a large number of 5\cent, 10\cent and 25\cent coins. In each experiment, you reach in and randomly pull out a fistful of coins, where each fistful contains a mix of these coins. The exact number of coins and their types vary from sample to sample.
    \begin{enumerate}
    \item Describe a suitable sample space $\Omega$ for this experiment. Be precise about what each element of $\Omega$ represents.
        \paragraph{Solution: }Denote the coins, $N$ for the number of nickels, $D$ for dimes, and $Q $ for quarters. Then our sample space is:
        \[
        \Omega=\{(N,D,Q):N,D,Q\in \mathbb{Z}_{\ge 0}\} 
        .\] 
    \item Define a random variable $X$ that represents the total monetary value (in cents) of the coins in a sample.
        \paragraph{Solution: }Define $X:\Omega\to V\subseteq \mathbb{R},$ where $X(N,D,Q)=5N+10D+25Q$ cents.
    \item Suppose a new discovery reveals that mixing the metal from any 5\cent and 10\cent coin together produces a valuable alloy. Define a random variable Y that represents the maximum number of 5\cent/10\cent pairs that can be formed from a given sample.
        \paragraph{Solution: }Define $Y:\Omega\to V\subseteq \mathbb{R}, Y(N,D,Q)=\min \{N,D\} $.
    \item Describe the events corresponding to:
        \begin{enumerate}
            \item $X=30$
            \item $X \leq 30$
            \item $Y = 0$
            \item $Y = 1$
        \end{enumerate}
        \paragraph{Solution: }
        \begin{enumerate}
            \item Describe the set with $X=30:$
                \[
                \{(N,D,Q)\in \Omega:5N+10D+25Q=30\}
                .\] 
                There are few enough elements that we are able to explicitly describe the set. If $X=30$, we must have $30=5N+10D+25Q$. Clearly  $Q\leq 1$, otherwise we would have more than fifty cents. If $Q=1$, then $N=1$ is the only completion to $30$.

                Then consider $Q=0$. We could have $(0,3,0),(2,2,0),(4,1,0),$ or $(6,0,0)$.
                So the events corresponding to  $X= 30$ is:
                \[
                    \{(1,0,1),(0,3,0),(2,2,0),(4,1,0),(6,0,0)\} 
                .\] 
            \item We can find the events corresponding to $X\leq 30$ by simply describing the set:
                 \[
                \{(N,D,Q)\in \Omega: 5N+10D+25Q\leq30\} 
                .\] 
            \item If $0=Y=\min \{N,D\} $, then either we have no dimes, or we have no nickels:
                \[
                    \{(0,D,Q)\in \Omega:D,Q\in \mathbb{Z}_{\ge 0}\} \cup \{N,0,Q:N,Q\in \mathbb{Z}_{\ge 0}\} 
                .\] 
                Or equivalently,
                \[
                \{(N,D,Q)\in \Omega:\min \{N,D\} =0\} 
                .\] 
            \item If $1=Y=\min \{N,D\} $, then we have either $N$ or $D$ equal to $1$, but neither is zero: % 
                \[
                    \{(1,D,Q)\in \Omega:D,Q\in \mathbb{Z}_{\ge 0}\text{ and }D>0\} \cup \{N,1,Q:N,Q\in \mathbb{Z}_{\ge 0}\text{ and }N>0\} 
                .\] 
                Or equivalently,
                \[
                \{(N,D,Q)\in \Omega:\min \{N,D\} =1\} 
                .\] 
        \end{enumerate}
    \end{enumerate}
\end{enumerate}
\end{document}
