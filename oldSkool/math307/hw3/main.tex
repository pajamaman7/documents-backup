\documentclass{article}
\usepackage[most,many,breakable]{tcolorbox}
\usepackage{amsmath}
\usepackage{amssymb}
\usepackage{amsthm}
\usepackage[]{thmbox}
\usepackage{blindtext}
\usepackage[utf8]{inputenc}
\usepackage{amsmath}
\usepackage{amsfonts}
\usepackage[]{graphicx}
\usepackage[legalpaper, portrait, margin = 1in]{geometry}
\usepackage{enumitem}


\usepackage{xcolor}

%\pagecolor[rgb]{0.2,0.19,0.18} 
%\color[rgb]{0.92,0.86,0.7}

\newtheorem[L]{le}{Lemma}[subsection]
\newtheorem[L]{th}[le]{Theorem}
\newtheorem[L]{df}[le]{Definition}
\newtheorem[L]{ex}[le]{Example}
\newtheorem[L]{pf}[le]{Proof}


\newcommand{\nl}{\newline}

\newcommand{\real}{\mathbb{R}}
\newcommand{\complex}{\mathbb{C}}
\newcommand{\integer}{\mathbb{Z}}
\newcommand{\rational}{\mathbb{Q}}
\newcommand{\lxor}{\oplus}
\newcommand{\then}{\Rightarrow}

\begin{document}
    \huge Homework 3 - Thomas Boyko - 30191728
    \normalsize
\begin{enumerate} 

    \item Consider the function of complex variable
        \[
        u(z)= \frac{\mathrm{Re}\,z}{|z|^2}
        .\] 

\begin{enumerate}[label= (\alph*)] 
    \item  Is $u (z)$ holomorphic in its domain?

        Begin by writing $z=x+iy$, so that $u(z)=\frac{x}{x^2+y^2}$.

        We check the Cauchy-Reimann equations:

        $$\frac{\partial u}{\partial x} =\frac{y^2-x^2}{(x^2+y^2)^2}$$
        $$\frac{\partial u}{\partial y} =-\frac{2xy}{(x^2+y^2)^2}$$
        And so we can see that these are not necessarily equal for the domain of $u$, so 
        $u$ is not differentiable in its domain and therefore not holomorphic in its domain.

    \item We know that $u (z)$ is the real part of a holomorphic function
        $f (z)$. Determine $f (z)$ as a function of the complex variable $z$.

        \begin{align*}
        \frac{\partial u}{\partial x} &= \frac{\partial v}{\partial y}  \\
                                      &=\frac{y^2-x^2}{(x^2+y^2)^2}\\
                        v&=\int  \frac{y^2-x^2}{(x^2+y^2)^2} \, dy\\
                        &= -\frac{y}{x^2+y^2}+g(x)
        .\end{align*}
        And we take the partial derivative w.r.t $x$, hoping to find $g(x)$
        \begin{align*}
                \frac{\partial v}{\partial x} &= -\frac{0-2xy}{x^2+y^2}+g'(x) \\
                &= \frac{2xy}{x^2+y^2} +g'(x)
        .\end{align*}
        But we already found 
        $\frac{\partial v}{\partial x}=-\frac{\partial u}{\partial y} =\frac{2xy}{x^2+y^2}$.
        So we equate
        \begin{align*}
            \frac{2xy}{x^2+y^2}&= \frac{2xy}{x^2+y^2} +g'(x)\\
            g'(x)&= 0 \\
            g(x)&= c \\
        .\end{align*}
        So $g$ is some real constant. This gives $v(z)=-\frac{y}{x^2+y^2}+c$
        And so:
        $$f(z)=\frac{\bar{z}}{|z|^2}+ic$$

    \item Determine the singularities of $f (z)$. Which are isolated? Which
        are non-isolated? Classify the isolated singularities of $f (z)$ if
        they exist.

        We can see that this function only has trouble when the denominator is zero, which happens
        only when $|z|=0$ and therefore when $z=0$. Since this is the only singularity,
        it must be an isolated singularity.

        So we check the modulus of $f$ as $z\to 0$. 

        \begin{align*}
            \lim_{z \to 0} |f(z)|&= \lim_{z \to 0} \left| \frac{\bar{z}}{z^2} \right|  \\
            &= \lim_{z \to 0} \frac{1}{|z|} \\
            &= \infty_{\mathbb{C}} 
        .\end{align*}
        And so this singularity is a pole.

\end{enumerate}
\item Classify the singularities of the following functions.
    \begin{enumerate}[label= (\alph*)] 
        \item \[ f(z)=\frac{1}{\sin z} .\]
            
            We know that this function is not defined when $\sin z=0$, so let us find all such points.

            Suppose $\sin z =0$. Then:
            \begin{align*}
                0&=\frac{e^{iz}-e^{-iz}}{2i}\\
                &= e^{iz}-e^{-iz} \\
                &= e^{i 2z}-1 \\
                1&= e^{2iz} \\
                 &= e^{2ix}e^{-2y}&\text{Writing }z=x+iy\\
            .\end{align*}
            And checking the modulus, we see that $|1|=|e^{2y}e^{2ix}|=|e^{2y}|$ So 
            $y=\mathrm{Re}\,z=0$.

            Now we have:
            \begin{align*}
                1&= e^{2ix} \\
                &= \cos 2x+ i\sin 2x \\
                &= \cos 2x 
            .\end{align*}

            And this is true for any $x=\pi k$ where $k\in \mathbb{Z}$.

            These singularities are isolated points along the real axis. We check the limit
            \[
            \lim_{z \to \pi n} \frac{1}{\sin z}
            .\] 
            With the change of variable $w(z)=\sin z$, $w(\pi n)=0$. So our limit becomes
            \[
            \lim_{w \to 0} \frac{1}{w}=\infty_{\mathbb{C}}
            .\] 
            And so this singularity is a pole.

        \item \[ f(z)=e^{\frac{1}{\sin z}} .\] 

            This function has the same singularities as in part a), since it has the same points 
            excluded from its domain. However the type of singularity may change. We examine the same
            limit,
            \[
            \lim_{z \to \pi n} e^{\frac{1}{\sin z}}
            .\] 
            Again, using the same change of variable $w(z)=\sin z$.

            \[
                \lim_{z \to \pi n} e^{\frac{1}{\sin z}}=\lim_{w \to 0} e^{\frac{1}{w}}
            .\] 
            We restrict along the real axis and examine the positive and negative limits. 
            \begin{align*}
                \lim_{t \to 0^{-}} e^{\frac{1}{t}}&= 0 \\
                \lim_{t \to 0^{+}} e^{\frac{1}{t}}&= \infty_{\mathbb{C}} 
            .\end{align*}
            And since we have found two limits upon which our function disagrees, the limit does
            not exist and we have an essential singulrity.


        \item \[ f(z)=\ln(\sin z) .\] 

            We know that $\ln z$ has problems with continuity when $z\leq 0$. So we find the points 
            where $\sin z$ is real and negative, since in part a) we already found its zeroes.

            First we set $\mathrm{Im}\,\sin z=0$.
            \begin{align*}
                \sin z &= \sin x\cosh y+i\cos x \sinh y\\
                \mathrm{Im}\,\sin z&= \cos x\sinh y 
            .\end{align*}
            This gives us two cases. Either $\sinh y=0$ or $\cos x = 0$

            Suppose $\sinh y=0$. In this case we must have $y=0$.
            This means our original function becomes
            \[
            \sin z= \sin(x+i0)=\sin x
            .\] 
            And in this case our function is real and negative in the set:
            $$S=\{\pi<x+2k\pi<2\pi\}.$$
            All the points contained in this are cluster points, so they are non-isolated singularities.

            I ran out of time to write an argument for the case $\cos x=0$ :(
    \end{enumerate}
\item Consider the power series
    \[
    \sum_{n=1}^{\infty} \frac{e^{\alpha n}}{n} z^n
    .\] 
where $\alpha \in  \mathbb{C}$.
\begin{enumerate}[label= (\alph*)] 
    \item Discuss its radius of convergence.

        We use the ratio test for absolute convergence:
        \begin{align*}
            \lim_{n \to \infty} \left| \frac{e^{\alpha(n+1)}z^{n+1}}{n+1}\frac{n}{z^{n}e^{\alpha n}} \right| &= \lim_{n \to \infty} \frac{n}{n+1}\left| e^{\alpha}z \right|  \\
            &= e^{\mathrm{Re}\,\alpha}|z|\lim_{n \to \infty} \frac{1}{\frac{1}{n}+1} \\
            &= e^{\mathrm{Re}\,\alpha}|z|
        .\end{align*}
        Since we want this to converge we must set this $<1$. With $e^{\mathrm{Re}\,\alpha}|z|<1$ we 
        see that $|z|<e^{-\mathrm{Re}\,\alpha}$, so our radius of convergence is $R=e^{-\mathrm{Re}\,\alpha}$.

    \item Identify the function $f (z) = \sum_{n=1}^{\infty} \frac{e^{\alpha n}}{n} z^n$
    for $\alpha\in \mathbb{C}$, inside the open disk of convergence.

    We begin by reindexing our sum, $k=n-1$:
    \[
    f(z)=\sum_{n=1}^{\infty} \frac{e^{\alpha n}}{n}z^{n}=\sum_{k=0}^{\infty} \frac{e^{\alpha(k+1)}}{k+1}z^{k+1}
    .\] 
    Now we take the formal derivative since the sum starts at 0.
    \[
    f'(z)=e^{\alpha}\sum_{k=1}^{\infty} e^{\alpha k}z^{k}=e^{\alpha}\sum_{k=1}^{\infty} (e^{\alpha}z)^{k}
    .\] 
    And we can see that this series is geometric, convergent when $|z|<e^{-\mathrm{Re}\,\alpha}$.
    When it does converge:
    \[
    f'(z)=\frac{e^{\alpha }}{1-e^{\alpha}z}
    .\] 
    Now consider the function:
    \[
    g(z)=-\ln(1-e^{\alpha}z).\]

    So
    \[
    g'(z)=\frac{e^{\alpha }}{1-e^{\alpha}z}
    .\] 
    We want to show $f=g$, so take $h(z)=g(z)-f(z)$. Then $h'(z)=f'(z)-g'(z)$ which is 0, giving us that $h'(z)=0$ and $h(z)=c$ for some constant $c\in \mathbb{C}$. Now we test at $0$:
    \begin{align*}
        c&= g(0)-f(0) \\
         &= -\ln(1-e^{\alpha}0)-\sum_{n=1}^{\infty} e^{\alpha n}\frac{0^{n}}{n}\\
         &= 0-0
    .\end{align*}

    So $c=0$ and $f=g$.
    Therefore:
    \[
         -\ln(1-e^{\alpha}z)=\sum_{n=1}^{\infty} e^{\alpha n}\frac{z^{n}}{n}
    .\] 
\end{enumerate}
\end{enumerate}
\end{document}
