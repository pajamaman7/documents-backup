\documentclass{article}
\usepackage[most,many,breakable]{tcolorbox}
\usepackage{amsmath}
\usepackage{amssymb}
\usepackage{amsthm}
\usepackage[]{thmbox}
\usepackage{blindtext}
\usepackage[utf8]{inputenc}
\usepackage{amsmath}
\usepackage{amsfonts}
\usepackage[]{graphicx}
\usepackage[legalpaper, portrait, margin = 1in]{geometry}
\usepackage{enumitem}


\usepackage{xcolor}

%\pagecolor[rgb]{0.2,0.19,0.18} 
%\color[rgb]{0.92,0.86,0.7}

\newtheorem[L]{le}{Lemma}[subsection]
\newtheorem[L]{th}[le]{Theorem}
\newtheorem[L]{df}[le]{Definition}
\newtheorem[L]{ex}[le]{Example}
\newtheorem[L]{pf}[le]{Proof}


\newcommand{\nl}{\newline}

\newcommand{\real}{\mathbb{R}}
\newcommand{\complex}{\mathbb{C}}
\newcommand{\integer}{\mathbb{Z}}
\newcommand{\rational}{\mathbb{Q}}
\newcommand{\lxor}{\oplus}
\newcommand{\then}{\Rightarrow}

\begin{document}
    \huge Homework 4 - Thomas Boyko - 30191728
    \normalsize
\begin{enumerate} 
\item Consider the function $z^n$, where $n \in  \mathbb{Z}$.
\begin{enumerate}[label= (\alph*)] 
    \item Calculate the integral 
        \[
            \oint_{\gamma}z^{n} \, dz 
        .\] 
        \[
        \Re
        .\] 
        Where $\gamma$ is the boundary of the
pentagon whose vertices are the 5th roots of 1, traversed once
clockwise.

For $n\ge 0$, we know that $z^{n}$ is holomorphic in its domain $\mathbb{C}$, so $\gamma$ will be homotopic to
a constant contour and the integral will be zero.

For $n<-1$, we notice that our curve $\gamma$ is homotopic to the unit circle traveled clockwise, the integral over which will be
the negative of the typical unit circle parameterization, call it $\delta$.

\begin{align*}
    \oint_{\gamma}z^{n} \, dz &= -\oint_{\delta} z^{n} \, d z  \\
    &= -\int_{0}^{2\pi} e^{nti}ie^{it} \, d t  \\
    &= -i\int_{0}^{2\pi} e^{(n+1)it)}\, d t  \\
    &= -i \left( \frac{e^{it(n+1)}}{n+1} \right)_{0}^{2\pi} \\
    &= -i\left( \frac{e^{i_2\pi(n+1)}}{n+1}-\frac{e^{0}}{n+1} \right)  \\
    &= -i\left(\frac{1}{n+1}-\frac{1}{n+1}\right) \\
    &= 0
.\end{align*}
We can see in our calculations that the primitive we found in this example is undefined for $n=-1$, which
motivates a seperate case.

For $n=-1$, we have the same homotopy and we can use the same parameterization; so we calculate
\begin{align*}
    \oint_{\delta} \frac{1}{z} \, d z &= -i\int_{0}^{2\pi} \frac{e^{it}}{e^{it}} \, d z  \\
    &= -i \int_{0}^{2\pi} 1 \, d z  \\
    &= -2\pi i 
.\end{align*}

So for $n=-1$, the integral is $-2\pi i$, and for any other $n\in \mathbb{Z}$, the integral is 0.

\item Calculate the integral of $z^n$ , where $n \in \mathbb{Z}$, on the first side of the same pentagon, originating from 1 and moving clockwise.

We start by finding the necessary 5th roots of unity, which are given by $z_0=1$ and $z_1=e^{\frac{8}{5}\pi i}$. First let's take the case $n\neq -1$, where our integral is not path-dependent. So we only depend on the endpoints, $1$ and $e^{i\frac{5\pi}{8}}$, and a primitive for $z^{n}$ which is $\frac{z^{n+1}}{n+1}$ for $n\neq -1$.
\[
    \int_{\gamma} z^{n} \, dz=F(e^{\frac{8\pi}{5}i})-F(1)=\frac{1}{n+1}\left( e^{\frac{8\pi}{5}(n+1)}-1 \right) 
.\] 
For $n=-1$, we notice that though $\frac{1}{z}$ does not have a primitive in its domain, it does have
a primitive in the domain of $\ln$, $\mathbb{C}\setminus\mathbb{R}_{\leq 0}$ since this domain is simply connected. This primitive is given by $\ln z$; and so we can evaluate the integral:
\[
\int_{\gamma}^{} \frac{1}{z} \, dz  =\ln e^{\frac{8\pi}{5}i}- \ln 1=\frac{8\pi}{5}i
.\] 

\end{enumerate} 

\newpage

\item Calculate the integral
    \[
    \oint_{\gamma} \frac{1}{z^2-4z+3} \, dz 
    .\] 
where $\gamma$ is the circle centred on 0 and of radius 2, traversed once counterclockwise.

We can see singularities of this function at $z=1,3$. We also note that $\gamma$ is a positively oriented Jordan curve. Now if we let $f(z)=\frac{1}{z-3}$,
we can modify our integral:
\begin{align*}
    \oint_{\gamma} \frac{1}{z^2-4z+3} \, d z&= \oint_{\gamma} \frac{f(z)}{z-1} \, d x  \\ 
    &=\frac{2\pi i}{2\pi i} \oint_{\gamma} \frac{f(z)}{z-1} \, d z  \\
    &= 2\pi i f(1) \\
    &= \frac{2\pi i}{1-3} \\
    &= -\pi i
.\end{align*}
So;
\[
    \oint_{\gamma} \frac{1}{z^2-4z+3} \, dz = -\pi i
.\] 

\end{enumerate}
\end{document}
