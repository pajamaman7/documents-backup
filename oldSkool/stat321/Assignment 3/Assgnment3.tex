\documentclass[]{article}
\usepackage[most,many,breakable]{tcolorbox}
\usepackage{amsmath}
\usepackage{amssymb}
\usepackage{amsthm}
\usepackage[]{thmbox}
\usepackage{blindtext}
\usepackage[utf8]{inputenc}
\usepackage{amsmath}
\usepackage{amsfonts}
\usepackage[]{graphicx}
\usepackage[legalpaper, portrait, margin = 1in]{geometry}
\usepackage{enumitem}


\usepackage{xcolor}

%\pagecolor[rgb]{0.2,0.19,0.18} 
%\color[rgb]{0.92,0.86,0.7}

\newtheorem[L]{le}{Lemma}[subsection]
\newtheorem[L]{th}[le]{Theorem}
\newtheorem[L]{df}[le]{Definition}
\newtheorem[L]{ex}[le]{Example}
\newtheorem[L]{pf}[le]{Proof}


\newcommand{\nl}{\newline}

\newcommand{\real}{\mathbb{R}}
\newcommand{\complex}{\mathbb{C}}
\newcommand{\integer}{\mathbb{Z}}
\newcommand{\rational}{\mathbb{Q}}
\newcommand{\lxor}{\oplus}
\newcommand{\then}{\Rightarrow}

\begin{document}
\huge Assignment 3 - Thomas Boyko - 30191728
\normalsize
\begin{enumerate} 

\item An urn contains two red balls, four blue balls, and three green balls. You randomly select a ball from the urn, record its color, and do not return it to the urn. You do this process twice (for a total of two recorded colors). Let $X$ represent the number of red balls that you observed.

Suppose $X\sim Hypergeometric(2,9,2)$
\begin{enumerate}[label= (\alph*)] 
\item Create a probability distribution table for $X$.
\begin{verbatim}
    dhyper(0:2,2,7,2)
\end{verbatim}
\begin{center}
\begin{tabular}{|c|c|c|c|}
    \hline
    x & 0 & 1 & 2\\
    \hline
    P(X=x)&0.58333333&0.38888889&0.02777778\\
    \hline 
\end{tabular}
\end{center}

\item How many red balls should you expect to select in this process?
\[E[X]=\frac{nr}{N}=\frac{2(2)}{9}=\frac{4}{9}\]
So the expected value is $0.44444$.
\end{enumerate}
\item A certain region of the Florida coast experiences 4.04 hurricanes per year.

Suppose $X\sim Poisson(4.04)$.
\begin{enumerate}[label= (\alph*)] 
\item What is the probability that this region experiences exactly 10 hurricanes in a two-year period?

Modify the distribution to consider a two year period rather than one (See part b). This gives $\lambda = 8.08$.
\begin{verbatim}
    dpois(10,8.08)
    #0.1012165
\end{verbatim}

\item How many hurricanes would we expect this region to experience in a two-year
period?

\[E[2X]=2E[X]=2\lambda=8.08\]

\end{enumerate}
\item Let $f(x) = \begin{cases} 
2x - 1 &1 \leq x \leq 2\\
0& elsewhere
\end{cases}$

Show that $f(x)$ is not a valid pdf.

Consider: 
\begin{align*} 
\int_{1}^{2}2x-1 dx&=x^2-x|^{2}_{1}\\
&=(2^2-2)-(1^2-1)\\
&=2-0\\
&=2\neq1
\end{align*}
So $f$ is not a valid pdf since the integral over all nonzero values is not 1.
\item The moment generating function for a Poisson random variable $X$ with parameter $\lambda$ is $M_X(t) = e^{\lambda(e^t-1)}$
Use this moment generating function to show that $E[X] =\lambda $.

\begin{align*}
    E[X]&=\frac{d}{dx}M_X(t)\\
    &=\frac{d}{dx}(e^{\lambda(e^t-1)})\\
    &=(\lambda e^t e^{\lambda(e^t-1)})\\
    t=0:\quad\quad&=\lambda e^0 e^{\lambda(e^0-1)}\\
    &=\lambda 1 e^{\lambda(0)}\\
    &=\lambda
\end{align*}
\item Let $f(x) = \begin{cases} 
2x & 0 \leq x \leq 1\\
0 & elsewhere
\end{cases}$ 
\begin{enumerate}[label= (\alph*)] 
\item Find $E[X]$.

\begin{align*} 
    E[X]&=\int_{0}^{1}xf(x)dx\\
    &=\int_{0}^{1}2x^2dx\\
    &=\frac{2x^3}{3}|^{1}_{0}\\
    &=\frac{2}{3}
\end{align*}
So the expected value is $\frac{2}{3}$.
\item Find $P(X \leq \frac{1}{2})$

\begin{align*} 
    P(X\leq \frac{1}{2})&=F(\frac{1}{2})\\
    &=\int_{0}^{\frac{1}{2}}2xdx\\
    &=x^2|_0^\frac{1}{2}\\
    &=\frac{1}{4}
\end{align*}
So $P(X\leq\frac{1}{2})=\frac{1}{4}$.
\end{enumerate}

\item The duration of a pulse of light from a device follows a uniform distribution with a = 10
seconds and b = 20 seconds. Let X represent the duration of the pulse (in seconds). If
the duration of a specific pulse is at least 13 seconds long, what is the probability that
the duration of the pulse is at most 18 seconds long?

We can say: $X\sim Uniform(10,20)$ and $f(x)=\frac{1}{10}$.

\begin{align*} 
    P(X\leq18|X\geq13)&=\frac{P(13\leq X\leq18)}{P(X\geq13)}=\frac{F(18)-F(13)}{1-F(13)}\\
\end{align*}
\begin{verbatim}
    (punif(18,10,20)-punif(13,10,20))/(1-punif(13,10,20))
    #0.7142857
\end{verbatim}
So there is a $0.7142857$ chance that the pulse is at most $18$ seconds long, given that the pulse is at least $13$ seconds long
\newpage
\item The finishing time for a particular half-marathon is normally distributed with $\mu = 131.9$
minutes and $\sigma = 20.3$ minutes.

So we say $X\sim Normal(131.9,20.3)$.
\begin{enumerate}[label= (\alph*)] 
\item What is the probability that a given runner's finishing time is less than two hours?
\begin{verbatim}
    pnorm(120,131.9,20.3)
    #0.2788682
\end{verbatim}
So the probability that a runner's finishing time is less than two hours is $0.27887$.
\item What is the fastest a runner can finish the half-marathon and still be in the slowest
$20\%$ of finishers?
\begin{verbatim}
    qnorm(0.2,131.9,20.3)
    #114.8151
\end{verbatim}
So the fastest a runner can finish while being in the slowest $20\%$ of runners is $113.8151$ minutes.
\item Suppose 10 random finishers are selected. What is the probability that exactly four
of the finishers selected took less than two hours to finish the half-marathon?

Each runner's time being under two hours can be considered a Bernouli trial with $p=0.27887$. So we can model our 10 trials with a Binomial Distribution.

Suppose $Y\sim Binomial(0.27887,10)$.

\begin{verbatim}
    dbinom(4,0.2788682,10)
    #0.1786089
\end{verbatim}

So the probability that exactly four of the ten runners selected took less than two hours to complete the half-marathon is $0.17861$.

\end{enumerate}
\item The number of times per hour a printer is used at a busy office can be modeled with a
Poisson distribution with $\lambda = 4.3$. Let $W$ represent the amount of time (in hours) that
passes between consecutive uses.
\begin{enumerate}[label= (\alph*)] 
\item What type of distribution can be used to model the behavior of $W$? State any
relevant parameter values as well.

$W\sim Exponential\left(\frac{1}{4.3}\right)$.

\item What is the probability that fewer than 30 minutes pass between consecutive uses
of this printer?
\begin{verbatim}
    pexp(0.5,4.3)
    #0.8835158
\end{verbatim}
So the probability that fewer than 30 minutes pass between consecutive uses of the printer is $0.10977$.
\end{enumerate}
\end{enumerate}
\end{document}