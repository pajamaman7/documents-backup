\documentclass{article}
\usepackage{amsmath}
\usepackage{amssymb}
\usepackage{amsthm}
\usepackage[utf8]{inputenc}
\usepackage{amsmath}
\usepackage{amsfonts}
\usepackage[]{graphicx}
\usepackage[a4paper, portrait, margin = 1in]{geometry}
\usepackage{enumitem}
\usepackage{xcolor}

%darkmode
%\pagecolor[rgb]{0.2,0.19,0.18} 
%\color[rgb]{0.92,0.86,0.7}

\newenvironment*{alphenum}{\begin{enumerate}[label= (\alph*)]}{\end{enumerate}}

\begin{document}
\huge Bonus Assignment - Thomas Boyko - 30191728
\normalsize
\begin{enumerate} 

\item Let $X$ and $Y$ be continuous random variables and let
\[f(x,y)=\begin{cases} 
\frac{x}{5}+cy&0\leq x\leq 1, 0\leq y \leq5\\
0 & elsewhere
 \end{cases}\]
Find the value of the constant $c$ that makes $f(x, y)$ a valid pdf.

For $f$ to be a valid pdf, the double integral over all nonzero values of $x,y$ must equal $1$.

\begin{align*} 
1&=\int_{0}^{1}\int_{0}^{5}\frac{x}{5}+cy dydx\\
&=\int_{0}^{1}\frac{25c}{2}dx\\
&=\frac{25cx}{2}|^{1}_{0}\\
&=\frac{25c}{2}\\
\frac{2}{25}&=c
\end{align*}

\item Let $X$ and $Y$ be continuous random variables with a joint probability density function
defined as follows:
\[f(x, y) =\begin{cases} 
2e^{-(x+y)} & 0\leq x, 0\leq y, x\leq y\\
0 & elsewhere
\end{cases}\]
Are $X$ and $Y$ independent random variables? Show why/why not.

$X,Y$ are independent $\iff f(x,y)=f(x)f(y)$.

\begin{align*} 
f(x)f(y)&=\int_{0}^{\infty}f(x,y)dy \cdot\int_{0}^{y}f(x,y)dx\\
&=\int_{0}^{\infty}2e^{-(x+y)}dy \cdot\int_{0}^{y}2e^{-(x+y)}dx\\
&=2e^{-x}\cdot(2e^{-y} -2e^{-2y})\neq2e^{-(x+y)}=f(x,y)
\end{align*}

Since $f(x,y)\neq f(x)f(y)$, the variables are not independent.

\item Let $X$ and $Y$ be continuous random variables with a joint probability density function
defined as follows:
\[f(x, y) = \begin{cases} 
xy &0 \leq x \leq 1, 0 \leq y \leq 2\\
0 & elsewhere
\end{cases}\]
Find $P(X \leq Y).$

\begin{align*} 
    P(X\leq Y) &= \int_{0}^{2}\int_{0}^{y}xydxdy\\
    &= \int_{0}^{2}\frac{yx^2}{2}|^{y}_{0}dy\\
    &=\int_{0}^{2}\frac{y^3}{2}dy\\
    &=\frac{y^4}{8}|^{1}_{0}\\
    &=\frac{1}{8}
\end{align*}

So $P(X\leq Y)=\frac{1}{8}$.

\item Suppose a game consists of rolling two fair four-sided die (a red one and a blue one)
and observing the number on the uppermost face of each die. In this game, you win \$2
if the blue die shows the same number as the red die, you win nothing (\$0) if the blue
die shows a higher number than the red die, and you lose \$1 on any other outcome.

Define two random variables as follows:
\begin{itemize}
\item Let $X$ be the number shown on the uppermost face of the blue die.
\item Let $Y$ be the amount of money you win (or lose) playing this game once.
\end{itemize}
\begin{enumerate}[label= (\alph*)] 
 \item Find the joint probability distribution of $X$ and $Y$ and display it as a joint probability
distribution table. Please be sure to clearly label which variable corresponds to the
row values and which variable corresponds to the column values.
\begin{center}
    \begin{tabular}{|c|c|c|c|c|c|}
        \hline
        $\downarrow Y\rightarrow X$ &1&2&3&4&Total\\
        \hline
        -1&0&1/16&2/16&3/16&6/16\\
        \hline
        0&3/16&2/16&1/16&0&6/16\\
        \hline
        2&1/16&1/16&1/16&1/16&4/16\\
        \hline
        Total & 4/16 &4/16&4/16&4/16&1\\
        \hline
    \end{tabular}
\end{center}
\item Find $COV(X, Y)$.
\begin{align*} 
    COV[X,Y]&=E[XY]-E[X]E[Y]\\
    &=
\end{align*}
\end{enumerate} 
\iffalse
Problem 5
At a health clinic, the wait time to see a doctor is known to follow an exponential
distribution with 𝛽 = 20.34 minutes. A receptionist at the clinic observes 14 patients and
records the waiting time (in minutes) for each of them.
1. Let 𝑥̅ represent the average waiting time of the 14 patients observed by the
receptionist. Describe the distribution of 𝑥̅ .
2. What is the probability that the average waiting time of the 14 patients is between 15
minutes and 20 minutes?
\fi
\end{enumerate}
\end{document}