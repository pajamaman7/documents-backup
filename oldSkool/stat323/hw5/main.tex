\documentclass{article}
\usepackage{amsmath}
\usepackage{amssymb}
\usepackage{amsthm}
\usepackage[utf8]{inputenc}
\usepackage{amsmath}
\usepackage{amsfonts}
\usepackage[]{graphicx}
\usepackage[a4paper, portrait, margin = 1in]{geometry}
\usepackage{enumitem}
\usepackage{xcolor}

%darkmode
%\pagecolor[rgb]{0.2,0.19,0.18} 
%\color[rgb]{0.92,0.86,0.7}

\newenvironment*{alphenum}{\begin{enumerate}[label= (\alph*)]}{\end{enumerate}}


\begin{document}
    \huge STAT 323 Assignment 5
    \normalsize
\begin{enumerate} 
    \item 1

        We set our two tests equal:
        \begin{align*}
            P(Y_1>.95|\theta=0)&= P(Y_1+Y_2>c) \\
            \int_{0.95}^{1} 1 \, d x &= P(Y_1+Y_2>c) \\
            0.05&= P(Y_1+Y_2>c) 
        .\end{align*}
        Now from example 1.1.6, $Y_1+Y_2=Z$ has the pdf given by:
        \[
        f_Z(z)=\begin{cases}
            z& 0\leq z\leq1\\
            -z+2 &1< z \leq2\\
            0 &\text{otherwise}
        \end{cases}
        .\] 
        And so we integrate;
        \begin{align*}
            0.95&= P(Z\leq c) \\
            &= \int_{0}^{1} z \, d z+\int_{1}^{c} -z+2 \, d z   \\
            &= ^{wolfram}1.68377 
        .\end{align*}
        Technically this gives us two roots, but we can reject the one greater than 2 since it is outside the support.
    \item 2 
    \item 3

        The power of our test $1-\beta$ is the probability that we reject ouir null hypothesis correctly;
        \[
        1-\beta=P(RH_0|H_0\text{ is false})=P(x_1x_2>0.75|\theta=2)
        .\] 
        To find this probability we must find the joint distribution of $x_1x_2$. Since our variables are independent,
        our joint distribution is given by $f_{x_1}f_{x_2}=4x_1x_2$, and integrating our support, 
        \[ \int_{\frac{3}{4}}^{1} \int_{\frac{3}{4x_2}}^{1} 4x_1x_2 \, d x_1  \, d x_2 =^{wolfram}0.113857\] 

    \item 4
    \item 5

        We load our data into R:
        \begin{verbatim}
> june=c(153.3, 155.9, 176.2, 189.9, 200.0, 214.9, 229.9, 231.5, 257.9, 299.9)
> december=c(151.1, 154.2, 169.9, 169.9, 185.9, 199.5, 229.9, 232.9, 279.9, 289.9)
        \end{verbatim}
        And we can use \verb|t.test| in R since our data is normally distributed. Our null hypothesis is that the mean selling 
        price of a condo is the same in December as in June, and our alternative is that the mean in December is less than June.

        Before using \verb|t.test| we must check the ratio of variances, which we can do using \verb|var.test|
        \begin{verbatim}
        > var.test(june,december)
        \end{verbatim}
        And the interval that this gives us does not contain 1, so we cannot conclude equal variance. So our \verb|t.test|:
        \begin{verbatim}
> t.test(december,june,alternative="less",var.equal=FALSE)
        \end{verbatim}
        This gives us a $p$-value of 0.5841, too high to reject the null hypothesis. Therefore there is not significant evidence to suggest
        that the mean price of a condo decreased from December to June.
    \item 6
    \item 7

        Let $\hat{p}_1$ be the proportion of defects while the water was contaminated. We know this to be 16 out of 414 births.
        Let $\hat{p}_2$ be the defects after the water was contaminated. We know this to be 2 out of 228 births. 
        Then our hypotheses are $H_0:p_1-p_2\leq 0$, and $H_a: p_1-p_2>0$. Our test statistic:
            $$Z_{calc}=\frac{\widehat{p}_1-\widehat{p}_2}{\sqrt{\widehat{p}(1-\widehat{p})\left(\frac1{n_1}+\frac1{n_2}\right)}}\sim Normal{\left(0,1\right)}$$
            Where $\hat{p}$ is our pooled sample proportion,
            when simplified with our above values gives us a test statistic of 2.194423. To find our p-value, the probability of observing
            a sample more extreme than this, is given by:
            \[
            P(Z>\text{Test Stat})=\verb|1-pnorm(2.194423)|=0.0141
            .\] 
            Which is less than our tolerance for type 1 error, so we conclude that the contaminated water had a significant effect on the proportion of 
            birth defects.
    
    \item 8
    \item 9
        Begin with neyman-pearson, using both sets of RVs,
        \begin{align*}
            \frac{L(H_0)}{L\left( H_a \right) }&=\frac{\prod_{i=1}^{n} 2^{y_i}e^{-{2}}/y_i! \prod_{j=1}^{m} 2^{x_i}e^{-2}/x_i!}
            {\prod_{i=1}^{n} 1 /2^{y_i}e^{-\frac{1}{2}}/y_i! \prod_{j=1}^{m} 3^{x_i}e^{-3}/x_i!} \\
            &=\frac{\prod_{i=1}^{n} 2^{y_i}e^{-{2}} \prod_{j=1}^{m} 2^{x_i}e^{-2}}
            {\prod_{i=1}^{n} 1 /2^{y_i}e^{-\frac{1}{2}} \prod_{j=1}^{m} 3^{x_i}e^{-3}} \\
            2^{\sum_{}^{} y_{i}}\left( \frac{1}{2} \right) ^{-\sum_{}^{} y_{i}}\left( \frac{2}{3} \right) ^{\sum_{}^{} x_{i}}e^{-2}e^{-2}e^{\frac{1}{2}}e^{3}&<k\\
            \left( \frac{2}{3} \right) ^{m\bar{x}}4^{n\bar{y}}&<k'
        .\end{align*}
        Taking the ln of both sides brings our variables out of the exponent but complicates the expression
        so I left this the way it was. 

        Our test statistic is
        \[
            \left( \frac{2}{3} \right) ^{m\bar{x}}4^{n\bar{y}}
        .\] 
    \item 10
    \item 11

        We use Neyman-Pearson; beginning with the ratio of our likelihood functions:
        \begin{align*}
            \frac{L(\sigma^2=2)}{L(\sigma^2=3)}&= \frac{\prod_{i=1}^{8} \frac{1}{\sqrt{4\pi} }\exp\left[ -\left( \frac{1}{4}(y_{i}-\mu)^2 \right)  \right] 
             }{\prod_{i=1}^{8} \frac{1}{\sqrt{6\pi} }\exp\left[ -\left( \frac{1}{6}(y_{i}-\mu)^2 \right)  \right] } \\
                &= \frac{ \frac{1}{(4\pi)^{4}}\prod_{i=1}^{8}\exp\left[ -\left( \frac{1}{4}(y_{i}-\mu)^2 \right)  \right] 
             }{ \frac{1}{(6\pi)^{4}}\prod_{i=1}^{8}\exp\left[ -\left( \frac{1}{6}(y_{i}-\mu)^2 \right)  \right] } \\
             &= \left( \frac{6\pi}{4\pi} \right) ^{4}\frac{\exp\left[ -\frac{1}{4}\sum_{i=1}^{8} (y_{i}-\mu)^2\right] 
             }{\exp\left[ -\frac{1}{6}\sum_{i=1}^{8} (y_{i}-\mu)^2 \right] 
             } \\
             \left( \frac{6\pi}{4\pi} \right) ^{4}\frac{\exp\left[ -\frac{1}{4}\sum_{i=1}^{8} (y_{i}-\mu)^2\right] 
             }{\exp\left[ -\frac{1}{6}\sum_{i=1}^{8} (y_{i}-\mu)^2 \right] }&<k \\
             \frac{\exp\left[ -\frac{1}{4}(8-1)S^2\right] 
             }{\exp\left[ -\frac{1}{6}(8-1)S^2 \right] }&<k' \\
             \exp\left[ \frac{7}{6}S^2-\frac{7}{4}S^2 \right] &<k'\\
             \frac{7}{6}S^2-\frac{7}{4}S^2 &<k''\\
             -\frac{14}{24}S^2&<k''\\
             S^2&>k'''
        .\end{align*}
        So we have our decision rule, to reject $H_0$ if $S^2$ is greater than our critical value. Now to find our critical value,
        setting the probability of Type I error to $0.05$:
        \begin{align*}
            0.05&= P(RH_0|H_0 \text{ is true})\\
            &= P(S^2>CV|\sigma ^2=2) \\
            &= P(\frac{7S^2}{2}>\frac{7}{2}CV) \\
            &= P(\chi^2_{7}>\frac{7}{2}CV) \\
        .\end{align*}
        And so in $R$ our CV is given by \verb|qchisq(.05,7)*(2/7)|=0.6192428. Our uniformly most powerful test is to reject the null hypothesis when our sample
        variance is greater than this value.
\end{enumerate}
\end{document}
