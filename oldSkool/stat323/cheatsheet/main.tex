\documentclass{article}
\usepackage[landscape]{geometry}
\usepackage{amsmath}
\usepackage{amssymb}
\usepackage{amsthm}
\usepackage[utf8]{inputenc}
\usepackage{amsmath}
\usepackage{amsfonts}
\usepackage[]{graphicx}
\usepackage[a4paper, portrait, margin = 1in]{geometry}
\usepackage{enumitem}
\usepackage{xcolor}

%darkmode
%\pagecolor[rgb]{0.2,0.19,0.18} 
%\color[rgb]{0.92,0.86,0.7}

\newenvironment*{alphenum}{\begin{enumerate}[label= (\alph*)]}{\end{enumerate}}


\begin{document}
\begin{multicols*}{3}

\tikzstyle{mybox} = [draw=black, fill=white, very thick,
    rectangle, rounded corners, inner sep=10pt, inner ysep=10pt]
\tikzstyle{fancytitle} =[fill=black, text=white, font=\bfseries]

% \huge Cheat Sheet - Thomas B
\normalsize
%------------ Method of Transformations -----------------
\begin{tikzpicture}
\node [mybox] (box){%
    \begin{minipage}{0.3\textwidth}
        Single variable: Let $Y=g(X)$. Find some $g^{-1}(Y)=X$. Then:
        $f_Y=f_X(g^{-1}(y))\left| \frac{d}{dy}g^{-1}(y) \right|$\\
        Multivariable: \\
        $f_{X_1,Y}(x_1,y)=f_{X_1,X_2}(x_1,g_{X_2}^{-1}(y,x_1))\left|\frac{d}{dy}g_{X_2}(y,x_1)\right|$\\
        $f_{X_2,Y}(x_2,y)=f_{X_2,X_1}(g_{X_1}^{-1}(y,x_2)),x_2)\left|\frac{d}{dy}g_{X_1}(y,x_2)\right|$\\
        $f_{Y_1,Y_2}(y_1,y_2)=f_{X_1,X_2}\left(g_1^{-1}\left(y_1,y_2\right),g_2^{-1}\left(y_1,y_2\right)\right)|J|$\\
        (then integrate out)
    \end{minipage}
};
%Method of Transformation
\node[fancytitle, right=10pt] at (box.north west) {Method of Transformations};
\end{tikzpicture}

\iffalse
% Dist. of statistics
\begin{tikzpicture}
\node [mybox] (box){%
    \begin{minipage}{0.3\textwidth}
        $\frac{(n-1)S^2}{\sigma^2}\sim\chi^2_{df=n-1}$\\
        $\bar{X}\sim N(\mu_X,\frac{\sigma^2}{n})$\\
        $\frac{S_{1}^{2}\sigma_{2}^{2}}{S_{2}^{2}\sigma_{1}^{2}}\sim F(n_{1}=1,n_{2}=1)$
    \end{minipage}
};
% Dist. of statistics
\node[fancytitle, right=10pt] at (box.north west) {Distributions of Statistics};
\end{tikzpicture}
\fi

%Bias Content
\begin{tikzpicture}
\node [mybox] (box){%
    \begin{minipage}{0.3\textwidth}
        For an estimator $\hat{\theta}$ of $\theta$:\\
        $B(\hat{\theta})=E[\hat{\theta}]-\theta$.\\
        $MSE(\hat{\theta})=B(\hat{\theta^2})+Var(\theta)$\\
    \end{minipage}
};

% Bias Header
\node[fancytitle, right=10pt] at (box.north west) {Bias and Mean Squared Error};
\end{tikzpicture}

%CI Content
\begin{tikzpicture}
\node [mybox] (box){%
    \begin{minipage}{0.3\textwidth}
        For sufficiently large or normal samples with mean $\mu$, variance $\sigma^2$:\\
        $\bar{X}\pm \frac{Z_{\frac{\alpha}{2}}\sigma}{\sqrt{ n }} $\\
        Population Proportion:\\
        $$\hat{p}\pm Z_{\frac{\alpha}{2}}\sqrt{\frac{\hat{p}(1-\hat{p}}{n}} $$
    \end{minipage}
};

% CI Header
\node[fancytitle, right=10pt] at (box.north west) {Confidence intervals};
\end{tikzpicture}

%CI single sample in R
\begin{tikzpicture}
\node [mybox] (box){%
    \begin{minipage}{0.3\textwidth}
        \textbf{To find a CI for $\mu$:}\\
        Have RAW DATA and know $\sigma$:\\
        \verb|z.test|\\
        (BSDA) Have SUMMARY DATA and do know $\sigma$ :\\
        \verb|zsum.test|\\
        Have RAW DATA and do not know $\sigma$:\\
        \verb|t.test|\\
        (BSDA) Have SUMMARY DATA and do not know $\sigma$:\\
        \verb|tsum.test|\\
        DescTools for variance CIs (only for raw data):\\
        \verb|VarCI|\\
        Use ? for more information on arguments
    \end{minipage}
};

% CI in R Header
\node[fancytitle, right=10pt] at (box.north west) {CIs and Hypothesis testing in R};
\end{tikzpicture}

%ORDER STATS
\begin{tikzpicture}
\node [mybox] (box){%
    \begin{minipage}{0.3\textwidth}
$f_{X_{(n)}}=n[F_X(x)]^{n-1}f_X(x)$\\
$f_{X_{(1)}}=n[1-F_X(x)]^{n-1}f_X(x)$\\
$f_{X_{(k)}}=\frac{n!}{(k-1)!(n-k)!}F_X(x)^{k-1}\left[ 1-F_X(x) \right] ^{n-k}f_X(x)$
    \end{minipage}
};

\node[fancytitle, right=10pt] at (box.north west) {Order Statistics};
\end{tikzpicture}

%Two Sample CI
\begin{tikzpicture}
\node [mybox] (box){%
    \begin{minipage}{0.3\textwidth}
        Use eg. \verb|?t.test| for info on arguments\\
        For $\frac{\sigma_1}{\sigma_2}$ (RAW only):\verb|var.test|\\
        CIs for $\mu_1-\mu_2$ (ensure you check equal variance): \\
        RAW: \verb|t.test|\\
        SUMMARY: \verb|tsum.test|\\
        CI for $\hat{p}_1-\hat{p}_2$:\\ 
        \verb|prop.test(x=c(success),n=c(n),conf=,correct=F)|
    \end{minipage}
};

\node[fancytitle, right=10pt] at (box.north west) {Two Sample CIs and Hypothesis testing in R};
\end{tikzpicture}

%Two Sample CI summary
\begin{tikzpicture}
\node [mybox] (box){%
    \begin{minipage}{0.3\textwidth}
        For $\frac{\sigma_1}{\sigma_2}$ with summary data: \\
$$\frac{S_1^2}{F_{1-\alpha/2,(n_1-1,n_2-1)}\cdot S_2^2}\leq\frac{\sigma_1^2}{\sigma_2^2}\leq\frac{S_1^2}{F_{\alpha/2,(n_1-1,n_2-1)}\cdot S_2^2}$$
    \end{minipage}
};

\node[fancytitle, right=10pt] at (box.north west) {Two Sample CIs};
\end{tikzpicture}

% Consistency
\begin{tikzpicture}
\node [mybox] (box){%
    \begin{minipage}{0.3\textwidth}
An estimator is said to be consistent if, for any positive integer $\epsilon$,
$$\lim\limits_{ n \to \infty } P(|\hat{\theta}_n-\theta|\leq\epsilon)=1$$
or,
$$\lim\limits_{ n \to \infty } P(|\hat{\theta}_n-\theta|>\epsilon)=0$$
   $$eff(\hat{\theta}_1,\hat{\theta}_2)=RE(\hat{\theta}_1,\hat{\theta}_2)=\frac{MSE(\hat{\theta}_1)}{MSE(\hat{\theta}_2)}$$
If efficiency is less than 1, $\hat{\theta}_1$ is a more efficient estimator.     
    \end{minipage}
};

\node[fancytitle, right=10pt] at (box.north west) {Consistency and Efficiency};
\end{tikzpicture}

%MoM
\begin{tikzpicture}
\node [mybox] (box){%
    \begin{minipage}{0.3\textwidth}
        \paragraph{Central Moments}
            Equate the first sample moment about the origin $M_1^{\prime}=\frac1n\sum_{i=1}^nX_i^1=X$ to the first theoretical moment $E(X).$
            Continue equating sample central moments, $M_k^*$ with the corresponding central theoretical moments $\bar{E}((X-\mu)^k),k=3,4,...$ until you have as many equations as you have parameters.
            Solve for the parameters.
        \paragraph{Raw Moments}
            Equate the first sample raw moment $M_1^{\prime}=\frac1n\sum_{i=1}^nX_i^1=\bar{X}$ to the first theoretical moment $E(X).$
            Continue equating raw sample moments, $M_k^{\prime}$, with the corresponding theoretical moments $E(X^k),k=3,4,...$ until you have as many equations as you have parameters.
            Solve for the parameters.
    \end{minipage}
};

\node[fancytitle, right=10pt] at (box.north west) {Method of Moments};
\end{tikzpicture}

%MLE
\begin{tikzpicture}
\node [mybox] (box){%
    \begin{minipage}{0.3\textwidth}
        Find pdf\\
        Find the likelihood function ${L}(\theta)=\prod_{i=1}^nf(x_i|\theta)$\\
        Optional: take $\ln$ of the function\\
        Take the first derivative and set it to 0\\
        Ensure that the zero is a maximum and not a minimum
    \end{minipage}
};

\node[fancytitle, right=10pt] at (box.north west) {MLEs};
\end{tikzpicture}

%Hypothesis Testing
\begin{tikzpicture}
\node [mybox] (box){%
    \begin{minipage}{0.3\textwidth}
        $\alpha=P(RH_0|H_0\text{ is true})$, type 1 error\\
        $\beta=P(FTRH_0|H_0\text{ is false})$, type 2 error\\
        Power$=1-\beta=P(RH_0|H_0\text{ is true})$
    \end{minipage}
};

\node[fancytitle, right=10pt] at (box.north west) {Hypothesis testing};
\end{tikzpicture}

%Test Statistics
\begin{tikzpicture}
\node [mybox] (box){%
    \begin{minipage}{0.3\textwidth}
$$Z_{calc}=\frac{\widehat{p}_1-\widehat{p}_2}{\sqrt{\widehat{p}(1-\widehat{p})\left(\frac1{n_1}+\frac1{n_2}\right)}}\sim Normal{\left(0,1\right)}$$
$$Z_{calc}=\frac{(\widehat{p}_1-\widehat{p}_2)-d_0}{\sqrt{\frac{\widehat{p}_1(1-\widehat{p}_1)}{n_1}+\frac{\widehat{p}_2(1-\widehat{p}_2)}{n_2}}}\sim Normal{\left(0,1\right)}$$
For difference in means or variance use R.
    \end{minipage}
};

\node[fancytitle, right=10pt] at (box.north west) {Test Statistics for proportion difference};
\end{tikzpicture}

%Neyman Pearson
\begin{tikzpicture}
\node [mybox] (box){%
    \begin{minipage}{0.3\textwidth}
        The uniformly most powerful test for testing $H_0:\theta=\theta_0$ vs. $H_a: \theta=\theta_a$:
        \[ \frac{L(\theta_0)}{L(\theta_a)}<k .\] 
    \end{minipage}
};

\node[fancytitle, right=10pt] at (box.north west) {Neyman Pearson};
\end{tikzpicture}

%LINEAR REGRESSION
\begin{tikzpicture}
\node [mybox] (box){
    \begin{minipage}{0.3\textwidth}
        We can find covariance and correlation respectively with \verb|cov, cor| in R\\
        For our probabalistic model of linear regression $Y_i=\beta_0+\beta_1X_i+\epsilon_i$,
        we use \verb|lm|, and for more information we wrap \verb|lm| in a \verb|summary()|\\
        Using \verb|anova| can see our SST and SSE.\\
        Using \verb|plot| we may be able to remove outliers from the dataset, and using \verb|plot(fit)|
        can give us several graphs to analyze our regression line.\\
        \verb|confint(lm)| can give us CIs for our estimate slope and intercepts of our regression model.
    \end{minipage}
};

\node[fancytitle, right=10pt] at (box.north west) {Covariance, Correlation and Linear Regression};
\end{tikzpicture}
%TODO: 
\end{multicols*}
\end{document}
