\documentclass{article}
\usepackage{amsmath}
\usepackage{amssymb}
\usepackage{amsthm}
\usepackage[utf8]{inputenc}
\usepackage{amsmath}
\usepackage{amsfonts}
\usepackage[]{graphicx}
\usepackage[a4paper, portrait, margin = 1in]{geometry}
\usepackage{enumitem}
\usepackage{xcolor}

%darkmode
%\pagecolor[rgb]{0.2,0.19,0.18} 
%\color[rgb]{0.92,0.86,0.7}

\newenvironment*{alphenum}{\begin{enumerate}[label= (\alph*)]}{\end{enumerate}}


\begin{document}
    \huge Assignment 3
    \normalsize
\begin{enumerate} 

    \item We can take the difference
        of the two means to find a point estimate for the mean of the two samples.
        This gives us our point estimate: $3.1-2.4=0.7$.
    \item 
    \item 3-
        \begin{enumerate}[label= (\alph*)] 
            \item First we must check whether the standard deviations of our 
                two samples are significantly different. Since R can only handle
                raw data, we do this with:

                \begin{align*}
                    \frac{S_1^2}{F_{1-\alpha/2,(n_1-1,n_2-1)}\cdot S_2^2}\leq\frac{\sigma_1^2}{\sigma_2^2}\leq\frac{S_1^2}{F_{\alpha/2,(n_1-1,n_2-1)}\cdot S_2^2}\\
                .\end{align*}
                \begin{verbatim}
                    > (8.03^2)/(qf(.025,364,297)*6.96^2)
                    [1] 1.651888
                    > (8.03^2)/(qf(1-.025,364,297)*6.96^2)
                    [1] 1.070024
                \end{verbatim}
                And we can't say the variances are equal, so now we find our CI
                using the R code:
                \begin{verbatim}
                    > tsum.test(18.5,8.03,265,16.5,6.96,298,var.equal=FALSE)
                \end{verbatim}
                We obtain the interval: 
                \[
                    (0.7484522,3.2515478)
                .\] 
            \item Since this interval does not contain 0, we can say with 95\% 
                confidence that there is a difference in the means of the 
                posttest scores.
        \end{enumerate}
    \item 
    \item 5- 
        \begin{enumerate}[label= (\alph*)] 
            \item Begin with
            \begin{align*}
                X&\sim N(\mu_1,\sigma^2)\\
                Y&\sim N(\mu_2,3\sigma^2)
            .\end{align*}
            And let $W=2\bar{X}+\bar{Y}$. We want to find $M_W(t)$.

            \begin{align*}
                M_W(t)&= E\left[ \exp(2t\bar{X}+t\bar{Y} )\right] \\
                      &= E\left[ \exp(2t\bar{X})]E[\exp(t\bar{Y}) \right]&\bar{X}\perp\bar{Y}\\
                      &= M_{\bar{X}}(2t)M_{\bar{Y}}(t)
            .\end{align*}
            And we know that:
            \begin{align*}
                M_{\bar{X}}(2t)&=\exp\left(2t\mu_1+(\frac{2t^2\sigma^2}{n}\right)  \\
                M_{\bar{Y}}(t)&= \exp\left( t\mu_2 +\frac{3t^2\sigma^2}{m}\right) 
            .\end{align*}
            Since each $X_i,Y_i$ is normal.
            \begin{align*}
                M_W(t)&= \exp\left(2t\mu_1+(\frac{2t^2\sigma^2}{n}\right) \exp\left( t\mu_2 +\frac{3t^2\sigma^2}{m}\right) \\
                &= \left( t(2\mu_1+\mu_2)+t^2\left( \frac{2\sigma^2}{n} + \frac{3\sigma^2}{m} \right)  \right) 
            .\end{align*}
            And we recognise the MGF of a normal distribution, and say
            $W\sim N(2\mu_1+\mu_2,\sigma^2\left( \frac{2}{n}+\frac{1}{m} \right) $

        \item To find our confidence interval we first notice that since $W$ is 
            normally distributed:
            \[
        \frac{W-2\mu_1-\mu_2}{\sigma^2\left( \frac{4}{n}+\frac{3}{m} \right) }=Z
        .\]
            And now we begin with the probability statement:
            \begin{align*}
                1-\alpha&=P(a<2\mu_1+\mu_2<b)\\
                    &=P(a-W<-W+2\mu_1+\mu_2<b-W)\\
                    &=P(W-b<W-(2\mu_1+\mu_2)<W-a)\\
                    &= P\left( \frac{W-b}{\sigma^2\left( \frac{4}{n}+\frac{3}{m} \right) }<\frac{W-(2\mu_1+\mu_2)}{\sigma^2\left( \frac{4}{n}+\frac{3}{m} \right) }<\frac{W-a}{\sigma^2\left( \frac{4}{n}+\frac{3}{m} \right) } \right)\\
                    &= P\left( \frac{W-b}{\sigma^2\left( \frac{4}{n}+\frac{3}{m} \right) }<Z<\frac{W-a}{\sigma^2\left( \frac{4}{n}+\frac{3}{m} \right) } \right)
            .\end{align*}
            Which gives us:
            \begin{align*}
                Z_{\frac{\alpha}{2}}&=\frac{W-b}{\sigma^2\left( \frac{4}{n}+\frac{3}{m} \right) }\\
                -Z_{\frac{\alpha}{2}}&=\frac{W-a}{\sigma^2\left( \frac{4}{n}+\frac{3}{m} \right) }
            .\end{align*}
            And after isolating for $a,b$ and substituting $W=2\bar{X}+\bar{Y}$,
            we get $(1-\alpha)100\%$ confidence interval for $2\mu_1+\mu_2$:
            \[
                (2\bar{X}+\bar{Y})\pm Z_{\frac{\alpha}{2}}\sigma^2\left( \frac{4}{n}+\frac{3}{m} \right) 
            .\] 
        \end{enumerate}
    \item 
    \item 7- 
        Seasonal ranges (in hectares) for alligators were monitored on a lake outside Gainesville, Florida, by biologists
        from the Florida Game and Fish Commission. Five alligators monitored in the spring showed ranges of 8.0,
        12.1, 8.1, 18.2, and 31.7. Four different alligators monitored in the summer showed ranges of 102.0, 81.7,
        54.7, and 50.7.
        Assume the sd in summer is higher than in the spring.


        We must assume that the gator range is normally distributed in order for CLT to work.
        Then we simply use \verb|t.test|.
        \begin{verbatim}
            > spring=c(8.0, 12.1, 8.1, 18.2, 31.7)
            > summer=c(102.0, 81.7, 54.7, 50.7)
            > t.test(spring,summer,var.equal=FALSE)
        \end{verbatim}

        Which gives the CI:

        $$(-93.04715,-20.26285)$$

        Which tells us that the average range of an alligator in spring is between $20$ and $93$ hectares less than that in summer, 95\% of the time.
    \item 
    \item 9-
        A Time magazine article reported about a survey of men’s attitudes, and noted that ‘when talking about
their problems, young men are more comfortable than older men.’ The survey reported that of 129 randomly
chosen males between the ages of 18 and 24 years, 80 were comfortable talking about their problems. Of
184 randomly chosen between the ages of 25 to 34 years, 98 were comfortable talking about their problems.
Using a confidence coefficient (or confidence level) of 95\%, Find a confidence interval that will allow you to
compare the two proportions of these two populations. 

Since we are testing two sample proportions we can use \verb|prop.test| for this.
        \begin{verbatim}
            > prop.test(prop.test(x=c(80,98),n=c(129,184),conf.level = .95,correct = F)
        \end{verbatim}
        Which gives our interval
        \[
            \left( -0.0229613,0.1980540 \right) 
        .\] 
        Which contains 0, which suggests with 95\% confidence that there is no difference in the true population proportions.
    \item 
\item We begin by checking whether $\sigma_1$ and $\sigma_2$ are significantly different. Running 
    \\\verb|var.test(mydata$DelaysAA,mydata$DelaysUA)| in R gives the interval:
    \[
        \left( 0.4946789,1.7401896 \right) 
    .\] 
    And since our interval contains $1$, we can assume the variances to be equal.
    
    Now we can run 
    \verb|t.test(mydata$DelaysAA,mydata$DelaysUA,var.equal=TRUE)|
    , which gives us the interval:
    \[
        (-14.86492, 14.52444)
    .\] 
    Which contains $0$, so we cannot say that the sample means are significantly 
    different.
    \item 

\end{enumerate}
\end{document}
