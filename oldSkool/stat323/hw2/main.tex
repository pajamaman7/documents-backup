\documentclass{article}
\usepackage{amsmath}
\usepackage{amssymb}
\usepackage{amsthm}
\usepackage[utf8]{inputenc}
\usepackage{amsmath}
\usepackage{amsfonts}
\usepackage[]{graphicx}
\usepackage[a4paper, portrait, margin = 1in]{geometry}
\usepackage{enumitem}
\usepackage{xcolor}

%darkmode
%\pagecolor[rgb]{0.2,0.19,0.18} 
%\color[rgb]{0.92,0.86,0.7}

\newenvironment*{alphenum}{\begin{enumerate}[label= (\alph*)]}{\end{enumerate}}


\begin{document}
    \huge Title - Thomas Boyko - 30191728
    \normalsize
\begin{enumerate} 
\item  Let $Y1, Y2, \ldots , Yn$ denote a random sample of size $N$ from a population whose density is given by
$$f_Y (y) =\begin{cases}
    3\beta^3y^{-4}& \beta\leq y, \,\text{where $\beta>0$ is unknown}\\
    0&\text{otherwise}
\end{cases}$$

   \begin{enumerate}[label= (\alph*)]
    \item Derive the bias of the estimator $Y_{(1)}=\hat{\beta}$.
    
    Before we do anything, we find the cdf for $Y$:
        \begin{align*}
            F_Y(y)&=\int_{\beta}^{y} 3\beta^3y^{-4} \, d y \\
            &= -\beta^3y^{-3}\Big|_{\beta}^{y} \\
            &= -\left(\frac{\beta}{y}\right)^3-(-\beta^3\beta^{-3}) \\
            &= -\left(\frac{\beta}{y}\right)^3 +1
        \end{align*}

        So $F_Y(y)=-\beta^3y^{-3}+1$.

        Now we can calculate $f_{Y_{(1)}}(y)$.

        \begin{align*}
            f_{Y_{(1)}}(y)&= n\left[ 1-F_Y(y) \right]^{n-1}f_Y(y) \\
            &= n\left[1-\left( -\left(\frac{\beta}{y}\right)^{3}+1 \right) \right] ^{n-1}\frac{3\beta^3}{y^{4}} \\
            &= n\left[\left(\frac{\beta}{y}\right)^{3} \right] ^{n-1}\frac{3\beta^3}{y^{4}} \\
            &= n\left( \frac{\beta}{y} \right) ^{3n-3}\frac{3\beta^3}{y^{4}} \\
            &=\frac{3n}{y} \frac{\beta^{3n}}{y^{3n}}
        .\end{align*}
        With support $(\beta,\infty)$.

        We need to find $E[Y_{(1)}]$.
        \begin{align*}
            E[Y_{(1)}]&=\int_{\beta}^{\infty} y\frac{3n\beta^{3n}}{y^{3n+1}} \, d y \\
            &=\int_{\beta}^{\infty} 3n\beta^{3n}y^{-3n} \, d y \\
            &= \frac{3n}{1-3n}y^{-3n+1}\beta^{3n} \Big|_{\beta}^{\infty}\\
            &= \frac{3n}{3n-1}\beta
        .\end{align*}
        Now we find $B(\hat{\beta})$.
            \begin{align*}
                B(\hat{\beta})&= E[\hat{\beta}]-{\beta}\\
                &=  \frac{3n}{3n-1}\beta- \beta\\
                &=\frac{\beta}{3n-1}
            .\end{align*}
        \item Derive $MSE(\hat{\beta})$.

            We begin by finding $E[\hat{\beta}^2]$ which we will need to find $V[\hat{\beta}]$.
        \begin{align*}
            E[\hat{\beta}^2]&=\int_{\beta}^{\infty} y^2\frac{3n\beta^{3n}}{y^{3n+1}} \, d y \\
            &=\int_{\beta}^{\infty} 3n\beta^{3n}y^{-3n+1} \, d y \\
            &= \frac{3n}{2-3n}\beta^{3n}y^{2-3n}\big|_{\beta}^{\infty} \\
            &= 0-\frac{3n}{2-3n}\beta^{3n}\beta^{2-3n} \\
            &= \frac{3n}{3n-2}\beta^2. 
        \end{align*}

        And now we can find $MSE[\hat{\beta}]$.
        \begin{align*}
            MSE(\hat{\beta})&= B[\hat{\beta}]^2+V[\hat{\beta}] \\
            &= \left( \frac{\beta}{3n-1} \right) ^2+\left( \frac{3n\beta^2}{3n-2} \right)-\left( \frac{3n\beta}{3n-1} \right)^2 \\
            &= \frac{\beta^2-9n^2\beta^2}{(3n-1)^2}+\frac{3n\beta^2}{3n-2} \\
            &= \beta^2\left( \frac{1-9n^2}{(3n-1)^2}+\frac{3n}{3n-2} \right)  \\
            &= \beta^2\left( \frac{(3n+1)(3n-1)}{(3n-1)^2} +\frac{3n}{3n-2}\right)  \\
            &= \beta^2\left( \frac{3n+1}{3n-1}+\frac{3n}{3n-2} \right) 
        .\end{align*}
\end{enumerate}

 \item Q2: TODO 
     \begin{enumerate}[label= (\alph*)] 
         \item 

             We begin by finding the CDF for $X$:
             $$F_{X}(x) = \int_{0}^{\theta} \frac{\alpha x^{(\alpha - 1)}}{\theta^{\alpha}}dx = \frac{x^{\alpha}}{\theta^{\alpha}}$$

             Then we use our typical order statistic formula.
    $$f_{X_{(n)}}(x) = n\left(\frac{x^{(\alpha n - \alpha)}}{\theta^{(\alpha n - \alpha)}}\right)\left(\frac{\alpha x^{(\alpha-1)}}{\theta^{\alpha}}\right) = \frac{n\alpha x^{(\alpha n-1)}}{\theta^{\alpha n}} I(0 < x < \theta)I(\alpha > 0)$$

\item 
    Bias of $X_{(n)}$ is given by:
    $$B(X_{(n)}) = \int_{0}^{\theta} \frac{n \alpha x^{(\alpha n)}}{\theta^{\alpha n}}dx - \theta = \frac{\alpha n \theta}{\alpha n+1} - \theta = \frac{-\theta}{\alpha n+1}$$


 Now to find an unbiased estimator for $\theta$: Let $\frac{\alpha n \theta_{ub}}{\alpha n+1} = X_{(n)}$. 
    Then solving for $\theta_{ub}$ we get:

    $$\theta_{ub} = \frac{X_{(n)} (\alpha n+1)}{\alpha n}.$$
\item
        $$MSE(\theta_{ub}) = E(\theta_{ub}^2) - (E(\theta_{ub}))^2$$
Where: 
$$E(\theta_{ub}^2) = (\frac{\alpha*n+1}{\alpha*n})^2*\int_{0}^{\theta} \frac{x^2*n*\alpha*x^{(\alpha*n-1)}}{\theta^{(\alpha*n)}}dx = \frac{\theta^2(\alpha*n+1)^2}{\alpha*n(\alpha*n+2)}$$

So we have 
$$MSE(\theta_{ub}) = \frac{\theta^2(\alpha*n+1)^2-\theta^2\alpha*n(\alpha*n+2)}{\alpha*n(\alpha*n+2)} + \frac{\theta^2}{(\alpha*n+1)^2}$$
     \end{enumerate}
\item When estimating a population proportion, we can use $\hat{p}_1=\frac{x}{n}$ or 
$\hat{p}_2 = \frac{x+2}{n+4}$. The second version is
called the Agresti sample proportion. It does not need to be shown, but $\hat{p}_1$ is an unbiased estimator
and $\hat{p}_2$ is a biased estimator. Compare the MSE’s in the distributions of these two estimators. For a
given sample sizes $n = 1, 5, 10, 16, 100$, which estimator would you suggest ‘best’ estimates the value
of the population proportion, $p$? Ensure you explain you answer.

Since $\hat{p}_1$ is unbiased, its MSE is simply equal to the variance $Var(\hat{p}_1)$.
\begin{align*}
    MSE(\hat{p}_1)=Var(\hat{p}_1)&= Var\left( \frac{x}{n} \right)  \\
            &= \frac{Var(x)}{n^2} \\
            &= \frac{p(1-p)n}{n^2} \\
            &= \frac{p(1-p)}{n} 
.\end{align*}
Since $\hat{p}_2$ is a biased estimator, we must find its bias before calculating MSE.
\begin{align*}
    B(\hat{p}_2)&= E[\hat{p}]-p \\
                &= \frac{E[x+2]}{n+4} -p\\
                &= \frac{E[x]+2}{n+4} -p\\
                &= \frac{np+2}{n+4} -p\\
                &= \frac{np+2}{n+4}-\frac{p(n+4)}{n+4} \\
                &= \frac{2-4p}{n+4} 
.\end{align*}
And now we can find $MSE$
\begin{align*}
    MSE(\hat{p}_2)&= B(\hat{p}_2)^2+V(\hat{p}_2) \\
                &= \left(\frac{-4p+2}{n+4}\right)^2 +V\left( \frac{x+2}{n+4} \right) \\
                &= \left(\frac{-4p+2}{n+4}\right)^2 +V\left( x\right)\frac{1}{(n+4)^2} \\
                &= \left(\frac{(2-4p)^2}{(n+4)^2}\right) +\frac{np(1-p)}{(n+4)^2}\\
                &= \frac{(2-4p)^2+np(1-p)}{(n+4)^2}
.\end{align*}

Since $\hat{p}_1$ is unbiased we can say it generally estimates the value of $p$ better. However
the second estimator $\hat{p}_1$ has consistently less variance than $\hat{p}_1$.

\item

\item  Let $X_1, X_2, . . . , X_n$ be a random sample from a population of values that has an unknown distribution, in addition to an unknown mean $\mu$ and an unknown variance $\sigma^2$. Find an unbiased estimator for the variance of $\bar{X}$.

    An unbiased estimator for the variance of $\bar{X}$ is $\frac{S^2}{n}$. Check:
    $$B(\frac{S^2}{n})=E\left[\frac{S^2}{n}\right]-V\left( \bar{X} \right)= \frac{E[S^2]}{n}-\frac{1}{n^2}V\left( \sum_{i=1}^{n} X_i \right) =\frac{\sigma^2}{n}-\frac{1}{n^2}n\sigma^2=0$$

\item 

\item The .txt in this folder on D2L contains data that resulted from a random sample of $n = 70$
    professional hockey players who have contracts with NHL teams. The values are their salary in millions of dollars. Import
or Copy and paste the data into R, and answer the following questions.
\begin{enumerate}[label= (\alph*)] 
\item Find the sample mean and sample standard deviation, $\bar{X}$ and $S$.

    We begin by inputting the data into a vector x in R, and use:
    \begin{verbatim}
        > mean=mean(x)
        > mean
        [1] 2.141214
        > sd=sd(x)
        > sd
        [1] 1.911294
    \end{verbatim}

    So our sample mean is $\bar{X}= 2.131214$ and our sample standard deviation is $S=1.911294$.

\item Find a 95\% confidence interval for $\mu$, the mean salary of an NHL player for the season 2013-2014.

    For this we can again use \verb t.test(x,conf.level=.95) \\ in R, which gives us the interval 
    $(1.685482,2.596946)$.

Interpret the meaning of this interval in the context of the data.

We can say with 95\% confidence that the true mean salary during this season is between $1.685$ million and $2.597$ million dollars.

\item Find a 95\% confidence interval for $\sigma$, the standard deviation of the distribution of NHL player salaries
for the 2013-2014 season. 

We use DescTools' \verb{VarCI(x,conf.level=.95)){ which gives us the interval for variance:
        $$(2.685591 ,5.259562)$$
        And taking the square root of both yields the 95\% interval for $\sigma$:
        $$(1.638777,2.293373)$$

        So we can say with 95\% confindence that the true standard deviation of NHL player salaries
        is between $1.639$ million and $2.293$ million.

\end{enumerate}

\item A recent Angus Reid survey of $n = 1504$ Canadians was taken between November 25 to 28th, 2014. One of
the questions posed in the survey was the following:
Do you approve or disapprove of proposals to change the criminal code of Canada to allow physicians to
assist with the suicide of their patients by prescribing lethal drugs? Here are the findings: Strongly Approve
= 556, Moderately Approve = 632, Moderately/Strongly Disapprove = 271, Don’t Know/No Opinion = 45
(as reported by Angus Reid.)
Find a 95\% confidence interval for the proportion of all Canadians who ‘approve’ of proposals to change the
criminal code of Canada to allow physicians to assist with the suicide of their patients by prescribing lethal
drugs. Interpret the meaning of this interval in the context of the data.

Begin by writing our $\hat{p}=\frac{632+556}{1504}=0.7898936$.

Then we can build our 95\% CI with the formula:

\begin{align*}
\hat{p}\pm Z_{\frac{\alpha}{2}}\sqrt{ \frac{\hat{p}(1-\hat{p})}{n}}
.\end{align*}

And we can do this in R:

\begin{verbatim}
> phat=(632+556)/n
> c(phat+qnorm(.025)*sqrt(phat*(1-phat)/1054),
phat-qnorm(.025)*sqrt(phat*(1-phat)/1054))
[1] 0.7652995 0.8144878
\end{verbatim}

This gives the interval for the true value of $p$:
$$(0.7652995, 0.8144878)$$
So we can say with 95\% confidence that the true proportion of all Canadians who 'approve' of proposals to change the criminal code of Canada to allow physicians to assist with the suicide of their patients by prescribing lethal drugs is between $76.53\%$ and $81.45\%$.

\item What is the normal body temperature for healthy humans? A random sample of 130 healthy human body
temperatures provided by Allen Shoemaker yielded 98.25 degrees and standard deviation 0.73 degrees.
\begin{enumerate}[label= (\alph*)] 
\item Give a 99\% confidence interval for the average body temperature of healthy people.

We can find this interval using 
\begin{verbatim}
> zsum.test(98.25,sigma.x =.73,n.x = 130,conf.level = 0.99 ) 
\end{verbatim}
Which gives us a 99 percent confidence interval: $98.08508, 98.41492$

\item Does the confidence interval obtained in part (a) contain the value 98.6 degrees, the accepted average temperature cited by physicians and others? What conclusions can you draw? 

    The interval contains the accepted average, so we can say that there is no evidence to suggest that the true mean is not 98.6 degrees.
\end{enumerate}

\item Use the data:
$$16, 5, 21, 19, 10, 5, 8, 2, 7, 2, 4, 9$$
to construct a 98\% confidence interval estimate for the mean LC50 for DDT. In addition, ensure you (i) state
any condition(s) that are necessary for your CI to be valid and (ii) interpret the meaning of your confidence
interval, in the context of the data. 

Since the size of our sample is quite small, we must suppose it is normally distributed, and since we do not know the true variance, we use \verb|t.test()|

\begin{verbatim}
> x=c(16, 5, 21, 19, 10, 5, 8, 2, 7, 2, 4, 9)
> t.test(x,conf.level=.98)
\end{verbatim}

Which gives our 98\% CI:
$$\left( 3.959158,14.040842 \right) $$
So we can say with 98\% confidence that the true mean LC50 for DDT is between $3.96$ and $14.04$.

\item In a recent survey of $n = 1005$ Canadians between the ages of 18 and 34, the polling company Ipsos found that 723 indicated they “owe it to their parents to keep them comfortable in their retirement.”
    \begin{enumerate}[label= (\alph*)] 
        \item  Find a 95\% confidence interval for the proportion of all Canadians 18 to 34
            years of age who hold the same sentiment. 

            We begin with $\hat{p}=\frac{723}{1005}=0.719403$, and $\alpha=.05$
            We use our formula for CIs of qualitative data:
            \[
               \hat{p}\pm Z_{\frac{\alpha}{2}}\sqrt{ \frac{\hat{p}(1-\hat{p})}{n}}
            .\] 
            In R:
            \begin{verbatim}
                > phat=723/1005
                > phat+qnorm(.025)*sqrt(phat*(1-phat)/1005)
                [1] 0.6916255
                > phat-qnorm(.025)*sqrt(phat*(1-phat)/1005)
                [1] 0.7471805
            \end{verbatim}
            So we can say with 95\% confidence that the true proportion of all Canadians 18 to 34 who
            share this sentiment is between $69.162\%$ and $74.718\%$.

        \item  Polls often come with a ‘margin of error’. State what the margin of error is for this
            poll, and make a statement explaining what this margin of error means. 

            The population proportion's margin of error is the second piece of our R code:
            \begin{verbatim}
                > -qnorm(.025)*sqrt(phat*(1-phat)/1005)
                [1] 0.02777747
            \end{verbatim}
            Which tells us that our margin of error is about $2.8\%$, or the radius around our 
            $\hat{p}$ which we can expect the true population proportion $p$ to lie.
    \end{enumerate}

\item The profit of a new car sold by automobile dealer (in thousands of dollars) was recorded for 6 sales.
$$2.1, 3.0, 1.2, 6.2, 4.5, 5.1$$
For this question we assume that profit per car is normally distributed with mean $\mu$ and standard deviation $\sigma$.

\begin{enumerate}[label= (\alph*)] 
\item find a 95\% confidence interval estimate for $\mu$, the mean profit of a new car sold by automobile dealer.

    So we use the R code:
    \begin{verbatim}
    > x=c(2.1, 3.0, 1.2, 6.2, 4.5, 5.1)
    > xbar=mean(x)
    > n=length(x)
    > tsum.test(xbar,sd(x),n,conf.level=.95)
    \end{verbatim}
    Which gives the interval for our mean:
    $$ 1.683982,5.682685$$
\item a 95\% confidence interval for $\sigma$, the standard deviation of the profit of a new car sold by an automobile dealer

    We can simply use the R code: 

    \begin{verbatim}
        > VarCI(x,conf.level=.95)
    \end{verbatim}
    Which gives our interval for variance:
    $$1.414247,21.833590$$
    Which we convert to standard deviation:
    $$1.189221,4.672643$$

\end{enumerate}

\item The length of time between billing and receipt of payment was recorded for a sample of 100 of a certified public accountant (CPA) firm’s clients. The sample mean and standard deviation for the 100 accounts were 39.1 days and 17.3 days, respectively. Find a 99\% confidence interval for the mean time between billing and receipt of payment for all of the CPA firm’s accounts. State any conditions/assumptions required, and interpret the meaning of your interval in the context of the data.

    Since we have summary data and we are looking for the mean without knowing the true standard deviation, we can use \verb|tsum.test(39.1,17.3,100,conf.level=.99)|. This gives us the confidence interval:
    $$\left(34.55632,43.64368 \right) .$$
    So we can say with 99\% confidence that the true mean time between billing and reciept of payment is between 34.55 and 43.64 days.

\item $X_1, X_2, \ldots , X_n$ represent a random sample of Student ID numbers from University of
Calgary students. Assume $Xi \sim Uniform(0, N )$ where $N$ is the total number of U of C students.
Use the pivotal quantity $\frac{X_{(n)}}{N}$ to find a 95\% lower confidence bound for $N$ . That is, use a sample of student ID’s to estimate, with 95\% confidence, that the largest Student ID will be at least how large? 

\end{enumerate}
\end{document}
