\documentclass{article}
\usepackage{amsmath}
\usepackage{amssymb}
\usepackage{amsthm}
\usepackage[utf8]{inputenc}
\usepackage{amsmath}
\usepackage{amsfonts}
\usepackage[]{graphicx}
\usepackage[a4paper, portrait, margin = 1in]{geometry}
\usepackage{enumitem}
\usepackage{xcolor}

%darkmode
%\pagecolor[rgb]{0.2,0.19,0.18} 
%\color[rgb]{0.92,0.86,0.7}

\newenvironment*{alphenum}{\begin{enumerate}[label= (\alph*)]}{\end{enumerate}}

\pagestyle{fancy}
\lhead{Assignment \# $2$}
\rhead{Name: Thomas Boyko; UCID: 30191728}
\chead{}

\begin{document}

\begin{enumerate}
    
\item Let $S$ be a subset (not necessarily a subspace) of a finite dimensional inner product space $V$. Show that $(S^\perp)^\perp = \text{span } S$, where
$$\text{span } S := \left\{\sum_{j=1}^m \alpha_j s_j : \alpha_j \in \mathbb{C}, s_j \in S, m \in \mathbb{N}\right\}$$
is the smallest subspace of $V$ containing $S$ (think of this as the set of all possible linear combinations of vectors from $S$).
\paragraph{Lemma:} To prove the following fact we use the following:

For any subset $S$ of a finite dimensional inner product space $V$, $(\text{span }S)^{\perp}=S^{\perp}$.
\begin{proof} 
    

{$\subseteq $:}  Let $v\in (\text{span }S )^{\perp}$. Since $v$ is perpendicular to any element of $\text{span }S$, it must be perpendicular to any element of $S$ (since $S\subseteq \text{span }S$).

{$\supseteq$:} Let $v\in S^{\perp}$, and let $w=a_1s_1+\dots+a_ks_k$ for some spanning set $\{s_1,\dots,s_k\} \subseteq S$. Then take:
\[ \langle w,v \rangle = \langle \sum_{j=1}^{k} a_js_j,v \rangle=\sum_{j=1}^{k} a_j\langle s_j,v \rangle=\sum_{j=1}^{k} a_j0=0 .\] 
So $v\perp w$ and therefore $v\in \text{span }S$.
\end{proof}
\paragraph{Solution: }

Let $B=\{b_1,\dots,b_n\} $ be an orthonormal basis for $\text{span } S$.

\paragraph{$\subseteq $:} Let $v\in \text{span }S$. Then write $v=a_1b_1+\dots+a_nb_n$. Let $w\in S^{\perp}$ be given, and note that $w\perp b_j$ for any basis element in $S$. Take:
\[ \left<v,w \right> =  \left< a_1b_1+\dots+a_nb_n,w\right> = a_1\left< b_1,w\right>+\dots+a_n\left<b_n ,w\right> =  0+\dots+0=0 .\] 
And we see that $v\perp w$, so $v$ is perpendicular to any element of $S^{\perp}$, and $v\in {S^{\perp}}^{\perp}$.
\paragraph{$\supseteq $:} Recall the identity $V=(\text{span } S)^{\perp}\oplus  \text{span }S$. Then for any $v\in(S^{\perp})^{\perp}$, $v=x+y$ for some  $x\in (\text{span }S)^{\perp}=S^{\perp}$ and $y\in \text{span } S$.

Now take the inner product $\langle v,x \rangle=0$ since $v\perp x$. Expanding,
\[
0=\langle v,x \rangle=\langle x+y,x \rangle=\langle x,x \rangle+\langle y,x \rangle=\langle x,x \rangle+0=\|x\|^2
.\] 
Then since $\|x\|=0$ we must have $x=0$, meaning $v=y$ and $v\in \text{span }S$.

\item Let $V$ and $W$ be finite dimensional inner product spaces and suppose $\ker A = \{0\}$. Find a left inverse for $A$ in terms of $A$ and $A^*$.

\paragraph{Solution: } Begin with the identity,
\[ \{0\} =\ker A =\ker A^* A .\] 
So the composition of transformations $A^* A:V\to V$ has zero kernel and is injective, and by rank-nullity it must too surjective. Then this map is invertible, and if we take $(A^* A)^{-1}A^* A=I$, we see that $(A^* A)^{-1}A^* $ is a left inverse for $A$.

\newpage
\item Let $V$ be a finite dimensional inner product space.
\begin{enumerate}[label= (\alph*)] 
\item We can think of any $x \in V$ as a linear map from $\mathbb{C} \to V$ by setting $x(\lambda) := \lambda x$. You do not have to prove that this is linear. Show that $x^*: V \to \mathbb{C}$ satisfies
$$x^* y = \langle y, x\rangle.$$
Use this to deduce that the map $xy^*$ is given by $xy^* v = \langle v, y\rangle x$.
HINT: The inner product on $\mathbb{C}$ is assumed to be $\langle z, w\rangle = z\overline{w}$.
\item Show that if $T: V \to \mathbb{C}$ is any linear map, then there is a vector $y$ so that $T = y^*$.
\end{enumerate}
\paragraph{Solution: } 
\begin{enumerate}[label= (\alph*)] 
    \item Recall from the definition of an adjoint operator, that the adjoint $x^* :V\to \mathbb{C}$ is given by:
        \begin{align*}
            \left< x(\lambda),y \right>_V &=\left<\lambda,x^* (y) \right>_{\mathbb{C}}\\
            \left<\lambda x,y \right>_V&= \lambda \overline{x^* (y)} \\
            \lambda\left< x,y \right>_V&= \lambda \overline{x^* (y)} \\
            \overline{\lambda\left< x,y \right>_V}&= \overline{\lambda \overline{x^* (y)}} \\
            \overline{\lambda}\left< y,x \right>_V&= \overline{\lambda} {x^* (y)} \\
            \langle y, x\rangle&= x^* y
        .\end{align*}
        Then for the map $xy^* :V\to V$, 
        \[
            x(y^* (v))=x(\left<v,y \right>_V)=\left<v,y \right> x
        .\] 
    \item Choose $y=T^* (1).$ Then, for any $v\in V$, 
        \begin{align*}
            y^* (v)=\left<v,y \right>_V=\left<v,T^* (1) \right>_V = \left<Tv,1 \right>_{\mathbb{C}}=Tv
        .\end{align*}
        And $y^* =T$.
\end{enumerate}

\item Let $V$ and $W$ be finite dimensional vector spaces. You may find problem 3 useful here. \begin{enumerate}[label= (\alph*)] 
    \item Suppose $T: V \to W$ satisfies $\text{rank } T = 1$. Show that there are vectors $x \in W$ and $y \in V$ so that $T = xy^*$.
    \item Suppose $T: V \to W$ satisfies $\text{rank } T = k$. Show that $T$ is the sum of $k$ rank one operators. Hint: $PT = T$ where $P$ is the orthogonal projection onto $\text{ran } T$.
\end{enumerate}
\paragraph{Solution: }
\begin{enumerate}[label= (\alph*)] 
    \item Since the dimension of the image of $T$ has dimension 1, we must have $\text{ran }T=\text{span }\{b\} $ for some $b\in W$. Let $v\in V$, then $Tv=\alpha b$ for some $\alpha\in \mathbb{C}$. Choose $x=b$, and $y^*(v)=\alpha$. Now we have 
        \[
        xy^* (v)=x(y^* (v))=x(\alpha)=\alpha x=\alpha b= Tv
        .\] 

    \item Let $T$ be linear from $V$ to $W$ of rank $k$. Then let $\{b_1,\dots,b_k\} $ be an orthogonal basis for $\text{ran }T$. Then for $1\leq j\leq k$ and some $v\in V$, define $T_jv=P_{b_j}(Tv)$. Now:
        \[
            \sum_{j=1}^{k} T_k v=\sum_{j=1}^{k} P_{b_j}(Tv)=\sum_{j=1}^{k} \frac{\left<Tv,b_j \right>}{\|b_j\|}b_j=Tv
        .\] 
        The final equality is simply the projection of $Tv$ onto its range, which is $Tv$ itself.
\end{enumerate}
\newpage

\item Suppose that $A$ and $B$ are unitarily equivalent $n \times n$ matrices. That is, there is a unitary matrix $U$ so that $U^*AU = B$. Show that $E$ is an invariant subspace for $B$ if and only if $UE$ is invariant for $A$.
Recall that a subspace $E$ of $V$ is invariant for $T$ if $Tv \in E$ for all $v \in E$.
\paragraph{Solution: }Suppose that $ U^* AU=B$, and recall that from unitary equivalence, we can see that $AU=UB$;
\begin{align*}
    E \text{ is invariant under }B&\iff Bv\in E\quad\forall v\in E\\
                        &\iff Bv=w\in E\\
                         &\iff UBv=Uw\\
                         &\iff AUv=Uw\\
                         &\iff AUv\in UE\quad \forall Uv\in UE\\
                         &\iff  UE \text{ is invariant under } A
.\end{align*}
\end{enumerate}
\end{document}
