\documentclass{article}
\usepackage[most,many,breakable]{tcolorbox}
\usepackage{amsmath}
\usepackage{amssymb}
\usepackage{amsthm}
\usepackage[]{thmbox}
\usepackage{blindtext}
\usepackage[utf8]{inputenc}
\usepackage{amsmath}
\usepackage{amsfonts}
\usepackage[]{graphicx}
\usepackage[legalpaper, portrait, margin = 1in]{geometry}
\usepackage{enumitem}


\usepackage{xcolor}

%\pagecolor[rgb]{0.2,0.19,0.18} 
%\color[rgb]{0.92,0.86,0.7}

\newtheorem[L]{le}{Lemma}[subsection]
\newtheorem[L]{th}[le]{Theorem}
\newtheorem[L]{df}[le]{Definition}
\newtheorem[L]{ex}[le]{Example}
\newtheorem[L]{pf}[le]{Proof}


\newcommand{\nl}{\newline}

\newcommand{\real}{\mathbb{R}}
\newcommand{\complex}{\mathbb{C}}
\newcommand{\integer}{\mathbb{Z}}
\newcommand{\rational}{\mathbb{Q}}
\newcommand{\lxor}{\oplus}
\newcommand{\then}{\Rightarrow}

\begin{document}
    \huge Homework 3 - Thomas Boyko - 30191728
    \normalsize
\begin{enumerate} 
\item Compute $\oint_C \vec{F} d\vec{r}$ where $C$ is the ellipse $4x^2 +y^2 = 4$ (with counterclockwise
    motion) and 

    $\vec{F} (x, y) = \begin{pmatrix} \frac{3x^2}{2\sqrt{x^3+y^2} }+y^2, 
    && \frac{y}{\sqrt{ x^3+y^2}}+x^2 \end{pmatrix}^{T}$.

    Trying to find an "almost potential" for $\vec{F}$, we begin by integrating $F_1,\, F_2$:

    \[
    \int \frac{3x^2}{2\sqrt{x^3+y^2} }+y^2 \, dx = \sqrt{x^3+y^2} +xy^2+f_1(y)
    .\] 
    \[
    \int \frac{y}{\sqrt{x^3+y^2} }+x^2 \, dy = \sqrt{x^3+y^2} +x^2y+f_2(x) 
    .\] 
    And we can see that there is no $f_1,f_2$ which satisfy both equations.

    So we will take $\Phi=\sqrt{x^3+y^2} $ and $\vec G=(xy^2, x^2y)^{T}$, so that 
    $\nabla \Phi+\vec{G}=\vec{F}$. Now $\oint_C\vec{F}\,d\vec{r}=\oint_C\vec{G}\,d\vec{r}$.

    Can we find a potential $\Psi$ for $\vec{G}$? 
    \begin{align*}
        \Psi&= \int xy^2\,dx =\frac{x^2y^2}{2}+f_3(y)\\
        \Psi&= \int x^2y\,dy =\frac{x^2y^2}{2}+f_4(x)
    .\end{align*}
    And we can quickly see that if we take $\Phi=\frac{x^2y^2}{2}$ then $\nabla \Psi=\vec{G}$.

    So we have a potential for $\vec{G}$, meaning that $\vec{G}$ is conservative, and since our
    integral is over a closed curve, $\oint \vec{F} d\vec{r}=\oint \vec{G} d \vec{r}=0$.

\item Compute the surface integral $\iint_S (z+3)dS$ where $S$ is the part of the
paraboloid $z = 2x^2 + 2y^2 - 3$ that lies below the plane $z = 1$.

    We begin by parameterizing $S$. 
    Since $S$ is a function-type, we can write 
    $\vec r(x,y)=( x,y,2x^2+2y^2-3 )^{T}$.

    As for our bounds on $x,y$, we can just say that $(x,y)\in D$ where $D$ is the circle of radius
    $2$ centered at the origin.

    Now we find $\|\vec{n}\|$.
    \[
        \vec{n}=\begin{pmatrix} 4x \\ 4y \\ -1 \end{pmatrix} 
    \] 
    \[
        \|\vec{n}\|=\sqrt{16x^2+16y^2+1} 
    \] 
    We can write our integral now as:
    \[
    2\iint_D (x^2+y^2)\sqrt{16x^2+16y^2+1} \, dA
    .\] 
    Converting to cylindrical coordinates:
    \[
    2\int_0^2\int_0^{2\pi} r^3\sqrt{16r^2+1}  \, d\theta\,dr=4\pi \int_{0}^{2} r^3\sqrt{16r^2+1} dr 
    .\] 
    Let $u=16r^2+1$, so $dr=\frac{du}{32r}$, and $r^2=\frac{u-1}{16}$.
    Our endpoints now go from $u(0)=1$ to $u(2)=65$.
    \[
    \frac{\pi}{128}\int_{1}^{65}  (u-1)\sqrt{u} \,du=\frac{\pi}{128}\left( \frac{2}{5}u^{\frac{5}{2}}
    -\frac{2}{3}u^{\frac{3}{2}}\right) _{u=1}^{65}
    \] 
    \[
    =\frac{\pi}{128}\left(\frac{2}{5}65^\frac{5}{2}-\frac{2}{3}65^{\frac{3}{2}}+\frac{4}{15}\right)
    .\] 
    
\item Compute the surface integral $\iint_S y dS$ where $S$ is the part of the cylinder
$4x^2 + y^2 = 4$ bounded between the planes $z = -3$ and $z = 4y + x + 3$.

First we parameterize $S$ as:
\[
    \vec{r}(z,\theta)=\begin{pmatrix} \cos\theta&2\sin\theta&z \end{pmatrix} ^{T}
.\] 
Within a region $D$ given by $\theta\in [0,2\pi]$ and $z\in [-3,8\sin\theta+\cos\theta+7].$

Taking the cross product of the derivatives of $\vec{r}$ we get:
\[
    \vec{n}=\begin{pmatrix} 2\cos\theta & \sin\theta & 0 \end{pmatrix} ^{T}
.\] 
And so $\|\vec n\|=\sqrt{3\cos^2\theta+1} $, which gives us the integral
\begin{align*}
    \int_{0}^{2\pi}\int_{-3}^{8\sin\theta+\cos\theta+7}2\sin\theta\sqrt{3\cos^2\theta+1} dzd\theta&= 
    \int_{0}^{2\pi} (8\sin\theta+\cos\theta+10)\sqrt{\cos^2\theta+1} d\theta \\
    \approx 96.8845
.\end{align*}
I was not able to compute this final step but Wolfram was.

\item Compute the surface integral $\iint_S xz dS$ where $S$ is the part of the plane
$z = 4y + x + 3$ that lies inside the cylinder $4x^2 + y^2 = 4$.

First we parameterize $S$ with $\vec{r}(x,y)=(x,y,4y+x+3)^{T}$ with $0\le 4x^2+y^2\le 4$.
This gives our normal $\vec{n}=(1,4,-1)^{T}$, and $\|\vec{n}\|=\sqrt{18} =3\sqrt{2} $.

Our bounds on $x,y$ become $x\in [-1,1]$ and $y\in [-2\sqrt{1-x^2}  ,2\sqrt{1-x^2} ]$.

We transform our integral into:

\begin{align*}
    3\sqrt{2} \int_{-1}^1\int_{-2\sqrt{1-x^2} }^{2\sqrt{1-x^2} }4xy+x^2+3x\,dy\,dx&=
    3\sqrt{2} \int_{-1}^{1}4x^2\sqrt{1-x^2} +12x\sqrt{1-x^2} \,dx\\
    &= 3\sqrt{2} \int_{-1}^{1}(4x^2+12)\sqrt{x^2-1}
.\end{align*}
Wolfram helpfully evaluates this lengthy inverse trig substitution to give a final answer of $3\sqrt{2} \frac{\pi}{2}$
\end{enumerate}
\end{document}
