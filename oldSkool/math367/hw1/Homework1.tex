\documentclass{article}
\usepackage[most,many,breakable]{tcolorbox}
\usepackage{amsmath}
\usepackage{amssymb}
\usepackage{amsthm}
\usepackage[]{thmbox}
\usepackage{blindtext}
\usepackage[utf8]{inputenc}
\usepackage{amsmath}
\usepackage{amsfonts}
\usepackage[]{graphicx}
\usepackage[legalpaper, portrait, margin = 1in]{geometry}
\usepackage{enumitem}


\usepackage{xcolor}

%\pagecolor[rgb]{0.2,0.19,0.18} 
%\color[rgb]{0.92,0.86,0.7}

\newtheorem[L]{le}{Lemma}[subsection]
\newtheorem[L]{th}[le]{Theorem}
\newtheorem[L]{df}[le]{Definition}
\newtheorem[L]{ex}[le]{Example}
\newtheorem[L]{pf}[le]{Proof}


\newcommand{\nl}{\newline}

\newcommand{\real}{\mathbb{R}}
\newcommand{\complex}{\mathbb{C}}
\newcommand{\integer}{\mathbb{Z}}
\newcommand{\rational}{\mathbb{Q}}
\newcommand{\lxor}{\oplus}
\newcommand{\then}{\Rightarrow}

\begin{document}
    \huge Homework 1 - Thomas Boyko - 30191728

    \normalsize
\begin{enumerate} 

\item (5 points) Define $f : \mathbb{R}^2 \to \mathbb{R}^2$

$f(x,y)\begin{cases}0&(x,y)=(0,0)\\\frac{xy^2}{x^2+y^2}&(x,y)\neq(0,0)\end{cases}.$

\begin{enumerate}[label= (\alph*)] 
\item (1 point) Compute the partial derivatives for f at (0, 0).

Use the definition for directional derivatives:
\begin{align*} 
f_x(x,y)=D_{e_1}f(0,0)&=\lim_{t\to0}\frac{f(t,0)-f(0,0)}{t}\\
&=\lim_{t\to0}\frac{\frac{0}{t^2}-0}{t}\\
&=\lim_{t\to0}\frac{0}{t}=0
\end{align*}
And for $f_y$:
\begin{align*} 
f_y(x,y)=D_{e_2}f(0,0)&=\lim_{t\to0}\frac{f(0,t)-f(0,0)}{t}\\
&=\lim_{t\to0}\frac{\frac{0}{t^2}-0}{t}\\
&=0
\end{align*}

So we see that $f_x(x,y)=f_y(x,y)=0$.

Note for later that $Df(0,0)=\begin{bmatrix} 0&0 \end{bmatrix}$

\item (2 point) Determine which directional derivatives exist for $f$ at $(0, 0)$ and compute
their value.

Let $\vec v = \begin{bmatrix} v_1&v_2 \end{bmatrix}, \vec v \in \mathbb{R}^2$ with $||\vec v||=1 $. Note that $\sqrt{v_1^2+v_2^2}=1$ and $v_1^2+v_2^2=1$.

Then the directional derivatives at $(0,0)$ will exist when the following limit exists.

\begin{align*} 
D_{\vec v}f(x,y)&= \lim_{t\to0}\frac{f(v_1t,v_2t) - f(0,0)}{t}\\
&=\lim_{t\to0}\frac{\frac{v_1v_2^2t^3}{t^2(v_1^2+v_2^2)}-0}{t}\\
&=\lim_{t\to0}\frac{v_1v_2^2t}{t}\\
&=v_1v_2^2
\end{align*}

So the directional derivative exists for any value of $\vec v$.

\item (2 points) Is f differentiable at (0, 0)? Why or why not?

In order for $f$ to be differentiable at $(0,0)$, the following limit must be $0$, at any path to the origin.

\begin{align*}
\lim_{(x,y)\to(0,0)}\frac{f(x,y)-f(0,0)-\begin{bmatrix} 0&0 \end{bmatrix}\begin{bmatrix} x\\y\end{bmatrix}}{\sqrt{x^2-y^2}}&=\lim_{(x,y)\to(0,0)}\frac{\frac{xy^2}{x^2+y^2}}{(x^2+y^2)^{\frac{1}{2}}}\\
&=\lim_{(x,y)\to(0,0)}\frac{xy^2}{(x^2+y^2)^{\frac{3}{2}}}\\
\end{align*}
Choose the path to the origin where $x=y$ and $x\to 0^{+}$.

\begin{align*} \lim_{x\to0^+}\frac{x^3}{(2x^2)^{\frac{3}{2}}}&=\lim_{x\to0^+}\frac{x^3}{2x^3}\\
&=\frac{1}{2}
\end{align*}

Since this limit does not equal $0$ for our curve near the origin, the function $f$ is not differentiable at $(0,0)$.

\end{enumerate}
\item  (4 points) Define $f (x, y, z) = e^{x^2+yz}$ .
\begin{enumerate}[label= (\alph*)]  
\item (1 point) Compute the gradient for $f$ at the point $(1, 1, -1)$.

$$\nabla f(x,y,z)=\begin{pmatrix}f_x\\f_y\\f_z\end{pmatrix}=\begin{pmatrix}2xe^{x^2+yz}\\ze^{x^2+yz}\\ye^{x^2+yz}\end{pmatrix}$$

$$\nabla f(1,1,-1)=\begin{pmatrix}2e^{0}\\-e^{0}\\1e^{0}\end{pmatrix}=\begin{pmatrix}2\\-1\\1\end{pmatrix}$$

\item (1 point) Determine the maximal rate of ascent for $f$ at $(1, 1, -1)$. In which
direction does this occur?

The maximal rate of ascent will be given by $||\nabla f(1,1,-1)||$.

$$||\nabla f(1,1,-1)||=\sqrt{2^2+1^2+(-1)^2} = \sqrt{6}$$

The direction in which this occurs will be given by:

$$\frac{\nabla f}{||\nabla f||}=\frac{1}{\sqrt{6}}\begin{pmatrix}
    2\\-1\\1
\end{pmatrix}$$

\item (1 point) Compute the tangent plane for $f$ at $(1, 1, -1)$.
\begin{align*} 
\vec L(1,1,-1)&=f(1,1,-1)+\begin{pmatrix}2&-1&1\end{pmatrix}\begin{pmatrix}x-1\\y-1\\z+1\end{pmatrix}\\
&=1+(2x-2)+(1-y)+z+1\\
\vec L(1,1,-1)&=2x-y+z+1
\end{align*}
\item (1 point) Estimate the value $f (0.9, 1.1, -0.8)$ using part (c)

$$f (0.9, 1.1, -0.8)\approx 2(0.9)-1.1-0.8+1=0.9$$
\end{enumerate}

\item (6 points)Define the function $\vec f (x, y) = \begin{pmatrix}xy, x^2 + y^2\end{pmatrix}^T on \mathbb{R}^2$.
\begin{enumerate}[label= (\alph*)]  
\item (3 point) Determine on which set of points in $\mathbb{R}^2$ the hypotheses of the inverse
function theorem is satisfied.

Begin by finding $D\vec f(x,y)$.

$$D\vec f(x,y)=\begin{bmatrix}
    y&x\\
    2x&2y
\end{bmatrix}$$
This matrix is invertible whenever $\det(D\vec f(x,y))\neq0$.
$$\det(D\vec f(x,y))=2y^2-2x^2$$
So $2y^2-2x^2\neq0$.

So $\vec f$ has a local inverse wherever $y\neq |x|$

\item Apply the inverse function theorem to f at the point (2, 1) and use it
to find $D\vec g(2, 5)$ where $g$ is a local inverse for $f$ at the point $(2, 1)$.

From above, $D\vec f(2,1) =\begin{bmatrix} 
1&2\\
4&2
\end{bmatrix}$, and $\det(D\vec f(2,1))=2-8=-6$.

So $D\vec f^{-1}(2,1)=D\vec g(2,5)=-\frac{1}{6}\begin{bmatrix} 2&2\\-4&1 \end{bmatrix}$

\end{enumerate}
\item Consider the equations
\begin{align*} x^2 + y + 3z + u + v + 4 &= 0\\
xy^2 - z + u - v + 3 &= 0
\end{align*}
around the point $(x, y, z, u, v) = (1, 0 ,- 1, -3, 1)$.
Let $\vec F (x, y, z, u, v) = (x^2 + y + 3z + u + v + 4, xy^2 - z + u - v + 3)^T$ .
Use the implicit function theorem to show that there is a function $\vec g : U \to\mathbb{R}^2$ where
$U$ is an open set containing $(x, z, v) = (1, -1, 1)$, so that $g_1(x, z, v) = y, g_2(x, z, v) = u$
and $\vec F (x, g_1(x, z, v), z, g_2(x, z, v), v) = (3, 2)^T$ . Find $D\vec g(1, 2, 1)$


First we must find $D\vec F_{(1,0,-1,-3,1)}= \begin{pmatrix}2x&1&3&1&1\\y^2&2xy&-1&1&-1\end{pmatrix}_{(1,0,-1,-3,1)}$

Evaluating at our chosen point we get $D\vec F_{(1,0,-1,-3,1)}= \begin{pmatrix}2 &1&3&1&1\\0&0&-1&1&-1\end{pmatrix}$.

If an implicit function exists then the matrix $B=\begin{pmatrix}1&1\\0&1\end{pmatrix}$ will be invertible. Clearly this matrix is invertible since it is upper triangular and $\det B=1$ which is the product of the main diagonal. And the inverse of $B$ is the matrix $B^{-1}=\begin{pmatrix}1&-1\\0&1\end{pmatrix}$.

So now we can take the remaining columns of our large matrix and create the matrix $A=\begin{pmatrix}2&3&1\\0&-1&-1\end{pmatrix}$. Finally we can compute the product $$-B^{-1}A=-\begin{pmatrix}1&-1\\0&1\end{pmatrix}\begin{pmatrix}2&3&1\\0&-1&-1\end{pmatrix}=\begin{pmatrix}-2&-4&-2\\0&1&1\end{pmatrix}$$

And by the implicit function theorem, $D\vec g(1,2,1)=\begin{pmatrix}-2&-4&-2\\0&1&1\end{pmatrix}$.

\end{enumerate}

\end{document}
