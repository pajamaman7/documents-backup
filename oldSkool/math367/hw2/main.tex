\documentclass{article}
\usepackage{amsmath}
\usepackage{amssymb}
\usepackage{amsthm}
\usepackage[utf8]{inputenc}
\usepackage{amsmath}
\usepackage{amsfonts}
\usepackage[]{graphicx}
\usepackage[a4paper, portrait, margin = 1in]{geometry}
\usepackage{enumitem}
\usepackage{xcolor}

%darkmode
%\pagecolor[rgb]{0.2,0.19,0.18} 
%\color[rgb]{0.92,0.86,0.7}

\newenvironment*{alphenum}{\begin{enumerate}[label= (\alph*)]}{\end{enumerate}}

\usepackage{amsthm}
\usepackage{scsnowman}

\renewcommand{\qedsymbol}{\scsnowman[hat,snow]}
\newtheorem{theorem}{Theorem}

\begin{document}
    \huge Homework Set 2 - Thomas Boyko - 30191728
    \normalsize
\begin{enumerate} 
\item Define a parametric curve $\vec r(t) = (2t, \frac{t^2}{2}, 2 \ln t)^{T}$ where $t > 0$.
    \begin{enumerate}[label= (\alph*)] 
        \item Compute the arclength of the curve between the points $(2, 0.5, 0)$ and
        $(8, 8, 2 \ln 4)$.

        First we begin by calculating $\vec{v}(t)=D\vec{r}(t)=(2,t,\frac{2}{t})^{T}$.

        We can normalize this to find $v(t)=\sqrt{4+t^2+\frac{4}{t^2}}$, which we can simplify to 
        $\frac{t^2+2}{t}$ since $t>0$.

        Notice that $\vec{r}(1)=(2,0.5,0)$ and $\vec{r}(4)=(8,8,2\ln 4)^{T}$.
        So the arc length between the two points is given by:
        \[
        \int_{1}^{4} t+\frac{2}{t} dt=\left(\frac{t^2}{2}+2\ln t \right)_1^{4}=\frac{15}{2}+2\ln 4
        .\] 

        \item Compute the velocity, speed, acceleration, tangent vector, normal vector, binormal vector,
        curvature, torsion, and the tangential and normal components of acceleration for the curve at
        an arbitrary $t > 0$.

        With the functions of $t$ we already have.
        \[
        \vec{T}(t)=\frac{\vec{v}(t)}{v(t)}=\frac{t}{t^2+2}
        \begin{pmatrix} 2 \\t\\\frac{2}{t}\end{pmatrix} 
        .\] 
        Above we calculated $\vec{v}(t)$, and we will use this now to calculate 
        $\vec{a}(t)=(0,1,-\frac{2}{t^{2}})^{T}$.

        Now we find $\|\vec a(t)\|=\sqrt{1+\frac{4}{t^4}} = \frac{t^4+4}{t^2}$.

        Another necessary calculation will be $\vec{v}\times \vec{a}$.
        \[
            \vec{v}\times \vec{a}=
            \det\begin{bmatrix} 
                i&j&k\\
                2&t&\frac{2}{t}\\
                0&1&-\frac{2}{t^2}
            \end{bmatrix} 
            =\begin{pmatrix} 
                -\frac{4}{t}\\
                \frac{4}{t^2}\\
                2
            \end{pmatrix} 
        .\] 

        And normalizing this vector gives us:
        \[
        \|\vec{v}\times \vec{a}\|=\sqrt{\frac{16}{t^2}+\frac{16}{t^4}+4}
        =\frac{2(t^2+2)}{t^2}
        .\] 
        With these two quantities we can find:
        \[
            \vec{B}=\frac{\vec{v}\times \vec{a}}{\|\vec{v}\times \vec{a}\|}
            =
            \frac{1}{t^2+2}
            \begin{pmatrix} 
                -2t\\
                2\\
                t^2
            \end{pmatrix} 
        .\]     
        And we can use $\vec{B}\times \vec{T}=\vec{N}$.
        \[
            \vec{N}=\frac{t}{(t^2+2)^2}
            \det\begin{bmatrix} 
                i&j&k\\
                -2t&2&t^2\\
                2&t&\frac{2}{t}\\
            \end{bmatrix}
            =\frac{1}{t^2+2}\begin{pmatrix} 2-t^2\\2t\\-2t \end{pmatrix}  
        .\] 
        Now we find $\tau$ with $\tau= \frac{(\vec{v}\times \vec{a})\cdot \vec{a}'}{\|\vec{v}\times \vec{a}\|^{2}}$.
        \[
            \tau=\frac{t^4}{4(t^2+2)^2}\begin{pmatrix} -\frac{4}{t}\\\frac{4}{t^2}\\2 \end{pmatrix} 
            \cdot \begin{pmatrix} 0\\0\\ \frac{4}{t^3} \end{pmatrix} 
            =\frac{2t}{(t^2+2)^2}
        .\] 
        Using $\kappa =\frac{1}{v^3}\|\vec v \times \vec a\|$:
        \[
            \kappa=\left( \frac{t}{t^2+2} \right)^{3}2\frac{t^2+2}{t^2}=\frac{2t}{(t^2+2)^{2}} 
        .\] 
        We can now also find $a_T$, $a_N$.
        \[
        a_T=v'=\frac{t^2-2}{t^2}
        \] 
        \[
        a_N=\kappa v^2=\frac{2t}{(t^2+2)^{2}}\left(\frac{t^2+2}{t}  \right) ^2=\frac{2}{t}
        .\]            


    \end{enumerate}
\item
    \begin{enumerate}[label= (\alph*)] 
        \item Suppose $y = f (x)$ is a planar curve where $f$ is twice differentiable.
        Show that $f$ has zero curvature everywhere if and only if $f$ is a line.
        \begin{proof} 
            Let $f$ be twice differentiable as above.

            $\implies$: Suppose $f$ has zero curvature. That is,
            \[
                \kappa=\frac{f''(x)}{{ (1+f'(x)^{2} })^\frac{3}{2}}=0.
            .\] 
            So $f''(x)=0$. Then $f'(x)=c$ for some $c\in\mathbb{R}$, and $f(x)=cx+d$, for some 
            $d\in \mathbb{R}$. This is the equation of a line.

            So if $f$ has zero curvature, $f$ must be a  line!

            $\impliedby$: Suppose $f$ is a line. That is $f(x)=ax+b$ for some $a,b\in\mathbb{R}$.
            Then $f'(x)=a$ and $f''(x)=0$. Using our formula:
            $$\kappa=\frac{f''(x)}{{ (1+f'(x)^{2} })^\frac{3}{2}}
            =\frac{0}{{ (1+a^{2} })^\frac{3}{2}}=0.$$
            We can see that if $f$ is a line, then $\kappa=0$. 

            Therefore, curvature is zero $\iff$ $f$ is a line.
        \end{proof}

        \item Suppose $C$ is a parametric curve with parametrization $\vec{r}(t)$. If $C$ has
        zero curvature at all points, is it the case that $\vec{r}$ is a line? Provide a proof or a
        counterexample

        It is not the case that if $C$ has zero curvature that $\vec{r}$ is a line. Consider the example 
        $\vec{r}=\begin{pmatrix} e^{t}&e^{t}&0 \end{pmatrix}$. Note that $\vec{r}$ is not a line.

        We can find that $\vec{r}=\vec{v}=\vec{a}$. So, calculating $\vec{v}\times \vec{a}$, we see:
        \[
            \vec{v}\times \vec{a}=\det\begin{bmatrix} \vec{i}&\vec{j}&\vec{k}\\
                e^{t}&e^{t}&0\\
                e^{t}&e^{t}&0\\
            \end{bmatrix} = \begin{pmatrix} 0\\0\\0 \end{pmatrix} 
        .\] 
        And so $\|\vec{v}\times \vec{a}\|=0$. Since $\kappa=\frac{1}{v^3}\|\vec{v}\times \vec{a}\|$, 
        $\kappa=\frac{0}{2e^{2t}}=0$. So $C$ has zero curvature but $\vec{r}$ is not a line.

        \item Suppose $C$ is a parametric curve with parametrization $\vec{r} (t)$. If $C$ has
        zero torsion at all points, is it the case that $\vec{r}$ is a line? Provide proof or a
        counterexample.

        Choose $\vec{r}=\begin{pmatrix} t^2&t&3t^2 \end{pmatrix}^{T}$.

        Then we have:
        \begin{align*}
            \vec{v}(t)&=\begin{pmatrix} 2t&1&6t \end{pmatrix} ^{T}\\
            \vec{a}(t)&= \begin{pmatrix} 2&0&6 \end{pmatrix}^{T}  \\
            \vec{a'}(t)&= 0 
        .\end{align*}

        Now we find $\vec{v}\times \vec{a}$.
        \[
            \vec{v}\times \vec{a}=
            \det\begin{bmatrix} 
                \vec{i}&\vec{j}&\vec{k}\\
                2t&1&6t\\
                2&0&6
            \end{bmatrix} 
            =\begin{pmatrix} 6&0&2 \end{pmatrix} ^{T}.
        \] 
        And from this we find $\|\vec{v}\times \vec{a}\|=2\sqrt{10} $. From our formula for $\tau$:
        \[
            \tau=\frac{(\vec{v}\times \vec{a})\cdot\vec{a}'}{\|\vec{v}\times \vec{a}\|^2}
            =\frac{0}{2\sqrt{10} }=0
        .\] 
        So $C$ has zero torsion but $r$ is not a line.
        \newpage
    \end{enumerate}
\item
    \begin{enumerate}[label= (\alph*)] 
        \item Suppose $C$ is a curve parametrized at by $r (t)$. Suppose we know that
        $\vec a = (4, -2, 6)^T$ ,  $\vec T = \frac{1}{\sqrt{3} } (1, -1, 1)^T$ 
        and  $\vec N = \frac{1}{\sqrt{5} } (1, 0, 2)^T$ at some point on the
        curve. Determine ${a}_T$ and ${a}_N$ there.

            We can find $a_T$ by taking $\vec{a}\cdot\vec{T}$ at our point. So:
            \[
            a_T=\frac{4}{\sqrt{3} }+\frac{2}{\sqrt{3} }+\frac{6}{\sqrt{3} }=\frac{12}{\sqrt{3} }=4\sqrt{3} 
            .\] 

            Likewise, we can find $a_N$ by taking $\vec{a}\cdot\vec{N}$ at our point:
            \[
            a_N=\frac{4}{\sqrt{5} }+\frac{12}{\sqrt{5} }=\frac{16}{\sqrt{5} }
            .\] 

        \item Suppose $C$ is a curve parametrized at by $\vec r (t)$. Is it possible for $a_T = 0$
        at all points? If so, what kind of curve is $C$?

        $a_T$ can equal $0$ for all $t$. If so, we can see $v'(t)=0$ which implies that $v(t)=c$ for some
        $c\in\mathbb{R}$. That is, the speed is constant (we have zero acceleration).

        \item Suppose $C$ is a curve parametrized at by $\vec r (t)$. Is it possible for $a_N = 0$
        at all points? If so, what kind of curve is C?

        If $a_N=0$, $v^2\kappa=0$. This means that either $v=0$ or $\kappa=0$. 

        Intuitively, we know that normal acceleration represents the change in direction of velocity. So 
        in the case where we have zero normal acceleration we have no change in the direction of velocity.
\end{enumerate}
\end{enumerate}
\end{document}
