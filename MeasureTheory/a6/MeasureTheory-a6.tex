\documentclass{article}
\usepackage[most,many,breakable]{tcolorbox}
\usepackage{amsmath}
\usepackage{amssymb}
\usepackage{amsthm}
\usepackage[]{thmbox}
\usepackage{blindtext}
\usepackage[utf8]{inputenc}
\usepackage{amsmath}
\usepackage{amsfonts}
\usepackage[]{graphicx}
\usepackage[legalpaper, portrait, margin = 1in]{geometry}
\usepackage{enumitem}


\usepackage{xcolor}

%\pagecolor[rgb]{0.2,0.19,0.18} 
%\color[rgb]{0.92,0.86,0.7}

\newtheorem[L]{le}{Lemma}[subsection]
\newtheorem[L]{th}[le]{Theorem}
\newtheorem[L]{df}[le]{Definition}
\newtheorem[L]{ex}[le]{Example}
\newtheorem[L]{pf}[le]{Proof}


\newcommand{\nl}{\newline}

\newcommand{\real}{\mathbb{R}}
\newcommand{\complex}{\mathbb{C}}
\newcommand{\integer}{\mathbb{Z}}
\newcommand{\rational}{\mathbb{Q}}
\newcommand{\lxor}{\oplus}
\newcommand{\then}{\Rightarrow}
\pagestyle{fancy}
\lhead{Assignment \# $6$}
\rhead{Thomas Boyko}
\chead{}

\begin{document}
\begin{exe}
    Assume that $\mu$ is a measure on $(\mathbb{R},\mathcal{B}(\mathbb{R}))$, which satisfies that 
    \[
        \mu((-\infty,x])=\mu([x,\infty))<\infty\quad \text{for all }x\in \mathbb{R}
    .\] 
    Show then that $\mu(B)=\mu(-B)$ for any Borel set $B$.
\end{exe}
\paragraph{Solution: }Suppose $\mu$ is a measure as above. Define a new measure $\nu$ by $\nu(B)=\mu(-B)$.

It is not too difficult to see $\nu$ is a measure; since $\nu(\varnothing)=\mu(-\varnothing)=\mu(\varnothing)=0$
And, if $(A_n)_{n\in \mathbb{N}}$ is a sequence of subsets of $\mathbb{R}$, we have:
\[
\nu\left( \bigcup_{n\in \mathbb{N}} A_n \right) =\mu\left( -\bigcup_{n\in \mathbb{N}} A_n \right) = \mu\left( \bigcup_{n\in \mathbb{N}} -A_n \right) =\sum_{n\in \mathbb{N}}^{} \mu(-A_n)=\sum_{n\in \mathbb{N}}^{} \nu(A_n)
.\] 
So $\nu$ is a measure.
We wish to apply Theorem 2.2.2 on the system $\mathcal{S}$:
\[
    \mathcal{S}=\{(-\infty,x]:x\in \mathbb{R}\}
.\] 
It's discussed in the textbook that the system is $\cap $-stable, and that the system generates $\mathcal{B}(\mathbb{R})$.

From the assumption we have:
\[
    \mu((-\infty,x])=\mu([x,\infty))=\mu(-(-\infty,x])=\nu((-\infty,x])
.\] 
So $\mu,\nu$ agree on $\mathcal{S}$ (and are finite). They also agree on the sequence $A_n=(-\infty,n]$ in $\mathcal{S}$, which has $\bigcup_{n\in \mathbb{N}} A_n=\mathbb{R}$.

Since all the conditions in 2.2.2 are satisfied, we have:
\[
    \mu(B)=\nu(B)=\mu(-B)
.\] 
\end{document}
