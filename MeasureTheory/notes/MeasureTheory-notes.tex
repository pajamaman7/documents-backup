\documentclass{article}
\usepackage[most,many,breakable]{tcolorbox}
\usepackage{amsmath}
\usepackage{amssymb}
\usepackage{amsthm}
\usepackage[]{thmbox}
\usepackage{blindtext}
\usepackage[utf8]{inputenc}
\usepackage{amsmath}
\usepackage{amsfonts}
\usepackage[]{graphicx}
\usepackage[legalpaper, portrait, margin = 1in]{geometry}
\usepackage{enumitem}


\usepackage{xcolor}

%\pagecolor[rgb]{0.2,0.19,0.18} 
%\color[rgb]{0.92,0.86,0.7}

\newtheorem[L]{le}{Lemma}[subsection]
\newtheorem[L]{th}[le]{Theorem}
\newtheorem[L]{df}[le]{Definition}
\newtheorem[L]{ex}[le]{Example}
\newtheorem[L]{pf}[le]{Proof}


\newcommand{\nl}{\newline}

\newcommand{\real}{\mathbb{R}}
\newcommand{\complex}{\mathbb{C}}
\newcommand{\integer}{\mathbb{Z}}
\newcommand{\rational}{\mathbb{Q}}
\newcommand{\lxor}{\oplus}
\newcommand{\then}{\Rightarrow}
\pagestyle{fancy}
\lhead{Measure and Integration}
\rhead{Thomas Boyko}
\chead{}

\begin{document}

\tableofcontents
\newpage

\section{$\sigma$-algebras and Measures}

We begin by discussing the concept of a measure, or how to 'measure' space on the real line, and in $\mathbb{R}^{n} $. Then we extend the concept, to allow measuring of non-Euclidean space. 
\begin{defn}[Area in $\mathbb{R}^2$]
    Consider a set function $\lambda_2:\mathcal{P}(\mathbb{R}^2)\to [0,\infty]$ satisfying:
    \begin{enumerate}
        \item $\lambda_2(\varnothing)=0$
        \item  $\lambda_2\left(\bigcup_{i\in I} A_i\right)=\sum_{i\in I} \lambda_2(A_i)$ for disjoint $A_i$ and countable $I$.
        \item $\lambda_2((a_1,b_1)\times (a_2,b_2))=(b_1-a_1)(b_2-a_2)$
        \item Measure is invariant of translation and rotation.
    \end{enumerate}
\end{defn}

\begin{defn}[$\sigma$-algebra in $\mathbb{R}^2$]
    A $\sigma$-algebra is a system $\mathcal{E}$ of subsets of $\mathbb{R}^2$ satisfying:
    \begin{enumerate}
        \item $\mathbb{R}^2\in \mathcal{E}$
        \item If $A\in \mathcal{E}$, then $A^{c}\in \mathcal{E}$.
        \item if $\{A_n\} _{n\in \mathbb{N}}$ is a sequence of sets in from $\mathcal{E}$, then $\bigcup_{n\in \mathbb{N}} A_n\in \mathcal{E}$
    \end{enumerate}
    The common choice for a $\sigma-$algebra in $\mathbb{R}^2$ is the Borel Algebra $\mathcal{B}(\mathbb{R}^2)$, which is the smallest $\sigma$-algebra containing all open rectangles $(a_1,b_1)\times (a_2,b_2)\subseteq \mathbb{R}^2$.
\end{defn}

\begin{defn}[$\sigma$-algebra]
    A $\sigma$-algebra in a set $X$ is a nonempty collenction $\mathcal{E}$ of subsets of $X$ which contains $X$, and is closed under complement and countable union; if $A_1,A_2,\dots \in \mathcal{E}$; then
    \begin{enumerate}
        \item $X,\varnothing\in \mathcal{E}$
        \item $A^{c}\in \mathcal{E}$
        \item $\bigcup_{i=1}^\infty A_i\in \mathcal{E}$
        %\item $\bigcap_{1\leq i} A_i\in \mathcal{E}$
    \end{enumerate}
    Elements of $\mathcal{E}$ are called measureable sets, or just measurable. $X$ is a measurable space.

    We may consider only finite unions, in which case we call $\mathcal{E}$ just an Algebra (This is equivalent to only requiring $A\cup B\in \mathcal{E}$ when $A,B\in \mathcal{E}$).
\end{defn}

\begin{defn}[Measure]
    Let $\mathcal{E} $ be a  $\sigma $-algebra in a set $X$. A measure on $\mathcal{E}$ is a set function $\mu:\mathcal{E}\to [0,\infty]$, satisfying the following:
    \begin{enumerate}
        \item $\mu(\varnothing)=0$
        \item $\mu\left( \bigcup_{n=1} ^{\infty}A_n \right) =\sum_{n=1}^{\infty} \mu(A_n)$
    \end{enumerate}
    A measure satisfying $\mu(X)=1$ may be called a probability measure.
\end{defn}

\begin{prop}
    Let $\mathcal{E}$ be an algebra on $X$. Then:
    \begin{enumerate}
        \item $\varnothing\in \mathcal{E}$.
        \item If $A,B\in \mathcal{E}$, then $A\cup B\in \mathcal{E}$.
        \item If $A,B\in \mathcal{E}$ then $A \setminus B\in \mathcal{E}$.
        \item If $\mathcal{E}$ is a $\sigma$-algebra, and $A_1,A_2,\dots \in \mathcal{E}$, then $\bigcap_{i=1} ^{\infty}A_n\in \mathcal{E}$.
    \end{enumerate}
\end{prop}

\begin{exmp}
    Below are some examples of $\sigma$-algebras, for some set $X$
    \begin{enumerate}
        \item $\mathcal{E}_{min}=\{\varnothing,X\} $
        \item $\mathcal{E}_{max}=\mathcal{P}(X)$
        \item $\mathcal{E}=\{\varnothing,A,A^{c},X\} $
        \item If $A_1,\dots,A_n$ are disjoint but cover $X,$ $\mathcal{E}=\left\{\bigcup\limits_{j\in I} A_j:I\subseteq \{1,\dots,n\} \right\} $
    \end{enumerate}
\end{exmp}

\begin{thm}[Generated $\sigma$-algebras]
    Let $\{\mathcal{E}_i\} _{i\in I}$ be a family of $\sigma$-algebras in $X$. Then the system:
    \[
    \bigcap_{i\in I} A_i=\{A\subseteq X:A\in \mathcal{E}_i\forall i\in I\} 
    .\] 
    Is a $\sigma$-algebra in $X$. In fact this is the smallest $\sigma$-algebra in $X$. Members of this $\sigma$-algebra are sometimes called Borel sets.
\end{thm} 

\end{document}
