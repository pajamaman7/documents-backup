\documentclass{article}
\usepackage[most,many,breakable]{tcolorbox}
\usepackage{amsmath}
\usepackage{amssymb}
\usepackage{amsthm}
\usepackage[]{thmbox}
\usepackage{blindtext}
\usepackage[utf8]{inputenc}
\usepackage{amsmath}
\usepackage{amsfonts}
\usepackage[]{graphicx}
\usepackage[legalpaper, portrait, margin = 1in]{geometry}
\usepackage{enumitem}


\usepackage{xcolor}

%\pagecolor[rgb]{0.2,0.19,0.18} 
%\color[rgb]{0.92,0.86,0.7}

\newtheorem[L]{le}{Lemma}[subsection]
\newtheorem[L]{th}[le]{Theorem}
\newtheorem[L]{df}[le]{Definition}
\newtheorem[L]{ex}[le]{Example}
\newtheorem[L]{pf}[le]{Proof}


\newcommand{\nl}{\newline}

\newcommand{\real}{\mathbb{R}}
\newcommand{\complex}{\mathbb{C}}
\newcommand{\integer}{\mathbb{Z}}
\newcommand{\rational}{\mathbb{Q}}
\newcommand{\lxor}{\oplus}
\newcommand{\then}{\Rightarrow}
\pagestyle{fancy}
\lhead{Assignment \# $2$}
\rhead{Thomas Boyko}
\chead{}

\begin{document}

\begin{exe}
    Consider the measure space $(\mathbb{R},\mathcal{B}(\mathbb{R}),\lambda)$, where $\lambda $ denotes Lebesgue measure on $\mathbb{R}$.
    \begin{enumerate}
        \item Show that $\lambda(\{a\})=0 $ for any $a\in \mathbb{R}.$  
        \item Show that for any $a<b$ in $\mathbb{R}$, the follwing holds:
            \[
                \lambda((a,b))=\lambda((a,b])=\lambda([a,b))=\lambda([a,b])=b-a
            .\] 
        \item Show that for any $a$ in $\mathbb{R}$, the follwing holds:
            \[
                \lambda((-\infty,a))=\lambda((-\infty,a])=\lambda([a,\infty))=\lambda([a,\infty])=\infty
            .\] 
    \end{enumerate}
\end{exe}

\paragraph{Student's note} For this assignment we take the convention $0\not\in \mathbb{N}.$ I am not sure whether this is usual for the course (:
\begin{enumerate}
    \item Let $a\in \mathbb{R}$. Take the sequence of decreasing sets:
        \[
        \left(a-\frac{1}{n},a+\frac{1}{n}\right)
        .\] 

        So that:
        \[
            \{a\} =\bigcap_{n\in \mathbb{N}} \left(a-\frac{1}{n},a+\frac{1}{n}\right)
        .\] 
        Then we have:
        \begin{align*}
        \lambda(\{a\}) &=\lambda\left( \bigcap_{n\in \mathbb{N}}  \left(a-\frac{1}{n},a+\frac{1}{n}\right) \right)\\
        &= \lim_{n \to \infty} \lambda\left( \left( a-\frac{1}{n},a+\frac{1}{n} \right) \right) &\text{ From prop. 1.3.4 (vi)}\\
        &= \lim_{n \to \infty} a+\frac{1}{n}-\left( a-\frac{1}{n} \right)  \\
        &= \lim_{n \to \infty} \frac{1}{2n} \\
        &= 0
        .\end{align*}
    \item 
        We split into 4 parts:
        \begin{enumerate}[label= (\alph*)] 
            \item $\lambda((a,b))=b-a$
            \item $\lambda((a,b])=b-a$
            \item $\lambda([a,b))=b-a$
            \item $\lambda([a,b])=b-a$
        \end{enumerate}
        \paragraph{Solution:} 
        \begin{enumerate}[label= (\alph*)] 
            \item This is true by definition
            \item Let $a<b$ be reals. We use the same strategy again. Take the sequence of decreasing sets:
        \[
        \left(a,b+\frac{1}{n}\right)
        .\] 
        So that:
        \[
            (a,b] =\bigcap_{n\in \mathbb{N}} \left(a,b+\frac{1}{n}\right)
        .\] 
        Then we have:
        \begin{align*}
            \lambda((a,b])&=\lambda\left( \bigcap_{n\in \mathbb{N}} \left(a,b+\frac{1}{n}\right) \right) \\
            &= \lim_{n \to \infty} \lambda\left( \left( a,b+\frac{1}{n} \right)  \right)  &\text{ From prop. 1.3.4 (vi)}\\
            &= \lim_{n \to \infty} b+\frac{1}{n}-a \\
            &= b-a
        .\end{align*}

    \item Let $a<b$ be reals. We use the same strategy again. Take the sequence of decreasing sets:
        \[
        \left(a-\frac{1}{n},b\right)
        .\] 
        So that:
        \[
            [a,b) =\bigcap_{n\in \mathbb{N}} \left(a-\frac{1}{n},b\right)
        .\] 
        Then we have:
        \begin{align*}
            \lambda([a,b))&=\lambda\left( \bigcap_{n\in \mathbb{N}} \left(a-\frac{1}{n},b\right) \right) \\
            &= \lim_{n \to \infty} \lambda\left( \left( a-\frac{1}{n},b \right)  \right)  &\text{ From prop. 1.3.4 (vi)}\\
            &= \lim_{n \to \infty} b-\frac{1}{n}-a \\
            &= b-a
        .\end{align*}
    \item Let $a<b$ be reals. We use the same strategy again. Take the sequence of decreasing sets:
        \[
        \left(a-\frac{1}{n},b\right]
        .\] 
        So that:
        \[
            [a,b] =\bigcap_{n\in \mathbb{N}} \left(a-\frac{1}{n},b\right]
        .\] 
        Then we have:
        \begin{align*}
            \lambda([a,b))&=\lambda\left( \bigcap_{n\in \mathbb{N}} \left(a-\frac{1}{n},b\right] \right) \\
            &= \lim_{n \to \infty} \lambda\left( \left( a-\frac{1}{n},b \right]  \right)  &\text{ From prop. 1.3.4 (vi)}\\
            &= \lim_{n \to \infty} b-\frac{1}{n}-a \\
            &= b-a
        .\end{align*}
    \end{enumerate}
    \newpage
\item 
        We split into 4 parts:
        \begin{enumerate}[label= (\alph*)] 
            \item $\lambda((-\infty,a))=\infty$
            \item $\lambda((-\infty,a])=\infty$
            \item $\lambda((a,\infty))=\infty$
            \item $\lambda([a,\infty))=\infty$
        \end{enumerate}
        \paragraph{Solution: }
        \begin{enumerate}
            \item 
    Rewrite:
        \[
            (-\infty,a)=\bigcup_{n\in \mathbb{N}} [a-n,a-n+1)
        .\] 
        Then since these are all disjoint subsets of $\mathbb{R}$, we have:
        \begin{align*}
            \lambda\left( (-\infty,a) \right) &= \lambda(\bigcup_{n\in \mathbb{N}} [a-n,a-n+1) \\
            &= \sum_{n=1}^{\infty} \lambda([a-n,a-n+1)) \\
            &= \sum_{n=1}^{\infty}(a-n+1)-(a-n)&\text{From 2.}\\
            &= \sum_{n=1}^{\infty}1\\
            &=\infty
        .\end{align*}
    \item Rewrite:
        \[
            (-\infty,a)=\bigcup_{n\in \mathbb{N}} [a-n,a-n+1)
        .\] 
        Then since these are all disjoint subsets of $\mathbb{R}$, we have:
        \begin{align*}
            \lambda\left( (-\infty,a) \right) &= \lambda(\bigcup_{n\in \mathbb{N}} [a-n,a-n+1) \\
            &= \sum_{n=1}^{\infty} \lambda([a-n,a-n+1)) \\
            &= \sum_{n=1}^{\infty}(a-n+1)-(a-n)&\text{From 2.}\\
            &= \sum_{n=1}^{\infty}1\\
            &=\infty
        .\end{align*}

        \newpage
            \item 
    Rewrite:
    \[
        (a,\infty)=\bigcup_{n\in \mathbb{N}} (a+n-1,a+n]
    .\] 
    Then since these are all disjoint subsets of $\mathbb{R}$, we have:
    \begin{align*}
        \lambda\left( a,\infty \right) &= \lambda\left( \bigcup_{i \in  I} (a+n-1,a+n] \right)  \\
                                       &= \sum_{n=1}^{\infty} \lambda\left(  (a+n-1,a+n] \right)  \\
                                       &= \sum_{n=1}^{\infty} (a-n+1)-(a-n) &\text{From 2.}\\
                                       &= \sum_{n=1}^{\infty} 1 \\
                                       &= \infty 
    .\end{align*}

        \item Rewrite:
    \[
        (a,\infty)=\bigcup_{n\in \mathbb{N}} (a+n-1,a+n]
    .\] 
    Then since these are all disjoint subsets of $\mathbb{R}$, we have:
    \begin{align*}
        \lambda\left[ a,\infty \right) &= \lambda\left( \bigcup_{i \in  I} [a+n-1,a+n) \right)  \\
                                       &= \sum_{n=1}^{\infty} \lambda\left(  [a+n-1,a+n) \right)  \\
                                       &= \sum_{n=1}^{\infty} (a-n+1)-(a-n) &\text{From 2.}\\
                                       &= \sum_{n=1}^{\infty} 1 \\
                                       &= \infty 
    .\end{align*}


\end{enumerate}
\end{enumerate}
\end{document}
