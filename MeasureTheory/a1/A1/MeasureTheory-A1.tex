\documentclass{article}
\usepackage{amsmath}
\usepackage{amssymb}
\usepackage{amsthm}
\usepackage[utf8]{inputenc}
\usepackage{amsmath}
\usepackage{amsfonts}
\usepackage[]{graphicx}
\usepackage[a4paper, portrait, margin = 1in]{geometry}
\usepackage{enumitem}
\usepackage{xcolor}

%darkmode
%\pagecolor[rgb]{0.2,0.19,0.18} 
%\color[rgb]{0.92,0.86,0.7}

\newenvironment*{alphenum}{\begin{enumerate}[label= (\alph*)]}{\end{enumerate}}

\pagestyle{fancy}
\lhead{Assignment \# $1$}
\rhead{Name: Thomas Boyko; UCID: 30191728}
\chead{}

\begin{document}

\begin{exmp}

    Let $X\neq \varnothing$, and $B\subseteq X$ be given. Show that the set:
    \[
    \mathcal{E}_B=\{A\subseteq X:B\subseteq A \text{ or }B\subseteq A^{c}\} 
    .\] 
    Is a $\sigma$-algebra in $X$.
    
\end{exmp}

\paragraph{$\sigma 1$:} $X\in \mathcal{E}_B$ since $B\subseteq X$ by assumption.
\paragraph{$\sigma 2$:} If $A\in \mathcal{E}_B$, then either $B\subseteq A$ or $B\subseteq A^{c}$. If the first holds, then clearly $B\subseteq (A^{c})^{c}=A$, and $A^{c}\in \mathcal{E}_B$. On the other hand, if $B\subseteq A^{c}$, then $A^{c}\in \mathcal{E}_B$. Therefore $\mathcal{E}_B$ is closed under complements.

\paragraph{$\sigma 3$:} Suppose $\{A_n\} _{n\in \mathbb{N}}$ is a sequence of elements in $\mathcal{E}_B$. Partition $\mathbb{N}$ into disjoint subsets $I,J$ so that $B\subseteq A_i$ for $ i\in I$, and $B\subseteq A_j^{c}, $ for $ j\in J$. The $A_j$ require some consideration. By $\sigma2$, we know that each $A_j$ must also be in $\mathcal{E}_B$. Then $B\subseteq A_i$, for all $i$, and $B\subseteq J$ for all $j$. Since $I\cup J=\mathbb{N}  ,$ we have
\[
B\subseteq \bigcup_{i \in  I} A_i\cup \bigcup_{j\in J} A_j  =\bigcup_{n\in \mathbb{N}} A_n
.\] 
So $\mathcal{E}_B$ is closed under countable union, and is therefore a $\sigma$-algebra.

\end{document}
