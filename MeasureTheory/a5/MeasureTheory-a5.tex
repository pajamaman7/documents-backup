\documentclass{article}
\usepackage[most,many,breakable]{tcolorbox}
\usepackage{amsmath}
\usepackage{amssymb}
\usepackage{amsthm}
\usepackage[]{thmbox}
\usepackage{blindtext}
\usepackage[utf8]{inputenc}
\usepackage{amsmath}
\usepackage{amsfonts}
\usepackage[]{graphicx}
\usepackage[legalpaper, portrait, margin = 1in]{geometry}
\usepackage{enumitem}


\usepackage{xcolor}

%\pagecolor[rgb]{0.2,0.19,0.18} 
%\color[rgb]{0.92,0.86,0.7}

\newtheorem[L]{le}{Lemma}[subsection]
\newtheorem[L]{th}[le]{Theorem}
\newtheorem[L]{df}[le]{Definition}
\newtheorem[L]{ex}[le]{Example}
\newtheorem[L]{pf}[le]{Proof}


\newcommand{\nl}{\newline}

\newcommand{\real}{\mathbb{R}}
\newcommand{\complex}{\mathbb{C}}
\newcommand{\integer}{\mathbb{Z}}
\newcommand{\rational}{\mathbb{Q}}
\newcommand{\lxor}{\oplus}
\newcommand{\then}{\Rightarrow}
\pagestyle{fancy}
\lhead{Assignment \# $5$}
\rhead{Thomas Boyko}
\chead{}

\begin{document}
\begin{exe}
    Let $K\in (0,\infty),\alpha\in \mathbb{R}$. Show that:
    \[
    \int_{0}^{K} x^{\alpha}\,\lambda(  d x )=
    \begin{cases}
        \infty&\text{if }\alpha\leq -1\\
        (\alpha+1)^{-1}K^{\alpha+1}&\text{if }\alpha> -1\\
    \end{cases}
    .\] 

\end{exe}
\paragraph{Solution: }
Use the relationship between the Lebesgue and Riemann integral again.
\[
    I_\alpha=R \int_{0}^{K}x^{\alpha}  \, d x 
.\] 
Then proceed by cases. If $\alpha=-1$, we integrate as if it were high school.
\begin{align*}
    I_{-1}&=\lim_{n\to\infty}R \int_{\frac{1}{n}}^{K}x^{-1}  \, d x \\
    &= \lim_{n\to\infty} \left( \ln x \right) _{\frac{1}{n}}^{K} \\
    &= \lim_{n\to\infty} \ln K-\ln \frac{1}{n} \\
    &=\infty
.\end{align*}

If $\alpha\neq -1,$ use the usual power rule:
\begin{align*}
    I_\alpha&=\lim_{n\to\infty} R \int_{\frac{1}{n}}^{K} x^{\alpha} \, d x \\
    &=\lim_{n\to\infty}  \left(\left( \alpha+1 \right) ^{-1}x^{\alpha+1} \right)_{\frac{1}{n}}^{K}\\
    &=(\alpha+1)^{-1}\lim_{n\to\infty} \left( x^{\alpha+1} \right) ^{K}_{\frac{1}{n}}\\
    &=\lim_{n\to\infty} (\alpha+1)^{-1}\left( K^{\alpha+1}-\frac{1}{n^{\alpha+1}} \right) 
.\end{align*}
Now we must again split into cases. If $\alpha<-1$, then $\alpha+1$ is negative and we will have the limit:
\[
    \lim_{n\to\infty} \frac{1}{n^{\alpha+1}}=\infty
.\] 
And $I_\alpha=\infty$.

If $\alpha>-1, $ then $\alpha+1$ is positive;
\[
    \lim_{n\to\infty} \frac{1}{n^{\alpha+1}}=0
.\] 
And $I_\alpha=\frac{K^{\alpha+1}}{\alpha+1}$

\newpage
\begin{exe}

    Let $K\in (0,\infty),\alpha\in \mathbb{R}$. Show that:
    \[
    \int_{K}^{\infty} x^{\alpha}\,\lambda(  d x )=
    \begin{cases}
        \infty&\text{if }\alpha\leq -1\\
        -(\alpha+1)^{-1}&\text{if }\alpha> -1\\
    \end{cases}
    \] 


\end{exe}

\paragraph{Solution: } Rewrite as a Riemann integral again:
\begin{align*}
    I_\alpha&=  \int_{K}^{\infty} x^{\alpha} \, d \lambda\\
            &= \lim_{n\to\infty} R \int_{K}^{n}x^{\alpha}  \, d \lambda
.\end{align*}
If $\alpha=-1,$ again we have:
\begin{align*}
    I_{-1}&=\lim_{n\to\infty}  \left( \ln x \right) ^{n}_{K} \\
    &= \lim_{n\to\infty} \ln n-\ln K
    &= \infty
.\end{align*}
Now if $\alpha\neq -1$, 
\[
I_\alpha=\lim_{n\to\infty} \left( \frac{x^{\alpha+1}}{\alpha+1} \right) ^{n}_{K}=\left( \alpha+1 \right) ^{-1}\lim_{n\to\infty} n^{\alpha+1}-K^{\alpha+1}
.\] 
If $\alpha<-1$, we have $\alpha+1$ negative, and 
\[
\lim_{n\to\infty} n^{\alpha+1}=0
.\] 
And in turn 
\[
I_\alpha=-\frac{K^{\alpha+1}}{\alpha+1}
.\] 
Now if $\alpha>-1$, then $\alpha+1$ is positive, and 
\[
\lim_{n\to\infty} n^{\alpha+1}=\infty
.\] 

\end{document}
