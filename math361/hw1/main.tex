\documentclass{article}
\usepackage{amsmath}
\usepackage{amssymb}
\usepackage{amsthm}
\usepackage[utf8]{inputenc}
\usepackage{amsmath}
\usepackage{amsfonts}
\usepackage[]{graphicx}
\usepackage[a4paper, portrait, margin = 1in]{geometry}
\usepackage{enumitem}
\usepackage{xcolor}

%darkmode
%\pagecolor[rgb]{0.2,0.19,0.18} 
%\color[rgb]{0.92,0.86,0.7}

\newenvironment*{alphenum}{\begin{enumerate}[label= (\alph*)]}{\end{enumerate}}

\pagestyle{fancy}
\lhead{Assignment \#$1$}
\rhead{Name: Thomas Boyko; UCID: 30191728}
\chead{}

\begin{document}
\begin{enumerate} 
    \item Let $V$ and $W$ be finite dimensional vector spaces with given bases $\mathcal{B} = \{\vec{b}_1,\ldots,\vec{b}_n\}$ and $\mathcal{D} = \{\vec{d}_1,\ldots,\vec{d}_m\}$, respectively.
      \begin{enumerate}
        \item For a given $\vec{x} \in V$, there are unique scalars so that $\vec{x} = a_1\vec{b}_1 + \cdots + a_n\vec{b}_n$. Define the vector $[\vec{x}]_{\mathcal{B}} := (a_1,\ldots,a_n)^T \in \mathbb{C}^n$. Show that the map $\vec{x} \mapsto [\vec{x}]_{\mathcal{B}}$ is a linear isomorphism from $V$ into $\mathbb{C}^n$.

          \paragraph{Linearity: }Let $\vec{x},\vec{y}\in V$. Then write $\vec{x} = a_1\vec{b}_1 + \cdots + a_n\vec{b}_n$ and $\vec{y} = c_1\vec{b}_1 + \cdots + c_n\vec{b}_n$. Now:
          \[
              [\vec{x}+\vec{y}]_{\mathcal{B}}=\begin{bmatrix} a_1+c_1\\\vdots\\ a_n+c_n \end{bmatrix} 
            =\begin{bmatrix} a_1\\\vdots\\ a_n \end{bmatrix} 
            +\begin{bmatrix} c_1\\\vdots\\ c_n \end{bmatrix} 
            =[\vec{x}   ]_{\mathcal{B}}
            +[\vec{y}   ]_{\mathcal{B}}
          \] 
           \[
               [\alpha\vec{x}]_\mathcal{B}=\begin{bmatrix} \alpha a_1\\\vdots \\ \alpha a_n \end{bmatrix} 
               =\alpha \begin{bmatrix} a_1\\\vdots\\ a_n \end{bmatrix} 
               =\alpha[\vec{x}]_{\mathcal{B}}
          .\] 
          So $[\cdot ]_{\mathcal{B}}$ is linear. 

          \paragraph{Isomorphism: } Since $\dim V=\dim \mathbb{C}^{n}=n$, it will suffice to show that this mapping is injective. We do so by showing  $\ker [\cdot ]_\mathcal{B}=\{0\} $. Clearly $0$ is in the kernel since $[0]_{\mathcal{B}}=[0\vec{x}]_\mathcal{B}=0[\vec{x}]_\mathcal{B}=0$. For inclusion the other way, let $\vec{x}\in \ker [\cdot ]_\mathcal{B} $. Then $[\vec{x}]_\mathcal{B}=0$; meaning the basis representation of $\vec{x}$ is through zero coefficients; and
          \[
          \vec{x}=0\vec{b}_1+\ldots+0\vec{b}_n=0
          .\] 
          So $\ker[\cdot ]_\mathcal{B}=\{0\} $, and this map is injective. But since the spaces are of the same dimension it must also be surjective thanks to Rank-Nullity. So the map is a linear isomorphism from $V$ to $\mathbb{C}^{n}$. 
        \item Let $T : V \to W$ be a linear map. In class, we defined the matrix representation of $T$ with respect to $\mathcal{B}$ and $\mathcal{D}$ as the $m \times n$ matrix $[T]_{\mathcal{B}\mathcal{D}} = [[T\vec{b}_1]_{\mathcal{D}},\ldots,[T\vec{b}_n]_{\mathcal{D}}].$ In other words, the $j$-the column of $[T]_{\mathcal{B}\mathcal{D}}$ is $[T\vec{b}_j]_{\mathcal{D}}.$ Show that $[T]_{\mathcal{B}\mathcal{D}}[\vec{x}]_{\mathcal{B}}=[T\vec{x}]_{\mathcal{D}}$ for any $\vec{x} \in V$.
            \paragraph{Solution: }Let $T:V\to W$ be linear, then write $\vec{x}=a_1\vec{b}_1+\ldots+a_n\vec{b_n}$.
            \begin{align*}
                [T]_{\mathcal{B}\mathcal{D}}[\vec{x}]_{\mathcal{B}}
                &= \left[ [T\vec{b}_1]_\mathcal{D}\ldots[T\vec{b}_n ]_\mathcal{D} \right]
                \begin{bmatrix} a_1\\ \vdots \\ a_n \end{bmatrix} \\
                &= a_1[T\vec{b}_1]_\mathcal{D}+\ldots+ a_n[T\vec{b}_n ]_\mathcal{D}  \\
                &= [a_1 T\vec{b}_1 +\ldots+ a_n T\vec{b}_n ]_\mathcal{D}  &\text{By linearity of }[\cdot ]_{\mathcal{D}}\\
                &= \left[T(a_1 \vec{b}_1 +\ldots+ a_n \vec{b}_n) \right]_\mathcal{D}  &\text{By linearity of }T\\
                &=[T\vec{x}]_{\mathcal{D}}
            .\end{align*}
            \newpage
      \item Show that $[T]_{\mathcal{B}\mathcal{D}}$ is a linear isomorphism from $L(V,W)$ (the vector space of linear maps from $V$ to $W$) to $M_{mn}(\mathbb{C})$ (vector space of $m \times n$ complex matrices).
          \paragraph{Linearity: }Let $T,S$ be linear from $V$ to $W$. Then:
          \begin{align*}
              [T+S]_{\mathcal{B}\mathcal{D}}&= \left[ [(T+S)\vec{b}_1]_\mathcal{D} \ldots [(T+S)\vec{b}_n]_\mathcal{D}  \right] \\
               &= \left[ [(T\vec{b}_1+S\vec{b}_1)]_\mathcal{D} \ldots [(T\vec{b}_n+S\vec{b}_n)]_\mathcal{D}  \right]\\
               &= \left[ [T\vec{b}_1]_\mathcal{D}\ldots [T\vec{b}_n]_\mathcal{D} \right] + \left[ [S\vec{b}_1]_\mathcal{D}\ldots [S\vec{b}_n]_\mathcal{D} \right] &\text{By Linearity of }[\cdot]_\mathcal{D}\\
               &=  [S]_{\mathcal{B}\mathcal{D}} +[T]_{\mathcal{B}\mathcal{D}} 
          .\end{align*}
          And then letting $\alpha\in \mathbb{C}$,
           \begin{align*}
               \alpha[T]_{\mathcal{B}\mathcal{D}}&=\alpha\left[ [T\vec{b}_1]_\mathcal{D}\ldots[T\vec{b}_n]_\mathcal{D} \right] \\
                         &=\left[ \alpha[T\vec{b}_1]_\mathcal{D} \ldots\alpha[T\vec{b}_n]_\mathcal{D}  \right]&\text{Linearity of }[\cdot]_\mathcal{D}\\
                         &=\left[ [\alpha T\vec{b}_1]_\mathcal{D} \ldots[\alpha T\vec{b}_n]_\mathcal{D} \right] \\
               &= [\alpha T]_{\mathcal{B}\mathcal{D}} 
          .\end{align*}
          \paragraph{Injective: } 
          Clearly  $0\in \ker[\cdot ]_{\mathcal{BD}}$; take any transformation $T$ and $[0]_{\mathcal{BD}}=[T-T]_{\mathcal{BD}}=[T]_{\mathcal{BD}}-[T]_{\mathcal{BD}}=0$ .

      Then let $\mathcal{B}=\{b_1,\ldots ,b_n\} $ and $T\in \ker [\cdot ]_\mathcal{BD} $. Then:
      \begin{align*}
          [T]_{\mathcal{BD}}&=0\\
          \left[ [Tb_1]_\mathcal{D}\ldots [Tb_n]_\mathcal{D}\right] &=[0\ldots 0]
      .\end{align*} 
      Then $[Tb_i]_{\mathcal{D}}=0$ for any basis vector $b_i$. In particular this means that $TB_i=0$, since $[\cdot ]_\mathcal{D}$ is an isomorphism. Now for any arbitrary $v\in V,$ write $v=a_1b_1+\ldots a_nb_n$. Then $Tv=T(a_1b_1+\ldots a_n b_n)=a_1Tb_1+\ldots+a_nTb_n=0+\ldots+0=0$ and $T=0$.

      Therefore $\ker[\cdot ]_{\mathcal{BD}}=\{0\} $.
      \paragraph{Surjective:} The argument that $\dim L(V,W)=\dim M_{mn}(\mathbb{C})$ proves difficult, so instead we show directly that $[L(V,W)]_{BD}=M_{mn}(\mathbb{C})$.

      Let $A\in M_{mn}(\mathbb{C})$, and write $A=\begin{bmatrix} \vec{a}_1\ldots\vec{a}_n \end{bmatrix} $, where $\vec{a}_j$ are column vectors in $\mathbb{C}^m$. Then take the inverse map for $[\cdot ]_{D}$ (which was shown to exist in 1(a)), denote it $[\cdot ]_{D}^{-1}$ and define $T$ on the basis vectors in $\mathcal{B}$ such that $Tb_j=[\vec{a}_j]_\mathcal{D}^{-1}$. Then: 
      \begin{align*}
          [T]_{\mathcal{BD}}&= \left[[Tb_1]_\mathcal{D}\ldots[Tb_n]_\mathcal{D}\right] \\
                            &= \left[\left[ [\vec{a}_1]_\mathcal{D}^{-1}\right]_\mathcal{D}\ldots\left[[\vec{a}_n]_\mathcal{D}^{-1}\right]_\mathcal{D} \right] \\
                            &= \begin{bmatrix} \vec{a}_1\ldots\vec{a}_n \end{bmatrix}  \\
                            &=A
      .\end{align*}
      So every arbitrary matrix has a preimage in the space of linear transformations, and therefore the mapping is onto. Since $[\cdot ]_{\mathcal{BD}}$ is bijective and linear, it must then be an isomorphism. 

      \end{enumerate}
      \newpage
    \item Let $V$, $W$ and $U$ be finite dimensional vector spaces with given bases: 

        $\mathcal{B} = \{\vec{b}_1,\ldots,\vec{b}_n\}, \mathcal{D} = \{\vec{d}_1,\ldots,\vec{d}_m\}, \text{ and } \mathcal{F} = \{f_1,\ldots,f_k\}$, respectively. Suppose $T : V \to W$ and $S : W \to U$ are linear. Prove or disprove the following statement for the composition linear map  $ST : V \to U$:
      \[
      [ST]_{\mathcal{B}\mathcal{F}}=[S]_{\mathcal{DF}}[T]_{\mathcal{B}\mathcal{D}}
      .\] 

      \paragraph{Solution: } 
      We make great use of the property shown in 1(b). Where it is used will be marked with $(*)$. Let $v\in V$ be arbitrary and recall that $[v]_{\mathcal{B}}$ is unique since $[\cdot ]_\mathcal{B}$ is an isomorphism. 
      %The plan was to mark this property wherever it was used, however it is used in every equality in the next formula.
      \begin{align*}
          [ST]_{\mathcal{B}\mathcal{F}}[v]_\mathcal{B}&= [STv]_\mathcal{F}&(*) \\
                                              &= [S]_{\mathcal{D}\mathcal{F}}[Tv]_{\mathcal{D}}&(*) \\
                                              &= [S]_{\mathcal{D}\mathcal{F}}[T]_{\mathcal{B}\mathcal{D}}[v]_\mathcal{B}&(*) 
      \end{align*}
      So we have shown that these matrices $[ST]_{\mathcal{B}\mathcal{F}}$ and  $[S]_{\mathcal{DF}}[T]_{\mathcal{B}\mathcal{D}} $ agree upon all vectors in the image of $[\cdot ]_\mathcal{B}$. However since this particular mapping is onto, we know this to be all of  $\mathbb{C}^{n}$. This means the matrices agree upon all of $\mathbb{C}^{n}$ and therefore they must be equal. 
  \item Let $V$ be a finite dimensional vector space and $T : V \to V$ be linear. Show that $\sigma(T)=\sigma([T]_{\mathcal{B}})$ where $\mathcal{B}$ is any basis for $V$.
      \paragraph{Linearity of inverse:} To show this we use the fact that $[\cdot ]_\mathcal{B}^{-1}$ is linear from $L(V,V)$ to $M_{mn}(\mathbb{C})$. Included is a brief demonstration of this fact. 
      Let $T,S\in L(V,V)$ and $\alpha\in \mathbb{C}$ .
      \begin{align*}
          [T]_\mathcal{B}^{-1}+[S]_{\mathcal{B}}^{-1}&=
          [[[T]_\mathcal{B}^{-1}+[S]_{\mathcal{B}}^{-1}]_\mathcal{B}]_\mathcal{B}^{-1}\\
         &=[[[T]_\mathcal{B}^{-1}]_\mathcal{B}+[[S]_{\mathcal{B}}^{-1}]_\mathcal{B}]_\mathcal{B}^{-1}\\
         &=[T+S]_\mathcal{B}^{-1}
      .\end{align*}
      \begin{align*}
          \alpha[T]_\mathcal{B}^{-1}&= [[\alpha[T]_\mathcal{B}^{-1}]_\mathcal{B}]_\mathcal{B}^{-1} \\ 
                                    &= [\alpha[[T]_\mathcal{B}^{-1}]_\mathcal{B}]_\mathcal{B}^{-1} \\ 
                                    &= [\alpha T]_\mathcal{B}^{-1} 
      .\end{align*}
    \paragraph{Solution:} $\subseteq $: Let $ \lambda\in \sigma(T)$, and let $\vec{v}$ be an associated eigenvector. We show that $[\vec{v}]_\mathcal{B}$ is an eigenvector for $\lambda$ under $[T]_\mathcal{B}$.
    \begin{align*}
        [T]_\mathcal{B}[\vec{v}]_\mathcal{B}&= [T\vec{v}]_\mathcal{B}&\text{By 1(b)} \\
                                            &= [\lambda \vec{v}]_\mathcal{B} \\
                                            &= \lambda[ \vec{v}]_\mathcal{B} &[\cdot ]_\mathcal{B}\text{ is linear }
    .\end{align*}

    $\supseteq$: Let $\tau$ be an eigenvalue of $[T]_B$ with associated eigenvector  $\vec{y}$. Since  $[\cdot ]_\mathcal{B}$ is an isomorphism, $\vec{y}$ has a unique preimage under the mapping, say $\vec{x}$ so that $[\vec{x}]_\mathcal{B}=\vec{y}$. Recall that $[\cdot ]_\mathcal{B}$ also must have an inverse. Denote this $[\cdot ]^{-1}_\mathcal{B}:M_{mn}(\mathbb{C})\to L(V,W)$. 
    \begin{align*}
        T\vec{x}&=[[T\vec{x} ]_\mathcal{B}]^{-1}_\mathcal{B}\\
                &= [[T]_\mathcal{B}[\vec{x}]_\mathcal{B}]_\mathcal{B}^{-1}&\text{Again by 1(b)} \\
                &=  [[T]_\mathcal{B}\vec{y}]_\mathcal{B}^{-1}\\
                &= [\tau\vec{y}]_\mathcal{B}^{-1} \\
                &= \tau[\vec{y}]_\mathcal{B}^{-1} &[\cdot ]_\mathcal{B} \text{ is linear }\\
                &= \tau\vec{x} 
    .\end{align*}
    Therefore $\sigma(T)=\sigma([T]_\mathcal{B})$.
    \newpage
\item Let $A$ be an $n \times n$ complex matrix with $\sigma(A) = \{1\}$. Show that $A$ is diagonalizable if and only if $A$ is the identity matrix.
    \paragraph{$\implies$}: Let $A$ be a diagonalizable matrix and $\sigma(A)=\{1\} $. Then there exists some invertible $S$ so that $S^{-1}AS=D=\text{diag}\{1,\ldots,1\}=I$. Multiply both sides:
    \begin{align*}
        S^{-1}AS&= I \\
        SS^{-1}ASS^{-1}&=SIS^{-1}\\
        A&=SS^{-1}\\
        A&= I
    .\end{align*}

    \paragraph{$\impliedby$:} Conversely, if $A=I$, then take the invertible matrix $I$, so that $IAI^{-1}=A=I$, and since $I$ is diagonal, $A$ is diagonalizable. 

\item Determine whether or not the derivative map $D : P_n \rightarrow P_n$ given by $Dp(z) = p'(z)$ is diagonalizable.
    \paragraph{Claim: }The $k+1$-th derivative of a polynomial of degree $k\in \mathbb{C}[x]$ is identically zero. 
    \begin{proof} [Proof of claim:]
        
  Proceed by induction on the degree of $p$. 
  \paragraph{Base case:} If $p$ has degree $0$, then $p$ is constant and has zero derivative, and as such, any subsequent derivative will be zero.
  \paragraph{Inductive hypothesis:} Suppose that the $k$-th derivative of any  $p \in \mathbb{C}[x]$ with 

  $\deg p=k-1$ is identically zero.
  \paragraph{Inductive step:} Let $f(x)\in \mathbb{C}[x]$ be of degree $k$. Write $f(x)=a_0+a_1x+\ldots+a_kx^{k}$ for complex $a_i$ . Then $Df(x)=a_1+2a_2+\ldots+ka_nx^{k-1}$. And since this polynomial $Df$ is of degree $k-1$, by the inductive hypothesis the $k$-th derivative of this must be zero, and \newline $D^{k+1}f=D(D^{k}f)=0$. Therefore by induction on $\deg f$, the derivative operator is nilpotent on $\mathbb{C}[x]$
    \end{proof}
  \paragraph{Solution: } The derivative map is nilpotent on $P_n$ for $n\geq 1$ (For $n=0$, $D^{1}=D=0$, a diagonal operator), and since nilpotent operators are not diagonalizable, the derivative operator is not diagonalizable for $n\geq 1$.

\end{enumerate}
\end{document}
