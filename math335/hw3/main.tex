\documentclass{article}
\usepackage[most,many,breakable]{tcolorbox}
\usepackage{amsmath}
\usepackage{amssymb}
\usepackage{amsthm}
\usepackage[]{thmbox}
\usepackage{blindtext}
\usepackage[utf8]{inputenc}
\usepackage{amsmath}
\usepackage{amsfonts}
\usepackage[]{graphicx}
\usepackage[legalpaper, portrait, margin = 1in]{geometry}
\usepackage{enumitem}


\usepackage{xcolor}

%\pagecolor[rgb]{0.2,0.19,0.18} 
%\color[rgb]{0.92,0.86,0.7}

\newtheorem[L]{le}{Lemma}[subsection]
\newtheorem[L]{th}[le]{Theorem}
\newtheorem[L]{df}[le]{Definition}
\newtheorem[L]{ex}[le]{Example}
\newtheorem[L]{pf}[le]{Proof}


\newcommand{\nl}{\newline}

\newcommand{\real}{\mathbb{R}}
\newcommand{\complex}{\mathbb{C}}
\newcommand{\integer}{\mathbb{Z}}
\newcommand{\rational}{\mathbb{Q}}
\newcommand{\lxor}{\oplus}
\newcommand{\then}{\Rightarrow}
\pagestyle{fancy}
\lhead{Assignment \# $3$}
\rhead{Name: Thomas Boyko; UCID: 30191728}
\chead{}
\makeqedsnowman

\begin{document}
\begin{enumerate} 

\item \begin{enumerate}
        \item Prove that the series $$\sum_{n=2}^\infty\frac1{(\log_2n)^{p(\log_2n)}}$$
        is convergent for all $p>1.$ Here $\log_2x$ denotes the logarithm base 2 of $x.$ You may assume
        that $\log_2n$ is increasing in $n.$

        \begin{proof} 
            We attempt to satisfy the criterion in Rudin Theorem 3.27. Rewrite the series; 
            %TODO Need to show that the general term is decreasing (denominator \ge 1)
            
            \begin{align*}
                \sum_{k=1}^{\infty} \frac{2^k}{\left(\log_2 2^k\right)^{p\log_2 2^k}}&= \sum_{k=1}^{\infty} \frac{2^{k}}{k^{pk}}\\
                &= \sum_{k=1}^{\infty} \left( \frac{2}{k^{p}} \right) ^{k} % Use theorem 3.25 to show that this is less than a p-series
            .\end{align*}
        \end{proof}

        \item For $a>0$ find the sum of the series
        $$\sum_{k=2}^\infty\left(\frac a{a+1}\right)^k\quad\text{(show your work)}$$

        \paragraph{Solution: }We notice a geometric series; since $a>0$,we can say $a<a+1$ and $\frac{a}{a+1}<1$.
        Then the sum is given by:
        \begin{align*}
            \left(\frac{a}{a+1}\right)^2 \frac{1}{1-\frac{a}{a+1}}
            &= \left(\frac{a}{a+1}\right)^2 \frac{1}{\frac{a+1}{a+1}-\frac{a}{a+1}}\\
            &= \left(\frac{a}{a+1}\right)^2 \frac{1}{\frac{1}{a+1}}\\
            &= \left(\frac{a}{a+1}\right)^2 (a+1)\\
            &=\frac{a^2}{a+1}
        .\end{align*}

    \end{enumerate}
    \newpage
\item \begin{enumerate}
        \item Prove that $f\left(x\right)=\sin\left(x^{2}\right)$ is not uniformly continuous in $[0,\infty).$
            $f$ is uniformly continuous on $E\subset X$ if and only if $\forall\varepsilon>0$, $\exists \delta>0$, $d(x,y)<\delta\implies d(f(x),f(y))<\varepsilon$

            $f$ is NOT uniformly continuous on $E\subset X$ if and only if $\exists\varepsilon>0$, $\forall  \delta>0$, we can choose $x,y$ so that $d(x,y)<\delta$ and $ d(f(x),f(y))\geq\varepsilon=1$
            \begin{proof} 

                %TODO epsilon = 1 may be simpler in order to bound delta
                Choose $\varepsilon=1$, and let $\delta>0$. Then we must choose $|x-y|<\delta$ but 
                $|\sin x^2-\sin y^2|\ge 1$

                We attempt to choose $x,y$ so that the function's value on one is $0$, and on the other
                is $\pm 1$. Then let $x^2=n\pi$ for some $n\in \mathbb{N}$, and $y^2= n\pi+\frac{\pi}{2}$,
                and our final choice is 
                \[
                x=\sqrt{n\pi},\quad y=\sqrt{n\pi+\frac{\pi}{2}}
                .\] 
                Then regardless of our choice of $n$, $|f(x)-f(y)|
                =|\sin(n\pi)-\sin\left(n\pi-\frac{\pi}{2}\right)|$. If $n$ is odd, then 
                $|\sin(n\pi)-\sin\left(n\pi-\frac{\pi}{2}\right)|=|\pm 1 -0|=1$, and if $n$ is even,
                $|\sin(n\pi)-\sin\left(n\pi-\frac{\pi}{2}\right)|=|0 -\pm 1|=1$

                We now have guaranteed that $|f(x)-f(y)|=1$ for any $n$, so now we must choose $n$
                so that $|x-y|<\delta$, for any given $\delta$. 
                \begin{align*}
                    |y-x|&=y-x&\text{Since }y>x\\
                    &= \sqrt{n\pi+\frac{\pi}{2}} -\sqrt{n\pi} \\
                    &= \frac{n\pi+\frac{\pi}{2}-n\pi}{\sqrt{n\pi+\frac{\pi}{2}} +\sqrt{n\pi}} \\
                    &= \frac{\frac{\pi}{2}}{\sqrt{n\pi+\frac{\pi}{2}} +\sqrt{n\pi}} \\
                    &< \frac{2}{\sqrt{n\pi+\frac{\pi}{2}} +\sqrt{n\pi}} &\text{Since }\frac{\pi}{2}<2\\
                    &< \frac{2}{\sqrt{n}\left(\sqrt{\pi+\frac{\pi}{2\sqrt{n}}} +\sqrt{\pi}\right)}\\
                    &< \frac{2}{\sqrt{n}\left(2\sqrt{\pi}\right)}\\
                    &< \frac{1}{\sqrt{n}}
                .\end{align*}
                So then choose $n>\frac{1}{\delta^2}$ (Which we can do by the Archimedian Property), 
                \begin{align*}
                    |y-x|&= y-x \\
                    &= \sqrt{n\pi+\frac{\pi}{2}} -\sqrt{n\pi}  \\
                    &< \sqrt{\frac{\pi}{\delta^2}+\frac{\pi}{2}} -\sqrt{\frac{\pi}{\delta^2}} \\
                .\end{align*}

            \end{proof}

        \item Show an example of a continuous function in $(0,1)$ which is not uniformly
            continuous (no proof necessary).

            \paragraph{Solution:} $f(x)=\sin\left( \frac{1}{x^2} \right) $ is continuous in $(0,1)$ however
            it is not uniformly continuous (as shown in class)
    \end{enumerate}

\end{enumerate}
\end{document}
