\documentclass{article}
\usepackage[most,many,breakable]{tcolorbox}
\usepackage{amsmath}
\usepackage{amssymb}
\usepackage{amsthm}
\usepackage[]{thmbox}
\usepackage{blindtext}
\usepackage[utf8]{inputenc}
\usepackage{amsmath}
\usepackage{amsfonts}
\usepackage[]{graphicx}
\usepackage[legalpaper, portrait, margin = 1in]{geometry}
\usepackage{enumitem}


\usepackage{xcolor}

%\pagecolor[rgb]{0.2,0.19,0.18} 
%\color[rgb]{0.92,0.86,0.7}

\newtheorem[L]{le}{Lemma}[subsection]
\newtheorem[L]{th}[le]{Theorem}
\newtheorem[L]{df}[le]{Definition}
\newtheorem[L]{ex}[le]{Example}
\newtheorem[L]{pf}[le]{Proof}


\newcommand{\nl}{\newline}

\newcommand{\real}{\mathbb{R}}
\newcommand{\complex}{\mathbb{C}}
\newcommand{\integer}{\mathbb{Z}}
\newcommand{\rational}{\mathbb{Q}}
\newcommand{\lxor}{\oplus}
\newcommand{\then}{\Rightarrow}
\pagestyle{fancy}
\lhead{Assignment \# $3$}
\rhead{Name: Thomas Boyko; UCID: 30191728}
\chead{}
\makeqedsnowman

\begin{document}
\begin{enumerate} 

\item \begin{enumerate}
        \item Prove that the series $$\sum_{n=2}^\infty\frac1{(\log_2n)^{p(\log_2n)}}$$
        is convergent for all $p>1.$ Here $\log_2x$ denotes the logarithm base 2 of $x.$ You may assume
        that $\log_2n$ is increasing in $n.$

        \begin{proof} 
            We compare with a $p$-series. %TODO show the general term is less than n^-p, however this may be difficult since log_2 n > n.. Perhaps by induction?
        \end{proof}

        \item For $a>0$ find the sum of the series
        $$\sum_{k=2}^\infty\left(\frac a{a+1}\right)^k\quad\text{(show your work)}$$

        \paragraph{Solution: }We notice a geometric series; since $a>0$,we can say $a<a+1$ and $\frac{a}{a+1}<1$.
        Then the sum is given by:
        \[
        \left(\frac{a}{a+1}\right)^2 \frac{1}{1-\frac{a}{a+1}}=
        \left(\frac{a}{a+1}\right)^2 \frac{1}{\frac{a+1}{a+1}-\frac{a}{a+1}}=
        \left(\frac{a}{a+1}\right)^2 \frac{1}{\frac{1}{a+1}}=
        \left(\frac{a}{a+1}\right)^2 (a+1)=\frac{a^2}{a+1}
        .\] 

    \end{enumerate}
\item \begin{enumerate}
        \item Prove that $f\left(x\right)=\sin\left(x^{2}\right)$ is not uniformly continuous in $[0,\infty).$

            $f$ is uniformly continuous on $E\subset X$ if and only if $\forall\varepsilon>0$, $\exists \delta>0$, $d(x,y)<\delta\implies d(f(x),f(y))<\varepsilon$

            $f$ is NOT uniformly continuous on $E\subset X$ if and only if $\exists\varepsilon>0$, $\forall  \delta>0$, we can choose $x,y$ so that $d(x,y)<\delta$ and $ d(f(x),f(y))\geq\varepsilon=1$
            \begin{proof} 

                Choose $\varepsilon=2$, and let $\delta>0$. Then we must choose $|x-y|<\delta$ but 
                $|\sin x^2-\sin y^2|=2$

                %TODO more clarity on choice of x,y
                We choose $x<y-\delta$, say $x^2=n\pi+\frac{\pi}{2}$ for $n\in \mathbb{N}$ so 
                $x=\sqrt{n\pi+\frac{\pi}{2}}$ (taking the positive root since we care only about the positive reals.
                Then we want $y^2-x^2=\pi$ so that the difference $\sin y^2-\sin x^2=2$.
                So choose $y^2= (n+1)\pi+\frac{\pi}{2}$, and then $y=\sqrt{(n+1)\pi+\frac{\pi}{2}} $.

                We now have guaranteed that $|f(x)-f(y)|=2$, and we must choose $n$ so that $|x-y|<\delta$ for any given
                $\delta$. Rewrite with the assumptions $y>x$, and with our expressions for $x,y$ above, so we may
                find an expression for $n$ in terms of $\delta$.
                \begin{align*}
                    y-x&<\delta\\
                    y^2+x^2-2yx&<\delta^2\\
                    (n\pi+\pi)+\frac{\pi}{2}-(n\pi+\frac{\pi}{2})-2yx&<\delta^2\\
                    \pi-2yx &<\delta^2\\
                    2yx&>\pi-\delta^2\\
                    2\sqrt{\left( n\pi+\frac{\pi}{2} \right)\left( (n-1)\pi+\frac{\pi}{2} \right)  } &>\pi-\delta^2\\
                .\end{align*}

            \end{proof}

        \item Show an example of a continuous function in $(0,1)$ which is not uniformly
            continuous (no proof necessary).

            \paragraph{Solution:} $f(x)=\sin\left( \frac{1}{x^2} \right) $ is continuous in $(0,1)$ however
            it is not uniformly continuous (as shown in class)
    \end{enumerate}

\end{enumerate}
\end{document}
