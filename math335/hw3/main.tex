\documentclass{article}
\usepackage[most,many,breakable]{tcolorbox}
\usepackage{amsmath}
\usepackage{amssymb}
\usepackage{amsthm}
\usepackage[]{thmbox}
\usepackage{blindtext}
\usepackage[utf8]{inputenc}
\usepackage{amsmath}
\usepackage{amsfonts}
\usepackage[]{graphicx}
\usepackage[legalpaper, portrait, margin = 1in]{geometry}
\usepackage{enumitem}


\usepackage{xcolor}

%\pagecolor[rgb]{0.2,0.19,0.18} 
%\color[rgb]{0.92,0.86,0.7}

\newtheorem[L]{le}{Lemma}[subsection]
\newtheorem[L]{th}[le]{Theorem}
\newtheorem[L]{df}[le]{Definition}
\newtheorem[L]{ex}[le]{Example}
\newtheorem[L]{pf}[le]{Proof}


\newcommand{\nl}{\newline}

\newcommand{\real}{\mathbb{R}}
\newcommand{\complex}{\mathbb{C}}
\newcommand{\integer}{\mathbb{Z}}
\newcommand{\rational}{\mathbb{Q}}
\newcommand{\lxor}{\oplus}
\newcommand{\then}{\Rightarrow}
\pagestyle{fancy}
\lhead{Assignment \# $3$}
\rhead{Name: Thomas Boyko; UCID: 30191728}
\chead{}
\makeqedsnowman

\begin{document}
\begin{enumerate} 

\item \begin{enumerate}
        \item Prove that the series $$\sum_{n=2}^\infty\frac1{(\log_2n)^{p(\log_2n)}}$$
        is convergent for all $p>1.$ Here $\log_2x$ denotes the logarithm base 2 of $x.$ You may assume
        that $\log_2n$ is increasing in $n.$

        \begin{proof} 
            We attempt to satisfy the criterion in Rudin Theorem 3.27. Rewrite the series; 
            and let $c_n:\mathbb{N}\to \mathbb{R}$:
            \[
            c_n=\begin{cases}
                0&n=1\\
                \frac1{(\log_2n)^{p(\log_2n)}}&n\ge 2
            \end{cases}
            .\] 
            Then our summation becomes
            \begin{align*}
                \sum_{n=2}^\infty\frac1{(\log_2n)^{p(\log_2n)}}&= 0+\sum_{n=2}^\infty\frac1{(\log_2n)^{p(\log_2n)}} \\
                &= \sum_{n=1}^{\infty} c_n
            .\end{align*}

            %TODO Need to show that the general term is decreasing to apply thm (denominator \ge 1)
            Now that our sum is indexed from $1$ and we have shown that the general term is decreasing,
            we can apply Rudin theorem 3.27. Our series of $c_n$ converges if and only if the following 
            series converges.
            \begin{align*}
                \sum_{k=0}^{\infty} 2^kc_k
                &= 0\cdot 2^0+  \sum_{k=1}^{\infty} \frac{2^k}{\left(\log_2 2^k\right)^{p\log_2 2^k}} \\
                &= \sum_{k=1}^{\infty} \left( \frac{2}{k^{p}} \right) ^{k} 
            .\end{align*}   
            %TODO want to use 3.25, show this is less than a convergent series 2/k^p

        \end{proof}

        \item For $a>0$ find the sum of the series
        $$\sum_{k=2}^\infty\left(\frac a{a+1}\right)^k\quad\text{(show your work)}$$

        \paragraph{Solution: }We notice a geometric series; since $a>0$,we can say $a<a+1$ and $\frac{a}{a+1}<1$.
        Then the sum is given by:
        \begin{align*}
            \left(\frac{a}{a+1}\right)^2 \frac{1}{1-\frac{a}{a+1}}
            &= \left(\frac{a}{a+1}\right)^2 \frac{1}{\frac{a+1}{a+1}-\frac{a}{a+1}}\\
            &= \left(\frac{a}{a+1}\right)^2 \frac{1}{\frac{1}{a+1}}\\
            &= \left(\frac{a}{a+1}\right)^2 (a+1)\\
            &=\frac{a^2}{a+1}
        .\end{align*}

    \end{enumerate}
    \newpage
\item \begin{enumerate}
        \item Prove that $f\left(x\right)=\sin\left(x^{2}\right)$ is not uniformly continuous in $[0,\infty).$
            \begin{proof} 

                Choose $\varepsilon=1$, and let $\delta>0$. Then let $k>\frac{1}{\delta^2}$ 
                which we can do by the Archimedian Property.

                We attempt to choose $x,y$ so that the function's value on one is $0$, and on the other
                is $\pm 1$. Then let $x^2=k\pi$ for some $k\in \mathbb{N}$, and $y^2= k\pi+\frac{\pi}{2}$, 
                and our final choice is 
                \[
                x=\sqrt{k\pi-\frac{\pi}{2}},\quad y=\sqrt{k\pi}
                .\] 
                Then regardless of our choice of $k$, 
                \[
                |f(x)-f(y)|=
                \left|\sin\left(\left(\sqrt{ k\pi}\right)^2\right)
                -\sin\left(\left(\sqrt{ k\pi-\frac{\pi}{2}}\right)^2\right)\right|
                =\left|\sin(k\pi)-\sin\left(k\pi-\frac{\pi}{2}\right)\right|.
                \] 
                If $n$ is odd, then 
                $|\sin(k\pi)-\sin\left(k\pi-\frac{\pi}{2}\right)|=|\pm 1 -0|=1$, and if $k$ is even,
                $|\sin(k\pi)-\sin\left(k\pi-\frac{\pi}{2}\right)|=|0 -\pm 1|=1$.

                We now have guaranteed that $|f(x)-f(y)|=1$ for any $k$. 
                It aids us to note that thanks to our choice of $k$, we can say that 
                $\frac{1}{k}<\delta^2$ and $\frac{1}{\sqrt{k} }<\delta$. So then we proceed
                on $|y-x|$.

                \begin{align*}
                    |y-x|&=y-x&\text{Since }y>x\\
                    &= \sqrt{\pi k} -\sqrt{\pi k-\frac{\pi}{2}}  \\
                    &= \frac{\left(\sqrt{\pi k} -\sqrt{\pi k-\frac{\pi}{2}}\right)
                    \left(\sqrt{\pi k} +\sqrt{\pi k-\frac{\pi}{2}}\right)  }
                    {\left(\sqrt{\pi k} +\sqrt{\pi k-\frac{\pi}{2}}\right) } \\
                    &= \frac{\pi k -\left({\pi k-\frac{\pi}{2}}\right)}
                    {\left(\sqrt{\pi k} +\sqrt{ \pi k-\frac{\pi}{2}}\right) } \\
                    &= \frac{\frac{\pi}{2}} {\left(\sqrt{\pi k} +\sqrt{ \pi k-\frac{\pi}{2}}\right) } \\
                    &= \frac{\pi} {2\left(\sqrt{\pi k} +\sqrt{ \pi k-\frac{\pi}{2}}\right) } \\
                    &= \frac{\pi} {2\sqrt{\pi} \left(\sqrt{k} +\sqrt{  k-\frac{1}{2}}\right) } \\
                    &< \frac{\sqrt{ \pi}} {2\left(\sqrt{k} +\sqrt{  k}\right) } \\
                    &<\frac{1}{4\sqrt{k} }\\
                    &< \frac{1}{\sqrt{k}}\\
                    &<\delta
                .\end{align*}
            \end{proof}

        \item Show an example of a continuous function in $(0,1)$ which is not uniformly
            continuous (no proof necessary).

            \paragraph{Solution:} $f(x)=\sin\left( \frac{1}{x^2} \right) $ is continuous in $(0,1)$ however
            it is not uniformly continuous (as shown in class).
    \end{enumerate}

\end{enumerate}
\end{document}
