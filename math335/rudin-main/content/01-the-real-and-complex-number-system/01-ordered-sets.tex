\section{Ordered Sets}\label{sec:ordered-sets}

Often when we talk about collections of things (people, cars, dogs, etc.) we
talk about how they compare to each other (height, top speed, cuteness
respectively). \emph{But}, we can only do this because we have a way in which
these objects relate to each other. My dog is \emph{of course} cuter than
yours, so I might say my dog is better than yours. Symbolically, I might write
this as
\begin{equation*}\label{eq:doges}
    \text{your dog} < \text{my dog}
\end{equation*}
where ``$<$'' can be read as ``is less cute than'' in this particular scenario,
but in others, it might mean ``has a lower top speed'' or really anything else
you can think of. We could even take all dogs and compare them in lots of
different ways, such as by weight, or tail length, or number of hairs, or\ldots

There are a lot of ways, but the idea is that with a collection of objects we
often like to talk about how they relate to each other and how we can compare
objects of this underlying collection or set. The following definition puts
this in terms we will use through the rest of the book.

\begin{definition}\label{def:order}
    If we let $S$ be a set, then an \emph{order} on $S$ is a relation, often
    denoted $<$, with two extra properties.
    \begin{itemize}
        \item If $x$ and $y$ are in $S$, then \emph{only one} of the following
              is true.
              \[ x < y, \qquad x = y, \qquad y < x\]
        \item If $x$, $y$ and $z$ are in $S$ and $x < y$ and $y < z$, then $x <
                  z$.
    \end{itemize}\todo{change bullets to match with rudin}
\end{definition}

Note that we did not define what the symbol $>$ means, but as is often done we
will use it because mathematics is nothing without some abuses of notation. If
we write $x > y$, take that to mean $y < x$, but instead you may read it as $x$
is ``greater than'' $y$ or $x$ is ``larger than'' $y$. Along with this
notation, we will use $x \le y$ to mean that $x$ is either less than $y$ or it
is equal to $y$, but we don't know which. Similarly with $\ge$.

While in English (and many other languages) we rely on context to understand
what set and order people are using when they talk, in mathematics we have are
damn pedantic. Hence the following definition.

\begin{definition}\label{def:ordered-set}
    An \emph{ordered set} is a set, together with an order defined on said set.
\end{definition}

Going back to the dogs, people often say ``Nate, you're dog is \emph{the
    cutest}'' which would imply, if taken literally, that there is no dog that is
cuter than mine. People love these kind of extremes. We have a whole book
dedicated to people who are the \emph{most} at something (The Guinness World
Records) and it comes out every year. We also have the Olympics to find more of
the \emph{most} people. The fastest person on land, the fastest person in
water, the fastest person on land/water/wheel (triathlon). We love this kind of
thing, and of course some people are also interested in the slowest.

\begin{definition}\label{def:upper-bound}
    Suppose we have an ordered set $S$, and a subset $E\subset S$. If there
    exists a $\beta\in S$ such that for every $x\in E$ we have $x\leq\beta$,
    then we say $E$ is \emph{bounded above}, and call $\beta$ an \emph{upper
        bound} of $E$.
\end{definition}

A \emph{lower bound} can be defined in the same way, but with $\geq$ instead of
$\leq$.

The upper and lower bounds are super helpful when discussing using relative
terms. However, sometimes we wish to be more even more specific in nailing down
what we are talking about and how it relates to other upper and lower bounds.

\begin{definition}\label{def:least-upper-bound}
    Start off with taking $S$ an ordered set, $E\subset S$, and $E$ bounded
    above. Additionally, there exists an $\alpha\in S$ where
    \begin{itemize}
        \item $\alpha$ is an upper bound of $E$.
        \item If $\gamma < \alpha$ then $\gamma$ is not an upper bound of $E$.
    \end{itemize}
    Then $\alpha$ is called the \emph{least upper bound} of $E$. The term
    \emph{supremum} is also used, and symbolically we write
    \[\alpha = \sup{E}\]
\end{definition}

Similarly, we can define the \emph{greatest lower bound} or \emph{infimum} by
taking $E$ to be a subset of $S$ which is bounded below, and $\gamma > \alpha$
means $\gamma$ is not a lower bound. This is similarly written as
\[\alpha = \inf{E}\]

These definitions take the upper/lower bounds we defined in \cref{def:upper-bound} and uses them to build a new concept which captures a
measure of ``most'' which we discussed earlier. Before continuing onto more
definitions lets take a look at some examples.

\begin{example}
    \todo{references section previous section which isn't finished yet}
\end{example}

In \cref{def:least-upper-bound} we defined a concept

\begin{definition}\label{def:least-upper-bound-prop}
    A set $S$ has the \emph{least upper bound property} if
    \begin{itemize}
        \item for every subset $E\subset S$, with
        \item $E\neq\emptyset$, and
        \item $E$ being bounded above
    \end{itemize}
    then $\sup{E}$ exists and is in $S$.
\end{definition}

Having seen so many definitions in this section that come two-for-one, you might
guess that we also have definition for \emph{greatest lower bound property}. Do
you see what we might change in the above definition to create the full
definition?

The following theorem is vitally important as it really serves as a lesson that
we see all the time in mathematics; two ideas that seem more or less different,
are really shown to be not so different after all. Coming across the two
definitions makes the ideas feel somewhat different, but as we will see, they
are indeed equivalent.

\begin{theorem}
    Every set with the least upper bound property, also has the greatest lower
    bound property, and vice-versa.
\end{theorem}

\begin{proof}
    Suppose $S$ has the least upper bound property, and also that there is a
    non-empty subset $B\subset S$ that is bounded below. Define $L$ to be the
    collection of all lower bounds of $B$ (everything that is less than all of
    $B$). Symbolically, $L = \{x\in S \mid \forall b\in B, x\leq b\}$.

    Since $B$ is bounded below, we know $L$ is not empty, and $L$ is bounded
    above by all $b\in B$. Since $L\subset S$, and $S$ has the least upper bound
    property, $L$ has a least upper bound. We write $\alpha = \sup{L}$ to say
    $\alpha$ is the supremum of $L$.

    Now suppose we have a $\gamma < \alpha$. By \cref{def:least-upper-bound}
    $\gamma$ cannot be a least upper bound, and since it is less than $\alpha$,
    it must not be in $B$ ($\gamma\notin B$). If anything less than $\alpha$ is
    not in $B$, it follows that everything in $B$ must be greater than or equal
    to $\alpha$. That is for all $b\in B$, $b \geq \alpha$. By our definition of
    $L$ above, $\alpha\in L$.

    If we take a $\beta > \alpha$, it is not in $L$, since $\alpha$ is an upper
    bound. We also know $\alpha$ is a lower bound of $B$ since it is in $L$.
    These two facts together show us that $\alpha$ is the greatest lower bound
    of $B$, since anything bigger is not a lower bound. Thus the infimum exists
    in $S$, and in particular $\alpha = \inf{B}$.

    Thus the least upper bound property implies the greatest lower bound
    property. Proofing the other way round is nearly the same argument with some
    symbols reversed. Try it out.
\end{proof}

Orders are used heavily in analysis, and more generally in all sorts of places
in mathematics. They even have their own field of research called Order Theory.
It's a pretty crazy place with lots of crazy things going on, and we've only
just begin touching the surface. If you enjoyed this section, go check out some
more.

