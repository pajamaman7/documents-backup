\documentclass{article}
\usepackage[most,many,breakable]{tcolorbox}
\usepackage{amsmath}
\usepackage{amssymb}
\usepackage{amsthm}
\usepackage[]{thmbox}
\usepackage{blindtext}
\usepackage[utf8]{inputenc}
\usepackage{amsmath}
\usepackage{amsfonts}
\usepackage[]{graphicx}
\usepackage[legalpaper, portrait, margin = 1in]{geometry}
\usepackage{enumitem}


\usepackage{xcolor}

%\pagecolor[rgb]{0.2,0.19,0.18} 
%\color[rgb]{0.92,0.86,0.7}

\newtheorem[L]{le}{Lemma}[subsection]
\newtheorem[L]{th}[le]{Theorem}
\newtheorem[L]{df}[le]{Definition}
\newtheorem[L]{ex}[le]{Example}
\newtheorem[L]{pf}[le]{Proof}


\newcommand{\nl}{\newline}

\newcommand{\real}{\mathbb{R}}
\newcommand{\complex}{\mathbb{C}}
\newcommand{\integer}{\mathbb{Z}}
\newcommand{\rational}{\mathbb{Q}}
\newcommand{\lxor}{\oplus}
\newcommand{\then}{\Rightarrow}
\usepackage{tikz-cd}
\pagestyle{fancy}
\lhead{Assignment \# $5$}
\rhead{Name: Thomas Boyko; UCID: 30191728}
\chead{}
\scsnowmandefault{nose=orange,arms,broom=brown,hat=red,buttons,muffler,scale=1}
\renewcommand\qedsymbol{\scsnowman}

\begin{document}

    \begin{enumerate}
\item The goal of this problem is to produce a (particular) proof that the cyclotomic polynomials for a prime $p$ are irreducible.   Let $p$ be a prime.  The $p$-th cyclotomic polynomial is $\Phi_p(x)=x^{p-1}+x^{p-2}+\cdots +x^2+x+1$.  Let $u=e^{2\pi i/p}$.  Let $m(x)$ be the minimal monic polynomial for $u$ in ${\mathbb Q}(u)$.  Do not assume $m(x)=\Phi_p(x)$.

\begin{enumerate}
\item A {\it primitive} $p$-th root of unity is a complex number $\zeta$ such that $\zeta^p=1$ and $\zeta^k\neq 1$ for any $k<p$.  Prove that $u$ is a primitive $p$-th root of unity.
    \begin{proof} 
        Let $u=e^{i \frac{2\pi}{p}}$, then $u^{p}=e^{ip \frac{2\pi}{p}}=e^{2\pi i}=1$

        Recall that the $n$-th roots of unity form a group $G$ under complex multiplication, with $|G|=n$.
        Now suppose $u^{d}=1$. Then $d|p$ since the order of the element divides the order of the group,
        and either $d=1$ (in this case $u=1$), or $d=p$. So then $u$ must have order $p$, and $u$
        is a primitive $p$-th root of unity. In fact, any non-identity element in $G$ is a primitive root.
    \end{proof}
\item Verify that each $u^k$ for $k=1, \ldots , p-1$ is a root of $\Phi_p(x)$
    \paragraph{Solution:} Let $k$ be as above, and observe:

    \begin{align*}
        \Phi_p(x)(x-1)=(x-1)(x^{p-1}+\ldots+x+1)=x^{p}-x^{p-1}+x^{p-1}-x^{p-2}+\ldots-x+x+1=x^{p}-1
    .\end{align*}

    And since $(u^{k})^{p}-1=e^{\frac{2kp\pi i}{p}}-1=e^{2ik\pi}-1=1-1=0$, and $u^{k}$ is a root of this 
    product of polynomials. But since the linear polynomial $x-1$ is irreducible and $u^k\neq 1$ 
    for any of the given $k$, $u^{k}$ cannot be a root of $x-1$ and it must instead be a root of the 
    $p$th cyclotomic polynomial.

\item Prove that for any prime $q\neq p$, $m(u^q)=0$.

         Skipped as per the news item on D2L

\item Conclude that $m(x)=\Phi_p(x)$ and that therefore $\Phi_p(x)$ is irreducible.
    \paragraph{Solution:} Since $\Phi_p$ has a root $u$, the minimal polynomial $m(x)$ must divide $\Phi_p$.
    And $m(x)$ by assumption is irreducible and monic, and since it has $u^{q}$ as a root
    it must be the minimal polynomial for $u^{q} $ as well as $u$. Then take  $u^{k}$ for some 
    $k=1,\ldots,p-1$. This has a prime factorization $k=q_1q_2\ldots q_m$ and by repeating (c) on  $u$ with
    each prime, we can see that $m(u^{k})=0$ and $m$ is minimal for $u^{k}$.
    But since every root of $\Phi_p$ is a root of $m$, $\Phi_p|m$.
    Since the two polynomials divide each other, they must differ by at most a constant multiple.
    And since they are both monic and minimal, we must have $m(x)=\Phi_p(x)$.

\end{enumerate}

\newpage
\item Exercise 6.4.13 If $E$ is an extension of ${\mathbb Z}_p$ and $u\in E$ is a root of $f(x)\in {\mathbb Z}_p[x]$, show that $u^p$ is also a root.
    \paragraph{Solution:} Let $u$ be a root of $f(x)\in \mathbb{Z}_p[x]$. Let $f$ have degree $n$,
    and write it as $f(x)=a_0+a_1x+\ldots+a_nx^{n}$, with $a_i \in \mathbb{Z}_p$.

    Then let $\sigma:E\to E$ be the automorphism that fixes $\mathbb{Z}_p$ and has
    $\sigma(u)=u^{p}$. We know this automorphism to commute with polynomial functions;
    \begin{align*}
        f(\sigma(u))&= a_0+a_1\sigma(u)+\ldots+a_n\sigma(u)^{n}\\
        &=  \sigma(a_0)+\sigma(a_1)\sigma(u)+\ldots+\sigma(a_n)\sigma(u)^{n} \\
        &=  \sigma(a_0)+\sigma(a_1u)+\ldots+\sigma(a_nu^{n}) \\
        &=  \sigma(a_0+a_1u+\ldots+a_nu^{n}) \\
        &=  \sigma(f(u))\\
        &=\sigma(0)\\
        &=0
    .\end{align*}
    So $f(\sigma(u))=\sigma(f(u))=\sigma(0)=0$.

\item Exercise 10.1.8: If $E= {\mathbb Q}(\sqrt{2}, \sqrt{3} )$, show that $Gal(E:{\mathbb Q})\cong C_2\times C_2$.

    \paragraph{Solution:} First find the minimal monic polynomials. $x^2-2$ for $\sqrt{2} $ is 
    irreducible over $\mathbb{Q}$ by the quadratic equation (Its roots $\pm \sqrt{2} $ are real). For the 
    same reason $x^2-3$ is minimal for $\sqrt{3}$, it has roots $\pm\sqrt{3} $. So we have 
    four roots to permute, which tells us that our group must be either $C_2\times C_2$ or $C_4$ thanks
    to our classification of finite groups.
    %Mention fund. theorem arith? TODO

    Let $\sigma \in Gal(E:\mathbb{Q})$ such that $\sigma(\sqrt{2})=\sqrt{3}$ and 
    $\sigma(\sqrt{3} )=\sqrt{2} $. Then $\sigma^2(\sqrt{2} ) =\sigma(\sqrt{3})=\sqrt{2}$
    and $\sigma^2(\sqrt{3} )=\sigma(\sqrt{2} )=\sqrt{3} $. So $\sigma\cdot \sigma=\varepsilon$, and
    the order of $\sigma$ is $2$.

    Then pick $\tau$ so that $\tau(\sqrt{2})=-\sqrt{2} $ and $\tau(\sqrt{3})=\sqrt{3} $. Note that we
    can say that $\tau$ is distinct from $\sigma$ since they are uniquely determined by their action
    on $\sqrt{2} ,\sqrt{3} $. Then $\tau^2(\sqrt{2})=\tau(-\sqrt{2} ) =-\tau(\sqrt{2} )=\sqrt{2} $, and 
    $\tau^2(\sqrt{3})=\tau(\sqrt{3} )=\sqrt{3} $. So the order of $\tau$ is 2, which tells us the Galois
    group must be $C_2\times C_2$ since it has two distinct elements of order 2, which $C_4$ does not.

\end{enumerate}


\end{document}
