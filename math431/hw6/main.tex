\documentclass{article}
\usepackage{amsmath}
\usepackage{amssymb}
\usepackage{amsthm}
\usepackage[utf8]{inputenc}
\usepackage{amsmath}
\usepackage{amsfonts}
\usepackage[]{graphicx}
\usepackage[a4paper, portrait, margin = 1in]{geometry}
\usepackage{enumitem}
\usepackage{xcolor}

%darkmode
%\pagecolor[rgb]{0.2,0.19,0.18} 
%\color[rgb]{0.92,0.86,0.7}

\newenvironment*{alphenum}{\begin{enumerate}[label= (\alph*)]}{\end{enumerate}}

\pagestyle{fancy}
\lhead{Assignment \# $6$}
\rhead{Name: Thomas Boyko; UCID: 30191728}
\chead{}

\begin{document}
\begin{enumerate} 

\item Exercise 10.1.20. Let $\mathbb{F}$ be a field.
    \begin{enumerate}
        \item Show that the following are equivalent for a polynomial $f (x) \in  F[x].$
            \begin{enumerate}
                \item $f (x)$ has no repeated root in any extension field of $\mathbb{F}$.
                \item $f (x)$ has no repeated root in some splitting field over $\mathbb{F}$.
                \item $f (x)$ and $f '(x)$ are relatively prime in $\mathbb{F}[x]$
                    \paragraph{i.$\implies$ ii.} Suppose $f$ has no repeated root in any extension 
                    of $\mathbb{F}$. $f$ has a splitting field, and by assumption it must have no repeated
                    roots in this field. 
                    \paragraph{ii.$\implies$ iii.} Suppose $f$ has no repeated roots in an extension
                    $\mathbb{E}$ in which it splits. Then suppose for the sake of 
                    contradiction that there exists some $g\in \mathbb{F}[x]$ so that $g|f$ and $g|f'$,
                    and that $g$ is nonconstant. By Kronecker's Theorem, take a root $\alpha\in \mathbb{E}$
                    of $g$. Then $x-a|g$ and so $x-\alpha|f$, $x-\alpha|f'$. But if $x-\alpha|f'$, then
                    $(x-\alpha)^2|f$, a contradiction since we assumed that $f$ had no repeated root.

                    \paragraph{iii. $\implies$ i. } Suppose that $f,f'$ are relatively prime in
                    $\mathbb{F}[x]$. Then suppose for the sake of contradiction that there is some
                    extension $\mathbb{E}$ of $\mathbb{F}$ so that $f$ has a repeated root in $\mathbb{E}$.
                    Then $(x-\alpha)^2|f$. But then $(x-\alpha)$ would divide $f'$, contradicting
                    $f,f'$ being coprime. 
                    %TODO was this an iff??
                    
            \end{enumerate}   
            \item If $f (x)$ is as in (a), show that $f (x)$ is separable, but not conversely.

              \paragraph{Solution:} Let $f$ have no repeated roots in any extension $\mathbb{E}$ 
              of $\mathbb{F}$. 
              %TODO is this not the condition for seperability? if not what to use?

              Counterexample: 
    \end{enumerate}
\item Exercise 10.1.26 (a) (b)
    \begin{enumerate}
        \item Show that the following conditions are equivalent for a field $\mathbb{F}$ 
            (then called a perfect field):
        \begin{enumerate}
            \item Every algebraic extension of $\mathbb{F}$ is separable.
            \item Every finite extension of $\mathbb{F}$ is separable.
            \item Every irreducible polynomial in $\mathbb{F}[x]$ is separable.
        \end{enumerate}

          \paragraph{i. $\implies$ ii. } Suppose that every algebraic extension of $\mathbb{F}$ 
          is separable. Then in particular, if $\mathbb{E}$ were a finite extension, it would have to be
          algebraic and as such it would be separable.
          \paragraph{ii. $\implies$ iii. } Suppose that every extension of $\mathbb{F}$ is separable, and
          that $f$ is irreducible in $\mathbb{F}[x]$. % TODO

          \paragraph{iii. $\implies$ i. } Suppose that every irreducible polynomial in $\mathbb{F}[x]$ is 
          separable. Then let $\mathbb{E}$ be an algebraic extension of $\mathbb{F}$. 


        \item Show that every field of characteristic 0 is perfect.
          \paragraph{Solution:} Let $f$ be of characteristic $0$. Then an irreducible $p$ is separable 
          (Nicholson Chapter 10, Theorem 4), satisfying iii. Therefore $\mathbb{F}$ is perfect.
      \end{enumerate}
\item Exercise 10.2.12 If $\mathbb{E}$ is a finite extension of $\mathbb{F}$ and 
    $G = gal( \mathbb{E} : \mathbb{F})$, show that the extension $E$ of $F$ is Galois if and only if 
    $|G| = [\mathbb{E} : \mathbb{F}]$.
    \paragraph{$\implies$:} Let $\mathbb{E}$ be a finite extension of $\mathbb{F}$ and 
    $G=gal(\mathbb{E}:\mathbb{F})$. 
\end{enumerate}
\end{document}
