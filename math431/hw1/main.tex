\documentclass{article}
\usepackage{amsmath}
\usepackage{amssymb}
\usepackage{amsthm}
\usepackage[utf8]{inputenc}
\usepackage{amsmath}
\usepackage{amsfonts}
\usepackage[]{graphicx}
\usepackage[a4paper, portrait, margin = 1in]{geometry}
\usepackage{enumitem}
\usepackage{xcolor}

%darkmode
%\pagecolor[rgb]{0.2,0.19,0.18} 
%\color[rgb]{0.92,0.86,0.7}

\newenvironment*{alphenum}{\begin{enumerate}[label= (\alph*)]}{\end{enumerate}}

\title{Assignment 1}
\author{Thomas Boyko - 30191728}

\begin{document}
\maketitle
\begin{enumerate}
\item Exercises 8.3 \#14. Let $X= \mathbb{R}[x_1, \ldots , x_n]$ denote the set of all polynomials over $\mathbb{R}$ in the indeterminates $x_1, \ldots, x_n$.  If $\sigma\in S_n$ and $f=f(x_1, \ldots , x_n)\in X$, define $\sigma\cdot f=f(x_{\sigma 1}, \ldots , x_{\sigma n})$.  Show that this is an action and describe the fixer. If $n=3$, give three polynomials in the fixed set and compute $S_3\cdot g$ and $S(g)$, where $g(x_1, x_2, x_3)=x_1+x_2$.

    \paragraph{Solution: }Begin by checking axioms for a group action. Let $\varepsilon$ be identity in $S_{n}$, and let $f\in X$. Then
    \begin{align*}
        \varepsilon\cdot f&=f(x_{\varepsilon(1)},\ldots,x_{\varepsilon(n)})\\
                          &=f(x_1,\ldots,x_n)
    .\end{align*}

    Then let $\sigma,\tau\in S_n$. 
    \begin{align*}
        \sigma\cdot(\tau(f))&= \sigma\cdot(f(x_{\tau(1)},\ldots,x_{\tau(n)}) \\
        &=  f(x_{\sigma(\tau(1))},\ldots,x_{\sigma(\tau(n))} \\
        &= (\sigma\circ\tau)\cdot f
    .\end{align*}

    Claim: The fixer of $X$ is exactly. $\{\varepsilon\} $.

    Clearly $\varepsilon$ fixes any polynomial in $X$. Then let $\sigma$ be in the fixer, and $f\in S_n$.
    Then $\sigma\cdot f=f(x_{\sigma 1}, \ldots , x_{\sigma n})=f(x_1,x_2,\ldots,x_n)$. Then $\sigma(k)=k$ for any $k\in \{1,2,\ldots ,n\} $ and $\sigma=\varepsilon$.

    The constant polynomials $\{0,1,2\} $ are all clearly fixed by every element of  $S_n$, since there are no invariants to move around.
    
    Letting $g(x_1,x_2,x_3)=x_1+x_2$, we find $S_3\cdot g=\{x_1+x_2,x_1+x_3,x_2+x_3 \}$. We may have exactly two distinct invariants out of the total three, 
    without having to worry about the order in which they appear, so can be sure that we have found them all by noting ${\binom{3}{2}}=3$.

    Now we find $Stab(g)$. Thanks to the commutativity of real addition, we do not have to fix  $x_1,x_2$. However it's necessary to fix
    $x_3$. So $Stab(g)=\left\{\varepsilon,\begin{pmatrix} 1\,2\,3\\2\,1\,3 \end{pmatrix} \right\} $

\item Exercises 8.3 \#24.  Let $X$ be a $G$-set with just one orbit (called a transitive action).
\begin{enumerate}
\item If $K\triangleleft G$, show that $K\subseteq S(x)$ for some $x\in X$ if and only if $K$ is contained in the fixer.

    \paragraph{Solution: }
        Let $X$ be a $G$-set with only one orbit. Suppose $K\trianglelefteq G$.
        \paragraph{$\implies$:} Suppose $K\subseteq S(x)$ for some $x\in X$. Then let $k\in K$. Since $k\in S(x)$, $k\cdot x=x$. Let $y \in X$. Then $y$ must be in the orbit
        of $x$, since $X$ has only one orbit. So there exists some $g\in G$ so that $g\cdot x=y$. 

        Take $g^{-1}kg\cdot x$. By normality of $K$, and since the thing we are acting by is in $K$, contained within the stabilizer of $x$, this is simply $x$. 
        So 
        \begin{align*}
            x&=g^{-1}kg\cdot x\\
            g\cdot x&= kg\cdot x \\
            y&= k\cdot y 
        .\end{align*}
        So we see that $K$ is contained in the fixer.

        \paragraph{$\impliedby$:} Suppose $K$ is contained in the fixer. Then let $k\in K,x\in X$. We know then that $k\cdot x=x$, so clearly $k\in Stab(x)$
        and $K\subseteq Stab(x)$ for any $x\in X$.

\item If $|X|>2$, show that $g\in G$ exists such that $g\cdot x\neq x$ for all $x\in X$.
    \paragraph{Solution: }
    Nicholson points us to exercise \#10 in section 8.2, which points us to Theorem 2 in the same section.

    Exercise 10 asks us to show that for a finite group, $G\neq\bigcup_{a\in G}aHa^{-1}$, and hints us to use the fact that for a subgroup $H$ of a finite $G$,
    the number of conjugates of $H$ equals $[G:N(H)]$ (This is Theorem 2 from chapter 8.2).

    #TODO solve Ex10

    We take Exercise 10 as fact for the time being, and come back to it if I have time after finishing this assignment.

    Let $X$ be a $G$-set with only one orbit. We begin by showing that  $Stab(x)$ is conjugate to $Stab(y)$ for any $x,y\in X$.
    Since $x,y$ are in the same orbit, there exists some $g\in G$ so that $g\cdot x=y$, and it follows that $g^{-1}\cdot y=x$.
    Then we show that $h\in Stab(y)\iff h\in g^{-1}Stab(y)g$.
    \[
    h\in Stab(y)\iff h\cdot y=y\iff hg\cdot x= g^{-1}hg\cdot x=g^{-1}\cdot y=x\iff g^{-1}hg\in Stab(x)
    .\] 
    So every stabilizer subgroup is conjugate in $G$. Now we can use Exercise 10 to say for some fixed $x_0\in X$, we have.
    $\bigcup_{a\in G} aStab(x_0)a^{-1}=\bigcup_{x\in X} Stab(x)\neq G$. So there is some $g\in G$ that is not in any stabilizer, or
    there exists some $g\in G$ so that $g\cdot x\neq x$ for all $x\in X$.

\end{enumerate}
\item Exercises 8.4 \#1 Find all Sylow $3$-subgroups of $S_4$ and show explicitly that they are all conjugate.

    \paragraph{Solution: }
    The Sylow 3-subgroups are as follows. The order of each cycle is 3, and so the order
    of the generated subgroup is 3.
    \[ \left<(1\,2\,3)\right>,\left<(1\,3\,4)\right>,\left<(1\,2\,4)\right>,\left<(2\,3\,4)\right> .\] 
    Recall that conjugation preserves cycle structure and that conjugacy is an equivalence relation. 
    So we need only show that each subgroup is conjugate to another, or that each generator is conjugate to another.

    \begin{align*}
        (3\,4)(1\,2\,3)(3\,4)&= (1\,2\,4) \\
        (2\,3)(1\,2\,4)(2\,3)&= (1\,3\,4) \\
        (1\,2)(1\,3\,4)(1\,2)&= (2\,3\,4)
    .\end{align*}
    We can confirm that we have found every such subgroup by checking Sylow's third theorem. Let $n_3$ be the number of 
    Sylow 3-subgroups. Then since $|S_4|=24=3\cdot 8$:
    \begin{align*}
        n_3&\equiv 1\pmod{3} \\
        n_3 &|8
    .\end{align*}
    So we can have 1 or 4 Sylow 3-subgroups according to this rule. 

\item Exercises 8.3 \#1(b) If $|G|=28$, show that $G$ has a normal subgroup of order $7$.

    \paragraph{Solution: } Write $|G|=p^{k}m=7^1 4$, and denote the number of Sylow $7$-subgroups as $n$.
    Then, by Sylow's third theorem, we know that $n\equiv 1\pmod{7} $, and that $n|4$. The only $n$ that satisfies both of these is $n=1$.
    So there is exactly one Sylow $7$-subgroup, which by Sylow's second theorem, means this subgroup is normal in $G$.

\item If $N(P) = P$ for some Sylow $p$-subgroup $P$ of $G$, show that $N(Q) = Q$ for every Sylow $p$-subgroup $Q$ of $G$.

    \paragraph{Solution: } Begin by noting that $Q=gPg^{-1}$ for some $g\in G$, by Sylow's second theorem. Likewise, $N(Q)=N(gPg^{-1})$.

    Since $N(P)=P$, we can also say that $Q=gN(P)g^{-1}$. Now, if we can show that $N(gPg^{-1})=gN(P)g^{-1}$, we can show that $N(Q)=Q$.
    \begin{align*}
        h\in N(gpg^{-1})&\iff h(gPg^{-1})h^{-1}=gPg^{-1} &\text{By the definition of }N(gPg^{-1})\\
                        &\iff g^{-1}hgP(g^{-1}h^{-1}g=P&\text{Cancelling our RHS to get a lone }P\\
                        &\iff (g^{-1}hg)P(g^{-1}hg)^{-1}&\text{Rewriting}\\
                        &\iff g^{-1}hg\in N(P)\\
                        &\iff h\in gN(P)g^{-1} &\text{Coset logic}
    .\end{align*}
    So $Q=gPg^{-1}=gN(P)g^{-1}=N(gPg^{-1})=N(Q)$
\end{enumerate}
\end{document}
