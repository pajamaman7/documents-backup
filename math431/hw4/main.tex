\documentclass{article}
\usepackage[most,many,breakable]{tcolorbox}
\usepackage{amsmath}
\usepackage{amssymb}
\usepackage{amsthm}
\usepackage[]{thmbox}
\usepackage{blindtext}
\usepackage[utf8]{inputenc}
\usepackage{amsmath}
\usepackage{amsfonts}
\usepackage[]{graphicx}
\usepackage[legalpaper, portrait, margin = 1in]{geometry}
\usepackage{enumitem}


\usepackage{xcolor}

%\pagecolor[rgb]{0.2,0.19,0.18} 
%\color[rgb]{0.92,0.86,0.7}

\newtheorem[L]{le}{Lemma}[subsection]
\newtheorem[L]{th}[le]{Theorem}
\newtheorem[L]{df}[le]{Definition}
\newtheorem[L]{ex}[le]{Example}
\newtheorem[L]{pf}[le]{Proof}


\newcommand{\nl}{\newline}

\newcommand{\real}{\mathbb{R}}
\newcommand{\complex}{\mathbb{C}}
\newcommand{\integer}{\mathbb{Z}}
\newcommand{\rational}{\mathbb{Q}}
\newcommand{\lxor}{\oplus}
\newcommand{\then}{\Rightarrow}
\pagestyle{fancy}
\lhead{Assignment \# $4$}
\rhead{Name: Thomas Boyko; UCID: 30191728}
\chead{}

\begin{document}
\begin{enumerate} 

    \item Exercise 6.2 \#1(b) Show that $u=1+\sqrt{1+\sqrt[3]{2}}$ is algebraic over $\mathbb{Q}.$

        \paragraph{Solution:} Begin with algebraic trickery.
        \begin{align*}
            u&=1+\sqrt{1+\sqrt[3]{2}}\\
            u-1&= \sqrt{1+\sqrt[3]{2}} \\
            (u+1)^2&= 1+\sqrt[3]{2}  \\
            u^2+2u&= \sqrt[3]{2}  \\
            (u^2+2u)^3&= 2 \\
            u^6 + 6 u^5 + 12 u^4 + 8 u^3&= 2 \\
            u^6 + 6 u^5 + 12 u^4 + 8 u^3-2&= 0 \\
        .\end{align*}
        So we have found a polynomial $f(x)= x^6 + 6 x^5 + 12 x^4 + 8 x^3-2$ so that $f(u)=0$.

\item Exercise 6.2 \#7 Find the minimal polynomial of $u=\sqrt{3}-i$ over $\mathbb{Q}$ and also over
$\mathbb{R}.$

\paragraph{Solution: }Use the same strategy as above:
\begin{align*}
    u&= \sqrt{3}-i  \\
    u^2&= 3-1-2\sqrt{3} i \\
    u^2-2&= -2\sqrt{3} i \\
    u^{4}-4u^2+4&= -12 \\
    u^{4}-4u^2+16&= 0 
.\end{align*}

%TODO show irreducibility over Q and reduce over R then show that it is irreducible there.

\item Exercise 6.2 \#20 Let $\mathbb{K}$ be a field extension of $\mathbb{E}$ which is a field extension of F,
    and let $[\mathbb{E}:\mathbb{F}]$ be finite. Let $u\in\mathbb{K}$ be algebraic over E.
\begin{enumerate}              
    \item Show that [E($u):\mathbb{E}]\leq[\mathbb{F}(u):\mathbb{F}].$
        \paragraph{Solution:} Let $\mathbb{K}\supseteq \mathbb{E}\supseteq \mathbb{F}$, and $[\mathbb{E}:\mathbb{F}]=n$ be finite. Then let $u$ be algebraic over $\mathbb{K}$ with 
        minimal polynomial $m(x)\in \mathbb{F}[x]$. Now we proceed by cases. If $m$ is irreducible in $\mathbb{E}$, then
        it remains the minimal polynomial for $u$ in $\mathbb{E}$, and the degrees of the two extensions are equal.

        Now, if $m$ is reducible in $\mathbb{E}$, then there exists $f,g\in \mathbb{E}[x]$, both with degree less than $m$, and $m=fg$.
        We take the one which has $u$ as a root and repeat the process until we reach an irreducible polynomial $h\in  \mathbb{E}(u)$ with $u$ as a root.
        If this is not monic, we factor out the leading coefficient, and we will have a $h'\in \mathbb{E}(u)$ with $h'(u)=0$ that is monic and irreducible. This has degree
        strictly less than $m$, and is minimal for $u$.

        After all this we have $[\mathbb{E}(u):\mathbb{E}]=\deg  h' \leq\deg m=[\mathbb{F}(u):\mathbb{F}]$.
        %CHECK correct, check this over?

    \item Show that $[\mathbb{E}(u):\mathbb{F}(u)]\leq[\mathbb{E}:\mathbb{F}].$ (Hint: Theorem 6.1.6.)
        %6.1.6 discusses the size of a basis, didn't need this..
        Take the same minimal polynomials we found above; and rewrite both:
        \begin{align*}
            [\mathbb{E}(u):\mathbb{F}(u)]&=[\mathbb{E}(u):\mathbb{E}][\mathbb{E}:\mathbb{F}(u)]\\
                                         &=[\mathbb{E}(u):\mathbb{E}]\frac{[\mathbb{E}:\mathbb{F}]}{[\mathbb{F}(u):\mathbb{F}]}\\
                                         &= \deg h' \frac{n}{\deg m} 
        .\end{align*}
        Then since $\deg h' \leq \deg m$, $\frac{\deg h'}{\deg m}\leq 1$. So then $[\mathbb{E}(u):\mathbb{F}(u)]=\frac{n\deg h'}{\deg m}\leq n=[\mathbb{E}:\mathbb{F}].$
\end{enumerate}

\item Exercise 6.3 \#4(a) and 4(b). Find the splitting field $\mathbb{E}$ of $f(x)=x^3+1$ over $\mathbb{F}=\mathbb{Z}_2$
and factor $f(x)$ completely in $\mathbb{F}[x].$ Then, do the same thing but replace $\mathbb{F}=\mathbb{Z}_2$
with $\mathbb{F}=\mathbb{Z}_3$ (see the statement in the textbook).

\paragraph{Solution:} For $\mathbb{Z}_2$, we check the elements. $f(0)=0^3+1=1$, and $f(1)=1^3+1\equiv 0$, so $1$ is a root of the polynomial.
Rewrite $f(x)=(x+1)(x^2+x+1)$. Then let $f'(x)=x^2+x+1$, and check that this is also irreducible in $\mathbb{Z}_2$:
\[
f'(0)=0^2+0+1= 1\neq 0,\quad f'(1)=1^2+1+1=1\neq 0
.\] 
So this polynomial has no roots in $\mathbb{Z}_2$ and therefore is irreducible. Let $\alpha$ be such that $f'(\alpha)=0$. Then there extists (By Kronecker's Theorem) a field extension
of $F$ in which $\alpha$ is a root, and $\alpha^2+\alpha\equiv  1\pmod{2} $.
So we can factor $f'(x)=(x+\alpha)(x+\alpha+1)$. And so $f(x)=(x+1)(x+\alpha)(x+\alpha+1)$, so $x$ splits over $\mathbb{Z}_2(\alpha)$.
And since  $f'$ is monic and irreducible, it is the minimal polynomial for $\alpha$. Having degree $2$, we can say that $[\mathbb{Z}_2(\alpha):\mathbb{Z}_2]=2$, and since
$|\mathbb{Z}_2|=2$, by the multiplication theorem $|\mathbb{Z}_2(a)|=4$. Then we can finally say by the characterization of finite fields that
%CHECK better explanation for where \alpha came from?
$\mathbb{Z}_2(a)\cong \mathbb{F}_4$.

Now, in $\mathbb{Z}_3$ we can see that $f(2)=9\equiv 0\pmod{3} $, so $2\equiv -1$ is a root of $f$. Rewrite, $f(x)=(x+1)(x^2+2x+1)=(x+1)^3$. So the splitting field
for $f$ over $\mathbb{Z}_3$ is $\mathbb{Z}_3$.

\item Exercise 6.3 \#9 Let $f(x)$ and $g(x)$ be polynomials in $F[x].$ Show that $f(x)$ and $g(x)$
are relatively prime (have no common nonconstant factors) in $F[x]$ if and only if
they have no common root in any extension $E$ of $F.$

\paragraph{Solution: }
%CHECK this is wrong i think... The extension is not necessarily algebraic? which one might be?
\paragraph{$\implies$: }
Let $f,g\in F[x]$ be coprime, and suppose for the sake of contradiction that they have a common root $a$ in some extension $E$ of $F$.
Then $f(x)=f'(x)(x-a)$ and $g(x)=g'(x)(x-a)$. Then  $x-a$ is a common nonconstant factor, a contradiction!
Therefore $f,g$ must have no common factor in $F[x]$.
\paragraph{$\impliedby:$ }
Suppose that $f,g $ share no root in any $E\supseteq F$. Then suppose for the sake of contradiction that there exists some $h\in F[x],$ and $g=g'h$, $f=f'h$.
This polynomial must have a root in some extension of $F$, say $u\in K$. Then $g(u)=g'(u)h(u)=0=f'(u)h(u)=f(u)$. So $u$ is a root of both $f,g$, a contradiction. Then by contradiction
$f,g$ must be coprime. %Are they coprime in F or E?

\item Exercise 6.3 \#17 If $E$ over $F$ is an algebraic extension and every polynomial in
$F[x]$ splits over $E$, show that $E$ is algebraically closed. (Hint: Theorem 6.2.6)

\paragraph{Solution: }Let $E\supseteq F$ be an algebraic extension, where every polynomial in $F[x]$ splits over $E$. Then suppose for the sake of contradiction that 
$E$ is not algebraically closed. So there exists some $f(x)\in E[x]$, where $f$ has degree greater than one, and is irreducible. Write $f(x)=a_{0}+a_1x+\ldots+a_nx^{n}$,
with each $a_i\in E$. Take the field extension $F(a_0,\ldots,a_n)=F(a_0)(\ldots)(a_n)$. Since $E$ is an algebraic extension, each $a_i$ has a finite degree monic polynomial, 
each adjoined element on $F$ produces another finite degree extension (Theorem 6.2.6). Since this extension is finite, it must be algebraic. 
%TODO why does this imply there is some m in F[x]? what would be the contradiciton

\end{enumerate}
\end{document}
