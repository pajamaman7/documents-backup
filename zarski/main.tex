\documentclass{article}
\usepackage[most,many,breakable]{tcolorbox}
\usepackage{amsmath}
\usepackage{amssymb}
\usepackage{amsthm}
\usepackage[]{thmbox}
\usepackage{blindtext}
\usepackage[utf8]{inputenc}
\usepackage{amsmath}
\usepackage{amsfonts}
\usepackage[]{graphicx}
\usepackage[legalpaper, portrait, margin = 1in]{geometry}
\usepackage{enumitem}


\usepackage{xcolor}

%\pagecolor[rgb]{0.2,0.19,0.18} 
%\color[rgb]{0.92,0.86,0.7}

\newtheorem[L]{le}{Lemma}[subsection]
\newtheorem[L]{th}[le]{Theorem}
\newtheorem[L]{df}[le]{Definition}
\newtheorem[L]{ex}[le]{Example}
\newtheorem[L]{pf}[le]{Proof}


\newcommand{\nl}{\newline}

\newcommand{\real}{\mathbb{R}}
\newcommand{\complex}{\mathbb{C}}
\newcommand{\integer}{\mathbb{Z}}
\newcommand{\rational}{\mathbb{Q}}
\newcommand{\lxor}{\oplus}
\newcommand{\then}{\Rightarrow}
\pagestyle{fancy}
\lhead{The Zarski Topology}
\rhead{Thomas Boyko}
\chead{}

\usepackage{thmbox}
\hfuzz=8pt

\theoremstyle{plain}% default
\newtheorem[L]{thm}{Theorem}[section]
\newtheorem[L]{lem}[thm]{Lemma}
\newtheorem[L]{prop}[thm]{Proposition}
%\newtheorem*{cor}{Corollary}
\theoremstyle{definition}
\newtheorem[M]{defn}{Definition}[section]
\newtheorem[M]{exmp}{Example}[section]
\theoremstyle{remark}
%\newtheorem*{rem}{Remark}
%\newtheorem*{note}{Note}
\newtheorem{case}{Case}

\begin{document}
\section{The Spectrum and its Topology}
\begin{defn}[Spectrum of a Ring]
Let $R$ be a ring. Define the spectrum of $R$,
\[Spec(R)=\{\mathfrak{p}\trianglelefteq R: \mathfrak{p} \text{ is prime}\} .\] 
\end{defn}
\begin{defn}
    Upon $Spec(R)$ we define a topology by:
    \[ \mathcal{T}=V(I)=\{\mathfrak{p}\in Spec(R):I\subseteq \mathfrak{p}\} .\] 
\end{defn}
\begin{prop}
    $V(I)$ induces a topology on $Spec(R)$, define:
    \[
    \mathcal{T}=\{V(I):I\trianglelefteq R\} 
    .\] 
    \begin{enumerate}
        \item $\varnothing,Spec(R)\in \mathcal{T}$
        \item $\bigcup_{i=1}^{n}V(I_i)\in \mathcal{T}$
        \item $\bigcap_{\alpha}^{}V(I_i)\in \mathcal{T}$
    \end{enumerate}
\end{prop}
\begin{proof} 
\begin{enumerate}
\item Take $V(\{0\} )=\{\mathfrak{p}\trianglelefteq R \text{ prime}: \{0\} \subseteq \mathfrak{p}\} $. But naturally, the additive identity is contained within every prime ideal (which must be an abelian group w.r.t $+$), so we have  $V(\{0\} )=Spec(R)$, and $Spec(R)\in \mathcal{T}$

Now take $V(R)=\{\mathfrak{p}\trianglelefteq R \text{ prime}: R\subseteq \mathfrak{p}\} $. But by definition, no prime ideal can contain $R$, so $V(R)=\varnothing$ and $\varnothing\in \mathcal{T}$
\item It is sufficient to show that $V(I)\cup V(J)\in \mathcal{T}$ for any $V(I),V(J)\in \mathcal{T}$. Any finite union can be proven inductively using this result.

\item Take 
\end{enumerate}
\end{proof}
\end{document}
