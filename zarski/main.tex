\documentclass{article}
\usepackage{amsmath}
\usepackage{amssymb}
\usepackage{amsthm}
\usepackage[utf8]{inputenc}
\usepackage{amsmath}
\usepackage{amsfonts}
\usepackage[]{graphicx}
\usepackage[a4paper, portrait, margin = 1in]{geometry}
\usepackage{enumitem}
\usepackage{xcolor}

%darkmode
%\pagecolor[rgb]{0.2,0.19,0.18} 
%\color[rgb]{0.92,0.86,0.7}

\newenvironment*{alphenum}{\begin{enumerate}[label= (\alph*)]}{\end{enumerate}}

\pagestyle{fancy}
\lhead{The Zarski Topology}
\rhead{Thomas Boyko}
\chead{}

\usepackage{thmbox}
\hfuzz=8pt

\theoremstyle{plain}% default
\newtheorem[L]{thm}{Theorem}[section]
\newtheorem[L]{lem}[thm]{Lemma}
\newtheorem[L]{prop}[thm]{Proposition}
%\newtheorem*{cor}{Corollary}
\theoremstyle{definition}
\newtheorem[M]{defn}{Definition}[section]
\newtheorem[M]{exmp}{Example}[section]
\theoremstyle{remark}
%\newtheorem*{rem}{Remark}
%\newtheorem*{note}{Note}
\newtheorem{case}{Case}

\begin{document}
\section{The Spectrum and its Topology}
\begin{defn}[Spectrum of a Ring]
Let $R$ be a ring. Define the spectrum of $R$,
\[Spec(R)=\{\mathfrak{p}\trianglelefteq R: \mathfrak{p} \text{ is prime}\} .\] 
\end{defn}
\begin{exmp}
    Describe the spectrum $Spec(\mathbb{Z}_n)$.
\end{exmp}
\begin{defn}[Zarski Topology]
    Upon $Spec(R)$ we define a topology by letting our closed sets be of the form:
    \[ V(I)=\{\mathfrak{p}\in Spec(R):I\subseteq \mathfrak{p}\} .\] 
    For an ideal $I$.
\end{defn}
\begin{prop}
    $V(I)$ induces a topology on $Spec(R)$.
    \begin{enumerate}
        \item $\varnothing,Spec(R)$ are closed.
        \item $\bigcup_{i=1}^{n}V(I_i)$ is closed for any $n\in \mathbb{N}$.
        \item $\bigcap_{\alpha}^{}V(I_i)$ for any $\alpha\in \Delta$, some indexing set.
    \end{enumerate}
\end{prop}
\begin{proof} 
\begin{enumerate}
\item Take $V(\{0\} )=\{\mathfrak{p}\trianglelefteq R \text{ prime}: \{0\} \subseteq \mathfrak{p}\} $. But naturally, the additive identity is contained within every prime ideal (which must be an abelian group w.r.t $+$), so we have  $V(\{0\} )=Spec(R)$, and $Spec(R)\in \mathcal{T}$

Now take $V(R)=\{\mathfrak{p}\trianglelefteq R \text{ prime}: R\subseteq \mathfrak{p}\} $. But by definition, no prime ideal can contain $R$, so $V(R)=\varnothing$ and $\varnothing\in \mathcal{T}$
\item It is sufficient to show that $V(I)\cup V(J)\in \mathcal{T}$ for any $V(I),V(J)\in \mathcal{T}$. Any finite union can be proven inductively using this result. We claim that $V(I)\cup V(J)=V(IJ)$.

    Suppose $\mathfrak{p}\in V(I)$ without loss of generality. Then $\mathfrak{p}$ is a prime ideal containing $I$. But we have $IJ\subseteq I\cap J\subseteq I\subseteq \mathfrak{p}$, so $\mathfrak{p}\in V(IJ)$.

    Conversely, suppose $\mathfrak{p}\in V(IJ)$. Then $\mathfrak{p}$ is a prime ideal containing $IJ$. If $J\subseteq \mathfrak{p}$, we are done, so suppose that $J\not\subseteq \mathfrak{p}$. Then take $i\in I,j\in J\setminus \mathfrak{p}$. We know that $ij\in IJ\subseteq \mathfrak{p}$, and then since $\mathfrak{p}$ is a prime ideal, either $i$ or $j$ must be in $\mathfrak{p}$. Since we supposed it was not $j$, we know it must be $i$. Therefore, $I\subseteq \mathfrak{p}$, and $\mathfrak{p}\in V(I)$.

\item We claim that $\bigcap_{\alpha} V(I_{\alpha})=V\left(\sum_{\alpha}^{} I_\alpha\right)$.

    If $\mathfrak{p}\in \bigcap_{\alpha} V(I_\alpha)$, then $P\supseteq I_\alpha$ for all our $\alpha$. But $\sum_{\alpha}^{} I_\alpha\supseteq I_\alpha$ for any fixed $\alpha$, so $P\supseteq \sum_{\alpha}^{} I_\alpha\supseteq I\alpha$, and $\mathfrak{p}\in V\left( \sum_{\alpha}^{} I_\alpha \right) $.

    Conversely, if $\mathfrak{p}\in V\left( \sum_{\alpha}^{} I_\alpha \right), \mathfrak{p} \supseteq \sum_{\alpha}^{} I_\alpha$. Fix some  $\beta$ arbitrary in $\Delta$, then we know already $\mathfrak{p}\supseteq \sum_{\alpha}^{} I_\beta\supseteq I_\beta $. So $\mathfrak{p}\in V(I_\beta)$ for any $\beta$, and $\mathfrak{p}\in \bigcap_{\alpha}V(I_\alpha) $.

\end{enumerate}
\end{proof}
\begin{exmp}
    Describe the Zarski Topology on $\mathbb{Z}_n$.
\end{exmp}
\paragraph{Solution: }We know already that $Spec(\mathbb{Z}_n)=\{\mathbb{Z}_d:d|n\} $
\begin{prop}
    If $R$ is a ring, then the closed points of $Spec(R)$ correspond to $V(M)=\{M\} $, for the ideals of $R$.
\end{prop}
\begin{proof} 
    Recall closed points in $\mathfrak{p}\in Spec(R)$ are those for which $\{\mathfrak{p}\}$ is closed.

    Suppose $M$ is some such point. Then for some $I\trianglelefteq R$, we have $V(I)=\{M\} $, and $I\subseteq M\subseteq R$. Suppose we have some $J\trianglelefteq R$ so that $M\subseteq J\subseteq R$. Then $I\subseteq J$, and we would have to have  $J\in \{M\} $. Therefore $M$ is maximal. 
    %TODO more needed i think
\end{proof}
\begin{prop}
    For any $I\trianglelefteq R$, $V(\sqrt{I} )=V(I)$
\end{prop}
\end{document}
