\documentclass{article}
\usepackage[most,many,breakable]{tcolorbox}
\usepackage{amsmath}
\usepackage{amssymb}
\usepackage{amsthm}
\usepackage[]{thmbox}
\usepackage{blindtext}
\usepackage[utf8]{inputenc}
\usepackage{amsmath}
\usepackage{amsfonts}
\usepackage[]{graphicx}
\usepackage[legalpaper, portrait, margin = 1in]{geometry}
\usepackage{enumitem}


\usepackage{xcolor}

%\pagecolor[rgb]{0.2,0.19,0.18} 
%\color[rgb]{0.92,0.86,0.7}

\newtheorem[L]{le}{Lemma}[subsection]
\newtheorem[L]{th}[le]{Theorem}
\newtheorem[L]{df}[le]{Definition}
\newtheorem[L]{ex}[le]{Example}
\newtheorem[L]{pf}[le]{Proof}


\newcommand{\nl}{\newline}

\newcommand{\real}{\mathbb{R}}
\newcommand{\complex}{\mathbb{C}}
\newcommand{\integer}{\mathbb{Z}}
\newcommand{\rational}{\mathbb{Q}}
\newcommand{\lxor}{\oplus}
\newcommand{\then}{\Rightarrow}
\pagestyle{fancy}
\lhead{Assignment \# $2$}
\rhead{Name: Thomas Boyko; UCID: 30191728}
\chead{}

\begin{document}
\begin{enumerate} 

\item Let $\phi : \mathbb{R}^n \to \mathbb{R}$ be such that $\phi(x) = 0 \Leftrightarrow x = 0$ and $\phi(\lambda x) = |\lambda| \phi(x), \forall x \in \mathbb{R}^n, \forall \lambda \in \mathbb{R}$  Show that if the set $B = \left\{x \in \mathbb{R}^n | \phi(x) \leq 1 \right\}$ is convex, then $\phi$ defines a norm on $\mathbb{R}^n$. 
\paragraph{Solution: }Non-degeneracy and scalar linearity are given from the definition of $\phi$. So all that is left to prove is the triangle inequality and non-negativity.

Non-negativity: Suppose $x\in \mathbb{R}^{n}$ is nonzero, and 

Let $x,y\in \mathbb{R}^{n},$ and take $r=\max \{\phi(x),\phi(y)\} $. Then $\frac{x}{r},\frac{y}{r}\in B$ 
\item Let E be a compact set in $\mathbb{R}^n$ and let F be a closed set in $\mathbb{R}^n$ such that $E \cap F = \emptyset$.  

\begin{enumerate}[label= (\alph*)] 
    \item Show that there exists d > 0 such that $\|x - y\| > d$, $\forall x \in E$ and $\forall y \in F$.  
        \paragraph{Solution: }Take $d=\inf_{x\in E,y\in F}\|x-y\|$. Clearly this is less than any $\|x-y\|$ for $x\in E$, $y\in F$, and it cannot be negative since the norm is positive. So then $d\geq 0$.For contradiction suppose $d=0$.
    \item  Does the result you proved in the previous question remain true if E and F are closed, but neither is compact? Justify your answer.  
        \paragraph{Solution: }
\end{enumerate}
\item Let $E = \left\{(x,y) | y = \sin\left( \frac{1}{x} \right), x > 0 \right\}$. Is E open? Is it closed? What are the accumulation points of E?  

    \paragraph{Solution: }This set is not open. Take an arbitrary ball of radius $r$ about the point $p=\left(\frac{1}{\pi},0\right)\in E$. Then the point $q=\left(\frac{1}{\pi},\frac{r}{2}\right)\in B_r(p)$, but $q\not\in E$ since $\sin$ is well-defined. So any ball about $p$ contains points not in $E$, and $E$ is not open.

    By continuity of $\sin$ and $\frac{1}{x}$, all points of $E$ are accumulation points. %TODO better justification needed for points of E

    The accumulation points of $E$ not contained in $E$ are of the form $(0,a)$ for $a\in [-1,1]$. Take one such point, and some $r>0$, and consider the $r$-ball about $(0,a)$. Choose $k\in \mathbb{N}$ so that $\frac{1}{2\pi k}<r$, and let $x=\frac{1}{2\pi k+\arcsin a}\leq \frac{1}{2\pi k}<r$. Then: 
    \begin{align*}
        \frac{1}{x}&= 2\pi k +\arcsin a \\
        \frac{1}{x}-2\pi k&= \arcsin a \\
        \sin\left( \frac{1}{x}-2\pi k \right) &= a \\
        \sin\left( \frac{1}{x}\right) &= a
    .\end{align*}

    Then the point $(x,a)$ is in $E$, and $\|(x,a)-(0,a)\|=\|(x,0)\|=\sqrt{x^2} =x<r$, so $x$ is in the arbitrary open ball we chose around $(0,a)$, and so every open ball around $p$ contains a distinct point in $E$, and as such $p$ is an accumulation point of $E$.

    Clearly none of these accumulation points can be in $E$ thanks to the condition $x>0$, so $E$ does not contain all its limit points and is not closed.
    %TODO proof reading

\item Let $f : \mathbb{R}^n \to \mathbb{R}$ be a function in $C^1(\mathbb{R}^n)$, i.e., $f, \partial_{x_1} f, ..., \partial_{x_n} f$ are continuous in $\mathbb{R}^n$. Suppose  $f(tx) = t f(x), \forall x \in \mathbb{R}^n, \forall t \in \mathbb{R}$  Show that f is a linear function.  
    \paragraph{Solution: }

\item Given $u : \mathbb{R} \to \mathbb{R}$ a function in $C^2(\mathbb{R})$, define $f : \mathbb{R}^2 \to \mathbb{R}$ by  $f(x,y) = \begin{cases} u(y) - u(x) & \text{if } y \neq x \\ u'(x) & \text{if } y = x \end{cases}$  Show that f is differentiable at any point $(a,a)$.  
    \paragraph{Solution: }

\item Let $f : \mathbb{R}^2 \to \mathbb{R}$ be a function that is defined in an open set $\Omega$ in $\mathbb{R}^2$.  Show that if $\partial_x f(x,y), \partial_y f(x,y)$ and $\partial_{xy} f(x,y)$ are continuous in $\Omega$, then $\partial_{yx} f(x,y)$ exists in $\Omega$ and we have  $\partial_{yx} f(x,y) = \partial_{xy} f(x,y), \forall (x,y) \in \Omega$  Hint: Consider the expression $\Delta(s,t) = f(a+s,b+t) - f(a+s,b) - f(a,b+t) + f(a,b)$.  
    \paragraph{Solution: }

\item Compute the degree 3 Taylor polynomial $T_3(x,x_2)$ of the function $f: \mathbb{R}^2 \to \mathbb{R}$, defined by $f(x_1, x_2) = \frac{4x_1 + 6x_2 - 1}{2x_1 + 3x_2}$ at the point $(-1,1).$
    \paragraph{Solution: }
\end{enumerate}
\end{document}
