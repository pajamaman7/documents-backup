\documentclass{article}
\usepackage{amsmath}
\usepackage{amssymb}
\usepackage{amsthm}
\usepackage[utf8]{inputenc}
\usepackage{amsmath}
\usepackage{amsfonts}
\usepackage[]{graphicx}
\usepackage[a4paper, portrait, margin = 1in]{geometry}
\usepackage{enumitem}
\usepackage{xcolor}

%darkmode
%\pagecolor[rgb]{0.2,0.19,0.18} 
%\color[rgb]{0.92,0.86,0.7}

\newenvironment*{alphenum}{\begin{enumerate}[label= (\alph*)]}{\end{enumerate}}

\pagestyle{fancy}
\lhead{Assignment \# $2$}
\rhead{Name: Thomas Boyko; UCID: 30191728}
\chead{}
%\hfuzz=45.002pt 
\begin{document}
\begin{enumerate} 

\item Let $\phi : \mathbb{R}^n \to \mathbb{R}$ be such that $\phi(x) = 0 \Leftrightarrow x = 0$ and $\phi(\lambda x) = |\lambda| \phi(x), \forall x \in \mathbb{R}^n, \forall \lambda \in \mathbb{R}$  Show that if the set $B = \left\{x \in \mathbb{R}^n | \phi(x) \leq 1 \right\}$ is convex, then $\phi$ defines a norm on $\mathbb{R}^n$. 
\paragraph{Solution: }Non-degeneracy and scalar linearity are given from the definition of $\phi$. So all that is left to prove is the triangle inequality and non-negativity.

Non-negativity: For the sake of contradiction, suppose that there exists some $x\in \mathbb{R}^{n}$ so that $\phi(x)<0$. Then let $n\in \mathbb{N}$. Take $\phi(nx)=n\phi(x)<0\leq 1$. So if we take the set $\{nx:n\in \mathbb{N}\} $, which is clearly unbounded, we see that it is contained in $B$. However this is a contradiction since $B$ is bounded. So $\phi(x)>0\forall x\neq 0$.

Triangle inequality: Let $x,y\in \mathbb{R}^{n}$, and take $\lambda= \frac{\phi(x)}{\phi(x)+\phi(y)}\leq 1$. Note that $\phi\left( \frac{x}{\phi(x)} \right) =\frac{\phi (x)}{\phi(x)}=1$, and likewise for $y$, so we have $\frac{x}{\phi(x)},\frac{y}{\phi(y)}\in B$ and may take:
\begin{align*}
    \phi\left( \lambda \frac{x}{\phi(x)}+(1-\lambda) \frac{y}{\phi(y)} \right) 
    &= \phi\left(  \frac{x\phi(x)}{\phi(x)(\phi(x)+\phi(y))}+ \frac{y\phi(y)}{\phi(y)(\phi(x)+\phi(y))}\right)  \\
    &= \phi\left( \frac{x+y}{\phi(x)+\phi(y)} \right)  \\
    &= \frac{\phi(x+y)}{|\phi(x)+\phi(y)|}\leq 1 \\
    \phi(x+y)&\leq\phi(x)+\phi(y)
.\end{align*}
And so the triangle inequality is satisfied.

\item Let E be a compact set in $\mathbb{R}^n$ and let F be a closed set in $\mathbb{R}^n$ such that $E \cap F = \emptyset$.  

\begin{enumerate}[label= (\alph*)] 
    \item Show that there exists $d > 0$ such that $\|x - y\| > d$, $\forall x \in E$ and $\forall y \in F$.  
        \paragraph{Solution: }Take $d=\inf_{x\in E,y\in F}\|x-y\|$. Clearly this is less than or equal to any $\|x-y\|$ for $x\in E$, $y\in F$, and it cannot be negative since the norm is positive. So then $d\geq 0$.For contradiction suppose $d=0$. 

        Since this is an $\inf$, we can find a sequence $\{x_n-y_n\} _{n\ge 1}$, with $x_{i}\in E,y_{i}\in F$. Since $\{x_{n}\} _{n\ge 1}$ is a sequence in the bounded set $E$, we can find a convergent subsequence, say $x_{n_k}\to x$, with $x\in E$ by closure of $E$. But since $\|x_{n_k}-y_{n_k}\|\to 0$, we must have $y_{n_k}\to x$, meaning $x\in F$, giving us the contradiction we sought ($E\cap F=\varnothing$). Therefore $d$ is positive, and to ensure strict inequality, we simply take $d'=\frac{d}{2}<d\leq\|x-y\|$ for any $x\in E,y\in F$.

    \item  Does the result you proved in the previous question remain true if $E$ and $F$ are closed, but neither is compact? Justify your answer.  
        \paragraph{Solution: }This does not remain true. Take the sequence $\{e_n\}_{n\geq 2}$ given by $e_n=n$, and $E$ as its image. Take $\{f_n\}_{n\geq 2} :f_n=e_n+\frac{1}{n}$ and $F$ as its image (These sets contain only isolated points, and are closed). Then for any $d>0$, we can pick $N\in \mathbb{N}:\frac{1}{N}<d$; and the points $e_N$ and $f_N$ will have $|e_N-f_N|=\frac{1}{N}<d$, meaning we can have arbitrarily close points between the closed sets $E,F$, and no such $d$ can exist.
\end{enumerate}
\newpage
\item Let $E = \left\{(x,y) | y = \sin\left( \frac{1}{x} \right), x > 0 \right\}$. Is E open? Is it closed? What are the accumulation points of E?  

    \paragraph{Solution: }This set is not open. Take an arbitrary ball of radius $r$ about the point $p=\left(\frac{1}{\pi},0\right)\in E$. Then the point $q=\left(\frac{1}{\pi},\frac{r}{2}\right)\in B_r(p)$, but $q\not\in E$ since $\sin$ is well-defined. So any ball about $p$ contains points not in $E$, and $E$ is not open.

    Clearly each point of $E$ is an accumulation point.

    The accumulation points of $E$ not contained in $E$ are of the form $(0,a)$ for $a\in [-1,1]$. Take one such point, and some $r>0$, and consider the $r$-ball about $(0,a)$. Choose $k\in \mathbb{N}$ so that $\frac{1}{2\pi k}<r$, and let $x=\frac{1}{2\pi k+\arcsin a}\leq \frac{1}{2\pi k}<r$. Then: 
    \begin{align*}
        \frac{1}{x}&= 2\pi k +\arcsin a \\
        \frac{1}{x}-2\pi k&= \arcsin a \\
        \sin\left( \frac{1}{x}-2\pi k \right) &= a \\
        \sin\left( \frac{1}{x}\right) &= a
    .\end{align*}

    Then the point $(x,a)$ is in $E$, and $\|(x,a)-(0,a)\|=\|(x,0)\|=\sqrt{x^2} =x<r$, so $x$ is in the arbitrary open ball we chose around $(0,a)$, and so every open ball around $p$ contains a distinct point in $E$, and as such $p$ is an accumulation point of $E$.

    Clearly none of these accumulation points can be in $E$ thanks to the condition $x>0$, so $E$ does not contain all its limit points and is not closed.

\item Let $f : \mathbb{R}^n \to \mathbb{R}$ be a function in $C^1(\mathbb{R}^n)$, i.e., $f, \partial_{x_1} f, ..., \partial_{x_n} f$ are continuous in $\mathbb{R}^n$. Suppose  $f(tx) = t f(x), \forall x \in \mathbb{R}^n, \forall t \in \mathbb{R}$  Show that f is a linear function.  
    \paragraph{Solution: }Take the partial derivative with respect to $x_{i}$ for some $1\leq i\leq n$.
    \begin{align*}
        f(tx)&= tf(x) \\
        \partial_if(tx)&=\partial_i tf(x) \\
        tf_{x_i}(tx)&= tf_{x_{i}}(x) \\
        f_{x_i}(tx)&= f_{x_{i}}(x)
    .\end{align*}
    But this must mean any partial $f_{x_{i}}$ is constant, and combined with $f(0)=f(0x)=0f(x)=0$, we can write that:
    \[
    f(x_1,\dots,x_n)=a_1x_1+\dots+a_nx_n
    .\] For real constants $a_1,\dots,a_n$, so $f$ is linear.

\newpage

\item Given $u : \mathbb{R} \to \mathbb{R}$ a function in $C^2(\mathbb{R})$, define $f : \mathbb{R}^2 \to \mathbb{R}$ by  $f(x,y) = \begin{cases} \frac{u(y) - u(x)}{y-x} & \text{if } y \neq x \\ u'(x) & \text{if } y = x \end{cases}$  Show that f is differentiable at any point $(a,a)$.  
    \paragraph{Solution: }Take the partial derivative with respect to $y$,
    \begin{align*}
        \frac{\partial f}{\partial y}(a,a) &= \lim_{h \to 0} \frac{f(a,a+h)-f(a,a)  }{h} \\
        &= \lim_{h \to 0} \frac{ \frac{u(a+h)-u(a)}{a+h-a}-u'(a)}{h} \\
        &= \lim_{h \to 0} \frac{ \frac{u(a+h)-u(a)}{h}-u'(a)}{h}
    .\end{align*}
    Before we get too far, let's do the same for $x$:
    \begin{align*}
        \frac{\partial f}{\partial x}(a,a) &= \lim_{h \to 0} \frac{f(a+h,a)-f(a,a)  }{h} \\
        &= \lim_{h \to 0} \frac{ \frac{u(a)-u(a+h)}{a-a-h}-u'(a)}{h} \\
        &= \lim_{h \to 0} \frac{ \frac{u(a+h)-u(a)}{h}-u'(a)}{h}
    .\end{align*}
    And so now we have $\frac{\partial f}{\partial x} (a,a)=\frac{\partial f}{\partial y} (a,a) $

    Now we apply l'Hopital in $h$, noticing that the first term on the numerator tends to $u'(a)$:
    \begin{align*}
        \lim_{h \to 0} \frac{ \frac{u(a+h)-u(a)}{h}-u'(a)}{h}
        &=\lim_{h \to 0} \frac{hu'(a+h)-u(a+h)+u(a)}{h^2}
    .\end{align*}
    And again,
    \begin{align*}
        \lim_{h \to 0} \frac{hu'(a+h)-u(a+h)+u(a)}{h^2}&=\lim_{h \to 0} \frac{hu''(a+h)+u'(a+h)-u'(a+h)}{2h}\\
        &= \lim_{h \to 0}  \frac{u''(a+h)}{2} \\
        &= \frac{u''(a)}{2}
    .\end{align*}
    Therefore:
    \[
        \frac{\partial f}{\partial x}(a,a) = \frac{u''(a)}{2}= \frac{\partial f}{\partial y}(a,a) 
    .\] 
    And thanks to $u\in C^2(\mathbb{R})$, both our partials exist and are continuous at any $(a,a)$ and therefore $f$ is differentiable at any $(a,a)$.

\newpage
\item Let $f : \mathbb{R}^2 \to \mathbb{R}$ be a function that is defined in an open set $\Omega$ in $\mathbb{R}^2$.  Show that if $\partial_x f(x,y), \partial_y f(x,y)$ and $\partial_{xy} f(x,y)$ are continuous in $\Omega$, then $\partial_{yx} f(x,y)$ exists in $\Omega$ and we have  $\partial_{yx} f(x,y) = \partial_{xy} f(x,y), \forall (x,y) \in \Omega$  Hint: Consider the expression $\Delta(s,t) = f(a+s,b+t) - f(a+s,b) - f(a,b+t) + f(a,b)$.  
    \paragraph{Solution: } Let $(x,y)\in \Omega$, with $t,s$ real and small enough that $(x+s,y+t)\in \Omega$ and write $g(y)=f(x+s,y)-f(x,y).$ Apply the Mean Value Theorem in the interval $(y,y+t)$ to obtain some $\mu\in (0,1)$ so that:
    \begin{align*}
        \Delta(s,t)&= \left( f(x+s,y+t)-f(x,y+t) \right) -(f(x,y)-f(x+s),y) \\
        &=  g(y+t)-g(y)\\
        &= \frac{d}{dy}g(y+\mu t)(y+t-y) \\
        &= t\frac{\partial }{\partial y} \left( f(x+s,y+\mu t)- f(x,y+\mu t)\right) 
    .\end{align*}
    Now we take the function $h(x)=f_y(x,y+\mu t)$, and we use MVT again in conjunction with our above expression to find some $\tau\in (0,1)$:
    \begin{align*}
        \Delta(s,t) &= t(f_y(x+s,y+\mu t)- f_y(x,y+\mu t))\\
                    &=t(h(x+s)-h(x))\\
                    &= t (x+s-x) \frac{d}{dx}\left( h(x+\tau s) \right)  \\
                    &= t s\frac{d}{dx}\left( f_y(x+\tau s,y+\mu t) \right)  \\
                    &= tsf_{yx}(x+\tau s,y+\mu t) \\
        \frac{\Delta(s,t)}{st}&= f_{yx}(x+\tau s,y+\mu t) 
    .\end{align*}
    If we perform the same process on $x$ before $y$, we will obtain some $\lambda,\theta$ so that:
    \[ \frac{\Delta(s,t)}{st}= f_{xy}(x+\theta s,y+\lambda t) .\] 
    Then we simply take the limit:
    \[
    f_{yx}(x,y)=\lim_{t,s \to 0} f_{yx}(x+\tau s,y+\mu t)=\lim_{t,s \to 0} \frac{\Delta(s,t)}{st}=\lim_{t,s \to 0} f_{xy}(x+\theta s,y+\lambda t)=f_{xy}(x,y)
    .\] 

    \newpage

\item Compute the degree 3 Taylor polynomial $T_3(x,x_2)$ of the function $f: \mathbb{R}^2 \to \mathbb{R}$, defined by $f(x_1, x_2) = \frac{4x_1 + 6x_2 - 1}{2x_1 + 3x_2}$ at the point $(-1,1).$
    \paragraph{Solution: } Begin by computing all necessary partials, and evaluating at $(-1,1)$: 
    \begin{align*}
        f(x_1,x_2   )&= \frac{4x_1+6x_2-1}{2x_1+3x_2},&f(-1,1)&=1 \\
        \frac{\partial f}{\partial x_1} &= \frac{2}{(2x_1+3x_2)^2},& \frac{\partial f}{\partial x_1}(-1,1)&=2 \\
        \frac{\partial f}{\partial x_2} &= \frac{3}{(2x_1+3x_2)^2},& \frac{\partial f}{\partial x_2}(-1,1)&=3 \\
        \frac{\partial^2 f}{\partial x_1x_2} &= \frac{-12}{(2x_1+3x_2)^3},& \frac{\partial^2 f}{\partial x_1x_2}(-1,1)&=-12 \\
        \frac{\partial^2 f}{\partial x_1x_1} &= \frac{-8}{(2x_1+3x_2)^3},& \frac{\partial^2 f}{\partial x_1x_1}(-1,1)&=-8 \\
        \frac{\partial^2 f}{\partial x_2x_2} &= \frac{-18}{(2x_1+3x_2)^3},& \frac{\partial^2 f}{\partial x_2x_2}(-1,1)&=-18 \\
        \frac{\partial^3 f}{\partial x_1x_1x_1} &= \frac{48}{(2x_1+3x_2)^4},& \frac{\partial^3 f}{\partial x_1x_1x_1}(-1,1)&=48 \\
        \frac{\partial^3 f}{\partial x_2x_2x_2} &= \frac{162}{(2x_1+3x_2)^4},& \frac{\partial^3 f}{\partial x_2x_2x_2}(-1,1)&=162 \\
        \frac{\partial^3 f}{\partial x_1x_2x_2} &= \frac{72}{(2x_1+3x_2)^4},& \frac{\partial^3 f}{\partial x_1x_2x_2}(-1,1)&=72 \\
        \frac{\partial^3 f}{\partial x_1x_1x_2} &= \frac{108}{(2x_1+3x_2)^4},& \frac{\partial^3 f}{\partial x_1x_1x_2}(-1,1)&=108
    .\end{align*}
    Then begin expanding the first three terms of the Taylor expansion
    \begin{align*}
        f((-1,1)+x)&=f(-1,1)+\left(\sum\limits_{k=1}^{3}\frac{1}{k!}\left( (x_1+1)\frac{\partial }{\partial x_1} +(x_2-1)\frac{\partial }{\partial x_2}  \right) ^{k}f(-1,1)\right)\\
        &= 1+\frac{1}{1!}\left( (x_1+1)\frac{\partial }{\partial x_1} +(x_2-1)\frac{\partial }{\partial x_2}  \right) ^{1}f(-1,1)\\
        &\quad+\frac{1}{2!}\left( (x_1+1)\frac{\partial }{\partial x_1} +(x_2-1)\frac{\partial }{\partial x_2}  \right) ^{2}f(-1,1)\\
        &\quad+\frac{1}{3!}\left( (x_1+1)\frac{\partial }{\partial x_1} +(x_2-1)\frac{\partial }{\partial x_2}  \right) ^{3}f(-1,1)\\
        &= 1+(x_1+1)\frac{\partial f}{\partial x_1}(-1,1) +(x_2-1)\frac{\partial f}{\partial x_2}(-1,1) \\
        &\quad+\frac{1}{2}\bigg( (x_1+1)^2\frac{\partial^2 }{\partial x_1x_1} +2 (x_1+1)(x_2-1)\frac{\partial }{\partial x_1x_2}\\
        &\quad\quad+(x_2-1)^2\frac{\partial^2 }{\partial x_2x_2} \bigg)f(-1,1)  \\
        &\quad+\frac{1}{6}\bigg( (x_1+1)^3\frac{\partial^3 }{\partial x_1x_1x_1}+3(x_1+1)^2(x_2-1) \frac{\partial^3 }{\partial x_1x_1x_2}  \\
        &\quad\quad +3(x_1+1)(x_2-1)^2 \frac{\partial^3 }{\partial x_1x_2x_2}+(x_2-1)^3\frac{\partial^3 }{\partial x_2x_2x_2}  \bigg)\, f(-1,1)\\
        &= 1+2(x_1+1)+3(x_2-1) \\
        &\quad+\frac{1}{2}\left( -8(x_1+1)^2 -24 (x_1+1)(x_2-1)-18(x_2-1)^2\right) \\
        &\quad+\frac{1}{6}\bigg( 48(x_1+1)^3+3\cdot 108(x_1+1)^2(x_2-1) \\&\quad+3\cdot 72(x_1+1)(x_2-1)^2 +162(x_2-1)^3 \bigg)\\
        &= 2x_1-3x_2 -4(x_1+1)^2 -12 (x_1+1)(x_2-1)-9(x_2-1)^2 \\
        &\quad+ 8(x_1+1)^3+54(x_1+1)^2(x_2-1) \\&\quad+36(x_1+1)(x_2-1)^2 +27(x_2-1)^3 
    .\end{align*}
    %TODO how much to simplify??

\iffalse
\item Consider the function $f:\mathbb{R}^3\rightarrow \mathbb{R}^3$, defined by
$f(x,y,z) = (x^3-y-z, 2x+y+z, x+y-z)$

\begin{enumerate}[label= (\alph*)] 
\item Compute $Jf(x,y,z)$ and show that $\partial f_{(x,y,z)}$ is invertible for any $(x,y,z)\in\mathbb{R}^3$.

\item Find the largest open $U\subset \mathbb{R}^3$ where $f$ has a continuously differentiable inverse function $g$.
\end{enumerate}

\item Consider the system of equations:
$\left(\mathcal{S}\right) \quad 
\begin{cases}
x - y - u^2 + v^2 &= 0 \\
x + y - 2uv &= 0
\end{cases}$

\begin{enumerate}[label= (\alph*)] 
\item Show that the system $(\mathcal{S})$ can be solved for $u,v$ in term of $(x,y)$ near the point $(x,y,u,v)=(1,1,1,1)$.

\item Compute $\partial_x u(1,1)+\partial_y v(1,1)$
\end{enumerate}

\item Let $f:\mathbb{R}\to\mathbb{R}$ and $g:\mathbb{R}\to\mathbb{R}$ be such that $f(1)=g(1)=0$, and consider the system \begin{equation*}
\begin{cases}
f(xy)+g(yz) & = 0 \\
g(xy)+f(yz) & = 0
\end{cases}
\end{equation*}
Find conditions on $f$ and $g$ that guarantee the system $(S)$ can be uniquely solved for $y$ and $z$ as functions of $x$ near the point $(x,y,z)=(1,1,1)$.

\item Let $A$ be a positive definite $n\times n$ matrix. Interpreting $x\in\mathbb{R}^n$ as a column matrix, show that $\|.\|:\mathbb{R}^n\to\mathbb{R}$ defined by $\|x\|^2=x^tAx$ is a norm on $\mathbb{R}^n.$

    \paragraph{Solution: } %
    Non-negativity follows from the positive semi-definite condition; since $x^{T}Ax>0,$ we must have $\|x\|=\sqrt{x^{T}Ax} >0$ whenever $x\neq 0$.

    Non-degeneracy follows from the same condition, $x^{T}Ax=0\iff \sqrt{x^{T}Ax} =0\iff x=0$.

    Scalar linearity is quick; $\|\alpha x\|=\sqrt{(\alpha x)^{T}A(\alpha x)}=\sqrt{\alpha^2 x^{T}Ax} =|\alpha|\sqrt{x^{T}Ax} =|\alpha|\|x\| $.

    Finally, the triangle inequality. 

    \begin{align*}
        \|x+y\|^2&= (x+y)^{T}A(x+y)\\
                 &=(x^{T}+y^{T})(Ax+Ay) \\
                 &= x^{T}Ax+y^{T}Ax+x^{T}Ay+y^{T}Ay \\
                 &= \|x\|^2+\|y\|^2+x^{T}Ay+y^{T}Ax \\
    .\end{align*}
\fi

\end{enumerate}
\end{document}
