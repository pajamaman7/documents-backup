\documentclass{article}
\usepackage{amsmath}
\usepackage{amssymb}
\usepackage{amsthm}
\usepackage[utf8]{inputenc}
\usepackage{amsmath}
\usepackage{amsfonts}
\usepackage[]{graphicx}
\usepackage[a4paper, portrait, margin = 1in]{geometry}
\usepackage{enumitem}
\usepackage{xcolor}

%darkmode
%\pagecolor[rgb]{0.2,0.19,0.18} 
%\color[rgb]{0.92,0.86,0.7}

\newenvironment*{alphenum}{\begin{enumerate}[label= (\alph*)]}{\end{enumerate}}

\pagestyle{fancy}
\lhead{Assignment \# $1$}
\rhead{Name: Thomas Boyko; UCID: 30191728}
\chead{}

\begin{document}
\begin{enumerate} 

\item Let $\{f_n\}$ be the sequence of functions defined by
  \[
  f_n=\begin{cases}
    nx &\text{If } 0\leq x\leq \frac{1}{n}\\
    1 &\text{If } \frac{1}{n}<x<1-\frac{1}{n}\\
    n-nx &\text{If } 1-\frac{1}{n}\leq x \leq 1
  \end{cases}
  .\] 

  \begin{enumerate}
    \item Find the pointwise limit $f$ of the sequence.
      \paragraph{Solution: }Proceed by cases. If $x=0$, then the first case of the function will always be
      taken since $0\leq x$. So $f_n(0)=n0=0$. Likewise if $x=1$, then $f(1)=n-n 1=n-n=0$.

      %TODO weak reasoning
      Now, if $x\in (0,1)$, then we observe that $\frac{1}{n}\to 0$, and $1-\frac{1}{n}\to 1$. Therefore the middle case of our piecewise function gives us $f(x)=1$ for all $x$ in this open interval. 

    \item Does $f_n\xrightarrow[{[0,1]}]{c.u}f$? Justify your answer.
      \paragraph{Solution: }This sequence is not uniformly continuous. Pick $\varepsilon=\frac{1}{3}$, and let $N\in \mathbb{N}$, and $n>N$. Pick $x=\frac{1}{2n}$ so that $0\leq x\leq \frac{1}{n}$, and then $f_n(x)=nx=\frac{n}{2n}=\frac{1}{2}$. Then: $\left| f_n(x)-f(x) \right| = \left| \frac{1}{2}-1 \right|=\frac{1}{2}>\varepsilon$.

        Therefore the sequence is not uniformly continuous.

  \end{enumerate}

\item Let $f_n(x)=\left( \cos\left( \frac{2x}{n}\right)  \right)^{n^2}$

  \begin{enumerate}
    \item Compute the pointwise limit $f$ of the sequence $\{f_n\} $.
    \item Show that $f_n\xrightarrow[{[0,1]}]{c.u}f$.
  \end{enumerate}

\item Let $a\in \mathbb{R}_{+}$. Compute the limit
  \[
  \lim_{n \to \infty} \int_{a}^{\pi} \frac{\sin(nx)}{nx} \, d x 
  .\] 
  What happens if $a=0$?

  We begin by considering our sequence of functions within the integral, each of which is a quotient of continuous functions, and is itself continuous (for all but $x=0$). Call this $g_n(x) =\frac{\sin(nx)}{nx}$. Note that since $-1\leq\sin(nx)\leq 1$, we can find (for nonzero $x$) that $-\frac{1}{nx}\leq g_n(x)\leq \frac{1}{nx}$. Both the sequences bounding $g$ have a zero limit at infinity, so by the squeeze theorem on sequences of functions, we can say that for nonzero $x$, $g_n\to 0$.
  %TODO x=0 case, show if each g_n is integrable?
  Now since we have already shown that our sequence $g_n$ is bounded, and since each $g_n$ is integrable, we can say:

  \begin{align*}
  \lim_{n \to \infty} \int_{a}^{\pi} \frac{\sin(nx)}{nx} \, d x 
  &=\int_{a}^{\pi} \lim_{n \to \infty} \frac{\sin(nx)}{nx} \, d x \\
  &=\int_{a}^{\pi} 0 \, d x \\
  &= 0-0 \\
  &= 0
  .\end{align*}

\item Construct a sequence of functions defined in $[0,1]$, each of which is discontinuous at every point of $[0,1]$ and which converges uniformly to a function that is continuous at every point 

  \paragraph{Solution: }Take the series $\{f_n\} $ defined by:
  \[
    f_n(x)=\begin{cases}
      \frac{1}{n}&\text{If } x\in \mathbb{Q}\\
      0&\text{Otherwise}
    \end{cases}
  \] 
  %TODO Discontinuous everywhere?
  \paragraph{Claim:}  $\{f_n\} $ converges uniformly to the constant function $0$, which we know to be continuous on the real line, as well as $[0,1]$. 

  %TODO is this archimedes???
  Let $\varepsilon>0$, and choose (By Archimedian Principle), $N$ such that $0<\frac{1}{N}<\varepsilon$.
  Then any $n\geq N$ will have $0<\frac{1}{n}<\frac{1}{N}<\varepsilon$. Now by cases, if $x\in \mathbb{Q}$, then we have 
  \[
  |f_n(x)-f(x)|=\left|\frac{1}{n}-0\right|=\frac{1}{n}<\frac{1}{N}<\varepsilon
  .\] 
  And for $x\not\in \mathbb{Q}$, 
\[
  |f_n(x)-f(x)|=\left|0-0\right|=0<\varepsilon
.\] 
  Therefore $\{f_n\} $ is a sequence of functions which are continuous nowhere, convergent to the zero function which is continuous everywhere. 

\item Consider the series of functions $\sum_{n\geq 1}^{} \frac{x}{n(n+x)}$.
  \begin{enumerate}
    \item Show that the series converges uniformly in the interval $[0,b]$ for any $b>0$.

        %Bounded above by $\frac{b}{n^2}$
        \paragraph{Solution:} 
        \begin{align*}
            \frac{x}{n(n+x)}&= \frac{x}{n^2+nx} \\
                            &\leq \frac{x}{n^2}\\
                            &\leq \frac{b}{n^2}
        .\end{align*}
        Define $u_n=\frac{b}{n^2}$, then by the Weierstrass Comparison test, since $\sum_{n\geq 1}^{} u_n$ is convergent as a $p$-series with $p=2$, $\frac{x}{n(n+x)}\leq u_n$, this series must converge.

    \item Let $F(x)=\sum_{n\geq 1}^{} \frac{x}{n(n+x)}$. Show that $F'(x)=\sum_{n\geq 1}^{} \frac{1}{n(n+x)^2}$, $x\geq 0$.
        %TODO is it as easy as using the linearity of the derivative on the partial sums? Or the Cauchy m to n? 

  \end{enumerate}
\item Consider the series of functions $\sum_{n\geq 1}^{} \frac{x}{1+n^2x^2}$. Show that the series doesn't converge uniformly in $\mathbb{R}_+$.

  \textbf{Hint:} You could start by showing that $\frac{x}{1+n^2x^2}\geq \int_{n}^{n+1} \frac{x}{1+t^2x^2} \, d t, \quad \forall x\in \mathbb{R}$.
  %Consider refinements?
\end{enumerate}
\end{document}
