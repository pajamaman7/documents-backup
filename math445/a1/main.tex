\documentclass{article}
\usepackage{amsmath}
\usepackage{amssymb}
\usepackage{amsthm}
\usepackage[utf8]{inputenc}
\usepackage{amsmath}
\usepackage{amsfonts}
\usepackage[]{graphicx}
\usepackage[a4paper, portrait, margin = 1in]{geometry}
\usepackage{enumitem}
\usepackage{xcolor}

%darkmode
%\pagecolor[rgb]{0.2,0.19,0.18} 
%\color[rgb]{0.92,0.86,0.7}

\newenvironment*{alphenum}{\begin{enumerate}[label= (\alph*)]}{\end{enumerate}}

\pagestyle{fancy}
\lhead{Assignment \# $1$}
\rhead{Name: Thomas Boyko; UCID: 30191728}
\chead{}

\begin{document}
\paragraph{Student's note:} A couple times in this assignment, I used a result from the Conway text, the squeeze theorem for sequences of functions. Upon review, I realised that the proof of this fact was not covered in class. So here I present my own proof of this fact. 

\paragraph{Claim:} Let  $f_n,g_n,h_n:I\to \mathbb{R}$ be sequences of functions such that $g_n(x)\leq f_n(x)\leq h_n(x)$ for any $n\in \mathbb{N}$ and $x\in I$, and $g_n,h_n\xrightarrow[{I}]{c.u}f$, Then $f_n\xrightarrow[{I}]{c.u}f$
\begin{proof} 
    Let $f_n,g_n,h_n$ be as above and $\varepsilon>0$. Then there exist $N_1,N_2:$ $n_1>N_1\implies |g_n(x)-f(x)|<\varepsilon$ and $n_2<N_2\implies |h_n(x)-f(x)|<\varepsilon$ for any $x\in I $. Take $N=\max \{N_1,N_2\} $. Then we get the pair of double inequalities;
    \[
    f(x)-\varepsilon < h_n(x)<f(x)+\varepsilon,\quad
    f(x)-\varepsilon < g_n(x)<f(x)+\varepsilon
    .\] 
    And combining this with our other assumption,
    \[
    f(x)-\varepsilon <g_n(x)\leq f_n(x)\leq h_n(x)<f(x)+\varepsilon\implies|f_n(x)-f(x)<\varepsilon
    .\] 
    And therefore $f_n$ converges uniformly to $f$ on $I$.
\end{proof}
\begin{enumerate} 

\item Let $\{f_n\}$ be the sequence of functions defined by
  \[
  f_n=\begin{cases}
    nx &\text{If } 0\leq x\leq \frac{1}{n}\\
    1 &\text{If } \frac{1}{n}<x<1-\frac{1}{n}\\
    n-nx &\text{If } 1-\frac{1}{n}\leq x \leq 1
  \end{cases}
  .\] 

  \begin{enumerate}
    \item Find the pointwise limit $f$ of the sequence.
      \paragraph{Solution: }Proceed by cases. If $x=0$, then the first case of the function will always be
      taken since $0\leq x$. So $f_n(0)=n0=0$. Likewise if $x=1$, then $f(1)=n-n 1=n-n=0$.

      Now, if $x\in (0,1)$, then we observe that $\frac{1}{n}\to 0$, and $1-\frac{1}{n}\to 1$. Therefore the middle case of our piecewise function gives us $f(x)=1$ for all $x$ in this open interval. 

    \item Does $f_n\xrightarrow[{[0,1]}]{c.u}f$? Justify your answer.
      \paragraph{Solution: }This sequence is not uniformly convergent. Pick $\varepsilon=\frac{1}{3}$, and let $N\in \mathbb{N}$, and $n>N$. Pick $x=\frac{1}{2n}$ so that $0\leq x\leq \frac{1}{n}$, and then $f_n(x)=nx=\frac{n}{2n}=\frac{1}{2}$. Then: $\left| f_n(x)-f(x) \right| = \left| \frac{1}{2}-1 \right|=\frac{1}{2}>\varepsilon$.

        Therefore the sequence is not uniformly convergent.

  \end{enumerate}

  \newpage
\item Let $f_n(x)=\left( \cos\left( \frac{2x}{n}\right)  \right)^{n^2}$

  \begin{enumerate}
    \item Compute the pointwise limit $f$ of the sequence $\{f_n\} $.
        \textbf{Hint:} Use the following double inequalities:
    \[ 1-\frac{1}{2}t^2\leq \cos t\leq 1-\frac{1}{2}t^2+\frac{1}{24}t^4,\quad \forall t\in \mathbb{R} .\] 
        \[ -t-t^2\leq\ln(1-t)\leq-t,\quad\forall t\in \left[0,\frac{1}{2}\right] .\] 
        \paragraph{Solution: }Begin with the first inequality. 
        \begin{align*}
            1-\frac{4x^2}{2n^2}&\leq \cos \left(   \frac{2x}{n}\right)\leq 1-\frac{4x^2}{2n^2}+\frac{16x^4}{24n^{4}}\\
            \ln\left(  1-\frac{2x^2}{n^2}\right) &\leq\ln \left(  \cos \left(   \frac{2x}{n}\right)\right) \leq\ln\left( 1-\left( \frac{2x^2}{n^2}-\frac{2 x^4}{3n^{4}}\right)  \right) &\ln\text{ is increasing in }\mathbb{R}\\
            -\frac{2x^2}{n^2}-\frac{4x^4}{n^4}&\leq \ln \left(  \cos \left(   \frac{2x}{n}\right)\right)\leq-\frac{2x^2}{n^2}+\frac{2x^4}{3n^{4}}&\text{From the second inequality}\\
            -{2x^2}-\frac{4x^4}{n^2}&\leq n^2 \ln \left(  \cos \left(   \frac{2x}{n}\right)\right)\leq{-2x^2}+\frac{2x^4}{3n^{2}}\\
            -{2x^2}-\frac{4x^4}{n^2}&\leq \ln \left( \left( \cos \left(   \frac{2x}{n}\right)\right)^{n^2}\right)\leq{-2x^2}+\frac{2x^4}{3n^{2}}\\
            \exp\left( -{2x^2}-\frac{4x^4}{n^2} \right) &\leq \left( \cos \left(   \frac{2x}{n}\right)\right)^{n^2}\leq\exp\left( -{2x^2}+\frac{2x^4}{3n^{2}} \right) &\exp\text{ is increasing in }\mathbb{R}
        .\end{align*}
        Intuitively, according to squeeze theorem, it appears that the limit will become $\exp\left( e^{-2x^2} \right) $, however this idea needs some formalizing.
    \item Show that $f_n\xrightarrow[{[0,1]}]{c.u}f$.
        \paragraph{Lemma:} If $ f_n\xrightarrow[{[0,1]}]{c.u}f$, and $g:\mathbb{R}\to \mathbb{R}$ is uniformly continuous, then $ g\circ f_n\xrightarrow[{[0,1]}]{c.u}g\circ f$.

        Since $f_n$ is uniformly convergent, it must be uniformly bounded, say that $f_n<M$ for all $n\in \mathbb{N}$ and $x\in [0,1]$. Let $\varepsilon>0$, and by uniform continuity of $g,$ there exists some $\delta$ such that $|a-b|<\delta\implies|g(a)-g(b)|<\varepsilon$.

        Now, take $N\in \mathbb{N}$ so that $n>N\implies|f_n(x)-f(x)|<\delta$.

        Therefore $|g(f_n(x))-g(f(x))|<\varepsilon$, and $ g\circ f_n\xrightarrow[{[0,1]}]{c.u}g\circ f$.

        \paragraph{Solution: } For both our functions of the form $h_c(x)=-2x^2+ \frac{cx^{4}}{n^2}$, we know that they are continuous on the compact set $[0,1]$, and so they must be bounded on that interval, say by $M_c$. The exponential is continuous, so it must be uniformly continuous on the compact set $[-M_c,M_c]$, which contains $\exp(h_c([0,1]))$.  Since $-2x^2+\frac{2x^{4}}{3n^2}\xrightarrow[{[0,1]}]{c.u}{-2x^2} $, and $-2x^2- \frac{4x^{4}}{n^2}\xrightarrow[{[0,1]}]{c.u}{-2x^2} $, then by the above lemma both the bounds found for $f_n$ in part (a) must converge uniformly to  $\exp\left( -2x^2 \right) $. Then by squeeze theorem: \[ f_n\xrightarrow[{[0,1]}]{c.u}e^{-2x^2} .\] 

  \end{enumerate}

\item Let $a\in \mathbb{R}_{+}$. Compute the limit
  \[ \lim_{n \to \infty} \int_{a}^{\pi} \frac{\sin(nx)}{nx} \, d x .\] 
  What happens if $a=0$?

  \paragraph{Solution: }We begin by considering our sequence of functions within the integral, each of which is a quotient and composition of continuous functions, and is itself continuous (for all but $x=0$). Call this $g_n(x) =\frac{\sin(nx)}{nx}$. Note that since $-1\leq\sin(nx)\leq 1$, we can find (for nonzero $x$) that $-\frac{1}{nx}\leq g_n(x)\leq \frac{1}{nx}$. Both the sequences bounding $g$ have a zero limit at infinity, so by the squeeze theorem on sequences of functions, we can say that for nonzero $x$, $g_n\to 0$.
  Now since we have already shown that our sequence $g_n$ is bounded, and since each $g_n$ is integrable, we can say:

  \begin{align*}
  \lim_{n \to \infty} \int_{a}^{\pi} \frac{\sin(nx)}{nx} \, d x 
  &=\int_{a}^{\pi} \lim_{n \to \infty} \frac{\sin(nx)}{nx} \, d x \\
  &=\int_{a}^{\pi} 0 \, d x \\
  &= 0-0 \\
  &= 0
  .\end{align*}
  %
  \paragraph{a=0:} Intuitively, since each function is not defined at 0, which will be the endpoint of integration, the supremum or infimum of the function over the first range in any partition $[0,x_1]$ will remain zero for sufficiently large $n$. 

  \newpage 
\item Construct a sequence of functions defined in $[0,1]$, each of which is discontinuous at every point of $[0,1]$ and which converges uniformly to a function that is continuous at every point 

  \paragraph{Solution: }Take the series $\{f_n\} $ defined by:
  \[
    f_n(x)=\begin{cases}
      \frac{1}{n}&\text{If } x\in \mathbb{Q}\\
      0&\text{Otherwise}
    \end{cases}
  \] 
  \paragraph{Claim:}  $\{f_n\} $ converges uniformly to the constant function $0$, which we know to be continuous on the real line, as well as $[0,1]$. 

  Let $\varepsilon>0$, and choose $N$ such that $0<\frac{1}{N}<\varepsilon$.
  Then any $n\geq N$ will have $0<\frac{1}{n}<\frac{1}{N}<\varepsilon$. Now by cases, if $x\in \mathbb{Q}$, then we have 
  \[
  |f_n(x)-f(x)|=\left|\frac{1}{n}-0\right|=\frac{1}{n}<\frac{1}{N}<\varepsilon
  .\] 
  And for $x\not\in \mathbb{Q}$, 
\[
  |f_n(x)-f(x)|=\left|0-0\right|=0<\varepsilon
.\] 
Now we must show that each of these functions is continuous nowhere in $[0,1]$.  Suppose by way of contradiction that $f_n$ is continuous at some $c\in [0,1]$. Then for $\varepsilon=\frac{1}{n+1}$, there must be some $\delta$ such that if $|x-c|<\delta$, $|f_n(x)-f_n(c)|<\frac{1}{n+1}$. Take $B_\delta(c)$ the $\delta$-ball about $c$, and proceed by cases on $c$.
\paragraph{$c\in \mathbb{Q}$:} If $c$ is rational, find some $d\not\in\mathbb{Q} $ inside $B_{\delta}(c)$. Then we will have $f_n(c)=\frac{1}{n}$ and $f_n(d)=0$
\paragraph{$c\not\in \mathbb{Q}$:} If $c$ is irrational, find some $d\in\mathbb{Q}$ inside $B_{\delta}(c)$. Then we will have $f_n(d)=\frac{1}{n}$ and $f_n(c)=0$

Regardless of case, we will get $|f_n(c)-f_n(d)|=\frac{1}{n}>\frac{1}{n+1}$, and we have found our contradiction.

  Therefore $\{f_n\} $ is a sequence of functions which are continuous nowhere, convergent to the zero function which is continuous everywhere. 

  \newpage
\item Consider the series of functions $\sum_{n\geq 1}^{} \frac{x}{n(n+x)}$.
  \begin{enumerate}
    \item Show that the series converges uniformly in the interval $[0,b]$ for any $b>0$.

        %Bounded above by $\frac{b}{n^2}$
        \paragraph{Solution:} 
        \begin{align*}
            \frac{x}{n(n+x)}&= \frac{x}{n^2+nx} \\
                            &\leq \frac{x}{n^2}\\
                            &\leq \frac{b}{n^2}
        .\end{align*}
        Define $u_n=\frac{b}{n^2}$, then by the Weierstrass Comparison test, since $\sum_{n\geq 1}^{} u_n$ is convergent as a $p$-series with $p=2$, $\frac{x}{n(n+x)}\leq u_n$, this series must converge.

    \item Let $F(x)=\sum_{n\geq 1}^{} \frac{x}{n(n+x)}$. Show that $F'(x)=\sum_{n\geq 1}^{} \frac{1}{(n+x)^2}$, $x\geq 0$.
        \paragraph{Solution: }Begin by considering the derivative of the partial sums, using the linearity of the derivative.
        \[
        \frac{d}{dx}\sum_{k=1}^{n} \frac{x}{k(x+k)}=\sum_{k=1}^{n} \frac{d}{dx} \frac{x}{k(x+k)}=\sum_{k=1}^{n} \frac{1}{(k+x)^2}
        .\] 
        Then, since for $x\geq 0,$ $ \frac{1}{(x+k)^2}\leq \frac{1}{k^2}$, a convergent $p$-series, this series must converge uniformly. Since this term-differentiated sequence of partial sums converges, we can say that $F'(x)=\sum_{n\geq 1}^{} \frac{1}{(n+x)^2}$.
        %TODO perhaps there's a better way to cite this result
        
        \iffalse
        \begin{align*}
            F'(x)&= \frac{d}{dx}\left(  \sum_{k=1}^{\infty} \frac{x}{k(k+x)}\right)  \\
                &= \frac{d}{dx}\left( \lim_{n \to \infty} \sum_{k=1}^{n} \frac{x}{k(k+x)}\right)  \\
                &= \lim_{n \to \infty}\left( \frac{d}{dx}\sum_{k=1}^{n} \frac{x}{k(k+x)}  \right)\\
                &= \lim_{n \to \infty} \sum_{k=1}^{n} \frac{d}{dx}\frac{x}{k(k+x)}  \\
                &= \lim_{n \to \infty} \sum_{k=1}^{n} \frac{k(x+k)-kx}{k^2(k+x)^2}  \\
                &= \lim_{n \to \infty} \sum_{k=1}^{n} \frac{k}{k^2(k+x)^2}  \\
                &= \lim_{n \to \infty} \sum_{k=1}^{n} \frac{1}{k(k+x)^2}  \\
                &= \sum_{k=1}^{\infty} \frac{1}{k(k+x)^2} 
        .\end{align*}
    \fi

  \end{enumerate}
  \newpage
\item Consider the series of functions $\sum_{n\geq 1}^{} \frac{x}{1+n^2x^2}$. Show that the series doesn't converge uniformly in $\mathbb{R}_+$.

  \textbf{Hint:} You could start by showing that $\frac{x}{1+n^2x^2}\geq \int_{n}^{n+1} \frac{x}{1+t^2x^2} \, d t, \quad \forall x\in \mathbb{R}$.

  \paragraph{Solution: }Begin with the hint. If we take $P_{0}=\{n,n+1\} $, the trivial partition on $[n,n+1]$, then we will have the upper sum:
  \[
       U\left(P_0,\frac{x}{1+t^2x^2}\right)=\sum_{k=1}^{1} \sup_{t\in [n,n+1]}\left( \frac{x}{1+t^2x^2} \right) ((n+1)-n)\\
       = \frac{x}{1+n^2x^2} \\
  .\] 
  But from the definition of the Riemann integral, we have (For $\sigma$ the set of all partitions of $[n,n+1]$):
  \begin{align*}
      \int_{n}^{n+1} \frac{x}{1+t^2x^2} \, d t=&\inf_{P\in \sigma}U\left(P,\frac{x}{1+t^2x^2}\right)\\
                        &\leq U\left( P_0,\frac{x}{1+t^2x^2} \right) \\
      &= \frac{x}{1+n^2x^2}
  .\end{align*}
  Suppose by way of contradiction that the series does converge uniformly. Then there exists some $m$ such that, for any $x\in \mathbb{R}$
  \[
  \left| \sum_{n=1}^{\infty} \frac{x}{1+n^2x^2}-\sum_{n=1}^{m} \frac{x}{1+n^2x^2} \right| <\frac{1}{2}
  .\] 

  However,
  \begin{align*}
      \left| \sum_{n=1}^{\infty} \frac{x}{1+n^2x^2}-\sum_{n=1}^{m} \frac{1}{1+n^2x^2} \right| &= \left| \sum_{n=m+1}^{\infty} \frac{x}{1+x^2n^2} \right|  \\
                  &=\sum_{n=m+1}^{\infty} \frac{x}{1+x^2n^2}\\
                  &\geq \sum_{n=m+1}^{\infty} \int_{n}^{n+1} \frac{x}{1+t^2x^2} \, d t \\
                    &= \int_{m+1}^{\infty} \frac{x}{1+t^2x^2} \, d t &\text{Let }u=tx \\
                    &= \int_{(m+1)x}^{\infty} \frac{1}{1+u^2} \, d u &du=xdt \\
                    &= \left( \arctan u \right)_{u=(m+1)x}^{\infty} \\
                    &= \frac{\pi}{2}-\arctan ((m+1)x)\\
                    &= \frac{\pi}{2}-\arctan(1) &\text{Pick }x=\frac{1}{m+1} \\
                    &= \frac{\pi}{2}-\frac{\pi}{4} \\
                    &= \frac{\pi}{4}\\
                    & >\frac{1}{2}
  .\end{align*}
A contradiction, so our series cannot converge uniformly.

\end{enumerate}
\end{document}
