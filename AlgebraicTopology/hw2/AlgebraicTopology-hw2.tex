\documentclass{article}
\usepackage{amsmath}
\usepackage{amssymb}
\usepackage{amsthm}
\usepackage[utf8]{inputenc}
\usepackage{amsmath}
\usepackage{amsfonts}
\usepackage[]{graphicx}
\usepackage[a4paper, portrait, margin = 1in]{geometry}
\usepackage{enumitem}
\usepackage{xcolor}

%darkmode
%\pagecolor[rgb]{0.2,0.19,0.18} 
%\color[rgb]{0.92,0.86,0.7}

\newenvironment*{alphenum}{\begin{enumerate}[label= (\alph*)]}{\end{enumerate}}

\pagestyle{fancy}
\lhead{Assignment \# $2$}
\rhead{Thomas Boyko}
\chead{}

\usepackage{pgfplots}
\pgfplotsset{compat=1.18}

\begin{document}
\begin{exe}
    
    In the surface $M_g$ of genus $g$, let $C$ be a circle that separates $G_g$ into two compact subsurfaces $M'_h$ and $M_k'$ obtained from the closed surfaces $M_h$ and $M_k$ by deleting an open disk from each. Show that $M'_h$ does not retract onto its boundary circle $C$ and hence $M_g$ does not retract onto $C$. [Hint: abelianize $\pi_1$.] But show that $M_g$ does retract onto the nonseparating circle $C'$ in the figure (Hatcher).
\end{exe}
\paragraph{Solution: }
% These Mhk are subcomplexes with boundary on one of the 1-cells

%Build Mg differently as a cw complex by adding a 1-cell loop?

Begin with the computation of the fundamental group of the punctured surface of genus $h$, $M_h'$. Recall the construction of $M_h$ consisted of a single $0$-cell, $2h$ $1$-cells, and a single $2$-cell. Puncturing the $2$-cell with a hole allows a retract to the boundary of the cell. After gluing the boundary to the 1-cells, we will just be left with the $1$-cells, and we will have a retract onto $\bigvee^{2h}S^{1}$. And since we know the fundamental group of the wedge sum, we are left with: 
\[
    \pi_1(S^{1})=\overbrace{\pi_1(S^{1})*\dots*\pi_1(S^{1})}^{2h\text{ times}}=\overbrace{\mathbb{Z}*\dots* \mathbb{Z} }^{2h\text{ times}}
.\] 
Now suppose for the sake of contradiction that $M_h'$ did retract onto the boundary circle $C\simeq S^{1}$, for some retraction $r$. Then the homomorphism induced by the inclusion $\iota_*:\pi_1(S^1)\to \pi_1(M_h')$ would be injective, and the composition $r_*\iota_*$ would be identity. Then, applying $\pi_1$ and then $ab$,
\[\xymatrix{
        \pi_1(S^1)\ar@/_1pc/[rr]_{id_*}\ar@{->>}[r]^{\iota_*}&\pi_1(M_h')\ar@{->}[r]^{r_*}&\pi_1(S^{1})\\
    \mathbb{Z}\ar@/_1.5pc/[rr]_{id_*^{ab}}\ar@{->>}[r]^{\iota_*^{ab}}&\mathbb{Z}^{2h} \ar@{->}[r]^{r_*^{ab}}&\mathbb{Z}
}\]

However, since $\iota_*$ maps the generator of $\mathbb{Z}$ to some commutator $aba^{-1}b^{-1}$, and abelianization kills all the commutators through the quotient, we must have $\iota_*^{ab}$ identically zero (homomorphisms from a cyclic group are determined by their evaluation on generator). This contradicts $r_*^{ab}\iota_*^{ab}=id_*^{ab}$, and so $M_k'$ cannot retract onto $C$.


Now for $C'$. Early in Hatcher we identified $M_g$ as a $4g$ sided polygon, with sides 
\[
a_1b_1a^{-1}_1b^{-1}_1\dots a_gb_ga_g^{-1}b_g^{-1}
.\] 
(in that order). Take the quotient $M_g / \sim $, where $~$ is defined by $a_i\sim a_1$ and $b_i\sim b_1$ for all $i$. The quotient map  $q:M_g\to M_g /\sim =M_1$ then gives a retract of $M_g$ to $M_1$, the torus.

We further retract onto the circle $C'$, by viewing $M_1$ as the typical square with sides $aba^{-1}b^{-1}.$ Construct this retract $r$ by retracting a point in the square to the closest point in $a$. This is clearly continuous except perhaps at the line $a^{-1}$. However this is not a problem since $a^{-1}$ is identified with $a$ in the construction of the torus.

By composing the retracts,
\[\xymatrix{
        M_g\ar@/_1pc/[rr]_{rq}\ar@{->}[r]^{q}&M_1\ar@{->}[r]^{r}&C'
}\]
We have our retract $M_g\to C'$

\end{document}
