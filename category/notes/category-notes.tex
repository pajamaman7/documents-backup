\documentclass{article}
\usepackage{amsmath}
\usepackage{amssymb}
\usepackage{amsthm}
\usepackage[utf8]{inputenc}
\usepackage{amsmath}
\usepackage{amsfonts}
\usepackage[]{graphicx}
\usepackage[a4paper, portrait, margin = 1in]{geometry}
\usepackage{enumitem}
\usepackage{xcolor}

%darkmode
%\pagecolor[rgb]{0.2,0.19,0.18} 
%\color[rgb]{0.92,0.86,0.7}

\newenvironment*{alphenum}{\begin{enumerate}[label= (\alph*)]}{\end{enumerate}}

\pagestyle{fancy}
\renewcommand{\textbullet}{\ensuremath{\bullet}}
\lhead{Category Theory Notes}
\rhead{Thomas Boyko}
\chead{}

\begin{document}
\section{Categories}
\begin{defn}
    A Category is a collection of objects, together with a collection of morphisms satisfying:
    \begin{itemize}
        \item Each morphism $f$ has domain and codomain objects, $f:x\to y$
        \item For any object $x$, there exists an identity object $1_x$, satisfying $1_x x=x$.
        \item For any pair of morphisms, $f:x\to y$, $g:y\to z,$ there exists a composite morphism, $gf:x\to z$. This also introduces a law of associativity, $g(fx)=(gf)x$.    
    \end{itemize}
\end{defn}
\begin{exmp}
    \begin{itemize}
        \item $\mathrm{Set}$ forms a category, with its objects being sets, and morphisms being functions between sets.
        \item  $\mathrm{Top}$ has topological spaces as its objects and continuous functions as its morphisms.
        \item  $\mathrm{Group}$ has Groups as objects and Group Homomorphisms as morphisms. This example lent the general term “morphisms” to the data of an abstract category. The categories $\mathrm{Ring}$ of associative and unital Rings and Ring Homomorphisms and $\mathrm{Field}$ of Fields and field homomorphisms are defined similarly.
    \end{itemize}
\end{exmp}
\begin{exmp}
    Keep in mind that categories need not have objects as sets. Though many familiar categories will deal only with sets and functions, our only requirement for a category is that we have objects and morphisms.
    \begin{itemize}
        \item Take a poset $\mathrm{P}$. This forms a category, with objects of $P$ being objects in the category, and if $x\leq y$, we have a morphism $x\to y$. Transitivity allows for composition, and identity is guaranteed from reflexivity.
        \item A group defines a category $\mathrm{B}G$, with a single object. Morphisms are the elements of $G$, which can be composed, are associative, and we have an identity in any group $G$.
    \end{itemize}
    \iffalse
    \begin{rem}
        It may assist the reader to consider the single object in the category of a group as an object invariant under the action of the group. So consider a square, under the action of $D_4$. Every morphism of $\mathrm{B}D_4$ (element of $D_4$) leaves the square as it is, 
    \end{rem}
\fi
\end{exmp}

\end{document}
