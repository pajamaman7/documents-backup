\documentclass{article}
\usepackage[all]{xy}
\usepackage[most,many,breakable]{tcolorbox}
\usepackage{amsmath}
\usepackage{amssymb}
\usepackage{amsthm}
\usepackage[]{thmbox}
\usepackage{blindtext}
\usepackage[utf8]{inputenc}
\usepackage{amsmath}
\usepackage{amsfonts}
\usepackage[]{graphicx}
\usepackage[legalpaper, portrait, margin = 1in]{geometry}
\usepackage{enumitem}


\usepackage{xcolor}

%\pagecolor[rgb]{0.2,0.19,0.18} 
%\color[rgb]{0.92,0.86,0.7}

\newtheorem[L]{le}{Lemma}[subsection]
\newtheorem[L]{th}[le]{Theorem}
\newtheorem[L]{df}[le]{Definition}
\newtheorem[L]{ex}[le]{Example}
\newtheorem[L]{pf}[le]{Proof}


\newcommand{\nl}{\newline}

\newcommand{\real}{\mathbb{R}}
\newcommand{\complex}{\mathbb{C}}
\newcommand{\integer}{\mathbb{Z}}
\newcommand{\rational}{\mathbb{Q}}
\newcommand{\lxor}{\oplus}
\newcommand{\then}{\Rightarrow}
\pagestyle{fancy}
\lhead{Week \# $2$}
\rhead{Name: Thomas Boyko; UCID: 30191728}
\chead{}

\begin{document}
\begin{enumerate} 
    \item Find three examples of functors not mentioned above.

    \item Show that functors preserve isomorphism. That is, prove that if $F:\mathscr{A}\to \mathscr{B}$ is a functor and $A,A'\in \mathscr{A}$ with $A\cong A'$, then $F(A)\cong F(A')$.
        \begin{proof} 
            Suppose $F:\mathscr{A}\to \mathscr{B} $ is a functor, and $A\cong A'$ in $\mathscr{A}.$ Then there exists a pair of morphisms $f:A\to A'$ and $g:A'\to A$ with $fg=1_{A'}$ and $gf=1_A$. And, the functor $F$ gives another pair of morphisms $Ff,Fg$. Verify:
            \[
            ( Ff)( Fg)= F(fg)=F1_{A'}=1_{FA'}
            \] 
            and likewise:
            \[
                (Fg)(Ff)=F(gf)=F 1_A=1_{FA}
            .\] 
            And so we have $FA\cong FA'$. 
        \end{proof}

    \item Two categories $\mathscr{A}$ and $\mathscr{B}$ are isomorphic, written as $\mathscr{A}\cong \mathscr{B}$, if they are isomorphic as objects of $\mathrm{Cat}$.
        \begin{enumerate}
            \item Let $G$ be a group, regarded as a one-object category all of whose maps are isomorphisms. Then its opposite $G^{op}$ is also a one-object category all of whose maps are isomorphisms, and can therefore be regarded as a group too. What is $G^{op},$ in purely group-theoretic terms? Prove that $G$ is isomorphic to $G^{op}$.
                \begin{proof} 
                Take the functors $F:G\to G^{op}$, and $F':G^{op}\to G$. Define, for $g\in G$ and $h^{op}\in G^{op}$:
                \[
                F(g)=(g^{-1})^{op},\quad \quad F'(h^{op})=h^{-1}
                .\] 
                We first check that these functors compose to identity:
                \begin{align*}
                    FF'(g^{op})&= F(g^{-1}) \\
                    &= ((g^{-1})^{-1})^{op} \\
                    &= g^{op} \\
                    F F'&= 1_{G^{op}} \\
                    F'F(g)&= F'((g^{-1})^{op} )\\
                    &= (g^{-1})^{-1} \\
                    &= g \\
                    F'F&=1_{G}
                .\end{align*}
                And then we check that these mappings are indeed functors. Clearly $F,F'$ map the single object in $G$ to $G^{op}$, and vice versa. Then we check the morphism identities for $F$ and $F'$. Let $g,h \in G$;
                \begin{align*}
                    F(gh)&= \left( \left( gh \right) ^{-1} \right) ^{op} \\
                    &= \left( h^{-1} g ^{-1} \right) ^{op} \\
                    &= \left( g^{-1} \right)^{op}\left( h^{-1} \right) ^{op} \\
                    &= F(g)F(h)
                .\end{align*}
                Then, if $g^{op},h^{op}\in G^{op}$;
                \begin{align*}
                    F'(g^{op}h^{op})&= F'( (hg)^{op}) \\
                    &= (hg)^{-1} \\
                    &= g^{-1}h^{-1} \\
                    &= F(g^{op})F(h^{op})
                .\end{align*}
                And all that is left to verify is that $F,F'$ send identities to identities. Let $g\in G$, and $g^{op}\in G^{op}$. We wish to show that $F(1_G)= (1_G)^{op} =1_{G^{op}}$, and that $F'(1_{G^{op}}) =1_G$. Take $g^{op}\in G^{op}$, which we know to have a preimage $g^{-1}$ under $F$.
                \begin{align*}
                    (1_G)^{op}g^{op}&= F(1_G) g^{op}\\
                    &= F(1_G)F(g^{-1}) \\
                    &= F(1_Gg^{-1}) \\
                    &= F(g^{-1}) \\
                    &= g^{op} 
                .\end{align*}
                And so $1_{G^{op}}=(1_G)^{op}=F(1_G)$ (Since identity of right composition follows from the same argument). Now for $g\in G$,
                \begin{align*}
                    F'(1_{G^{op}})&= F'((1_G)^{op}) \\
                    &= 1_G^{-1} \\
                    &=1_G
                .\end{align*}
                So $F$ and $F'$ are functors which serve as inverses for one another, and $G\cong G^{op}$.
                \end{proof}
            \item Find a monoid which is not isomorphic to its opposite.
                \paragraph{Solution: }Take $\mathbb{N}$, %TODO
        \end{enumerate}

    \item Of the functors appearing in this section, which are faithful and which are full?

    \item Give an example of a functor that is full, faithful, both, and neither.
        \paragraph{Solution: }
        \begin{enumerate}
            \item The forgetful functor $F:\mathrm{CRing}\to \mathrm{Ring}$ that forgets commutativity is faithful, for distinct commutative rings will necessarily map to distinct rings. However it is not full; there exist rings which are not commutative ($\mathbb{M}_2(\mathbb{R})$)
            \item For a full but not faithful functor, we can take the categorical representation of the trivial group, and a functor $F:\mathrm{Set}\to \{e\} $, which maps every $X\in \mathrm{Set}$ to the single object, and morphisms map to the identity.
            \item A functor which is neither full nor faithful, we take $F:\mathrm{Set}\to  \mathrm{Set}$ defined by $F(X)=\varnothing$ for any $X\in \mathrm{Set}$, and $F(f)=1_{\varnothing}$ for any morphism in $\mathrm{Set}$
            \item The functors in the group exercise is both full and faithful, being bijections between the set of morphisms in $G$ and $G^{op}$.
        \end{enumerate}

    \item Let $A$ and $B$ be sets, and denote $B^{A}$ the set of functions from $A$ to $B$. Write down:
        \begin{enumerate}
            \item a canonical function $A\times B^A\to B$;
                \paragraph{Solution: }The God-Given function from $A\times B^{A}\to B$ is the function $f:A\times B^{A}\to B$, given by $f(a,g)=g(a)$ where $g:A\to B$.
            \item a canonical function $A \to B^{(B^A)}$.
                \paragraph{Solution: }The God-Given function from $A\to B^{(B^{A})}$ is the function $h(a)$, which for any $a\in A$ corresponds to a function $\mathrm{ev}_a$, which takes a function from $A \to B$ and outputs its value at $a$. That is, $h(a)$ gives the evaluation map on $B^{A}$ at $a$.
        \end{enumerate}

    \item In this exercise, you will prove Proposition 1.3.18. Let $F:\mathscr{A}\to \mathscr{B}$ be a functor.
        \begin{enumerate}
            \item Suppose that $F$ is an equivalence. Prove that $F$ is full, faithful and essentially surjective on objects. (Hint: prove faithfulness before fullness.)
                \begin{proof} 
                    Suppose that $F$ is an equivalence. Then there exists a functor $G:\mathscr{B}\to \mathscr{A}$, and natural isomorphisms $\eta:1_\mathscr{A}\to GF$ and $\varepsilon:FG\to 1_{\mathscr{B}}$.
                    \begin{multicols}{2}
                        \[ \begin{codi}
                            \obj{A&&A'\\|(GA)| GF(A)&&|(GAA)| GF(A')\\};
                            
                            \mor A 1_\mathscr{A}f=f:-> A' ;
                            \mor A' \eta_{A'}:->  GAA; 
                            \mor GA GF(f):-> GAA ;
                            \mor A \eta_{A}:->  GA; 
                            
                        \end{codi} \] 

                        \[ \xymatrix{
                        F(A) \ar[r]^{F(f)} \ar[d]_{\alpha_A}    &
                        F(A') \ar[d]^{\alpha_{A'}}      \\
                        G(A) \ar[r]_{G(f)}      &
                        G(A')
                        } .\] 
                    \end{multicols}
                    \paragraph{Faithfulness:} 
                \end{proof}
            \item Now suppose instead that $F$ is full, faithful and essentially surjective on objects. For each $B\in \mathscr{B}$, choose an object $G(B)$ of $\mathscr{A}$ and an isomorphism $\varepsilon_B : F(G(B)) \to B$. Prove that $G$ extends to a functor in such a way that $(\varepsilon_B)_{B\in \mathscr{B}}$ is a natural isomorphism $FG \to 1_B$. Then construct a natural isomorphism $1_A \to GF$, thus proving that $F$ is an equivalence.
        \end{enumerate}
    \item Kristaps' favorite:  If you understood the "groupoid with one object" example, determine what functors between two such groupoids correspond to in terms of groups.  Then, determine what natural transformations correspond to.
        \paragraph{Solution: }Let $G,H$ be groupoids with one object, and $\phi,\psi$ be functors $G\to H$. We know already for any elements $g,g'$ of the group $G$ (Morphisms in the sigle object category),
        \[
        \phi(gg')=\phi(g)\phi(g')
        .\] 
        Which we know already to be the identity required by a group homomorphism. Then let $\alpha$ be a natural transformation:
        \[ \xymatrix{
        \mathrm{A} \rtwocell^F_G{\alpha} &\mathrm{B}
        }\] 
\end{enumerate}
\end{document}
