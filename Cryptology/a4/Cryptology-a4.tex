\documentclass{article}
\usepackage[most,many,breakable]{tcolorbox}
\usepackage{amsmath}
\usepackage{amssymb}
\usepackage{amsthm}
\usepackage[]{thmbox}
\usepackage{blindtext}
\usepackage[utf8]{inputenc}
\usepackage{amsmath}
\usepackage{amsfonts}
\usepackage[]{graphicx}
\usepackage[legalpaper, portrait, margin = 1in]{geometry}
\usepackage{enumitem}


\usepackage{xcolor}

%\pagecolor[rgb]{0.2,0.19,0.18} 
%\color[rgb]{0.92,0.86,0.7}

\newtheorem[L]{le}{Lemma}[subsection]
\newtheorem[L]{th}[le]{Theorem}
\newtheorem[L]{df}[le]{Definition}
\newtheorem[L]{ex}[le]{Example}
\newtheorem[L]{pf}[le]{Proof}


\newcommand{\nl}{\newline}

\newcommand{\real}{\mathbb{R}}
\newcommand{\complex}{\mathbb{C}}
\newcommand{\integer}{\mathbb{Z}}
\newcommand{\rational}{\mathbb{Q}}
\newcommand{\lxor}{\oplus}
\newcommand{\then}{\Rightarrow}
\pagestyle{fancy}
\lhead{Assignment \# $4$}
\rhead{Thomas Boyko}
\chead{}

\usepackage{pgfplots}
\pgfplotsset{compat=1.18}

\begin{document}
\begin{exe}
For a bit string $X$, let $\bar{X}$ denote the complement of $\bar{X}$, that is, the string obtained by flipping all bits in $X$. Show that for any plaintext block $X$ and DES key $K$, it holds that if $Y = DES_K(X)$, then $\bar{Y} = DES_{\bar{K}}( \bar{X}).$ 
\end{exe}
DES encryption is a composition of a number of functions. If we can show that each of these functions has the property $\overline{F_K(X)}=F_{\bar{K}} (\bar{X})$ then we can infer that the whole encryption function will have the same property.

We work through the diagram of a single cycle in $DES_{\bar{K}} (\bar{X})$:
\[\xymatrix{
    &\bar{X}\ar@{->}[dr]^{}\ar@{->}[dl]_{}\\
        \bar{L}\ar@{->}[d]^{}&\bar{K}\ar@{->}[d]^{}&\bar{R}\ar@/^1pc/[lldd]^{}\ar@{->}[ld]^{}\\
    \oplus \ar@{->}[drr]^{}  &f_{\bar{K}}\ar@{->}[l]^{}\\
        L'&&R'
}\]
Clearly the projections of $\bar{X}$ onto the left and right halves will maintain the complement, as will the switching of the halves at the end. It's known as well that $\overline{A\oplus  B}=\bar{A}\oplus  \bar{B}$. So all that is left to show is that ${f}_{\bar{K}}(\bar{X})=\overline{f_K(X)}$. From the definition:
\[
    f_{\bar{K}}(\bar{R})=P(S(\bar{K}\oplus  E(\bar{R})))
.\] 
The first function we apply is $E$, which copies the input, duplicating a few select bits. So if a bit is flipped before being input, it will be copied and duplicated the same way. So we have $E(\bar{R})=\overline{E(R)}$.
\[
    f_{\bar{K}}(\bar{R})=P(S(\bar{K}\oplus  \overline{E(R)}))
.\] 
And, as already discussed, the operation $\oplus  $ maintains the complement;

\[
    f_{\bar{K}}(\bar{R})=P(S(\overline{{K}\oplus  E(R)}))
.\] 
Finally, we see that $S,P$ behave nicely with complements. The division of a bitstring into blocks, and the permutation of the blocks both do nothing to the bits themselves, only to their ordering.
\[
    f_{\bar{K}}(\bar{R})=\overline{P(S({K}\oplus  E(R)))}=\overline{f_K(R)}
.\] 
And so we have our desired result.


\begin{exe}
    Also show that, given a chosen plaintext attack where you may ask for the encryption of 2 plaintexts, you can use this property to do exhaustive key search in half the time it would normally take.
\end{exe}

Suppose the oracle chooses some key $K$. Choose any arbitrary plaintext $X$, and request the encryptions of $X$ and $X'$. Then begin brute force encrypting $X$ with each $K_i$, being sure to keep track and never try $\bar{K}_i$ for any $i$ we previously checked. After encrypting, we check each ciphertext $C_i$ against $E_K(X)$ and $\bar{C}_i$ against $\overline{E_K(X)}$. This cuts down half the keys needed to try.

\end{document}
