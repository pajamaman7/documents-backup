\documentclass{article}
\usepackage{amsmath}
\usepackage{amssymb}
\usepackage{amsthm}
\usepackage[utf8]{inputenc}
\usepackage{amsmath}
\usepackage{amsfonts}
\usepackage[]{graphicx}
\usepackage[a4paper, portrait, margin = 1in]{geometry}
\usepackage{enumitem}
\usepackage{xcolor}

%darkmode
%\pagecolor[rgb]{0.2,0.19,0.18} 
%\color[rgb]{0.92,0.86,0.7}

\newenvironment*{alphenum}{\begin{enumerate}[label= (\alph*)]}{\end{enumerate}}

\pagestyle{fancy}
\lhead{Assignment \# $6$}
\rhead{Thomas Boyko}
\chead{}

\begin{document}
\begin{exe}
    Show that for any $x\in \mathbb{Z}_n$, we have $D_{n,d}(E_{n,e}(x)) \equiv  x\pmod{n}$.\end{exe}
\paragraph{Solution: }
Let $x\in \mathbb{Z}_n$. Then, as per the RSA specification, we have
\[
ed\equiv 1\pmod{\varphi(n)} 
.\] 
So:
\[
\phi(n)|ed-1
.\] 
And there exists $k\in \mathbb{Z}$ so that:
\[
k\phi(n)=ed-1
,\] 
\[
k\phi(n)+1=ed
.\] 
By Fermat's little theorem (or Lagrange theorem if you like),
\begin{align*}
    x^{p-1}&\equiv 1\pmod{p} ,&
    x^{q-1}&\equiv 1\pmod{q} \\
    x^{(p-1)(q-1)}&\equiv 1\pmod{p} ,&
    x^{(p-1)(q-1)}&\equiv 1\pmod{q} \\
    x^{k\varphi(n)}&\equiv 1\pmod{p} ,&
    x^{k\varphi(n)}&\equiv 1\pmod{q} \\
    x^{k\varphi(n)+1}&\equiv 1\pmod{p} ,&
    x^{k\varphi(n)+1}&\equiv 1\pmod{q} \\
    x^{ed}&\equiv x\pmod{p} ,&
    x^{ed}&\equiv x\pmod{q} 
.\end{align*}
Then by the Chinese remainder theorem, we must have
\[
x^{ed}\equiv x\pmod{pq} 
.\] 

\end{document}
