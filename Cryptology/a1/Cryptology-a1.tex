\documentclass{article}
\usepackage[most,many,breakable]{tcolorbox}
\usepackage{amsmath}
\usepackage{amssymb}
\usepackage{amsthm}
\usepackage[]{thmbox}
\usepackage{blindtext}
\usepackage[utf8]{inputenc}
\usepackage{amsmath}
\usepackage{amsfonts}
\usepackage[]{graphicx}
\usepackage[legalpaper, portrait, margin = 1in]{geometry}
\usepackage{enumitem}


\usepackage{xcolor}

%\pagecolor[rgb]{0.2,0.19,0.18} 
%\color[rgb]{0.92,0.86,0.7}

\newtheorem[L]{le}{Lemma}[subsection]
\newtheorem[L]{th}[le]{Theorem}
\newtheorem[L]{df}[le]{Definition}
\newtheorem[L]{ex}[le]{Example}
\newtheorem[L]{pf}[le]{Proof}


\newcommand{\nl}{\newline}

\newcommand{\real}{\mathbb{R}}
\newcommand{\complex}{\mathbb{C}}
\newcommand{\integer}{\mathbb{Z}}
\newcommand{\rational}{\mathbb{Q}}
\newcommand{\lxor}{\oplus}
\newcommand{\then}{\Rightarrow}
\pagestyle{fancy}
\lhead{Assignment \# $1$}
\rhead{Thomas Boyko}
\chead{}

\begin{document}

We are given the ciphertext: 


\begin{verbatim}
    
KQEREJEBCPPCJCRKIEACUZBKRVPKRBCIBQCARBJCVFCUPKRIOFKPACUZQEPBKRXPEIIEAB
DKPBCPFCDCCAFIEABDKPBCPFEQPKAZBKRHAIBKAPCCIBURCCDKDCCJCIDFUIXPAFFERBIC
ZDFKABICBBENEFCUPJCVKABPCYDCCDPKBCOCPERKIVKSCPICBRKIJPKABI

\end{verbatim}

A simple internet tool gives us the letter distribution:

\begin{verbatim}
        C-32
        B-21
        K-20
        P-20
        I-16
        E-13
        A-13
        R-12
        F-10
        D-9
        J-6
        U-6
        Q-4
        Z-4
        V-4
        O-2
        X-2
        H-1
        N-1
        Y-1
        S-1
\end{verbatim}

We are instructed to follow the English distributions, so we suppose that the function $E_k$ maps E to C, and we must take a guess which of B,K,P is mapped to T. After trying to solve the associated systems of congruences, we find that only one system produces a valid plaintext:

\begin{align*}
    19a+b&\equiv 1\pmod{26}  \\
    4a+b&\equiv 2\pmod{26}
\end{align*}

Subtracting the equivalences, we get:
\[
11a\equiv 1\pmod{26} 
.\] 
And we solve for $a\equiv 19\pmod{26} $ using the Extended Euclidean Algorithm. Substituting back into our original congruences, we find $b\equiv 4\pmod{26} $. Then we can use an online decryption tool to decrypt our ciphertext:

\begin{verbatim}
ocanadaterredenosaieuxtonfrontestceintdefleuronsglorieuxcartonbrassaitporter
lepeeilsaitporterlacroixtonhistoireestuneepopeedesplusbrillantsexploitsettav
aleurdefoitrempeeprotegeranosfoyersetnosdroits
\end{verbatim}

A Canadaian elementary student would be ashamed to not recognize the French national anthem!
\end{document}
